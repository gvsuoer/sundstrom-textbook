\section*{Section \ref{S:compositionoffunctions}}

\begin{list}{\bf{\ref{exer64:notcommutative}.}}
\item $( {g \circ h} ):\mathbb{R} \to \mathbb{R}$  by  
$( {g \circ h} )( x ) = g( {h( x )} ) = g\!\left( {x^3 } \right) = 3x^3  + 2$.

$( {h \circ g} ):\mathbb{R} \to \mathbb{R}$  by  
$( {h \circ g} )( x ) = h( {g( x )} ) = h( {3x + 2} ) = ( {3x + 2} )^3 $.

\noindent
This shows that $h \circ g \ne g \circ h$ or that composition of functions is not commutative.
\end{list}


\begin{list}{\bf{\ref{exer:sec64-2}.}}
\item \begin{enumerate}
\item $F\left( x \right) = \left( {g \circ f} \right)\left( x \right)$, where 
$f\left( x \right) = e^x$  and $g\left( x \right) = \cos x$.

\item $G\left( x \right) = \left( {g \circ f} \right)\left( x \right)$ where 
$f\left( x \right) = \cos x$ and $g\left( x \right) = e^x $.

\item $H\left( x \right) = \left( {g \circ f} \right)\left( x \right)$, 
$f\left( x \right) = \sin x$, $g\left( x \right) = \frac{1}{x}$.

\item $K\left( x \right) = \left( {g \circ f} \right)\left( x \right)$, 
$f\left( x \right) = e^{-x^2}$, $g\left( x \right) = \cos x$.
\end{enumerate}
%\item \begin{list}{\bf{(a)}}
%\item $F( x ) = ( {g \circ f} )( x )$, $f( x ) = e^x$ , $g( x ) = \cos x$
%\end{list}
%\end{list}
%
%\begin{list}{}
%\item \begin{list}{\bf{(b)}}
%\item $G( x ) = ( {g \circ f} )( x )$, $f( x ) = \cos x$, $g( x ) = e^x $
%\end{list}
\end{list}


\vskip6pt
\begin{list}{\bf{\ref{exer:sec64-4}.}}
\item \begin{list}{\bf{(a)}}
\item For each  $x \in A$, $( {f \circ I_A } )( x ) = f( {I_A ( x )} ) = f( x )$.  Therefore, \\$f \circ I_A  = f$.
\end{list}
\end{list}



\vskip6pt
\begin{list}{\bf{\ref{exer:sec64-3}.}}
\item \begin{list}{\bf{(a)}}
\item $\left[ {( {h \circ g} ) \circ f} \right]( x ) = \sqrt[3]{{\sin ( {x^2 } )}}$; 
$\left[ {h \circ ( {g \circ f} )} \right]( x ) = \sqrt[3]{{\sin ( {x^2 } )}}$.  This proves that $(h \circ g) \circ f = h \circ (g \circ f)$ for these particular functions.
\end{list}
\end{list}


\begin{list}{\bf{\ref{exer:sec64-5}.}}
\item Start of a proof:  Let  $A$, $B$, and  $C$  be nonempty sets and let  $f\x A \to B$  and  
$g\x B \to C$.  Assume that  
$f$  and  $g$  are both injections.  Let  $x, y \in A$ and assume that  
$( {g \circ f} )( x ) = ( {g \circ f} )( y )$.  
\end{list}


\begin{list}{\bf{\ref{exer:sec64-8}.}}
\item \begin{list}{\bf{(a)}}
\item $f\x \mathbb{R} \to \mathbb{R}$ by $f ( x ) = x$, 
$g: \mathbb{R} \to \mathbb{R}$ by $g ( x ) = x^2$.  Then \\$g \circ f: \R \to \R$ by $(g \circ f)(x) = x^2$.  The function $f$ is a surjection, but $g \circ f$ is not a surjection.
\end{list}
\end{list}

\begin{list}{}
\item \begin{list}{\bf{(b)}}
\item $f: \mathbb{R} \to \mathbb{R}$ by $f \left( x \right) = x$, 
$g: \mathbb{R} \to \mathbb{R}$ by $g \left( x \right) = x^2$.  Then \\$g \circ f: \R \to \R$ by $(g \circ f)(x) = x^2$.  The function $f$ is an injection, but $g \circ f$ is not an injection.
\end{list}
\end{list}


\begin{list}{}
\item \begin{list}{\bf{(f)}}
\item By Part~(\ref{T:morecompositefunctions1}) of Theorem~\ref{T:morecompositefunctions}, this is not possible since if $g \circ f$ is an injection, then $f$ is an injection.
\end{list}
\end{list}
\hbreak


\endinput


