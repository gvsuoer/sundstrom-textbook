\subsection*{The Principle of Mathematical Induction}
Although we proved that Statement~(2) is false, in this text, we will not prove that Statement~(1) is true.  One reason for this is that we really do not have a formal definition of the natural numbers.  However, we should be convinced that Statement~(1) is true.  We resolve this by making Statement~(1) an axiom for the natural numbers so that this becomes one of the defining characteristics of the natural numbers.
\begin{flushleft}
\fbox{\parbox{4.68in}{
\textbf{The Principle of Mathematical Induction} \\
\index{Principle of Mathematical Induction}%
\index{mathematical induction!Principle}%
If  $T$  is a subset of  $\mathbb{N}$ such that

\begin{enumerate}
  \item $1 \in T$\!, and
  \item For every  $k \in \mathbb{N}$, if  $k \in T$\!, then  
$\left( {k + 1} \right) \in T$\!,
\end{enumerate}
then  $T = \mathbb{N}$.
}}
\end{flushleft}
%\hbreak
%
\subsection*{Using the Principle of Mathematical Induction}
The primary use of the Principle of Mathematical Induction is to prove statements of the form
\[
\left( {\forall n \in \mathbb{N}} \right)\left( {P\left( n \right)} \right),
\]
where  $P( n )$ is some open sentence.  Recall that a universally quantified statement like the preceding one is true if and only if the truth set  $T$  of the open sentence $P( n )$ is the set  $\N$.  So our goal is to prove that  $T = \N$, which is the conclusion of the Principle of Mathematical Induction.  To verify the hypothesis of the Principle of Mathematical Induction, we must

\begin{enumerate}
\item Prove that  $1 \in T$\!.  That is, prove that  $P( 1 )$ is true.

\item Prove that if  $k \in T$\!, then  $\left( {k + 1} \right) \in T$\!.  That is, prove that if  $P( k )$ is true, then  $P( {k + 1} )$ is true.

\end{enumerate}
The first step is called the \textbf{basis step}
\index{mathematical induction!basis step}%
\index{basis step}%
 or the \textbf{initial step}, and the second step is called the \textbf{inductive step}.
\index{mathematical induction!inductive step}%
\index{inductive step}%
  This means that a proof by mathematical induction will have the following form:
%\hbreak
\begin{flushleft}
\fbox{\parbox{4.68in}{
\textbf{Procedure for a Proof by Mathematical Induction} \\
\noindent
To prove:	$\left( {\forall n \in \mathbb{N}} \right)\left( {P\left( n \right)} \right)$

\begin{tabular}{r l}
                   &                                 \\
Basis step:        &    Prove  $P( 1 )$.  \\
                   &                                 \\
Inductive step:    &  Prove that for each  $k \in \mathbb{N}$, \\
                   &  if  $P( k )$ is true, then  $P( {k + 1} )$ is true. \\
\end{tabular}
\vskip10pt
\noindent
We can then conclude that  $P( n )$ is true for all  $n \in \mathbb{N}$.
}}
\end{flushleft}

%\hbreak
\newpar
Note that in the inductive step, we want to prove that the conditional statement ``for each $k \in \N$, if 
$P(k)$ then $P(k + 1)$''  is true.  So we will start the inductive step by assuming that   $P( k )$  is true.  This assumption is called the \textbf{inductive assumption}
\index{inductive assumption}%
 or the \textbf{inductive hypothesis.}
\index{inductive hypothesis}%

%\vskip6pt
The key to constructing a proof by induction is to discover how  $P( {k + 1} )$
 is related to  $P( k )$ for an arbitrary natural number  $k$.  For example, in 
\typeu Activity~\ref*{PA:exploringstatements}, one of the open sentences $P(n)$ was
\[
1^2  + 2^2  + \, \cdots \, + n^2  = \frac{{n(n + 1)(2n + 1)}}{6}.
\]
Sometimes it helps to look at some specific examples such as  $P( 2 )$ and  $P( 3 )$.  The idea is not just to do the computations, but to see how the statements are related.  This can sometimes be done by writing the details instead of immediately doing computations.
\begin{align*}
P(2) &\qquad \text{is} & 1^2  + 2^2  &= \dfrac{{2 \cdot 3 \cdot 5}}{6} \\
P(3) &\qquad \text{is} & 1^2  + 2^2  + 3^2  &= \dfrac{{3 \cdot 4 \cdot 7}}{6}
\end{align*}
%\begin{center}
%\begin{tabular}{l l l} \\
%$P( 2 )$  &  is  &	$1^2  + 2^2  = \dfrac{{2 \cdot 3 \cdot 5}}{6}$.  \\
%$P( 3 )$  & 	is  &	$1^2  + 2^2  + 3^2  = \dfrac{{3 \cdot 4 \cdot 7}}{6}$. \\
%\end{tabular}
%\end{center}
In this case, the key is the left side of each equation.  The left side of  $P( 3 )$
  is obtained from the left side of  $P( 2 )$  by adding one term, which is $3^2$.  This suggests that we might be able to obtain the equation for $P( 3 )$  by adding  
$3^2$  to both sides of  the equation in  $P( 2 )$.  Now for the general case, if  
$k \in \mathbb{N}$, we look at  $P( {k + 1})$ and compare it to  $P( k )$.
\begin{align*}
P(k)  &\quad \text{is} & 1^2  + 2^2  +  \cdots  + k^2  &= \dfrac{{k(k + 1)(2k + 1)}}{6} \\
P(k+1)  &\quad \text{is} & 1^2  + 2^2  +  \cdots  + \left( {k + 1} \right)^2  &= \dfrac{{\left( {k + 1} \right)\left[ {\left( {k + 1} \right) + 1} \right]\left[ {2\left( {k + 1} \right) + 1} \right]}}{6}
\end{align*}

%$P( k )$    is  	$1^2  + 2^2  +  \cdots  + k^2  = \dfrac{{k(k + 1)(2k + 1)}}{6}
%$.
%\vskip6pt
%$P( k+1 )$   	is  	$1^2  + 2^2  +  \cdots  + \left( {k + 1} \right)^2  = \dfrac{{\left( {k + 1} \right)\left[ {\left( {k + 1} \right) + 1} \right]\left[ {2\left( {k + 1} \right) + 1} \right]}}{6}$.
%\vskip6pt

\newpar
%The predicate $P( {k + 1} )$ was obtained by substituting  $n = k + 1$ into the equation for  $P( n )$.    
The key is to look at the left side of the equation for  $P( {k + 1} )$ and realize what this notation means.  It means that we are adding the squares of the first  $\left( {k + 1} \right)$ natural numbers.  This means that we can write
\[
1^2  + 2^2  + \, \cdots \, + \left( {k + 1} \right)^2  = 1^2  + 2^2  + \, \cdots \, + k^2  + \left( {k + 1} \right)^2.
\]
This shows us that the left side of the equation for  $P( {k + 1} )$ can be obtained from the left side of the equation for  $P( k )$ by adding  $\left( {k + 1} \right)^2 $.  This is the motivation for proving the inductive step in the following proof.
%\hbreak
%
\begin{proposition} \label{P:suminduction}
For each natural number  $n$, 
\[
1^2  + 2^2  + \, \cdots \, + n^2  = \dfrac{{n(n + 1)(2n + 1)}}{6}.
\]
\end{proposition}
%
\setcounter{equation}{0}
\begin{myproof}
We will use a proof by mathematical induction.  For each natural number $n$, we let
$P( n )$ be
\[
1^2  + 2^2  +  \cdots  + n^2  = \dfrac{{n(n + 1)(2n + 1)}}{6}.
\]
We first prove that $P ( 1 )$ is true.  Notice that  $\dfrac{{1\left( {1 + 1} \right)\left( {2 \cdot 1 + 1} \right)}}{6} = 1$.  This shows that
\[
1^2  = \dfrac{{1\left( {1 + 1} \right)\left( {2 \cdot 1 + 1} \right)}}{6},
\]
which proves that $P( 1 )$  is true.

\newpar
For the inductive step, we prove that for each $k \in \mathbb{N}$, if $P ( k )$ is true, then 
$P( k + 1 )$ is true.  So let  $k$  be a natural number and assume that  $P( k )$  is true.  That is, assume that
%
\begin{equation} \label{eq:suminduction1}
1^2  + 2^2  +  \cdots  + k^2  = \frac{{k(k + 1)(2k + 1)}}{6}.
\end{equation}
%
The goal now is to prove that  $P\left( {k + 1} \right)$ is true.  That is, it must be proved that
\begin{align} 
1^2  + 2^2  + \, \cdots \, + k^2 + (k + 1)^2  &= \frac{{(k + 1)\left[ {(k + 1) + 1} \right]\left[ {2(k + 1) + 1} \right]}}{6} \notag \\
                                        &= \frac{{(k + 1)\left( k + 2\right) \left( 2k + 3 \right)}}{6}. 
\label{eq:suminduction2}%
\end{align}
%
To do this, we add  $\left( {k + 1} \right)^2 $ to both sides of equation~(\ref{eq:suminduction1}) and algebraically rewrite the right side of the resulting equation.  This gives
%
\[
\begin{aligned}
  1^2  + 2^2  +  \cdots  + k^2  + (k + 1)^2  &= \frac{{k(k + 1)(2k + 1)}}{6} + (k + 1)^2  \\ 
                &= \frac{{k(k + 1)(2k + 1) + 6(k + 1)^2 }}{6} \\ 
                &= \frac{{(k + 1)\left[ {k(2k + 1) + 6(k + 1)} \right]}}{6} \\ 
                &= \frac{{(k + 1)\left( {2k^2  + 7k + 6} \right)}}{6} \\ 
                &= \frac{{(k + 1)(k + 2)(2k + 3)}}{6}. \\ 
                %&= \frac{{(k + 1)\left[ {(k + 1) + 1} \right]\left[ {2(k + 1) + 1} \right]}}{6}. \\ 
\end{aligned} 
\]
Comparing this result to equation~(\ref{eq:suminduction2}), we see that if  $P( k )$  is true, then  $P( {k + 1} )$ is true.  Hence, the inductive step has been established, and by  the Principle of Mathematical Induction, we have proved that for each natural number $n$, \linebreak
$1^2  + 2^2  + \, \cdots \, + n^2  = \dfrac{{n(n + 1)(2n + 1)}}{6}$.
\end{myproof}
\hbreak

\endinput
