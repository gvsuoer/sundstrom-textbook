\section*{Section \ref{S:introfunctions} Introduction to Functions}

\begin{enumerate}
\item \begin{enumerate}
\item 
$f \left( -3 \right) = 15$, 
$f \left( -1 \right) = 3$, 
$f \left( 1 \right) = -1$, 
$f \left( 3 \right) = 3$.

\item The set of preimages of  0  is $\{0, 2 \}$. The set of preimages of 4  is   
$\left\{ \dfrac{{2 - \sqrt {20} }}{2}, \dfrac{{2 + \sqrt {20} }}{2} \right\}$.  (Use the quadratic formula.)

%\item There are no preimages of  -2 since the equation $x^2 - 2x = -2$ has no real number solutions.

\addtocounter{enumii}{1}
\item $\text{range} \left( f \right) = \left\{ y \in \mathbb{R} \mid y \geq  - 1 \right\}$.
\end{enumerate}

\item \begin{enumerate}
\item $s \left( -3 \right) = 9$,
$s \left( -1 \right) = 1$,
$s \left( 1 \right) = 1$,
$s \left( 3 \right) = 9$.




\item The set of preimages of  0  is $\{ 0 \}$. The set of preimages of 2 is 
$\{ -\sqrt{2}, \sqrt{2} \}$.

\addtocounter{enumii}{1}
\item $\text{range} \left( s \right) = \left\{ y \in \mathbb{R} \mid y \geq  0 \right\} = 
\mathbb{R}^*$.
\end{enumerate}



%\item Only (a) can be used to represent a function from  $A$  to  $B$.  In Part~(a), each element of $A$ is associated with exactly one element of $B$.  In Part~(b), the element 2 in $A$ is associated with 2 elements in $B$, and in Part~(c), the element 3 in $A$ is not associated with an element in $B$.


\item \begin{enumerate}
\item 
$f \left( -7 \right) = 10$, 
$f \left( -3 \right) = 6$, 
$f \left( 3 \right) = 0$, 
$f \left( 7 \right) = -4$.

\item The set of preimages of 5 is $\{ -2 \}$. The set of preimages of 4 is $\{ -1 \}$.

\item $\text{range} \left( f \right) = \mathbb{Z}$.
\end{enumerate}




\item \begin{enumerate}
\item 
$f \left( -7 \right) = -13$, 
$f \left( -3 \right) = -5$, 
$f \left( 3 \right) = 7$, 
$f \left( 7 \right) = 15$.

\item The set of preimages of  5 is  $\left\{ 2 \right\}$.  There set of preimages of 4 is 
the empty set.

\item $\text{range} \left( f \right) = \left\{ y \in \mathbb{Z} \mid y \text{ is odd } \right\}$.  That is, the range of $f$ is the set of all odd integers.
\end{enumerate}




\item \begin{enumerate}
\item $d ( 1 ) = 1$, $d ( 2 ) = 2$, $d ( 3 ) = 2$, 
$d ( 4 ) = 3$, $d ( 8 ) = 4$, $d ( 9 ) = 3$

\item There is no natural number $n$ such that $d ( n ) = 1$ since every natural number has at least two divisors.

\item The only natural numbers  $n$  such that $d( n ) = 2$   are the prime numbers. The set of preimages of the natural number  2 is the set of prime numbers.

\item The statement is false.  A counterexample is $m = 2$ and $n = 3$ since 
$d( 2 ) = 2$ and $d( 3 ) = 2$.

\item $d \left( 2^0 \right) = 1$, $d \left( 2^1 \right) = 2$, $d \left( 2^2 \right) = 3$, 
$d \left( 2^3 \right) = 4$, $d \left( 2^4 \right) = 5$, \\
$d \left( 2^5 \right) = 6$, and $d \left( 2^6 \right) = 7$.

\item Let $P \left( n \right)$ be, ``$d \left( 2^n \right) = n + 1$.''  $P \left( 0 \right)$ is true.  Let $k \in \mathbb{Z}$ with $k \geq 0$ and assume that $P \left( k \right)$ is true.  Then,
\begin {center}
$d \left( 2^k \right) = k + 1$.
\end{center}
This means that $2^k$ has $k + 1$ divisors.  Now, any divisor of $2^k$ is also a divisor of 
$2^{k+1}$.  The only other divisor of $2^{k+1}$ is $2^{k+1}$.  Thus,
\[
\begin{aligned}
d \left( 2^{k + 1} \right)&= ( k + 1 ) + 1 \\
                      &= k + 2.
\end{aligned}
\]
This proves that if $P( k )$ is true, then $P( k + 1 )$ is true.
\end{enumerate}




\item \begin{enumerate}
\item $f( { - 3, 4} ) = 9$, $f( { - 2, - 7} ) =  - 23$

\item $\left\{ { {( {m, n} ) \in \mathbb{Z} \times \mathbb{Z} } \mid m = 4 - 3n} \right\}$
\end{enumerate}




\item \begin{enumerate}
\item $g ( 3, 5 ) = ( 6, -2 )$, \qquad
$g ( -1, 4 ) = ( -2, -5 )$.

\item $( 0, 0 )$ is the only preimage of $( 0, 0 )$.

\item The set of  preimages of $( 8, -3 )$ is $\left\{ ( 4, 7 ) \right\}$. 

\item The set of  preimages of $( 1, 1 )$ is $\emptyset$ since the domain is $\mathbb{Z} \times \mathbb{Z}$. 

\item Part~(d) shows that the statement is false since there does not exist an 
$( m, n ) \in \mathbb{Z} \times \mathbb{Z}$ such that 
$f ( m, n ) = ( 1, 1 )$.
\end{enumerate}

\item \begin{enumerate}
\item $\text{dom}( k ) = \left\{ {x \in \mathbb{R}  \mid x \geq 3} \right\}$,  
$\text{range} ( k ) = \left\{ y \in \mathbb{R} \mid y \geq 0 \right\}$.

\item $\text{dom} ( F ) = \left\{ {x \in \mathbb{R} \mid x > \frac{1}{2}} \right\}$,  
$\text{range} ( F ) = \mathbb{R}$.

\item $\text{dom} ( f ) = \mathbb{R}$,
$\text{range} ( f ) = \left\{ y \in \mathbb{R} \mid -1 \leq y \leq 1 \right\}$.

\item $\text{dom} ( g ) = \left\{ {x \in \mathbb{R} \mid x \ne 2\text{  and  }x \ne  - 2} \right\}$, \\
$\text{range} ( g ) = \left\{ { {y \in \mathbb{R} } \mid y > 0} \right\} \cup 
\left\{ {y \in \mathbb{R} \mid y \leqslant  - 1} \right\}$.
\end{enumerate}

\end{enumerate}
\hbreak

\endinput
