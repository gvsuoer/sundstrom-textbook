\begin{previewactivity}[\textbf{Functions Defined by a Congruence}]\label{PA:function-congruence} \hfill

Before beginning this activity, review the statements of  Theorem~\ref{T:congtorem}  and  Corollary~\ref{C:congtorem}   from Section~\ref{S:cases}.

Theorem~\ref{T:congtorem} and Corollary~\ref{C:congtorem} state that an integer is congruent (mod $n$) to its remainder when it is divided by  $n$.  (Recall that we always mean the remainder guaranteed by the Division Algorithm, which is the least nonnegative remainder.)  Since this remainder is unique and since the only possible remainders for division by $n$  are  $0, 1, 2,  \ldots , n - 1$, we then know that each integer is congruent, modulo $n$, to precisely one of the integers $0,1,2, \ldots ,n - 1$.  So for each natural number 
$n$, we will define a new set $\Z_n$ as follows:
\[
\Z_n = \{ 0, 1, 2, \ldots , n - 1 \}.
\]
For example, $\Z_4 = \{0, 1, 2, 3 \}$ and $\Z_6 = \{ 0, 1, 2, 3, 4, 5 \}$.
\begin{enumerate}
\item For each  $x \in \Z_6 $, compute  $x^2  + 3$ and then determine the value of  $r$  in  
$\Z_6 $ so that
\[
\left( {x^2  + 3} \right) \equiv r\pmod 6.
\]
For example,  $2^2  + 3 = 7$, and since $\mod {7}{1}{6}$, we obtain   
$\left( {2^2  + 3} \right) \equiv 1 \pmod 6$.  Organize your results in a table with one column for the value of $x$ and another column for the value of $r$, where $r \in \mathbb{Z}_6$ and $\left( {x^2  + 3} \right) \equiv r\pmod 6$. 
\label{PA:functioncongruence1}

\item Explain how your work in Part~(\ref{PA:functioncongruence1}) can be used to define a function from   $\mathbb{Z}_6 $ to  $\mathbb{Z}_6 $. \label{PA:functioncongruence2}
\end{enumerate}
We will now describe a way to specify the outputs of this type of a function using a formula involving a congruence.  This will be illustrated with another example.  Define  
$g\x \Z_4  \to  \Z_4 $  by  $g( x ) = r$,  where  
$r \in \mathbb{Z}_4 $ and  $x^3  \equiv r \pmod 4$.  Then
\begin{align*}
g( 0 ) &= 0 	\text{\qquad since \qquad}   0^3  \equiv 0 \pmod 4 \\
g( 1 ) &= 1 	\text{\qquad since \qquad}   1^3  \equiv 1 \pmod 4 \\
g( 2 ) &= 0 	\text{\qquad since \qquad}   2^3  \equiv 0 \pmod 4 \\
g( 3 ) &= 3 	\text{\qquad since \qquad}   3^3  \equiv 3 \pmod 4.
\end{align*}
Note that this information about the outputs of this function could also be communicated by means of an arrow diagram or by means of a table of values.
The verbal description and the notation for the outputs of this function we have used are quite cumbersome.  So we will use a more concise notation.  Instead of writing,   
``$g\x \mathbb{Z}_4  \to \mathbb{Z}_4 $  by  $g( x ) = r$,  where  $r \in \mathbb{Z}_4 $ and 
$x^3  \equiv r \pmod 4$,'' we will write
\[
g\x \mathbb{Z}_4  \to \mathbb{Z}_4 \text{  by  }g( x ) = x^3 \pmod 4.
\]
\setcounter{oldenumi}{\theenumi}
\begin{enumerate} \setcounter{enumi}{\theoldenumi}
\item Define the function from Parts~(\ref{PA:functioncongruence1}) and~(\ref{PA:functioncongruence2}) in a manner similar to the way the function $g$ was defined.
\end{enumerate}
\end{previewactivity}
\hbreak

\endinput
