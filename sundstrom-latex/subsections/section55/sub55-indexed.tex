\subsection*{The Union and Intersection of an Indexed Family of Sets}
One of the purposes of the \typel activities was to show that we often encounter situations in which more than two sets are involved, and it is possible to define the union and intersection of more than two sets.  In \typeu Activity~\ref*{PA:indexfamily}, we also saw that it is often convenient to ``index'' the sets in a family of sets.  In particular, if $n$ is a natural number and 
$\mathscr{A} = \left\{ A_1, A_2, \ldots , A_n \right\}$ is a family of $n$ sets, then the union of these $n$ sets, denoted by $A_1 \cup A_2 \cup \cdots \cup A_n$ or 
$\bigcup\limits_{j=1}^{n}A_j$, \label{sym:unionfiniteindex} is defined as
\[
\bigcup_{j=1}^{n}A_j = 
\left\{ x \in U \mid x \in A_j, \text{ for some } j \text{ with } 1 \leq j \leq n \right\}\!. 
\]
We can also define the intersection of these $n$ sets, denoted by 
$A_1 \cap A_2 \cap \cdots \cap A_n$ or $\bigcap\limits_{j=1}^{n}A_j$, 
\label{sym:interfiniteindex} as
\[
\bigcap_{j=1}^{n}A_j = 
\left\{ x \in U \mid x \in A_j, \text{ for all } j \text{ with } 1 \leq j \leq n \right\}\!. 
\]
We can also extend this idea to define the union and intersection of a family that consists of infinitely many sets.  So if $\mathscr{B} = \left\{ B_1, B_2, \ldots , B_n, \ldots \: \right\}$, then
\[
\begin{aligned}
\bigcup_{j=1}^{\infty}B_j &= 
\left\{ x \in U \mid x \in B_j, \text{ for some } j \text{ with } j \geq 1 \right\}\!, 
\label{sym:unioninfiniteindex} \text{ and} \\
\bigcap_{j=1}^{\infty}B_j &= 
\left\{ x \in U \mid x \in B_j, \text{ for all } j \text{ with } j \geq 1 \right\}\!. 
\end{aligned}
\]
\hbreak

\begin{prog}[\textbf{An Infinite Family of Sets}] \label{prog:infinitefamily} \hfill \\
For each natural number $n$, let $A_n = \left\{ 1, n, n^2 \right\}$.  For example,
\begin{multicols}{3}
$A_1 = \left\{ 1 \right\}$

$A_2 = \left\{ 1, 2, 4 \right\}$,

$A_3 = \left\{ 1, 3, 9 \right\}$,
\end{multicols}
\noindent
and
\begin{multicols}{2}
$\bigcup\limits_{j=1}^{3}A_j = \left\{1, 2, 3, 4, 9 \right\}$,

$\bigcap\limits_{j=1}^{3}A_j = \left\{1 \right\}$.
\end{multicols}

\noindent
Determine each of the following sets:
\begin{multicols}{3}
\begin{enumerate}
\item $\bigcup\limits_{j=1}^{6}A_j$
\item $\bigcap\limits_{j=1}^{6}A_j$
\item $\bigcup\limits_{j=3}^{6}A_j$
\item $\bigcap\limits_{j=3}^{6}A_j$
\item $\bigcup\limits_{j=1}^{\infty}A_j$
\item $\bigcap\limits_{j=1}^{\infty}A_j$
\end{enumerate}
\end{multicols}
\end{prog}
\hbreak

In all of the examples we have studied so far, we have used $\N$ or a subset of $\N$ to index or label the sets in a family of sets.  We can use other sets to index or label sets in a family of sets.  For example, for each real number $x$, we can define $B_x$ to be the closed interval $\left[x, x + 2 \right]$.  That is,
\[
B_x = \left\{ y \in \R \mid x \leq y \leq x + 2 \right\}.
\]
So we make the following definition.  In this definition, $\Lambda$ is the uppercase Greek letter lambda and $\alpha$ is the lowercase Greek letter alpha.

\begin{defbox}{D:indexfamily}{Let $\Lambda$ be a nonempty set and suppose that for each 
$\alpha \in \Lambda$, there is a corresponding set $A_\alpha$.  The family of sets 
$\left\{ A_\alpha \mid \alpha \in \Lambda \right\}$ \label{sym:indexfamily} is called an 
\textbf{indexed family of sets}
\index{indexed family of sets}%
\index{family of sets!indexed}%
\index{indexing set}%
 indexed by $\Lambda$.  Each $\alpha \in \Lambda$ is called an 
\textbf{index} and $\Lambda$ is called an \textbf{indexing set}.}
\end{defbox}

\begin{prog}[\textbf{Indexed Families of Sets}] \label{prog:indexfamily} \hfill \\
In each of the indexed families of sets that we seen so far, if the indices were different, then the sets were different.  That is, if $\Lambda$ is an indexing set for the family of sets 
$\mathscr{A} = \left\{ A_\alpha \mid \alpha \in \Lambda \right\}$, then if $\alpha, \beta \in \Lambda$ and $\alpha \ne \beta$, then $A_\alpha \ne A_\beta$.  (\note  The letter $\beta$ is the Greek lowercase beta.)

\begin{enumerate}
\item Let $\Lambda = \left\{ 1, 2, 3, 4 \right\}$, and for each $n \in \Lambda$, let 
$A_n = \left\{ 2n + 6, 16 - 2n \right\}$, and let 
$\mathscr{A} = \left\{A_1, A_2, A_3, A_4 \right\}$.  Determine $A_1$, $A_2$, $A_3$, and 
$A_4$.

\item Is the following statement true or false for the indexed family $\mathscr{A}$ in~(1)?
\begin{list}{}
\item For all $m, n \in \Lambda$, if $m \ne n$, then $A_m \ne A_n$.
\end{list}

\item Now let $\Lambda = \R$.  For each $x \in \R$, define 
$B_x = \left\{ 0, x^2, x^4 \right\}$.  Is the following statement true for the indexed family of sets $\mathscr{B} = \left\{ B_x \mid x \in \R \right\}$?
\begin{list}{}
\item For all $x, y \in \R$, if $x \ne y$, then $B_x \ne B_y$.
\end{list}  
\end{enumerate}
\end{prog}
\hbreak


We now restate the definitions of the union and intersection of a family of sets for an indexed family of sets.

\begin{defbox}{D:indexoper}{Let $\Lambda$ be a nonempty indexing set and let 
$\mathscr{A} = \left\{ A_\alpha \mid \alpha \in \Lambda \right\}$ be an indexed family of sets.  The \textbf{union over} $\mathbf{\mathscr{A}}$
\index{union!of an indexed family of sets}%
\index{indexed family of sets!union}%
 is defined as the set of all elements that are in at least one of sets $A_\alpha$, where $\alpha \in \Lambda$.  We write
\[
\bigcup_{\alpha \in \Lambda}^{}A_\alpha = \left\{ x \in U \mid \text{ there exists an } 
\alpha \in \Lambda \text{ with } x \in A_\alpha \right\}\!. \label{sym:unionindex}
\]
The \textbf{intersection over} $\mathbf{\mathscr{A}}$
\index{intersection!of an indexed family of sets}%
\index{indexed family of sets!intersection}%
 is the set of all elements that are in all of the sets $A_\alpha$ for each 
$\alpha \in \Lambda$.  That is,
\[
\bigcap_{\alpha \in \Lambda}^{}A_\alpha = \left\{ x \in U \mid \text{ for all } 
\alpha \in \Lambda,  x \in A_\alpha \right\}\!. \label{sym:interindex}
\]
}
\end{defbox}

\begin{example}[\textbf{A Family of Sets Indexed by the Positive Real Numbers}]\label{exam:indexfamily} \hfill \\
For each positive real number $\alpha$, let $A_\alpha$ be the interval 
$\left( -1, \alpha \right]$.  That is,
\[
A_\alpha = \left\{ x \in \R \mid -1 < x \leq \alpha \right\}\!.
\]
If we let $\R^+$ be the set of positive real numbers, then we have a family of sets indexed by $\R^+$.  We will first determine the union of this family of sets.  Notice that for each 
$\alpha \in \R^+$, $\alpha \in A_\alpha$, and if $y$ is a real number with $-1 < y \leq 0$, then $y \in A_\alpha$.  Also notice that if $y \in \R$ and $y < -1$, then for each $\alpha \in \R^+$, $y \notin A_\alpha$.  With these observations, we conclude that
\[
\bigcup_{\alpha \in \R^+}^{}A_\alpha = \left( -1, \infty \right) = 
\left\{ x \in \R \mid -1 < x \right\}\!.
\]
To determine the intersection of this family, notice that
\begin{itemize}
\item if $y \in \R$ and $y < -1$, then for each $\alpha \in \R^+$, $y \notin A_\alpha$;
\item if $y \in \R$ and $-1 < y \leq 0$, then $y \in A_\alpha$ for each $\alpha \in \R^+$; and
\item if $y \in \R$ and $y > 0$, then if we let $\beta = \dfrac{y}{2}$, $y > \beta$ and 
$y \notin A_\beta$.  
%Note:  The letter $\beta$ is the lowercase Greek letter ``beta.''
\end{itemize}
From these observations, we conclude that
\[
\bigcap_{\alpha \in \R^+}^{}A_\alpha = \left( -1, 0 \right] = 
\left\{ x \in \R \mid -1 < x \leq 0 \right\}\!.
\]
\end{example}
\hbreak


\begin{prog}[\textbf{A Continuation of Example~\ref{exam:indexfamily}}] \label{prog:indexfamily2} \hfill \\
Using the family of sets from Example~\ref{exam:indexfamily}, for each $\alpha \in \R^+$, we see that
\[
A_{\alpha}^c = \left(-\infty, -1 \right] \cup \left( \alpha, \infty \right)\!.
\]
Use the results from Example~\ref{exam:indexfamily} to help determine each of the following sets.  For each set, use either interval notation or set builder notation.
\begin{multicols}{2}
\begin{enumerate}
\item $\left( \bigcup\limits_{\alpha \in \R^+}^{}A_\alpha \right)^c$

\item $\bigcap\limits_{\alpha \in \R^+}^{}A_{\alpha}^c$

\item $\left( \bigcap\limits_{\alpha \in \R^+}^{}A_\alpha \right)^c$

\item $\bigcup\limits_{\alpha \in \R^+}^{}A_{\alpha}^c$
\end{enumerate}
\end{multicols}
\end{prog}


\endinput
