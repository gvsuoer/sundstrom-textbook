\documentclass[11pt]{article}
\usepackage{c://pctex/activity}

\lhead{}
\chead{\textbf{\large{Exercise 21 - Section 3.1\\Prime Numbers}}}
\rhead{}
\lfoot{\emph{Mathematical Reasoning: Writing and Proof, Third Ed.} \\Ted Sundstrom}
\cfoot{}
\rfoot{\copyright 2009 by Pearson Education, Inc.\\}


\begin{document}
\begin{enumerate}
  \item \textbf{Theorem}.  The only Pythagorean triple consisting of three consecutive natural numbers is 3, 4, and 5.

\begin{myproof}
To prove this, we let $n$ be a natural number and assume that $n$, $n + 1$, and $n + 2$ form a Pythagorean triple.  We will prove that $n = 3$.  Using the definition of a Pythagorean triple, we see that $n^2 + (n + 1)^2 = (n + 2)^2$.  Working with this equation, we obtain
\begin{align*}
n^2 + (n + 1)^2 &= (n + 2)^2 \\
n^2 + \left( n^2 + 2n + 1 \right) &= n^2 + 4n + 4 \\
2n^2 + 2n + 1 &= n^2 + 4n + 4
\end{align*}
We now solve the last equation for $n$ by first rewriting it into the standard form for a quadratic equation and then factoring the resulting expression.  This gives
\begin{align*}
n^2 - 2n - 3 &= 0 \\
(n + 1)(n - 3) &= 0
\end{align*}
From this, we conclude that $n + 1 = 0$ or $n - 3 = 0$, which means that $n = -1$ or $n = 3$.  Since $n$ is a natural number, we conclude that the only natural number solution is $n = 3$.  This proves that the only Pythagorean triple consisting of three consecutive natural numbers is 3, 4, and 5.
\end{myproof}


\newpage
  \item \textbf{Theorem}.  The only Pythagorean triple consisting of three natural numbers of the form $m$, $m + 7$, and $m + 8$ is 5, 12, and 13.

\begin{myproof}
To prove this, we let $m$ be a natural number and assume that $m$, $m+7$, and $m + 8$ form a Pythagorean triple.  We will prove that $m = 5$.  Using the definition of a Pythagorean triple, we see that $m^2 + (m + 7)^2 = (m + 8)^2$.  Working with this equation, we obtain
\begin{align*}
m^2 + (m + 7)^2 &= (m + 8)^2 \\
m^2 + \left( m^2 + 14n + 49 \right) &= m^2 + 16m + 64 \\
2m^2 + 14m + 1 &= m^2 + 16m + 64
\end{align*}
We now solve the last equation for $n$ by first rewriting it into the standard form for a quadratic equation and then factoring the resulting expression.  This gives
\begin{align*}
m^2 - 2m - 15 &= 0 \\
(m + 3)(m - 5) &= 0
\end{align*}
From this, we conclude that $m + 3 = 0$ or $m - 5 = 0$, which means that $m = -3$ or $m = 5$.  Since $m$ is a natural number, we conclude that the only natural number solution is $m = 5$.  This proves that the only Pythagorean triple consisting of three natural numbers of the form $m$, $m + 7$, and $m + 8$ is 5, 12, and 13.
\end{myproof}

\end{enumerate}
\end{document}
