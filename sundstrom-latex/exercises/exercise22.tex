\section*{Exercises for Section~\ref{S:logequiv}}
\begin{enumerate}
%
\xitem Write the converse and contrapositive of each of the following conditional statements. \label{exer:sec23-1}%
  \begin{enumerate}
    \item If  $a = 5$, then  $a^2  = 25$.  %(Assume that $a$ is some fixed integer.)
    \item If it is not raining, then Laura is playing golf.
    \item If  $a \ne b$, then  $a^4  \ne b^4 $. %(Assume that $a$ and $b$ are fixed integers.)
    \item If  $a$  is an odd integer, then  $3a$ is an odd integer. %(Assume that $a$ is a fixed integer.)
  \end{enumerate} 
%
%\pagebreak
\xitem Write each of the conditional statements in Exercise~(\ref{exer:sec23-1}) as a logically equivalent disjunction, and write the negation of each of the conditional statements in Exercise~(\ref{exer:sec23-1}) as a conjunction.\label{exer:sec23-2}
%
\item Write a useful negation of each of the following statements.  Do not leave a negation as a prefix of a statement.  For example, we would write the negation of ``I will play golf and I will mow the lawn'' as ``I will not play golf or I will not mow the lawn.'' \label{exer:sec23-3}

\begin{enumerate}
\yitem We will win the first game and we will win the second game.

\yitem They will lose the first game or they will lose the second game.

\yitem If you mow the lawn, then I will pay you \$20.

\yitem If we do not win the first game, then we will not play a second game.

\yitem I will wash the car or I will mow the lawn.

\item If you graduate from college, then you will get a job or you will go to graduate school.

\item If I play tennis, then I will wash the car or I will do the dishes.

\item If you clean your room or do the dishes, then you can go to see a movie.

\item It is warm outside and if it does not rain, then I will play golf.
\end{enumerate}

\item Use truth tables to establish each of the following logical equivalencies dealing with biconditional statements: \label{exer:sec23-bicond}
  \begin{enumerate}
    \item $\left( {P \leftrightarrow Q} \right) \equiv \left( {P \to Q} \right) \wedge \left( {Q \to P} \right)$ \label{exer:bicond} \label{exer:sec23-biconda}
    \item $\left( {P \leftrightarrow Q} \right) \equiv \left( {Q \leftrightarrow P} \right)$
    \item $\left( {P \leftrightarrow Q} \right) \equiv \left( {\mynot  P \leftrightarrow \mynot  Q} \right)$
  \end{enumerate}
%
\item Use truth tables to prove each of the \textbf{distributive laws} from Theorem~\ref{T:logequiv}. \label{exer:sec22-distrib}
\index{distributive laws!for statements}%
 %from Theorem~\ref{T:logequiv}.   
  \begin{enumerate}
    \item $P \vee \left( {Q \wedge R} \right) \equiv \left( {P \vee Q} \right) \wedge \left( {P \vee R} \right)$
    \item $P \wedge \left( {Q \vee R} \right) \equiv \left( {P \wedge Q} \right) \vee \left( {P \wedge R} \right)$
  \end{enumerate}



\item Use truth tables to prove the following logical equivalency from Theorem~\ref{T:logequiv}:
\[
\left[ {\left( {P \vee Q} \right) \to R} \right] \equiv \left( {P \to R} \right) \wedge \left( {Q \to R} \right). \label{exer:sec23-6}
\]



\xitem Use previously proven logical equivalencies to prove each of the following logical equivalencies about \textbf{conditionals with conjunctions:} \label{exer:sec23-7}
  \begin{enumerate}
    \item $\left[ {\left( {P \wedge Q} \right) \to R} \right] \equiv \left( {P \to R} \right) \vee \left( {Q \to R} \right)$
    \item $\left[ {P \to \left( {Q \wedge R} \right)} \right] \equiv \left( {P \to Q} \right) \wedge \left( {P \to R} \right)$
  \end{enumerate}



\item If $P$ and $Q$ are statements, is the statement 
$\left( P \vee Q \right) \wedge \mynot \left( P \wedge Q \right)$ logically equivalent to the statement $\left( P \wedge \mynot Q \right) \vee \left( Q \wedge \mynot P \right)$?  Justify your conclusion. \label{exer:sec23-9}


\item Use previously proven logical equivalencies to prove each of the following logical equivalencies:  \label{exer:sec23-8}
\begin{enumerate}
\item $\left[ {\mynot  P \to \left( {Q \wedge \mynot  Q} \right)} \right] \equiv P$
\item $\left( P \leftrightarrow Q \right) \equiv \left( \mynot P \vee Q \right) \wedge 
\left( \mynot Q \vee P \right)$
\item $\mynot \left( P \leftrightarrow Q \right) \equiv \left( P \wedge \mynot Q \right) \vee 
\left( Q \wedge \mynot P \right) $ 
\item $\left( P \to Q \right) \to R \equiv \left( P \wedge \mynot Q \right) \vee R$
\item $\left( P \to Q \right) \to R \equiv \left( \mynot P \to R \right) \wedge 
\left( Q \to R \right)$
\item $\left[ \left( P \wedge Q \right) \to \left( R \vee S \right) \right] \equiv 
\left[ \left( \mynot R \wedge \mynot S \right) \to \left( \mynot P \vee \mynot Q \right) \right]$
\item $\left[ \left( P \wedge Q \right) \to \left( R \vee S \right) \right] \equiv 
\left[ \left( P \wedge Q \wedge \mynot R \right) \to S \right]$
\item $\left[ \left( P \wedge Q \right) \to \left( R \vee S \right) \right] \equiv 
\left( \mynot P \vee \mynot Q \vee R \vee S \right)$
\item $\mynot \left[ \left( P \wedge Q \right) \to \left( R \vee S \right) \right] \equiv 
\left( P \wedge Q \wedge \mynot R \wedge  \mynot S \right)$
\end{enumerate} 





\xitem Let $a$ be a real number and let  $f$  be a real-valued function defined on an interval containing $x = a$.  Consider the following conditional statement: 
\label{exer:diffimpliescont}
\begin{list}{}
\item If $f$ is differentiable at $x = a$, then $f$ is continuous at $x = a$.
\end{list}
Which of the following statements have the same meaning as this conditional statement and which ones are negations of this conditional statement?  

\note  This is not asking which statements are true and which are false.  It is asking which statements are logically equivalent to the given statement.  It might be helpful to let $P$ represent the hypothesis of the given statement, $Q$ represent the conclusion, and then determine a symbolic representation for each statement.  Instead of using truth tables, try to use already established logical equivalencies to justify your conclusions.
\begin{enumerate}
\item If $f$ is continuous at $x = a$, then $f$ is differentiable at $x = a$. \label{exer:diffa}
\item If $f$ is not differentiable at $x = a$, then $f$ is not continuous at $x = a$. 
\label{exer:diffb}
%\item If $f$ is continuous at $x = a$, then $f$ is differentiable at $x = a$.
\item If $f$ is not continuous at $x = a$, then $f$ is not differentiable at $x = a$. 
\label{exer:diffc}
\item $f$ is not differentiable at $x = a$ or $f$ is continuous at $x = a$. \label{exer:diffd}
\item $f$ is not continuous at $x = a$ or $f$ is differentiable at $x = a$. \label{exer:diffe}
\item $f$ is differentiable at $x = a$ and $f$ is not continuous at $x = a$. \label{exer:difff}
\end{enumerate}

\item Let $a, b$, and $c$ be integers.  Consider the following conditional statement:
\begin{list}{}
\item If $a$ divides $bc$, then $a$ divides $b$ or $a$ divides $c$.
\end{list}
Which of the following statements have the same meaning as this conditional statement and which ones are negations of this conditional statement?  \label{exer:sec23-10}

The note for Exercise~(\ref{exer:diffimpliescont}) also applies to this exercise.

%\textbf{Note}:  This is not asking which statements are true and which are false.  It is asking which statements are logically equivalent to the given statement.  It might be helpful to let $P$ represent the hypothesis of the given statement, \linebreak
%$Q \vee R$ represent the conclusion, and then determine a symbolic representation for each statement.  Instead of using truth tables, try to use already established logical equivalencies to justify your conclusions.

\begin{enumerate}
\item If $a$ divides $b$ or $a$ divides $c$, then $a$ divides $bc$.
\item If $a$ does not divide $b$ or $a$ does not divide $c$, then $a$ does not divide $bc$.
\item $a$ divides $bc$, $a$ does not divide $b$, and $a$ does not divide $c$.
\yitem If $a$ does not divide $b$ and $a$ does not divide $c$, then $a$ does not divide $bc$.
\item $a$ does not divide $bc$ or $a$ divides $b$ or $a$ divides $c$.
\item If $a$ divides $bc$ and $a$ does not divide $c$, then $a$ divides $b$.
\item If $a$ divides $bc$ or $a$ does not divide $b$, then $a$ divides $c$.
\end{enumerate}

\item Let $x$  be a real number.  Consider the following conditional statement:

\begin{center}
If  $x^3  - x = 2x^2  + 6$, then  $x =  - 2$  or  $x = 3$.
\end{center}

Which of the following statements have the same meaning as this conditional statement and which ones are negations of this conditional statement?  Explain each conclusion.  (See the note in the instructions for  Exercise~(\ref{exer:diffimpliescont}).)

%Note:  You are not being asked which statements are true and which are false.  You are being asked which of the following statements are logically equivalent to the conditional statement and which are logically equivalent to its negation.

\begin{enumerate}
\item If  $x \ne  - 2$  and   $x \ne 3$, then  $x^3  - x \ne 2x^2  + 6$.

\item If  $x =  - 2$  or  $x = 3$, then  $x^3  - x = 2x^2  + 6$.

\item If  $x \ne  - 2$ or  $x \ne 3$, then  $x^3  - x \ne 2x^2  + 6$.

\item If  $x^3  - x = 2x^2  + 6$  and  $x \ne  - 2$, then  $x = 3$.

\item If  $x^3  - x = 2x^2  + 6$  or  $x \ne  - 2$, then  $x = 3$.

\item $x^3  - x = 2x^2  + 6$, $x \ne  - 2$, and  $x \ne 3$.

\item $x^3  - x \ne 2x^2  + 6$  or   $x =  - 2$  or  $x = 3$.
\end{enumerate}
\end{enumerate}

\subsection*{Explorations and Activities}
\setcounter{oldenumi}{\theenumi}
\begin{enumerate} \setcounter{enumi}{\theoldenumi}
  \item \textbf{Working with a Logical Equivalency}.  \label{A:workingeq} 
Suppose we are trying to prove the following for integers  $x$  and  $y$:

\begin{center}
If  $x \cdot y$ is even, then  $x$  is  even  or  $y$  is even.
\end{center}

\noindent
We notice that we can write this statement in the following symbolic form:
\[
P \to \left( {Q \vee R} \right),
\]
where $P$ is ``$x \cdot y$ is even,'' $Q$ is ``$x$ is even,'' and $R$ is ``$y$ is even.''

\begin{enumerate}
\item Write the symbolic form of the contrapositive of  $P \to \left( {Q \vee R} \right)$.  Then use one of De Morgan's Laws (Theorem~\ref{T:demorgan}) to rewrite the hypothesis of this conditional statement. \label{pr:workingeq2part2}
%
\item Use the result from Part~(\ref{pr:workingeq2part2}) to explain why the given statement is logically equivalent to the following statement:
  \begin{center}
    If  $x$  is  odd and  $y$  is odd, then  $x \cdot y$ is odd.
  \end{center}
\end{enumerate}

The two statements in this activity are logically equivalent.  We now have the choice of proving either of these statements.  If we prove one, we prove the other, or if we show one is false, the other is also false.  The second statement is Theorem~\ref{T:xyodd}, which was proven in Section~\ref{S:direct}.
\end{enumerate}
\hbreak
%
%\markboth{Chapter \ref{C:logic}. Logical Reasoning}{\ref{S:quantifier}. Quantifiers and %Negations}

\endinput

