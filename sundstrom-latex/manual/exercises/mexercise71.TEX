\section*{Section \ref{S:relations} Relations}

\begin{enumerate}
\item \begin{enumerate}
\item The set  $A \times B$  contains  9  ordered pairs.  
\[
A \times B = \left\{ \left( a, p \right), \left( a, q \right), \left( a, r \right), 
\left( b, p \right), \left( b, q \right), \left( b, r \right), 
\left( c, p \right), \left( c, q \right), \left( c, r \right) \right\}
\]
The set  $A \times B$  is a relation from  $A$  to  $B$  since $A \times B$  is a subset of  $A \times B$.

\item The set  $R$  is a relation from  $A$  to  $B$  since  $R \subseteq A \times B$.

\item $\text{dom}\left( R \right) = A$, $\text{range}\left( R \right) = \left\{ {p, q} \right\}$.

%\item $R^{ - 1}  = \left\{ {\left( {p, a} \right), \left( {q, b} \right), \left( {p, c} \right), \left( {q, a} \right)} \right\}$.
\end{enumerate}



\item \begin{enumerate}
\item The statement is false since $\left( c, c \right) \notin R$.
\item The statement is true since whenever $\left( x, y \right) \in R$, $\left( y, x \right)$ is also in $R$.
\item The statement is false since $\left( a, c \right) \in R$, $\left( c, b \right) \in R$, but 
$\left( a, b \right) \notin R$.
\item The statement is false since $\left( a, a \right) \in R$ and $\left( a, c \right) \in R$.
\end{enumerate}



\item \begin{enumerate}
\item The domain of  $D$  consists of the female citizens of the U.S. whose mother is a female citizen of the U.S.

\item The range of  $D$  consists of those female citizens of the U.S. who have a daughter that is a female citizen of the U.S.

\item Whether or not the relation $D$ is a function from $A$ to $A$ is somewhat ambiguous.  If we assume that $A$ is the set of all living female citizens of the U.S., then $D$ is not a function since there exist female citizens whose mothers are not living.  Even if we allow $A$ to contain females that are not living, it is quite likely that $D$ is not a function since it is virtually certain that there exist female citizens of the U.S. whose mothers are not citizens of the U.S.

%\item $D^{-1} = \left\{ \left( y, x \right) \in A \times A \mid y \text{ is the mother of } x \right\}$.
\end{enumerate}



\item \begin{enumerate}
\item $\left( {S, T} \right) \in R$ means that $S \subseteq T$.

\item The domain of the subset relation is $\mathcal{P} \left( U \right)$.

\item The range of the subset relation is $\mathcal{P} \left( U \right)$.

%\item $R^{-1} = \left\{ \left(T, S \right) \in \mathcal{P} \left( U \right) \times 
%\mathcal{P} \left( U \right) \mid S \subseteq T \right\}$.

\item The relation $R$ is not a function from $\mathcal{P} \left( U \right)$ to 
$\mathcal{P} \left( U \right)$ since any proper subset of $U$ is a subset of more than one subset of $U$.
\end{enumerate}



\item \begin{enumerate}
\item The domain of the ``element of'' relation is $U$.

\item The range of the ``element of'' relation is $\mathcal{P} \left( U \right)$.

%\item $R^{-1} = \left\{ \left(S, x \right) \in \mathcal{P} \left( U \right) \times U 
%\mid x \in S \right\}$.

\item If $U$ contains more than one element, then the relation $R$ is not a function from $U$ to $\mathcal{P} \left( U \right)$ since any element of $U$ would be an element of more than one subset of $U$.  If $U = \left\{ x \right\}$, then $R = \left\{ \left( x, U \right) \right\}$ and $R$ is a function from $U$ to $\mathcal{P} \left( U \right)$.
\end{enumerate}



\item $S = \left\{ {\left( {x, y} \right) \in \mathbb{R} \times \mathbb{R} \mid x^2  + y^2  = 100} \right\}$
\begin{enumerate}
\item $\left\{ {\left. {x \in \mathbb{R}\,} \right| \left( {x, 6} \right) \in S} \right\} = \left\{ { - 8, 8} \right\}$. \\ $\left\{ {\left. {x \in \mathbb{R}\,} \right| \left( {x, 9} \right) \in S} \right\} = \left\{ { - \sqrt {19} , \sqrt {19} } \right\}$.

\item The domain of the relation $S$ is the closed interval $\left[ -10, 10 \right]$.

The range of the relation $S$ is the closed interval $\left[ -10, 10 \right]$.

%\item $S^{-1} = S$.

\item The relation $S$ is not a function from $\mathbb{R}$ to $\mathbb{R}$.

\item The graph of the relation $S$ is the circle of radius 10 whose center is at the origin.
\end{enumerate}



\item $S = \left\{ {\left( {x, y} \right) \in \mathbb{R} \times \mathbb{R}   \mid y = \sqrt {100 - x^2 } } \right\}$
\begin{enumerate}
\item $\left\{ {\left. {x \in \mathbb{R}\,} \right| \left( {x, 6} \right) \in S} \right\} = \left\{ { - 8, 8} \right\}$. \\ $\left\{ {\left. {x \in \mathbb{R}\,} \right| \left( {x, 9} \right) \in S} \right\} = \left\{ { - \sqrt {19} , \sqrt {19} } \right\}$.

\item The domain of the relation $S$ is the closed interval $\left[ -10, 10 \right]$.

The range of the relation $S$ is the closed interval $\left[ 0, 10 \right]$.

%\item $S^{-1} = \left\{ {\left( {x, y} \right) \in \mathbb{R} \times \mathbb{R} \mid x^2  + y^2  = 100} \text{ and } y \geq 0 \right\}$

\item The relation $S$ is a function from $\mathbb{R}$ to $\mathbb{R}$.

\item The graph of the relation $S$ is the top half of the circle of radius 10 whose center is at the origin.
\end{enumerate}


\item \begin{enumerate}
\item $\text{dom}(R) = \{ x \in \R \mid -\sqrt{10} \leq x \leq \sqrt{10} \}$

$\text{range}(R) = \{ y \in \R \mid -\sqrt{10} \leq y \leq \sqrt{10} \}$

The graph of $R$ is a circle with radius $\sqrt{10}$ and center at the origin.

\item $\text{dom}(S) = \{ x \in \R \mid x \geq -10 \}$, 
$\text{range}(S) = \R$

The graph of $R$ is a parabola that opens to the right and has its vertex at $(-10, 0)$.

\item $\text{dom}(T) = \{ x \in \R \mid -10 \leq x \leq 10 \}$

$\text{range}(T) = \{ y \in \R \mid -10 \leq y \leq 10 \}$ \quad 
The graph of $T$ is a square with vertices at $(10, 0)$, $(0, 10)$, $(-10, 0)$, and $(0, -10)$. 

\item $\text{dom}(R) = \R$, $\text{range}(R) = \R$ \quad 
The graph of $R$ consists of the two straight lines whose equations are $y = x$ and $y = -x$.
\end{enumerate}


\item \begin{enumerate}
\item $R = \left\{ (a, b) \in \Z \times \Z \left| \left| a - b \right| \leq 2 \right. \right\}$

\item $\text{dom}(R) = \Z$ and $\text{range}(R) = \Z$

\item $\{ x \in \Z \mid x \mathrel{R} 5 \} = \{3, 4, 5, 6, 7 \} = \{ x \in \Z \mid 5 \mathrel{R} x \}$

\item The integers $x$ and $y$ for which $x \mathrel{R} 8$, $8 \mathrel{R} y$, but 
$x \mathrel{\not \negthickspace R} y$ are the following: 

\begin{multicols}{3}
$x = 6, y = 10$

$x = 6, y = 9$
 
$x = 7, y = 10$

$x = 10, y = 6$

$x = 9, y = 6$

$x = 7, y = 10$
\end{multicols}

\item $\{ x \in \Z \mid x \mathrel{R} a \} = \{a - 2, a - 1, a, a + 1, a + 2 \}$
\end{enumerate}



\item $R_{  < }  = \left\{ { {\left( {x, y} \right) \in \mathbb{R} \times \mathbb{R} } \mid x < y} \right\}$
\begin{enumerate}
\item The domain of $R_{  < }$ is $\mathbb{R}$.

\item The range of $R_{  < }$ is $\mathbb{R}$.

%\item $R_{<}^{-1} = \left\{ \left( {y, x} \right) \in \mathbb{R} \times \mathbb{R} \mid x <  y \right\} = \left\{ \left( {y, x} \right) \in \mathbb{R} \times \mathbb{R} \mid y > x \right\}$ 

\item The relation $R_{<}^{-1}$ is not a function.  For example, 
$\left( 0, 1 \right) \in R_{<}^{-1}$ and $\left( 0, 2 \right) \in R_{<}^{-1}$.
\end{enumerate}

%\item Let  $A$  and  $B$  be nonempty sets, and let  $R$  and  $S$  be relations from  $A$  to  $B$. If  $R \subseteq S$, then  $R^{ - 1}  \subseteq S^{ - 1} $.
%
%\textbf{\emph{Proof}.}  The key idea is that if $\left( b, a \right) \in R^{ - 1}$, then 
%$\left( a, b \right) \in R$.  Hence, $\left( a, b \right) \in S$ since $R \subseteq S$.  Thus, 
%$\left( b, a \right) \in S^{ - 1}$ and so, $R^{-1} \subseteq S^{-1}$.

\end{enumerate}




\subsection*{Explorations and Activities}
\setcounter{oldenumi}{\theenumi}
\begin{enumerate} \setcounter{enumi}{\theoldenumi}
\item \noindent
\textbf{Theorem~\ref{T:inverserelations}}.  Let  $R$  be a relation from the set  $A$  to the set  $B$.  Then,
\begin{enumerate}
\item The domain of  $R^{ - 1} $ is the range of  $R$.  That is, 
$\text{dom}\left( {R^{ - 1} } \right) = \text{range}\left( R \right)$.

\item The range of  $R^{ - 1} $  is the domain of  $R$.   That is, 
$\text{range}\left( {R^{ - 1} } \right) = \text{dom}\left( R \right)$.

\item The inverse of  $R^{ - 1} $  is  R.  That is, $\left( {R^{ - 1} } \right)^{ - 1}  = R$.
\end{enumerate}

\begin{myproof}
Let  $R$  be a relation from the set  $A$  to the set  $B$.  For Part (1), we will first prove that  $\text{dom}\left( {R^{ - 1} } \right) \subseteq \text{range}\left( R \right)$.  So, let  
$y \in \text{dom}\left( {R^{ - 1} } \right)$.  This means that there exists an  $x \in A$ such that
\[
\left( {y, x} \right) \in R^{ - 1}.
\]
Using the definition of the inverse relation, this means that  $\left( {x, y} \right) \in R$, and hence,  $y \in \text{range}\left( R \right)$.  This proves that  
$\text{dom}\left( {R^{ - 1} } \right) \subseteq \text{range}\left( R \right)$.

Now let  $y \in \text{range}\left( R \right)$.  Then, there exists an  $x \in A$ such that  
$\left( {x, y} \right) \in R$.  But this means that  
\[
\left( {y, x} \right) \in R^{ - 1} 
\]
and hence,  $y \in \text{dom}\left( {R^{ - 1} } \right)$.  Consequently,  
$\text{range}\left( R \right) \subseteq \text{dom}\left( {R^{ - 1} } \right)$ and hence, 
$\text{dom}\left( {R^{ - 1} } \right) = \text{range}\left( R \right)$.
\vskip6pt

We will now prove Part (2) that  
$\text{range}\left( {R^{ - 1} } \right) = \text{dom}\left( R \right)$.  This will also be done by proving that each set is a subset of the other.  So, let  
$x \in \text{range}\left( {R^{ - 1} } \right)$. This means that there exists a  $y \in B$ such that  
\[
\left( {y, x} \right) \in R^{ - 1}.
\]
This in turn means that
\[
\left( {x, y} \right) \in R,
\]
and hence that  $x \in \text{dom}\left( R \right)$.  This proves that  
$\text{range}\left( {R^{ - 1} } \right) \subseteq \text{dom}\left( R \right)$. 

Now let $x \in \text{dom}\left( R \right)$.  Then, there exists a  $y \in B$ such that  
$\left( {x, y} \right) \in R$.  Consequently,  
\[
\left( {y, x} \right) \in R^{ - 1} 
\]
and so, $x \in \text{range}\left( {R^{ - 1} } \right)$.  This proves that  
$\text{dom}\left( R \right) \subseteq \text{range}\left( {R^{ - 1} } \right)$ and hence that \\ $\text{range}\left( {R^{ - 1} } \right) = \text{dom}\left( R \right)$.
\vskip6pt

Part (3) is proven using the following set equalities:

\[
\begin{aligned}
\left( {R^{ - 1} } \right)^{ - 1}  &= \left\{ {\left. {\left( {x, y} \right) \in A \times B} \right| \left( {y, x} \right) \in R^{ - 1} } \right\} \\ 
                                   &= \left\{ {\left. {\left( {x, y} \right) \in A \times B} \right| \left( {x, y} \right) \in R} \right\} \\ 
                                   &= R  \\
\end{aligned} 
\]
\end{myproof}

\end{enumerate}
\hbreak

\endinput
