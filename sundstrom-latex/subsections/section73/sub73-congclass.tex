\subsection*{Congruence Modulo $\boldsymbol{n}$ and Congruence Classes}
In \typeu Activity~\ref*{PA:congruencemodulo3}, we used the notation  $C[ k ]$ for the set of all integers that are congruent to  $k$  modulo  3.  We could have used a similar notation for equivalence classes, and this would have been perfectly acceptable.  However, the notation  
$[ a ]$  is probably the most common notation for the equivalence class of  $a$.  We will now use this same notation when dealing with congruence modulo  $n$ when only one congruence relation is under consideration.

\begin{defbox}{congclass}{Let  $n \in \mathbb{N}$.  Congruence modulo  $n$  is an equivalence relation on  $\mathbb{Z}$.  So for  $a \in \mathbb{Z}$,
\[
[ a ] = \left\{ { {x \in \mathbb{Z} } \mid x \equiv a \pmod n} \right\}\!.
\]
In this case, $[ a ]$ 
\label{sym:conclass} is called the \textbf{congruence class of}
\index{congruence class}%
  $\boldsymbol{a}$  \textbf{modulo}  $\boldsymbol{n}$.}
\end{defbox}
%\end{example}
%\hbreak
%
We have seen that congruence modulo 3 divides the integers into three distinct congruence classes.  Each congruence class consists of those integers with the same remainder when divided by 3.  In a similar manner, if we use congruence modulo 2, we simply divide the integers into two classes.  One class will consist of all the integers that have a remainder of 0 when divided by 2, and the other class will consist of all the integers that have a remainder of 1 when divided by 2.  That is, congruence modulo 2 simply divides the integers into the even and odd integers.
\hbreak

\begin{prog}[\textbf{Congruence Modulo 4}] \label{prog:congmod4} \hfill \\
Determine all of the distinct congruence classes for the equivalence relation of congruence modulo 4 on the integers.  Specify each congruence class using the roster method.
\end{prog}
\hbreak

\endinput
