\section*{Exercises for Section~\ref{S:reviewproofs}}
\begin{enumerate}
  \item Let $h$ and $k$ be real numbers and let $r$ be a positive number.  The equation for a circle whose center is at the point $\left( h, k \right)$ and whose radius is $r$ is
\label{exer:circle-31}%

\[
\left( x - h \right)^2 + \left( y - k \right)^2 = r^2.
\]

\noindent
We also know that if $a$ and $b$ are real numbers, then

\begin{itemize}
\item The point $\left( a, b \right)$ is inside the circle if 
$\left( a - h \right)^2 + \left( b - k \right)^2 < r^2$.
\item The point $\left( a, b \right)$ is on the circle if 
$\left( a - h \right)^2 + \left( b - k \right)^2 = r^2$.
\item The point $\left( a, b \right)$ is outside the circle if 
$\left( a - h \right)^2 + \left( b - k \right)^2 > r^2$.
\end{itemize}

\noindent
Prove that all points on or inside the circle whose equation is 
$\left( x - 1 \right)^2 + \left( y - 2 \right)^2 = 4$ are inside the circle whose equation is 
$x^2 + y^2 = 26$.


\item (Exercise~(\ref{exer:insidecircle}), Section \ref{S:directproof}) Let $r$ be a positive real number. The equation for a circle of radius $r$ whose center is the origin is $x^2 + y^2 = r^2$.
\begin{enumerate}
\item Use implicit differentiation to determine $\dfrac{dy}{dx}$.

\item (Exercise~(\ref{exer:sec32-16}), Section~\ref{S:moremethods}) Let $\left(a, b \right)$ be a point on the circle with $a \ne 0$ and $b \ne 0$.  Determine the slope of the line tangent to the circle at the point $\left(a, b \right)$.

\item Prove that the radius of the circle to the point $\left(a, b \right)$ is perpendicular to the line tangent to the circle at the point $\left(a, b \right)$. \hint Two lines (neither of which is horizontal) are perpendicular if and only if the products of their slopes is equal to $-1$.
\end{enumerate}


\item Are the following statements true or false?  Justify your conclusions.
\begin{enumerate}
  \item For each integer $a$, if 3 does not divide $a$, then 3 divides $2a^2 + 1$.
  \item For each integer $a$, if 3 divides $2a^2 + 1$, then 3 does not divide $a$.
  \item For each integer $a$, 3 does not divide $a$ if and only if 3 divides $2a^2 + 1$.
\end{enumerate}



\item Prove that for each real number $x$ and each irrational number $q$, $(x + q)$ is irrational or 
$(x - q)$ is irrational.


\item Prove that there exist irrational numbers $u$ and $v$ such that $u^v$ is a rational number.

\hint We have proved that $\sqrt{2}$ is irrational.  For the real number $q = \sqrt{2}^{\sqrt{2}}$, either $q$ is rational or $q$ is irrational.  Use this disjunction to set up two cases.   



\item (Exercise~(\ref{exer:sec32-16}), Section~\ref{S:moremethods}) Let  $a$  and  $b$  be natural numbers such that  $a^2  = b^3 $.  Prove each of the propositions in Parts~(\ref{exer:a2equalsb3-a}) through~(\ref{exer:a2equalsb3-d}).  (The results of Exercise~(\ref{exer:ncubed}) and Theorem~\ref{T:n2isodd} from Section~\ref{S:moremethods} may be helpful.)
\label{exer:sec32-9}%

\begin{enumerate}
  \item If  $a$  is even, then  4  divides  $a$. 
\label{exer:a2equalsb3-a}%
  \item If  4  divides  $a$, then  4  divides  $b$.
  \item If  4  divides  $b$, then  8  divides  $a$.
  \item If  $a$  is even, then  8 divides  $a$.  
\label{exer:a2equalsb3-d}%
  \item Give an example of natural numbers  $a$  and  $b$  such that  $a$  is even and  $a^2  = b^3 $, but  $b$  is not divisible by  8.
\end{enumerate}



\item (Exercise~(\ref{exer:sec32-equation17}), Section~\ref{S:moremethods})  Prove the following proposition:
\label{exer:sec32-equation}%
\begin{list}{}
\item Let $a$ and $b$ be integers with $a \ne 0$.  If $a$ does not divide $b$, then the equation 
$ax^3 + bx + \left( b + a \right) = 0$ does not have a solution that is a natural number.
\end{list}
\hint It may be necessary to factor a sum  of cubes.  Recall that 
\[
u^3 + v^3 = \left( u + v \right) \left( u^2 - uv + v^2 \right).
\]



\item Recall that a \textbf{Pythagorean triple}
\index{Pythagorean triple}%
 consists of three natural numbers $a$, $b$, and $c$ such that 
$a < b < c$ and $a^2 + b^2 = c^2$.  Are the following propositions true or false?  Justify your conclusions.
\begin{enumerate}
\item For all $a, b, c \in \N$ such that $a < b < c$, if $a$, $b$, and $c$ form a Pythagorean triple, then 3 divides $a$ or 3 divides $b$.
\item For all $a, b, c \in \N$ such that $a < b < c$, if $a$, $b$, and $c$ form a Pythagorean triple, then 5 divides $a$ or 5 divides $b$ or 5 divides $c$.
\end{enumerate}

\item \begin{enumerate}
\item Prove that there exists a Pythagorean triple $a$, $b$, and $c$, where  $a = 5$ and $b$ and $c$ are consecutive natural numbers.

\item Prove that there exists a Pythagorean triple $a$, $b$, and $c$, where  $a = 7$ and $b$ and $c$ are consecutive natural numbers.

\item Let $m$ be an odd natural number that is greater than 1.  Prove that there exists a Pythagorean triple $a$, $b$, and $c$, where  $a = m$ and $b$ and $c$ are consecutive natural numbers.
\end{enumerate}

\item One of the most famous unsolved problems in mathematics is a conjecture made by Christian Goldbach in a letter to Leonhard Euler in 1742.  The conjecture made in this letter is now known as \textbf{Goldbach's Conjecture}. \label{exer:goldbach}
\index{Goldbach's Conjecture}%
  The conjecture is as follows:

\begin{list}{}
\item \emph{Every even integer greater than 2 can be expressed as the sum of two (not necessarily distinct) prime numbers}.
\end{list}

Currently, it is not known if this conjecture is true or false. %although most mathematicians believe it to be true. \label{exer:goldbach}

\begin{enumerate}
  \item Write 50, 142, and 150 as a sum of two prime numbers.

%\item Describe one way to prove that Goldbach's Conjecture is false.

\item Prove the following:
\begin{quote}
If Goldbach's Conjecture is true, then every integer greater than 5 can be written as a sum of three prime numbers.
\end{quote}

\item Prove the following:
\begin{quote}
If Goldbach's Conjecture is true, then every odd integer greater than 7 can be written as a sum of three odd prime numbers.
\end{quote}

%\item Describe at least two different ways that could be used to prove that Goldbach's Conjecture is false.
\end{enumerate}


\item Two prime numbers that differ by 2 are called \textbf{twin primes}.
\index{twin primes}%
  For example, 3 and 5 are twin primes, 5 and 7 are twin primes, and 11 and 13 are twin primes.  Determine at least two other pairs of twin primes.  Is the following proposition true or false?  Justify your conclusion.

\begin{list}{}
\item For all natural numbers $p$ and $q$ if $p$ and $q$ are twin primes other than 3 and 5, then 
$pq + 1$ is a perfect square and 36 divides $pq + 1$.
\end{list}



\item Are the following statements true or false?  Justify your conclusions.
\begin{enumerate}
  \item For all integers $a$ and $b$, $\mod{(a + b)^2}{\left(a^2 + b^2 \right)}{2}$.
  \item For all integers $a$ and $b$, $\mod{(a + b)^3}{\left(a^3 + b^3 \right)}{3}$.
  \item For all integers $a$ and $b$, $\mod{(a + b)^4}{\left(a^4 + b^4 \right)}{4}$.
  \item For all integers $a$ and $b$, $\mod{(a + b)^5}{\left(a^5 + b^5 \right)}{5}$.
\end{enumerate}
If any of the statements above are false, write a new statement of the following form that is true (and prove that it is true):
\begin{list}{}
\item For all integers $a$ and $b$, $\mod{(a + b)^n}{\left(a^n + \text{ something } + b^n \right)}{n}$.
\end{list}


\item Let $a$, $b$, $c$, and  $d$  be real numbers with $a \ne 0$ and let 
$f \left( x \right) = ax^3 + bx^2 + cx + d$.

\begin{enumerate}
\item Determine the derivative and second derivative of the cubic function $f$.

\item Prove that the cubic function $f$ has at  most two critical points and has exactly one inflection point.
\end{enumerate}

\end{enumerate}

\subsection*{Explorations and Activities}
\setcounter{oldenumi}{\theenumi}
\begin{enumerate} \setcounter{enumi}{\theoldenumi}
  \item \textbf{A Special Case of Fermat's Last Theorem}.  We have already seen examples of \textbf{Pythagorean triples},
\index{Pythagorean triple}%
which are natural numbers $a$, $b$, and $c$ where $a^2 + b^2 = c^2$.  For example, 3, 4, and 5 form a Pythagorean triple as do 5, 12, and 13.  One of the famous mathematicians of the 17th century was Pierre de Fermat (1601 -- 1665).
\index{Fermat, Pierre}%
  Fermat made an assertion that for each natural number $n$ with $n \geq 3$, there are no positive integers $a$, $b$, and $c$ for which $a^n + b^n = c^n$.  This assertion was discovered in a margin of one of Fermat's books after his death, but Fermat provided no proof.  He did, however, state that he had discovered a truly remarkable proof but the margin did not contain enough room for the proof.  

This assertion became known as \textbf{Fermat's Last Theorem} 
\index{Fermat's Last Theorem}%
but it more properly should have been called Fermat's Last Conjecture.  Despite the efforts of mathematicians, this ``theorem'' remained unproved until Andrew Wiles, a British mathematician, first announced a proof in June of 1993.  However, it was soon recognized that this proof had a serious gap, but a widely accepted version of the proof was published by Wiles in 1995.  Wiles' proof uses many concepts and techniques that were unknown at the time of Fermat.  We cannot discuss the proof here, but we will explore and prove the following proposition, which is a (very) special case of Fermat's Last Theorem.  

\noindent
\textbf{Proposition}.  There do not exist prime numbers $a$, $b$, and $c$ such that \newline $a^3 + b^3 = c^3$.

Although Fermat's Last Theorem implies this proposition is true, we will use a proof by contradiction to prove this proposition.  For a proof by contradiction, we assume that
\begin{center}
there exist prime numbers $a$, $b$, and $c$ such that $a^3 + b^3 = c^3$.
\end{center}
Since 2 is the only even prime number, we will use the following cases:  (1) $a = b = 2$; (2) $a$ and $b$ are both odd; and (3) one of $a$ and $b$ is odd and the other one is 2.

\begin{enumerate}
  \item Show that the case where $a = b = 2$ leads to a contradiction and hence, this case is not possible.
  \item Show that the case where $a$ and $b$ are both odd leads to a contradiction and hence, this case is not possible.
  \item We now know that one of $a$ or $b$ must be equal to 2.  So we assume that $b = 2$ and that $a$ is an odd prime.  Substitute $b = 2$ into the equation $b^3 = c^3 - a^3$ and then factor the expression $c^3 - a^3$.  Use this to obtain a contradiction.
  \item Write a complete proof of the proposition.
\end{enumerate}





%  \item A \textbf{divisibility test} 
%\index{divisibility test}%
%gives a necessary and sufficient condition for a natural number to be divisible by another natural number.  For example, we often say that a natural number $n$ is divisible by 2 if and only if the units digit in the decimal representation of $n$ is even.  The proof of this divisibility test is Exercise~.  In this exercise, we will prove a divisibility test for 4.
%
%We first observe that 4 divides $100 = 10^2$ since $100 = 4 \cdot 25$.
%\begin{enumerate}
%  \item Let $n \in \N$ with $n \geq 2$.  Notice that we can use a law of exponents to write 
%$10^n = 10^{n-2}\cdot 10^2$.  Use this to prove that for each natural number $n$ with $n \geq 2$, 4 divides 
%$10^n$ and that $\mod{10^n}{0}{4}$.
%  \item Prove that for each natural number $n$ with $n \geq 2$ and each integer $a$ with $a \geq 0$, 
%$\mod{a \cdot 10^n}{0}{4}$.
%\end{enumerate}
%One of the tools used to develop most divisibility tests is the standard (decimal) place value system.  For example, when we write the natural number 7324, we say that 4 is the units digit, 2 is the tens digit, 3 is the hundreds digit, and 7 is the thousands digit.  More formally, this means that
%\[
%7324 = \left(7 \times 10^3 \right) + \left( 3 \times 10^2 \right) + \left( 2 \times 10^1 \right) + \left( 4 \times 10^0 \right).
%\]

  \item The purpose of this exploration is to investigate the possibilities for which integers cannot be the sum of the cubes of two or three integers.
\begin{enumerate}
\item If $x$ is an integer, what are the possible values (between 0 and 8, inclusive) for $x^3$ modulo 9?
\item If $x$ and $y$ are integers, what are the possible values for $x^3 + y^3$ (between 0 and 8, inclusive) modulo 9?
\item If $k$ is an integer and $\mod{k}{3}{9}$, can $k$ be equal to the sum of the cubes of two integers?  Explain.
\item If $k$ is an integer and $\mod{k}{4}{9}$, can $k$ be equal to the sum of the cubes of two integers?  Explain.
\item State and prove a theorem of the following form:  For each integer $k$, if (conditions on $k$), then $k$ cannot be written as the sum of the cubes of two integers.  Be as complete with the conditions on $k$ as possible based on the explorations in Part~(b).
\item If $x$, $y$, and $z$ are integers, what are the possible values (between 0 and 8, inclusive) for $x^3 + y^3 + z^3$ modulo 9?
\item If $k$ is an integer and $\mod{k}{4}{9}$, can $k$ be equal to the sum of the cubes of three integers?  Explain.
%\item If $k$ is an integer and $\mod{k}{5}{9}$, can $k$ be equal to the sum of the cubes of three integers?  Explain.
\item State and prove a theorem of the following form:  For each integer $k$, if (conditions on $k$), then $k$ cannot be written as the sum of the cubes of three integers.  Be as complete with the conditions on $k$ as possible based on the explorations in Part~(f).
\end{enumerate}
\note Andrew Booker, a mathematician at the University of Bristol in the United Kingdom, recently discovered that 33 can be written as the sum of the cubes of three integers.  Booker used a trio of 16-digit integers, two of which were negative.  Following is a link to an article about this discovery.

\begin{center}
\url{http://gvsu.edu/s/10c}
\end{center}
\end{enumerate}


%\end{enumerate}
\hbreak
%\newpage


\endinput


