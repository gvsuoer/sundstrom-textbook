\subsection*{Using Counterexamples}
\index{counterexample|(}%
In Section~\ref{S:direct} and so far in this section, our focus has been on proving statements that involve universal quantifiers.  However, another important skill for mathematicians is to be able to recognize when a statement is false and then to be able to prove that it is false.  For example, suppose we want to know if the following proposition is true or false.

\begin{list}{}
\item For each integer $n$, if 5 divides $\left(n^2 - 1 \right)$, then 5 divides $\left( n - 1 \right)$.
\end{list}  

\newpar
Suppose we start trying to prove this proposition.  In the backward process, we would say that in order to prove that 5 divides $\left( n - 1 \right)$, we can show that there exists an integer $k$ such that
\[
Q_1:  n - 1 = 5k \quad \text{or} \quad n = 5k + 1.
\]
For the forward process, we could say that since 5 divides $\left(n^2 - 1 \right)$, we know that there exists an integer $m$ such that
\[
P_1:  n^2 - 1 = 5m \quad \text{or} \quad n^2 = 5m + 1.
\]
The problem is that there is no straightforward way to use $P_1$ to prove $Q_1$.  At this point, it would be a good idea to try some examples for $n$ and try to find situations in which the hypothesis of the proposition is true.  (In fact, this should have been done before we started trying to prove the proposition.)  The following table summarizes the results of some of these explorations with values for $n$.
$$
\BeginTable
\BeginFormat
| c | c | c | c | c |
\EndFormat
" $n$ | $n^2 - 1$ | Does 5 divide $\left( n^2 - 1 \right)$ | $n - 1$ | Does 5 divide $(n - 1)$ " \\ \_
" 1   |   0  |  yes  |  0  |  yes " \\ \_
" 2   |   3  |  no   |  1  | no " \\ \_
" 3   |  8  |  no  |  2 | no " \\ \_
" 4  |  15  | yes |  3  | no " \\ \_
\EndTable
$$
We can stop exploring examples now since the last row in the table provides an example where the hypothesis is true and the conclusion is false.  Recall from Section~\ref{S:quantifier} 
(see page~\pageref{D:counterexample2}) that a \textbf{counterexample} for a statement of the form 
$\left( \forall x \in U \right) \left( P(x) \right)$ is an element $a$ in the universal set for which $P(a)$ is false.  So we have actually proved that the negation of the proposition is true.

When using a counterexample to prove a statement is false, we do not use the term ``proof'' since we reserve a proof for proving a proposition is true.  We could summarize our work as follows:


%\newpage
\indent
\parbox{4.5in}{\textbf{Conjecture}.  For each integer $n$, if 5 divides $\left(n^2 - 1 \right)$, then 5 divides $\left( n - 1 \right)$.}

\vskip6pt
\indent
\parbox{4.5in}{The integer $n = 4$ is a counterexample that proves this conjecture is false.  Notice that when $n = 4$, $n^2 - 1 = 15$ and 5 divides 15.  Hence, the hypothesis of the conjecture is true in this case.  In addition, $n - 1 = 3$ and 5 does not divide 3 and so the conclusion of the conjecture is false in this case.  Since this is an example where the hypothesis is true and the conclusion is false, the conjecture is false.}

\newpar
As a general rule of thumb, anytime we are trying to decide if a proposition is true or false, it is a good idea to try some examples first.  The examples that are chosen should be ones in which the hypothesis of the proposition is true.  If one of these examples makes the conclusion false, then we have found a counterexample and we know the proposition is false.  If all of the examples produce a true conclusion, then we have evidence that the proposition is true and can try to write a proof.
\hbreak

\begin{prog}[\textbf{Using a Counterexample}] \label{pr:counterexample} \hfill \\
Use a counterexample to prove the following statement is false.
\begin{list}{}
\item For all integers $a$ and $b$, if 5 divides $a$ or 5 divides $b$, then 5 divides $(5a + b)$. 
\end{list}
\end{prog}
\index{counterexample|)}%
\hbreak


%Outside of mathematics, a counterexample is often used as a rebuttal to a proposed general statement. For example, if someone states that ``Students who do not do well in school are invariably failures later in life,'' someone might say that Albert Einstein is a counterexample for this statement.  In mathematics, we only deal with statements that are true or are false and so we often use a counterexample to prove that a universally quantified statement is false.

