%
\subsection*{Another Method of Establishing Logical Equivalencies}
We have seen that it is often possible to use a truth table to establish a logical equivalency.   
%Following are two other methods for establishing the logical equivalence
%\[
%\mynot  \left( {P \to Q} \right) \equiv \left( {P \wedge \mynot  Q} \right).
%\]
%\noindent
%\textbf{Method 1:  Consider Cases in Which Each Expression Is True} \hfill \\
%A conditional statement is false only when the hypothesis is true and the conclusion is false.  So $\ {P \to Q} $ is false only when  $P$  is true and  $Q$  is false. Consequently, $\mynot  \left( {P \to Q} \right)$ is true only when  $P$  is true and  $Q$  is false.
%
%A conjunction of two statements is true only when both statements are true.  So $\left( {P \wedge \mynot  Q} \right)$ is true only when  $P$  is true and  $\mynot  Q$ is true.  That is, $\left( {P \wedge \mynot  Q} \right)$ is true only when  $P$  is true and  $Q$  is false.
%
%Comparing the last sentences in the two preceding paragraphs, we see that  
%$\mynot  \left( {P \to Q} \right)$ is logically equivalent to  
%$\left( {P \wedge \mynot  Q} \right)$.
%
%\vskip6pt
%\noindent
%\textbf{Method 2:  Use Previously Established Logical Equivalencies} \hfill \\
However, it is also possible to prove a logical equivalency using a sequence of previously established logical equivalencies.  For example,
%
\begin{itemize}
\item $P \to Q$ is logically equivalent to  $\mynot  P \vee Q$.  So
\item $\mynot  \left( {P \to Q} \right)$  is logically equivalent to  $\mynot  \left( {\mynot  P \vee Q} \right)$.   
\item Hence, by one of De Morgan's Laws (Theorem~\ref{T:demorgan}), $\mynot  \left( {P \to Q} \right)$  is logically equivalent to  $\mynot  \left( {\mynot  P} \right) \wedge \mynot  Q$.  
\item This means that $\mynot  \left( {P \to Q} \right)$ is logically equivalent to  $P \wedge \mynot  Q$. 
\end{itemize}
%
The last step used the fact that  $\mynot  \left( {\mynot  P} \right)$ is logically equivalent to  $P$.
\hbreak




When proving theorems in mathematics, it is often important to be able to decide if two expressions are logically equivalent.  Sometimes when we are attempting to prove a theorem, we may be unsuccessful in  developing a proof for the original statement of the theorem.  However, in some cases, it is possible to prove an equivalent statement.  Knowing that the statements are equivalent tells us that if we prove one, then we have also proven the other.  In fact, once we know the truth value of a statement, then we know the truth value of any other logically equivalent statement.  This is illustrated in Progress Check~\ref{pr:workingeq2}.
%\hbreak
\begin{prog}[\textbf{Working with a Logical Equivalency}] \label{pr:workingeq2} \hfill \\
In Section~\ref{S:logop}, we constructed a truth table for  
$\left( {P \wedge \mynot  Q} \right) \to R$.  See page~\pageref{Ta:compoundtruthtable}.  
%
%We will use that truth table here.
%
%$$
%\BeginTable
%\BeginFormat
%|c|c|c|c|c|c|
%\EndFormat
%\_6
%    |    $P$  |  $Q$  \|6  $R$  |  $\mynot  Q$  |  $P \wedge \mynot  Q$  |  $\left( {P \wedge \mynot  Q} \right) \to R$ | \\+22 \_6
%    |    T | T \|6 T | F | F | T | \\
%    |    T | T \|6 F | F | F | T | \\ 
%    |    T | F \|6 T | T | T | T | \\ 
%    |    T | F \|6 F | T | T | F | \\ 
%    |    F | T \|6 T | F | F | T | \\ 
%    |    F | T \|6 F | F | F | T | \\ 
%    |    F | F \|6 T | T | F | T | \\ 
%    |    F | F \|6 F | T | F | T | \\ \_6
%\EndTable
%$$
\begin{enumerate}
\item Although it is possible to use truth tables to show that 
$P \to \left( {Q \vee R} \right)$ is logically equivalent to  
$\left( {P \wedge \mynot  Q} \right) \to R$, we instead use previously proven logical equivalencies to prove this logical equivalency.  In this case, it may be easier to start working with $\left( {P \wedge \mynot  Q} \right) \to R$.  Start with
\[
\left( {P \wedge \mynot  Q} \right) \to R \equiv \mynot \left( P \wedge \mynot Q \right) \vee R,
\]
which is justified by the logical equivalency established in Part~(\ref{PA:logequiv6}) of 
\typeu Activity~\ref*{PA:logequiv}.  Continue by using one of De Morgan's Laws on  
$\mynot \left( P \wedge \mynot Q \right)$.

\item Let $a$  and  $b$  be integers.  Suppose we are trying to prove the following:
\begin{itemize} 
\item If  3  is a factor of  $a \cdot b$, then  3  is a factor of  $a$  or  3  is a factor of   $b$.
\end{itemize}
Explain why we will have proven this statement if we prove the following:
\begin{itemize}
\item If  3  is a factor of $a \cdot b$ and  3  is not a factor of   $a$, then  3  is a factor of   $b$.
\end{itemize}
\end{enumerate}
\end{prog}
\hbreak
