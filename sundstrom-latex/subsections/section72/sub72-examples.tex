\subsection*{Examples of Other Equivalence Relations}
\begin{enumerate}
\item The relation $\sim$ on $\Q$ from Progress Check~\ref{prog:example-equiv} is an equivalence relation.


\item Let  $A$  be a nonempty set.  The \textbf{equality relation on}
\index{equality relation}%
\index{relation!equality}%
\index{identity relation}%
\index{relation!identity}%
  $\boldsymbol{A}$  is an equivalence relation.  This relation is also called the \textbf{identity relation on}  $\boldsymbol{A}$    and is denoted by  $I_A $, where
\[
I_A  = \left\{ {\left( {x, x} \right)   \mid x \in A} \right\}\!.
\]

\item Define the relation  $\sim$  on  $\mathbb{R}$  as follows:

\begin{list}{}
\item For  $a, b \in \mathbb{R}$,  $a \sim b$ if and only if there exists an integer  $k$  such that  $a - b = 2k\pi$.
\end{list}
\vskip10pt

We will prove that the relation  $\sim$  is an equivalence relation on  $\mathbb{R}$.  The relation  $\sim$  is reflexive on  $\mathbb{R}$ since for each  $a \in \mathbb{R}$,  
$a - a = 0 = 2 \cdot 0 \cdot \pi $.

%\vskip10pt
Now, let  $a, b \in \mathbb{R}$ and assume that  $a \sim b$.  We will prove that  $b \sim a$.  Since  $a \sim b$, there exists an integer  $k$  such that
\[
a - b = 2k\pi.
\]
By multiplying both sides of this equation by $-1$, we obtain
\[
\begin{aligned}
  ( { - 1} )( {a - b} ) &= ( { - 1} )( {2k\pi } ) \\ 
                  b - a &= 2( { - k} )\pi . \\ 
\end{aligned}
\]
Since  $ - k \in \mathbb{Z}$, the last equation proves that  $b \sim a$.  Hence, we have proven that if  $a \sim b$, then  $b \sim a$ and, therefore, the relation  $\sim$  is symmetric.

%\vskip10pt
To prove transitivity, let  $a, b, c \in \mathbb{R}$ and assume that  $a \sim b$ and  
$b \sim c$.  We will prove that  $a \sim c$.  Now, there exist integers  $k$  and  $n$  such that
\[
a - b = 2k\pi \text{  and  }b - c = 2n\pi .
\]
By adding the corresponding sides of these two equations, we see that
\[
\begin{aligned}
( {a - b} ) + ( {b - c} ) &= 2k\pi  + 2n\pi  \\ 
                    a - c &= 2( {k + n} )\pi . \\ 
\end{aligned} 
\]
By the closure properties of the integers,  $k + n \in \mathbb{Z}$.  So this proves that  
$a \sim c$ and, hence the relation  $\sim$  is transitive.

%\vskip10pt
We have now proven that  $\sim$  is an equivalence relation on  $\mathbb{R}$.  This equivalence relation is important in trigonometry.  If   $a \sim b$, then there exists an integer  $k$  such that  $a - b = 2k\pi $ and, hence,  $a = b + k( {2\pi } )$.  Since the sine and cosine functions are periodic with a period of  $2\pi $, we see that
\[
\begin{aligned}
  \sin a &= \sin( {b + k( {2\pi } )} ) = \sin b,\text{ and} \\ 
  \cos a &= \cos( {b + k( {2\pi } )} ) = \cos b. \\ 
\end{aligned} 
\]
Therefore, when  $a \sim b$, each of the trigonometric functions have the same value at  $a$  and  $b$.

\item For an example from Euclidean geometry, we define a relation  $P$  on the set  $\mathcal{L}$ of all lines in the plane as follows:

\begin{list}{}
\item For  $l_1 , l_2  \in \mathcal{L}$,  $l_1 \mathrel{P}l_2 $  if and only if  $l_1 $  is parallel to  
$l_2 $  or  $l_1  = l_2 $.
\end{list}
%\vskip6pt

We added the second condition to the definition of  $P$  to ensure that  $P$  is reflexive on  $\mathcal{L}$.  Theorems from Euclidean geometry tell us that if  $l_1 $  is parallel to  $l_2 $, then  $l_2 $  is parallel to  $l_1 $, and if $l_1 $  is parallel to  $l_2 $ and  $l_2 $  is parallel to  $l_3 $, then  $l_1 $  is parallel to  $l_3 $.  (Drawing pictures will help visualize these properties.)  This tells us that the relation  $P$  is reflexive, symmetric, and transitive and, hence, an equivalence relation on  $\mathcal{L}$.
\end{enumerate}
\hbreak

\begin{prog}[\textbf{Another Equivalence Relation}] \label{prog:anotherequiv} \hfill \\
Let  $U$  be a finite, nonempty set and let  $\mathcal{P}\left( U \right)$ be the power set of  $U$.  Recall that $\mathcal{P}\left( U \right)$ consists of all subsets of $U$. (See page~\pageref{D:powerset}.)   Define the relation  $ \approx $  on  $\mathcal{P}\left( U \right)$  as follows:

\begin{center}
For  $A, B \in \mathcal{P}\left( U \right)$,  $A \approx B$  if and only if  
$\card(A) = \card(B)$.
\end{center}
For the definition of the cardinality of a finite set, see page~\pageref{D:cardinality}.  This relation states that two subsets of $U$  are equivalent provided that they have the same number of elements. Prove that 
$\approx$ is an equivalence relation on the power set $\mathcal{P}\left( U \right)$.
\end{prog}
%\hbreak

\endinput
