\begin{previewactivity}[\textbf{Set Operations}]\label{PA:setops} \hfill \\
Before beginning this section, it would be a good idea to review sets and set notation, including the roster method and set builder notation, in Section~\ref{S:predicates}.

In Section~\ref{S:logop}, we used logical operators (conjunction, disjunction, negation) to form new statements from existing statements.  In a similar manner, there are several ways to create new sets from sets that have already been defined.  In fact, we will form these new sets using the logical operators of conjunction (and), disjunction (or), and negation (not).  For example, if the universal set is the set of natural numbers $\N$ and 
\[
A = \{ 1, 2, 3, 4, 5, 6 \} \qquad \text{and} \qquad B = \{ 1, 3, 5, 7, 9 \},
\]
\begin{itemize}
  \item The set consisting of all natural numbers that are in $A$ and are in $B$ is the set $\{1, 3, 5 \}$;
  \item The set consisting of all natural numbers that are in $A$ or are in $B$ is the set 
         $\{ 1, 2, 3, 4, 5, 6, 7, 9 \}$; and
  \item The set consisting of all natural numbers that are in $A$ and are not in $B$ is the set 
         $\{ 2, 4, 6 \}$.
\end{itemize}
These sets are examples of some of the most common set operations, which are given in the following definitions.
\begin{defbox}{intersection}{Let  $A$  and  $B$ be subsets of some universal set  $U$\!.  
The \textbf{intersection}
\index{intersection!of two sets}%
\index{set!intersection}%
of  $A$  and  $B$, written  $A \cap B$ and read ``$A$ intersect $B$,''  is the set of all elements that are in both  $A$  and  $B$.  That is,
\[
A \cap B = \left\{ {x \in U} \mid {x \in A \text{  and  } x \in B}  \right\}\!.
\] 
\label{sym:intersect}
The \textbf{union}
\index{union!of two sets}%
\index{set!union}%
of  $A$  and  $B$, written  $A \cup B$ and read ``$A$ union $B$,'' is the set of all elements that are in  $A$  or in  $B$.  That is,
\[
A \cup B = \left\{ {x \in U} \mid {x \in A \text{  or  } x \in B}  \right\}.
\]} 
\label{sym:union}
\end{defbox}

\begin{defbox}{setdiff}{Let  $A$  and  $B$ be subsets of some universal set  $U$\!.  The 
\textbf{set difference}
\index{difference of two sets}%
\index{set!difference}%
 of  $A$  and  $B$, or \textbf{relative complement}
\index{relative complement}%
\index{set!relative complement}%
 of  $B$  with respect to  $A$, written  $A - B$ and read ``$A$ minus $B$'' or ``the complement of  $B$  with respect to  $A$,'' is the set of all elements in  $A$  that are not in  $B$.  That is,
\[
A - B = \left\{ {x \in U} \mid {x \in A} \text{ and } {x \notin B} \right\}\!.
\] \label{sym:setdiff}
The \textbf{complement}
\index{complement of a set}%
\index{set!complement}%
 of the set  $A$, written  $A^c $ and read ``the complement of $A$,'' is the set of all elements of  $U$  that are not in  $A$.  That is,
\[
A^c  = \left\{ {x \in U} \mid {x \notin A} \right\}\!.
\]} \label{sym:complement}
\end{defbox}

\noindent
For the rest of this \typel activity, the universal set is 
$U = \left\{ {0, 1, 2, 3,  \ldots , 10} \right\}$, and we will use the following subsets of  $U$:
\begin{center}
$A = \left\{ {0, 1, 2, 3, 9} \right\}$ \qquad and \qquad  $B = \left\{ {2, 3, 4, 5, 6} \right\}$.
\end{center}
%Use the roster method to list all of the elements of  $U$  that are in the truth set of each of the following predicates.
So in this case, $A \cap B = \left\{ {x \in U} \mid {x \in A \text{  and  } x \in B}  \right\} = \{2, 3 \}$.  Use the roster method to specify each of the following subsets of $U$.
\begin{multicols}{3}
\begin{enumerate}
%  \item $A \cap B$
  \item $A \cup B$
  \item $A^c$
  \item $B^c$
\end{enumerate}
\end{multicols}

\noindent
We can now use these sets to form even more sets.  For example,
\[
A \cap B^c = \{0, 1, 2, 3, 9 \} \cap \{0, 1, 7, 8, 9, 10 \} = \{0, 1, 9 \}.
\]
Use the roster method to specify each of the following subsets of $U$.
\begin{multicols}{4}
\setcounter{oldenumi}{\theenumi}
\begin{enumerate} \setcounter{enumi}{\theoldenumi}
  \item $A \cup B^c$
  \item $A^c \cap B^c$
  \item $A^c \cup B^c$
  \item $(A \cap B)^c$
\end{enumerate}
\end{multicols}

%  \item $x \in A \text{  and  } x \in B$
%  \item $x \in A \text{  or  } x \in B$
%  \item $x \notin A$
%  \item $x \notin B$
%  \item $x \in A\text{  and  }x \notin B$
%  \item $x \notin A \text{  or  } x \notin B$
%  \item $\mynot  \left( {x \in A\text{  and  }x \in B} \right)$
%  \item $\mynot  \left( {x \in A\text{  or  }x \in B} \right)$
%\end{enumerate}
%\end{multicols}


\end{previewactivity}
\hbreak
\endinput

