\documentclass[11pt]{article}
\usepackage{c://pctex/activity}

\lhead{}
\chead{\textbf{\large{Exercise 13 - Section 1.2 \\Pythagorean Triples}}}
\rhead{}
\lfoot{\emph{Mathematical Reasoning: Writing and Proof, Third Ed.} \\Ted Sundstrom}
\cfoot{}
\rfoot{\copyright 2009 by Pearson Education, Inc.\\}


\begin{document}
\begin{enumerate}
  \item Verify that
\begin{align*}
3^2 + 4^2 &= 5^2   &  8^2 + 15^2 &= 17^2  &  12^2 + 35^2 &= 37^2 \\
6^2 + 8^2 &= 10^2  &  10^2 + 24^2 &= 26^2 &  14^2 + 48^2 &= 50^2
\end{align*}

  \item One possible conjecture is that if $m$ is a natural number and $m \geq 2$, then  $2m$, $m^2 - 1$, and $m^2 + 1$ form a Pythagorean triple.  For $m = 10$, it can be verified that 20, 99, and 101 is a Pythagorean triple.
  \item \textbf{Proposition}.  If $m$ is a natural number and $m \geq 2$, then  $2m$, $m^2 - 1$, and $m^2 + 1$ form a Pythagorean triple.
\begin{myproof}
We let $m$ be a natural number with $m \geq 2$ and will prove that $2m$, $m^2 - 1$, and $m^2 + 1$ form a Pythagorean triple.  First note that since $m \geq 2$, $m^2 - 1 \geq 3$, and $m^2 + 1 \geq 5$, and so 
$2m$, $m^2 - 1$, and $m^2 + 1$ are natural numbers.  In addition,
\begin{align*}
\left( 2m \right)^2 + \left( m^2 - 1 \right)^2 &= 4m^2 + \left( m^4 -2m^2 + 1 \right) \\
                                               &= m^4 + 2m^2 + 1 \\
                                               &= \left( m^2 + 1 \right)^2
\end{align*}
This proves that $\left( 2m \right)^2 + \left( m^2 - 1 \right)^2 = \left( m^2 + 1 \right)^2$ and so we have proved that if $m$ is a natural number and $m \geq 2$, then  $2m$, $m^2 - 1$, and $m^2 + 1$ form a Pythagorean triple.
\end{myproof}
\end{enumerate}


\end{document}
