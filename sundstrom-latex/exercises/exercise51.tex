\section*{Exercises for Section~\ref{S:setoperations}}
%
\begin{enumerate}
\xitem Assume the universal set is the set of real numbers.  Let \label{exer:sec41-1}
\begin{align*}
A &= \left\{ { - 3, - 2,2,3} \right\}\!, &	B &= \left\{ {\left. {x \in \mathbb{R}} \right|x^2  = 4\text{  or  }x^2  = 9} \right\}\!,  \\
C &= \left\{ {x \in \mathbb{R}\left.  \right|x^2  + 2 = 0} \right\}\!,  &	D &= \left\{ {x \in \mathbb{R}\left.  \right|x > 0} \right\}\!.
\end{align*}
Respond to each of the following questions.  In each case, explain your answer.
\begin{enumerate}
  \item Is the set  $A$  equal to the set  $B$?
  \item Is the set  $A$  a subset of the set  $B$?
  \item Is the set  $C$  equal to the set  $D$?
  \item Is the set  $C$  a subset of the set  $D$?
  \item Is the set  $A$  a subset of the set  $D$?
\end{enumerate}

\xitem \label{exer41-equalsets}
\begin{enumerate} \item Explain why the set  $\left\{ {a,b} \right\}$  is equal to the set  $\left\{ {b,a} \right\}$. 
  \item Explain why the set  $\left\{ {a,b,b,a,c} \right\}$ is equal to the set  $\left\{ {b,c,a} \right\}$.
\end{enumerate}

\xitem Assume that the universal set is the set of integers. Let \label{exer:sec41-3}
\begin{align*}
A &= \left\{ { - 3, - 2,2,3} \right\}\!,  &	B &= \left\{ {x \in \mathbb{Z}\left.  \right|x^2  \leq 9} \right\}\!, \\
C &= \left\{ {\left. {x \in \mathbb{Z}} \right|x \geq  - 3} \right\}\!,  &	D &= \left\{ {1,2,3,4} \right\}\!.
\end{align*}
In each of the following, fill in the blank with one or more of the symbols  $ \subset$ , $\subseteq$ , $\not \subseteq$,  $=$ , $\ne$, $\in$, $\text{or } \notin $ so that the resulting statement is true.  For each blank, include all symbols that result in a true statement.  If none of these symbols makes a true statement, write nothing in the blank.
\begin{center}
\begin{tabular}{r p{0.6in} l p{0.3in} r p{0.6in} l }
  $A$   &   & $B$   &   & $\emptyset$   &   & $A$ \\ \cline{2-2} \cline{6-6}
   5    &   & $C$   &   &  $\left\{ 5 \right\}$  & &  $C$ \\ \cline{2-2} \cline{6-6}
  $A$   &   & $C$   &   &  $\left\{ {1,2} \right\}$  &  &  $B$ \\ \cline{2-2} \cline{6-6}
  $\left\{ {1,2} \right\}$ &  &  $A$ &  &  $\left\{ {3,2,1} \right\}$ & & $D$ \\ \cline{2-2} \cline{6-6}
  4     &  &  $B$ &  &  $D$  &  & $\emptyset$ \\ \cline{2-2} \cline{6-6}
  $\card(A)$ &  &  $\card(D)$  &  &  $\card(A)$ &  &  $\card(B)$  \\ \cline{2-2} \cline{6-6}
  $A$  &  &  $\mathcal{P}( A )$  &  &  $A$  &  &  $\mathcal{P}( B )$ \\ \cline{2-2} \cline{6-6}
\end{tabular}
\end{center}

\xitem Write all of the proper subset relations that are possible using the sets of numbers  $\mathbb{N}$, $\mathbb{Z}$, $\mathbb{Q}$, and $\mathbb{R}$. \label{exer:sec51-numbers}

\xitem For each statement, write a brief, clear explanation of why the statement is true or why it is false. \label{exer:sec41-5}
  \begin{enumerate}
    \item The set  $\left\{ {a,b} \right\}$ is a subset of  $\left\{ {a,c,d,e} \right\}$.
    \item The set  $\left\{ { - 2,0,2} \right\}$ is equal to  $\left\{ {x \in \mathbb{Z} \mid  
x\text{ is even  and  }x^2  < 5} \right\}$.
    \item The empty set  $\emptyset $  is a subset of   $\left\{ 1 \right\}$.
    \item If  $A = \left\{ {a,b} \right\}$, then the set  $\left\{ a \right\}$  is a subset of  $\mathcal{P}( A )$.
  \end{enumerate}


%\item Using the definitions of intersection and union, we can say that 
%\label{exer:sec41-8}
%%
%\begin{itemize}
%  \item $x \in A \cap B$  if and only if  $x \in A\text{  and  }x \in B$.
%  \item $x \in A \cup B$  if and only if  $x \in A\text{  or  }x \in B$.
%  \item $x \in A - B$  if and only if  $x \in A\text{  and  }x \notin B$.
%\end{itemize}
\item Use the definitions of set intersection, set union, and set difference to write useful negations of these definitions.  That is, complete each of the following sentences
\label{exer:sec41-8}%
%
\begin{enumerate}
\yitem $x \notin A \cap B$  if and only if  $ \ldots .$
\item $x \notin A \cup B$  if and only if  $ \ldots .$
\item $x \notin A - B$  if and only if  $ \ldots .$
\end{enumerate}


\xitem Let  $U = \left\{ {1,2,3,4,5,6,7,8,9,10} \right\}$, and let \label{exer:sec41-6}
%
\begin{align} \notag
A &= \left\{ {3,4,5,6,7} \right\}\!,  &  B &= \left\{ {1,5,7,9} \right\}\!, \\ \notag
C &= \left\{ {3,6,9} \right\}\!,      &  D &= \left\{ {2,4,6,8} \right\}\!.  \notag
\notag
\end{align}
%
Use the roster method to list all of the elements of each of the following sets.
%
\begin{multicols}{2}
\begin{enumerate}
  \item $A \cap B$
  \item $A \cup B$
  \item $\left( {A \cup B} \right)^c $
  \item $A^c  \cap B^c $
  \item $\left( {A \cup B} \right) \cap C$
  \item $A \cap C$
  \item $B \cap C$
  \item $\left( {A \cap C} \right) \cup \left( {B \cap C} \right)$
  \item $B \cap D$
  \item $\left( {B \cap D} \right)^c $
  \item $A - D$
  \item $B - D$
  \item $\left( {A - D} \right) \cup \left( {B - D} \right)$
  \item $\left( {A \cup B} \right) - D$
\end{enumerate}
\end{multicols}


\item Let  $U = \N$, and let
\label{exer:sec41-6x}%
%
\begin{align} \notag
A &= \left\{ x \in \N \mid x \geq 7 \right\}\!,  &  B &= \left\{ x \in \N \mid x \text{ is odd} \right\}\!, \\ \notag
C &= \left\{ x \in \N \mid x \text{ is a multiple of 3} \right\}\!,      &  
D &= \left\{ x \in \N \mid x \text{ is even} \right\}\!.  \notag
\notag
\end{align}
%%
Use the roster method to list all of the elements of each of the following sets.
%
\begin{multicols}{2}
\begin{enumerate}
  \item $A \cap B$
  \item $A \cup B$
  \item $\left( {A \cup B} \right)^c $
  \item $A^c  \cap B^c $
  \item $\left( {A \cup B} \right) \cap C$
%  \item $A \cap C$
%  \item $B \cap C$
  \item $\left( {A \cap C} \right) \cup \left( {B \cap C} \right)$
  \item $B \cap D$
  \item $\left( {B \cap D} \right)^c $
  \item $A - D$
  \item $B - D$
  \item $\left( {A - D} \right) \cup \left( {B - D} \right)$
  \item $\left( {A \cup B} \right) - D$
\end{enumerate}
\end{multicols}

%\item In previous mathematics courses, we have frequently used subsets of the real numbers called \textbf{intervals}. 
%\index{interval}%
%  There are some common names and notations for intervals.  These are given in the following table, where it is assumed that  $a$  and  $b$  are real numbers  and  $a < b$. \label{exer:sec41-7}
%\begin{center}
%\begin{tabular}[h]{r l l}
%Interval \\Notation     &     Set Notation     &     Name  \\ \hline
%$\left( {a,b} \right) = $  &  $\left\{ {x \in \mathbb{R}\left.  \right|a < x < b} \right\}$  &  Open interval
%\index{open interval}%
%\index{interval!open}%
% from  $a$  to  $b$  \\
%$\left[ {a,b} \right] = $  &	$\left\{ {x \in \mathbb{R}\left.  \right|a \leq x \leq b} \right\}$  &  	Closed interval
%\index{closed interval}%
%\index{interval!closed}%
% from  $a$  to  $b$  \\
%$\left[ {a,b} \right) = $  &	$\left\{ {x \in \mathbb{R}\left.  \right|a \leq x < b} \right\}$  &  Half-open interval
%\index{half-open interval}%
%\index{interval!half-open}%
%  \\
%$\left( {a,b} \right] = $  &  $\left\{ {x \in \mathbb{R}\left.  \right|a < x \leq b} \right\}$  &  Half-open interval\\
%$\left( {a, + \infty } \right) = $  &  $\left\{ {\left. {x \in \mathbb{R}} \right|x > a} \right\}$  &  Open ray  \\
%$\left( { - \infty ,b} \right) = $  &  $\left\{ {\left. {x \in \mathbb{R}} \right|x < b} \right\}$  &  Open ray
%\index{open ray}%
%\index{ray!open}%
% \\
%$\left[ {a, + \infty } \right) = $  &  $\left\{ {\left. {x \in \mathbb{R}} \right|x \geq a} \right\}$  &  Closed ray
%\index{closed ray}%
%\index{ray!closed}%
%  \\
%$\left( { - \infty ,b} \right] = $  &  $\left\{ {\left. {x \in \mathbb{R}} \right|x \leq b} \right\}$  &  Closed ray \\
%\end{tabular}
%\end{center}
%
%\begin{enumerate}
%\item Is  $\left( {a,b} \right)$ a proper subset of   $\left( {a,b} \right]$?  Explain.
%\item Is  $\left[ {a,b} \right]$ a subset of   $\left( {a, + \infty } \right)$?  Explain.
%\item Use interval notation to describe the intersection of  the interval  $\left[ { - 3,7} \right]$ with the interval  $\left( {5,9} \right]$.
%\item Write the set  $\left\{ {x \in \mathbb{R}} \mid \left| x \right| \leq 0.01 \right\}$ using interval notation.
%\item Write the set  $\left\{ x \in \mathbb{R} \mid \left| x \right| > 2 \right\}$
%as the union of two intervals.
%\end{enumerate}

%\item Using the definitions of intersection and union, we can say that \label{exer:sec41-8}
%%
%\begin{itemize}
%  \item $x \in A \cap B$  if and only if  $x \in A\text{  and  }x \in B$.
%  \item $x \in A \cup B$  if and only if  $x \in A\text{  or  }x \in B$.
%  \item $x \in A - B$  if and only if  $x \in A\text{  and  }x \notin B$.
%\end{itemize}
%Write a useful negation of each of these sentences.  That is, complete each of the following sentences:
%%
%\begin{enumerate}
%\item $x \notin A \cap B$  if and only if  $ \ldots $
%\item $x \notin A \cup B$  if and only if  $ \ldots $
%\item $x \notin A - B$  if and only if  $ \ldots $
%\end{enumerate}
%




\item Let $P$, $Q$, $R$, and $S$ be subsets of a universal set $U$.  Assume that \\
$\left( P - Q \right) \subseteq \left( R \cap S \right)$.
\label{exer:sec41-4sets}
\begin{enumerate}
\item Complete the following sentence: \label{exer:subset-defa}
\begin{list}{}
\item For each $x \in U$, if $x \in \left( P - Q \right)$, then $\ldots .$
\end{list}
\yitem Write a useful negation of the statement in Part~(\ref{exer:subset-defa}).
\item Write the contrapositive of the statement in Part~(\ref{exer:subset-defa}).
\end{enumerate}
%
\item Let $U$ be the universal set.  Consider the following statement: \label{exer:sec41-10}
\begin{center}
For all $A$ and $B$ that are subsets of $U$, if $A \subseteq B$, then $B^c \subseteq A^c$.
\end{center}
\begin{enumerate}
\yitem Identify three conditional statements in the given statement.
\item Write the contrapositive of this statement.
\item Write the negation of this statement.
%\item Write the given statement in symbolic form using quantifiers. \label{exer:subset-def2}
%\item Write the negation of the statement in Part~(\ref{exer:subset-def2}) in symbolic form.
%\item Write the contrapositive of the statement in Part~(\ref{exer:subset-def2}) in symbolic form.
\end{enumerate}
%
\item Let  $A$, $B$, and  $C$  be subsets of some universal set  $U$.  Draw a Venn diagram for each of the following situations. \label{exer:sec41-11}
\begin{enumerate}
\item $A \subseteq C$
\item $A \cap B = \emptyset $
\item $A \not \subseteq B,B \not \subseteq A,C \subseteq A,\text{and }C \not \subseteq B$
\item $A \subseteq B,C \subseteq B,\text{and }A \cap C = \emptyset $
\end{enumerate}
%
\item Let  $A$, $B$, and  $C$  be subsets of some universal set  $U$.  For each of the following, draw a general Venn diagram for the three sets and then shade the indicated region. \label{exer:sec41-12}
\begin{multicols}{2}
\begin{enumerate}
  \item $A \cap B$
  \item $A \cap C$
  \item $\left( {A \cap B} \right) \cup \left( {A \cap C} \right)$
  \item $B \cup C$
  \item $A \cap \left( {B \cup C} \right)$
  \item $\left( {A \cap B} \right) - C$
\end{enumerate}
\end{multicols}



\item We can extend the idea of consecutive integers (See Exercise~(\ref{exer:sec34-2}) in Section~\ref{S:divalgo}) to represent four consecutive integers as $m$, $m + 1$, $m + 2$, and $m + 3$, where $m$ is an integer.  There are other ways to represent four consecutive integers.  For example, if $k \in \Z$, then $k - 1$, $k$, $k + 1$, and $k + 2$ are four consecutive integers.
\label{exer:congruence41}%
\begin{enumerate}
\item Prove that for each $n \in \Z$, $n$ is the sum of four consecutive integers if and only if $n \equiv 2 \pmod 4$.

\item Use set builder notation or the roster method to specify the set of integers that are the sum of four consecutive integers.

\item Specify the set of all natural numbers that can be written as the sum of four consecutive natural numbers.

\item Prove that for each $n \in \Z$, $n$ is the sum of eight consecutive integers if and only if $n \equiv 4 \pmod 8$.

\item Use set builder notation or the roster method to specify the set of integers that are the sum of eight consecutive integers.

\item Specify the set of all natural numbers can be written as the sum of eight consecutive natural numbers.

\end{enumerate}




\item One of the properties of real numbers is the so-called \textbf{Law of Trichotomy}, 
\index{Law of Trichotomy}%
which states that if $a, b \in \R$, then exactly one of the following is true:  

\begin{multicols}{3}
\begin{itemize}
\item $a < b$;
\item $a = b$;
\item $a > b$.
\end{itemize}
\end{multicols}

\noindent
Is the following proposition concerning sets true or false?  Either provide a proof that it is true or a counterexample showing it is false.

\noindent
If $A$ and $B$ are subsets of some universal set, then exactly one of the following is true: 

\begin{multicols}{3}
\begin{itemize}
\item $A \subseteq B$;
\item $A = B$;
\item $B \subseteq A$.
\end{itemize}
\end{multicols}

\end{enumerate}
%\markboth{Chapter~\ref{C:settheory}. Set Theory}{\ref{S:provingset}. Proving Set Relationships}

\subsection*{Explorations and Activities}
\setcounter{oldenumi}{\theenumi}
\begin{enumerate} \setcounter{enumi}{\theoldenumi}
  \item \textbf{Intervals of Real Numbers}. \label{A:intervals}  
In previous mathematics courses, we have frequently used subsets of the real numbers called \textbf{intervals}. 
\index{interval}%
  There are some common names and notations for intervals.  These are given in the following table, where it is assumed that  $a$  and  $b$  are real numbers  and  $a < b$. \label{exer:sec41-7}
\begin{center}
\begin{tabular}[h]{r l l}
Interval \\Notation     &     Set Notation     &     Name  \\ \hline
$\left( {a,b} \right) = $  &  $\left\{ {x \in \mathbb{R}\left.  \right|a < x < b} \right\}$  &  Open interval
\index{open interval}%
\index{interval!open}%
 from  $a$  to  $b$  \\
$\left[ {a,b} \right] = $  &	$\left\{ {x \in \mathbb{R}\left.  \right|a \leq x \leq b} \right\}$  &  	Closed interval
\index{closed interval}%
\index{interval!closed}%
 from  $a$  to  $b$  \\
$\left[ {a,b} \right) = $  &	$\left\{ {x \in \mathbb{R}\left.  \right|a \leq x < b} \right\}$  &  Half-open interval
\index{half-open interval}%
\index{interval!half-open}%
  \\
$\left( {a,b} \right] = $  &  $\left\{ {x \in \mathbb{R}\left.  \right|a < x \leq b} \right\}$  &  Half-open interval\\
$\left( {a, + \infty } \right) = $  &  $\left\{ {\left. {x \in \mathbb{R}} \right|x > a} \right\}$  &  Open ray  \\
$\left( { - \infty ,b} \right) = $  &  $\left\{ {\left. {x \in \mathbb{R}} \right|x < b} \right\}$  &  Open ray
\index{open ray}%
\index{ray!open}%
 \\
$\left[ {a, + \infty } \right) = $  &  $\left\{ {\left. {x \in \mathbb{R}} \right|x \geq a} \right\}$  &  Closed ray
\index{closed ray}%
\index{ray!closed}%
  \\
$\left( { - \infty ,b} \right] = $  &  $\left\{ {\left. {x \in \mathbb{R}} \right|x \leq b} \right\}$  &  Closed ray \\
\end{tabular}
\end{center}

\begin{enumerate}
\item Is  $\left( {a,b} \right)$ a proper subset of   $\left( {a,b} \right]$?  Explain.
\item Is  $\left[ {a,b} \right]$ a subset of   $\left( {a, + \infty } \right)$?  Explain.
\item Use interval notation to describe
\begin{enumerate}
\item the intersection of  the interval  $\left[ { - 3,7} \right]$ with the interval  $\left( {5,9} \right]$;
\item the union of  the interval  $\left[ { - 3,7} \right]$ with the interval  $\left( {5,9} \right]$;
\item the set difference $\left[ -3, 7 \right] - \left( 5, 9 \right]$.
\end{enumerate}
\item Write the set  $\left\{ {x \in \mathbb{R}} \mid \left| x \right| \leq 0.01 \right\}$ using interval notation.
\item Write the set  $\left\{ x \in \mathbb{R} \mid \left| x \right| > 2 \right\}$
as the union of two intervals.
\end{enumerate}

\item \textbf{More Work with Intervals}.  For this exercise, use the interval notation described in Exercise~\ref{A:intervals}.
\label{exer:intervals41}%
\begin{enumerate}
\item Determine the intersection and union of  $[2, 5]$ and  $[-1, +\infty)$.
\item Determine the intersection and union of  $[2, 5]$ and  $[3.4, +\infty)$.
\item Determine the intersection and union of  $[2, 5]$ and  $[7, +\infty)$.
\end{enumerate}
Now let $a$, $b$, and $c$ be real numbers with $a < b$.
\begin{enumerate} \setcounter{enumii}{3}
\item Explain why the intersection of $[a, b]$ and $[c, +\infty)$ is either a closed interval, a set with one element, or the empty set.
\item Explain why the union of $[a, b]$ and $[c, +\infty)$ is either a closed ray or the union of a closed interval and a closed ray.
\end{enumerate}


\item \textbf{Proof of Theorem~\ref{T:powerset}}.  \label{exer:powerset}  To help with the proof by induction of Theorem~\ref{T:powerset}, we first prove the following lemma. (The idea for the proof of this lemma was illustrated with the discussion of power set after the definition on page~\pageref{D:powerset}.)

\begin{lemma} \label{L:inductivestepforsubsets}
Let $A$ and $B$ be subsets of some universal set.  If  $A = B \cup \left\{ x \right\}$, where  $x \notin B$, then any subset of  $A$  is either a subset of  $B$  or a set of the form   
$C \cup \left\{ x \right\}$, where  $C$  is a subset of  $B$.
\end{lemma}
%
\begin{myproof}
Let $A$ and $B$ be subsets of some universal set, and assume that  $A = B \cup \left\{ x \right\}$ where  $x \notin B$.  Let  $Y$  be a subset of  $A$.  We need to show that  $Y$  is a subset of  $B$  or that   $Y = C \cup \left\{ x \right\}$, where  $C$   is some subset of  $B$.  There are two cases to consider:  (1)  $x$  is not an element of  $Y$\!, and (2)  $x$  is an element of  $Y$\!.
\vskip6pt
\noindent
\textit{Case 1}:   Assume that  $x \notin Y$\!.  Let  $y \in Y$.  Then  $y \in A$  and  
$y \ne x$.  Since  
\[
A = B \cup \left\{ x \right\}\!,
\]
this means that  $y$  must be in  $B$.  Therefore,  $Y \subseteq B$\!.
\vskip6pt
\noindent
\textit{Case 2}:  Assume that  $x \in Y$\!. In this case, let  $C = Y - \left\{ x \right\}$.  Then every element of  $C$  is an element of  $B$. Hence, we can conclude that  $C \subseteq B$  and that  $Y = C \cup \left\{ x \right\}$.
\vskip10pt
\noindent
Cases (1) and (2) show that if  $Y \subseteq A$, then  $Y \subseteq B$  or  
$Y = C \cup \left\{ x \right\}$,  where  \linebreak
$C \subseteq B$.
\end{myproof}

\index{power set}%
\index{power set!cardinality}%
To begin the induction proof of Theorem~\ref{T:powerset}, for each nonnegative integer $n$, we let $P ( n )$ be, ``If   a finite set has exactly $n$  elements, then  that set  has exactly  $2^n $ subsets.''
\begin{enumerate}
\item Verify that $P ( 0 )$ is true.  (This is the basis step for the induction proof.)

\item Verify that $P( 1 )$ and $P( 2 )$ are true.

\item Now assume that  $k$  is a nonnegative integer and assume that $P( k )$ is true.  That is, assume that if a set has  $k$  elements, then that set has  $2^k $  subsets.  (This is the inductive assumption for the induction proof.)

Let  $T$  be a subset of the universal set with  $\card (T) = k + 1$, and let  
$x \in T$.  Then the set  $B = T - \left\{ x \right\}$ has  $k$  elements.

Now use the inductive assumption to determine how many subsets  $B$  has.  Then use  Lemma~\ref{L:inductivestepforsubsets} to prove that  $T$  has twice as many subsets as  $B$.  This should help complete the inductive step for the induction proof.
\end{enumerate}

\end{enumerate}
\hbreak


\endinput
