\documentclass[11pt]{article}
\usepackage{../../styles/activity}

\usepackage{xr}
\externaldocument{0-MR}

\lhead{}
%\chead{\textbf{\Large{\hspace{0pt}Beginning Activities for Section~4.2}}\\\hspace{0pt}\emph{Mathematical Reasoning: Writing and Proof}}
\bahead{4.2}
\rhead{}
\lfoot{}
\rfoot{}
\cfoot{\hspace{0pt}\scalebox{0.4}{\includegraphics{cc-by-nc-sa.eps}}}

\begin{document}

\subsection*{Beginning Activity 1 (Exploring a Proposition about Factorials)}
Notice how the numerical examples calculated in part~1 are used to explore the open sentence ``$n! > 2^n$.''
\begin{enumerate}
\item
\begin{tabular}[t]{| c | c | c | c | c | c | c | c |} \hline
$n$  &  1  &  2  &  3  &  4  &  5  &  6  &  7  \\ \hline
$2^n$  &  2  &  4  &  8  &  16  &  32  &  64  &  128  \\ \hline
$n!$  &  1  &  2  &  6  &  24  &  120  &  720  &  5040  \\ \hline
\end{tabular}
\item $P(1), P(2), P(3)$ are false.  $P(4), P(5), P(6), P(7)$ are true.
\item Based on the evidence so far, the following proposition appears to be true:  For each natural number  
$n$ with $n \geq 4$, $2^n > n!$. 
\item Since $(k + 1)!$ is the product of the first $k + 1$ natural numbers, we can think of this as the product of the first $k$ natural numbers times $(k + 1)$.  This means that $(k + 1)!$ is equal to $(k + 1)$ times $k!$.
\item If we multiply both sides of the inequality $(k + 1) > 2$ by $2^k$, we obtain 
\begin{align*}
(k + 1)2^k &> 2 \cdot 2^k \\
(k + 1)2^k &> 2^{k+1}
\end{align*}
\item We have
\begin{equation} 
(k + 1)\cdot k! > (k + 1) 2^k 
\end{equation}
and from part (5),
\begin{equation}
(k + 1)2^k > 2^{k+1}
\end{equation}
Inequalities (1) and (2) then imply that $(k + 1)\cdot k! > 2^{k+1}$ or that $(k + 1)! > 2^{k+1}$.

\eighth
If the argument (and algebra) in parts 4 through 6 are hard to understand, first try to follow along with a specific example such as $k = 5$.  We know that $5! > 2^5$.  Although it is possible to do the calculation to show that $6! > 2^6$, pretend we cannot do this and use this numerical example to illustrate the algebraic argument.  We would have
\setcounter{equation}{0}
\begin{align}
5! &> 2^5 \notag \\ 
6\cdot 5! &> 6\cdot 2^5 \notag \\
6! &> 6\cdot 2^5
\end{align}
In addition, $6 > 2$ and so
\begin{align}
6 \cdot 2^5 &> 2 \cdot 2^5 \notag \\
6 \cdot 2^5 &> 2^6
\end{align}
So using inequalities (1) and (2), we see that $6! > 2^6$.

%\item What appears to be a true proposition can be formed by adding the condition that  $n$  be greater than or equal to 4.  That is, the following proposition seems to be true:
%For each natural number  $n$  with $n \geq 4$, $n! > 2^n$.
\end{enumerate}
\hbreak



\subsection*{Beginning Activity 2 (Prime Factors of a Natural Number)}
\begin{enumerate}
\setcounter{enumi}{1}
\item 
\begin{tabular}[t]{p{1in} p{2in} p{1in} p{2in}}
$20 = 2^2 \cdot 5$ &  $40 = 2 \cdot 20$  & $50 = 2 \cdot 5^2$ & $150 = 3 \cdot 50$ \\
                   &  $40 = 2 ( {2^2 \cdot 5} )$  &  & $150 = 3 ( {2 \cdot 5^2} )$ \\
                   &  $40 = 2^3 \cdot 5$  &  & $150 = 2 \cdot 3 \cdot 5^2$ \\
\end{tabular}

%\item 
%\begin{tabular}[t]{p{1in} p{2in}}
%$50 = 2 \cdot 5^2$  & $150 = 3 \cdot 50$ \\
%                   &  $150 = 3 ( {2 \cdot 5^2} )$ \\
%                   &  $150 = 2 \cdot 3 \cdot 5^2$ \\
%\end{tabular}

\addtocounter{enumi}{1}
\item A natural number  $n$  is a composite number provided that there exists a natural number  
$d$  such that  $d$  divides  $n$  and  $d \ne 1$  and  $d \ne n$.  This means that there exists a natural number  $m$  such that  $n = m \cdot d$, $1 < d < n$, and  $1 < m < n$.

\item In this section, we will see how to use induction to prove that any composite number can be written as a product of primes.  The idea will be to factor a composite number as  
$n = m \cdot d$,  where  $1 < d < n$, and  $1 < m < n$.  We will then use induction to conclude that  $m$  and  $d$  can be factored as a product of primes.  (This was illustrated in 
Part~(2).)  We will need the Second Principle of Mathematical Induction, which is introduced in this section.
\end{enumerate}
\hbreak



%\subsection*{Beginning Activity 3 (Subsets of a Set with Four Elements)}
%\begin{enumerate}
%\item The eight sets are all subsets of  $A$  since for each set, every element in the set is in  $A$.
%
%\item \begin{multicols}{4}
%$\left\{ x \right\}$	
%
%$\left\{ {a, x} \right\}$	
%
%$\left\{ {b, x} \right\}$	
%
%$\left\{ {a, b, x} \right\}$
%
%$\left\{ {c, x} \right\}$	
%
%$\left\{ {a, c, x} \right\}$	
%
%$\left\{ {b, c, x} \right\}$	
%
%$\left\{ {a, b, c, x} \right\}$
%\end{multicols}
%
%\item Eight subsets of  $A$  were formed in Part (2).  This gives a total of 16 subsets of  $A$.  These are all the subsets of   $A$  since the original list contains those subsets of  $A$  that do not contain  $x$  and those in Part (2) are those subsets of  $A$  that contain  $x$.
%
%\item The work so far shows that  $S$  is a subset of  $A$  if and only if  $S$  is a subset of  
%$B$  or  $S = C \cup \left\{ x \right\}$ where  $C$  is a subset of  $B$.  Since  $S \subseteq A$
% means  $S \in \mathcal{P} ( A )$  and  $C \subseteq B$ means  
%$C \in \mathcal{P} ( B )$, this translates to: 
%\[
%\mathcal{P} ( A ) = \mathcal{P} ( B ) \cup \left\{ {C \cup \left\{ x \right\} \left| { C \in \mathcal{P} ( B )} \right.} \right\}.
%\]
%\end{enumerate}
%\hbreak




\end{document}

