
\chapter{Preface} \label{fm:preface_old}
\markboth{Preface}{Preface}
\emph{Mathematical Reasoning:  Writing and Proof} is designed to be a text for the first course in the college mathematics curriculum that introduces students to the processes of constructing and writing proofs and focuses on the formal development of mathematics.    The primary goals of the text are as follows:
\begin{itemize}
\item To help students learn how to read and understand mathematical definitions and proofs;
\item To help students learn how to construct mathematical proofs;
\item To help students learn how to write mathematical proofs according to accepted guidelines so that their work and reasoning can be understood by others; and 
\item To provide students with material that will be needed for their further study of mathematic
\end{itemize}

This type of course is becoming a standard part of the mathematics major at most colleges and universities.  It is often referred to as a ``transition course'' from the calculus sequence to the upper-level courses in the major.  The transition is from the problem-solving orientation of calculus to the more abstract and theoretical upper-level courses.  This is needed today because %the principal goals of most calculus courses are developing students' understanding of the concepts of calculus and improving their problem-solving skills.  Consequently, 
many students complete their study of calculus without seeing a formal proof or having constructed a proof of their own.  This is in contrast to many upper-level mathematics courses, where the emphasis is on the formal development of abstract mathematical ideas, and the expectations are that students will be able to read and  understand proofs and be able to construct and write coherent, understandable mathematical proofs.

\section*{Important Features of the Book}
\emph{Mathematical Reasoning: Writing and Proof} was written to assist students with the transition from calculus to upper-level mathematics courses.  Students should be able to use this text with a background of one semester of calculus.  Following are some of the important ways this text will help with this transition.

\begin{enumerate}
\item \textbf{Emphasis on Writing in Mathematics}

Issues dealing with writing mathematical exposition are addressed throughout the book.  Guidelines for writing mathematical proofs are incorporated into the book.  These guidelines are introduced as needed and begin in Chapter~\ref{C:intro}.  Appendix~\ref{C:writingguides}
 contains a summary of all the guidelines for writing mathematical proofs that are introduced throughout the text.  In addition, every attempt has been made to ensure that every completed proof presented in this text is written according to these guidelines.  This provides students with examples of well-written proofs.

%One of the motivating factors for writing this book was to develop a textbook for the course ``Communicating in Mathematics''  at Grand Valley State University.  This course is part of the university's Supplemental Writing Skills Program, and there was no text that dealt with writing issues in mathematics that was suitable for this course. It is because of this course that some of the writing guidelines in the text deal with the use of a word processor that is capable of producing the appropriate mathematical symbols and equations.  However, the writing guidelines can easily be implemented for courses where students do not have access to this type of word processing.  
%
%The course is a sophomore level course with a Calculus I prerequisite that is required for all mathematics majors and minors.  Not every proof that a student completes must be written using a word processor, but there is one major writing assignment in which students are required to use a word processor (available on the campus network) that is capable of producing the appropriate mathematical symbols and equations.  Many faculty call this assignment the ``Proofs Portfolio.''  Since this course is part of the University's Writing Skills Program, students must have an opportunity to work on a writing assignment, get feedback from the instructor, and then have the opportunity to revise their work.  We use the ``Proofs Portfolio'' to satisfy this requirement.  This portfolio consists of ten proofs (or propositions to be proven or disproven).  Students may hand in each proof to the professor two times to be critiqued.  More information about this type of assignment can be obtained by contacting the author and is also available in the Instructor's Manual.

%The issue of writing mathematical exposition is addressed throughout the book.  Guidelines for writing mathematical proofs are incorporated into the text.  These guidelines are introduced as needed and begin in Chapter~\ref{C:intro}.  Appendix~\ref{C:writingguides} contains a summary of all the guidelines for writing mathematical proofs that are introduced in the text.  In addition, every attempt has been made to ensure that each proof presented in this text is written according to these guidelines in order to provide students with examples of well-written proofs.

\item \textbf{Instruction in the Process of Constructing Proofs}

One of the primary goals of this book is to develop students' abilities to construct mathematical proofs.  Another goal is to develop their abilities to write the proof in a coherent manner that conveys an understanding of the proof to the reader.  These are two distinct skills.

Instruction on how to write proofs begins in Section~\ref{S:direct} and is developed further in Chapter~\ref{C:proofs}.  In addition, Chapter~\ref{C:induction} is devoted to developing students' abilities to construct proofs using mathematical induction.  
%Students are taught to organize their thought processes when attempting to construct a proof with a so-called know-show table. (See Sections~\ref{S:direct} and~\ref{S:directproof}.)  Students use this table to work backward from what it is they are trying to prove while at the same time working forward from the assumptions of the problem.

Students are introduced to a method to organize their thought processes when attempting to construct a proof that uses a so-called know-show table. (See Section~\ref{S:direct} and Section~\ref{S:directproof}.)  Students use this table to work backward from what it is they are trying to prove while at the same time working forward from the assumptions of the problem.  The know-show tables are used quite extensively in Chapters~\ref{C:intro} and~\ref{C:proofs}.  However, the explicit use of know-show tables is gradually reduced and these tables are rarely used in the later chapters.  One reason for this is that these tables may work well when there appears to be only one way of proving a certain result.  As the proofs become more complicated or other methods of proof (such as proofs using cases) are used, these know-show tables become less useful.


So the know-show tables are not to be considered an absolute necessity in using the text.  However, they are useful for students beginning to learn how to construct and write proofs.  They provide a convenient way for students to organize their work.  More importantly, they introduce students to a way of thinking about a problem.  Instead of immediately trying to write a complete proof, the know-show table forces students to stop, think, and ask questions such as

\begin{itemize}
\item Just exactly what is it that I am trying to prove?
\item How can I prove this?
\item What methods do I have that may allow me to prove this?
\item What are the assumptions?
\item How can I use these assumptions to prove the result?
\end{itemize}

Being able to ask these questions is a big step in constructing a proof.  The next task is to answer the questions and to use those answers to construct a proof.


\item \textbf{Emphasis on Active Learning}

One of the underlying premises of this text is that the best way to learn and understand mathematics is to be actively involved in the learning process.  However, it is unlikely that  students will learn all the mathematics in a given course on their own.  Students actively involved in learning mathematics need appropriate materials that will provide guidance and support in their learning of mathematics.  This text provides these opportunities by
\begin{itemize}
\item Incorporating two Preview Activities at the beginning of each section.  These Preview Activities should be completed by the students prior to the classroom discussion of the section.  The purpose of the Preview Activities is to prepare students to participate in the classroom discussion of the section.  Some Preview Activities will review prior mathematical work that is necessary for the new section.  This prior work may contain material from previous mathematical courses or it may contain material covered earlier in this text.  Other preview activities will introduce new concepts and definitions that will be used when that section is discussed in class.

\item Including several Progress Checks throughout each section.  These are either short exercises or short activities designed to help the students determine if they are understanding the material as it is presented.  Some progress checks are also intended to prepare the student for the next topic in the section.  Answers to the Progress Checks are provided in Appendix~\ref{C:progress}.

\item Including explorations and activities at the end of the exercises of each section.  These activities can be done individually or in a collaborative learning setting, where students work in groups to brainstorm, make conjectures, test each others' ideas, reach consensus, and, it is hoped, develop sound mathematical arguments to support their work.  These activities can also be assigned as homework in addition to the other exercises at the end of each section.
\end{itemize}


\item \textbf{Other Important Features of the Book}
\begin{itemize}
\item Several sections of the text include exercises called Evaluation of Proofs.  (The first such exercise appears in Section~3.1.)  For these exercises, there is a proposed proof of a proposition.  However, the proposition may be true or may be false.  If a proposition is false, the proposed proof is, of course, incorrect, and the student is asked to find the error in the proof and then provide a counterexample showing that the proposition is false.  However, if the proposition is true, the proof may be incorrect or not well written.  In keeping with the emphasis on writing, students are then asked to correct the proof and/or provide a well-written proof according to the guidelines established in the book.

\item To assist students with studying the material in the text, there is a summary at the end of each chapter.  The summaries usually list the important definitions introduced in the chapter and the important results proven in the chapter.  If appropriate, the summary also describes the important proof techniques discussed in the chapter.

\item Answers or hints for several exercises are included in an appendix.  This was done in response to suggestions from many students at Grand Valley and some students from other institutions who were using the book.  In addition, those exercises with an answer or a hint in the appendix are preceded by a star 
$\left( ^\star \right)$.
\end{itemize}


%The Preview Activities at the beginning of each section should be completed by the students prior to the classroom discussion of the section.  The purpose of the Preview Activities is to prepare students to participate in the classroom discussion of the section.  Some Preview Activities will review prior mathematical work that is necessary for the new section.  This prior work may contain material from previous mathematical courses or it may contain material covered earlier in this text.  Other preview activities will introduce new concepts and definitions that will be used when that section is discussed in class.

%In addition to the Preview Activities, each section of the text contains two or three activities related to the material contained in that section.  These activities can be used for in-class group work or can be assigned as homework in addition to the exercises at the end of each section.

%\item \textbf{Mathematical Content}

%Mathematical content is needed as a vehicle for learning how to construct and write proofs.  The mathematical content for this text is drawn primarily from elementary number theory including congruence arithmetic; elementary set theory; functions, including injections, surjections, and the inverse of a function; and relations and equivalence relations.  This material is needed for upper level mathematics courses.

%\item \textbf{The Role of Logic}

%In order to learn how to construct mathematical proofs, students need to learn some logic and  gain experience in the traditional language and proof methods used in mathematics.  Since this is a text that deals with constructing and writing mathematical proofs, the logic that is presented is intended to aid in the construction of proofs.  The goals are to provide students with a thorough understanding of conditional statements, quantifiers, and logical equivalencies.  Emphasis is placed on writing correct and useful negations of statements, especially those involving quantifiers.  The logical equivalencies that are presented are those that provide the logical basis for some of the standard proof techniques such as proving the contrapositive, proof by contradiction, and proof using cases.
\end{enumerate}

\section*{Content and Organization}

Mathematical content is needed as a vehicle for learning how to construct and write proofs.  The mathematical content for this text is drawn primarily from elementary number theory, including congruence arithmetic; elementary set theory; functions, including injections, surjections, and the inverse of a function; relations and equivalence relations; further topics in number theory such as greatest common divisors and prime factorizations; and cardinality of sets, including countable and uncountable sets.  This material was chosen because it can be used to illustrate a broad range of proof techniques and it is needed as a prerequisite for many upper-level mathematics courses.

The chapters in the text can roughly be divided into the following classes:

\begin{itemize}
\item Constructing and Writing Proofs:  Chapters~\ref{C:intro}, \ref{C:proofs}, and~\ref{C:induction}
\item Logic: Chapter~\ref{C:logic}
\item Mathematical Content: Chapters~\ref{C:settheory}, \ref{C:functions}, \ref{C:equivrelations}, \ref{C:numbertheory}, and~\ref{C:topicsinsets}
\end{itemize}

The first chapter sets the stage for the rest of the book.  It introduces students to the use of conditional statements in mathematics, begins instruction in the process of constructing a direct proof of a conditional statement, and introduces many of the writing guidelines that will be used throughout the rest of the book.  This is not meant to be a thorough introduction to methods of proof.  Before this is done, it is necessary to introduce the students to the parts of logic that are needed to aid in the construction of proofs.  This is done in 
Chapter~\ref{C:logic}.  

Students need to learn some logic and gain experience in the traditional language and proof methods used in mathematics. Since this is a text that deals with constructing and writing mathematical proofs, the logic that is presented in Chapter~\ref{C:logic} is intended to aid in the construction of proofs.  The goals are to provide students with a thorough understanding of conditional statements, quantifiers, and logical equivalencies.  Emphasis is placed on writing correct and useful negations of statements, especially those involving quantifiers.  The logical equivalencies that are presented provide the logical basis for some of the standard proof techniques, such as proof by contrapositive, proof by contradiction, and proof using cases.

The standard methods for mathematical proofs are discussed in detail in Chapter~\ref{C:proofs}.  The mathematical content that is introduced to illustrate these proof methods includes some elementary number theory, including congruence arithmetic.  These concepts are used consistently throughout the text as a way to demonstrate ideas in direct proof, proof by contrapositive, proof by contradiction, proof using cases, and proofs using mathematical induction.  This gives students a strong introduction to important mathematical ideas while providing the instructor a consistent reference point and an example of how mathematical notation can greatly simplify a concept.

The three sections of Chapter~\ref{C:induction} are devoted to proofs using mathematical induction.  Again, the emphasis is not only on understanding mathematical induction but also on developing the ability to construct and write proofs that use mathematical induction.




The last five chapters are considered ``mathematical content'' chapters.     Concepts of set theory are introduced in Chapter~\ref{C:settheory}, and the methods of proof studied in Chapter~\ref{C:proofs} are used to prove results about sets and operations on sets.  The idea of an ``element-chasing proof'' is also introduced in Section~\ref{S:provingset}.

Chapter~\ref{C:functions} provides a thorough study of functions.  Functions are studied before relations in order to begin with the more specific notion with which students have some familiarity and move toward the more general notion of a relation.  The concept of a function is reviewed but with attention paid to being precise with terminology and is then extended to the general definition of a function.  Various proof techniques are employed in the study of injections, surjections, composition of functions,  inverses of functions, and functions acting on sets.  

%\enlargethispage{\baselineskip}
Chapter~\ref{C:equivrelations} introduces the concepts of relations and equivalence relations.  %\linebreak 
Section~\ref{S:modulararithmetic} is included to provide a link between the concept of an equivalence relation and the number theory that has been discussed throughout the text.  

Chapter~\ref{C:numbertheory} continues the study of number theory.  The highlights include problems dealing with greatest common divisors, prime numbers, the Fundamental Theorem of Arithmetic, and linear Diophantine equations.  

Finally, Chapter~\ref{C:topicsinsets} deals with further topics in set theory, focusing on cardinality, finite sets, countable sets, and uncountable sets.

\subsection*{Designing a Course}
Most instructors who use this text will design a course specifically suited to their needs and the needs of their institution.  However, a standard one-semester course in constructing and writing proofs could cover the first six chapters of the text and at least one of Chapter~\ref{C:equivrelations}, Chapter~\ref{C:numbertheory}, or Chapter~\ref{C:topicsinsets}.  A class consisting of well-prepared and motivated students could cover two of the last three chapters.  If either of these options is a bit too ambitious, Sections~\ref{S:recursion}, \ref{S:indexfamily}, \ref{S:functionsonsets}, \ref{S:modulararithmetic}, 
and~\ref{S:diophantine} can be considered optional sections.  These are interesting sections that contain important material, but the content of these sections is not used in the rest of the book.  

%A class consisting of well-prepared and motivated students could cover two of the last three chapters.  In addition, there are a few options that an instructor could choose to tailor the course to her or his needs.  For example,
%
%\begin{itemize}
%\item Chapter~\ref{C:induction} can be covered before Chapter~\ref{C:settheory} if it is desired to cover all methods of proof before beginning the ``content'' portion of the course.  The only part of Chapter~\ref{C:induction} that would need to be skipped is the material in Section~\ref{S:otherinduction} dealing with the cardinality of the power set.  If desired, this material could be included when the power set is discussed in Chapter~\ref{C:settheory}.
%
%\item Instructors who would like to cover topics in both Chapters~\ref{C:equivrelations} and~\ref{C:numbertheory} can omit a few selected sections from earlier chapters.  Although it is an important and interesting section, Section~\ref{S:recursion} is not used in the remainder of the book.  The same is true for Sections~\ref{S:indexfamily},  
%\ref{S:functionsonsets}, \ref{S:modulararithmetic}, 
%and~\ref{S:diophantine}.  
%\end{itemize}


\section*{Changes in the Third Edition}
\begin{itemize}
  \item Many parts of the text have been rewritten or expanded due to comments and questions from students at Grand Valley State University during the 2009-2010 academic year.
  %\item There are no preview activities in Section 1.1.  This was done so that the assignment for the second class day can be to read Section 1.1.
  \item With the exception of Sections~\ref{S:prop} and~\ref{S:reviewproofs}, each section now has exactly two preview activities.  For many sections, this necessitated revisions in the previews and in the development of the material for that section.  In addition, many of the preview activities have been rewritten to include more content and hence,  to make them more of an integral part of the development of the material in each section.  
  \item The activities at the end of each section have been moved to the end of the exercises.  This was done because most of the activities were not an essential part of the development of the material in the text.  Each set of exercises will have a section at the end called ``Explorations and Activities.''
  \item The introduction to quantifiers in Section~\ref{S:predicates} of the second edition has been moved to the preview activities of Section~\ref{S:quantifier}.  The importance of quantifiers remains throughout the rest of the text, including the importance of being able to negate statements with quantifiers.
  \item A subsection on using counterexamples has been added to Section~\ref{S:directproof}.  The concept of a counterexample is introduced in Section~\ref{S:quantifier} and the material in Section~\ref{S:directproof} adds to this by showing students how to use counterexamples and especially, how to present a counterexample in writing.
  \item A subsection has been added to Section~\ref{S:contradiction} (Proof by Contradiction) discussing how to prove that something does not exist.
  \item A new section, Section~\ref{S:reviewproofs} -- Review of Proof Methods, has been added.  This section is different than other sections in the text.  There are no preview activities, and this section is intended to be a review of the proof methods studied in Chapter~\ref{C:proofs}.  In particular, the exercises will provide the students an opportunity to work on problems that are not in a specific section, and so they must decide what proof method to use.
  \item In the second edition, Chapter~\ref{C:induction} was ``Set Theory,'' and Chapter~\ref{C:settheory} was ``Mathematical Induction.'' In the third edition, the order of these two chapters has been reversed.  So there is now Chapter~\ref{C:induction}, ``Mathematical Induction,'' and Chapter~\ref{C:settheory}, ``Set Theory.''  This was done so that the basic proof methods are studied in Chapters~\ref{C:intro}, \ref{C:proofs}, and~\ref{C:induction} and the remaining chapters then use these proof methods to help study a particular area of mathematics.

Because the the order of the chapters on set theory and induction were reversed, Section 4.2 in the third edition no longer contains the induction proof for the result that a set with $n$ elements has $2^n$ subsets.  This is now in the new Section~5.1.
  \item Venn diagrams are now introduced in the preview activities of Section~\ref{S:setoperations} rather than at the end of the section.
  \item The material in the first few sections of Chapter~\ref{C:functions}~(Functions) has been reorganized.  One of the main differences is that the material on functions of two variables has been moved from Section~\ref{S:introfunctions} to Section~\ref{S:moreaboutfunctions}.  In addition, a subsection on working with a function of two variables has been added to Section~\ref{S:typesoffunctions} -- Injections, Surjections, and Bijections.
  \item The material on the inverse of a relation in Section~\ref{S:relations}~(Relations) has been moved and is now in the Explorations and Activities part of the exercises.  A new subsection dealing with visual representations of relations has been added.  So the concept of a directed graph for a relation is now in Section~\ref{S:relations}.
\end{itemize}



\section*{Supplementary Materials for the Instructor}
The instructor's manual for this text includes suggestions on how to use the text, how to incorporate writing into the course, and how to use the preview activities and activities.  The manual also includes solutions for all of the preview activities, activities, and exercises.  In addition, for each section, there is a description of the purpose of each preview activity and how it is used in the corresponding section, and there are suggestions about how to use each activity in that section.  The intention is to make it as easy as possible for the instructor to use the text in an active learning environment.  These activities can also be used in a more traditional lecture-discussion course.  In that case, some of the activities would be discussed in class or assigned as homework.

%The instructor's manual is available by contacting the editorial offices at Prentice Hall in Upper Saddle River, NJ. 
%or by emailing a request to george\_lobell@prenhall.com.

PDF files are also available to instructors who use the text to assist them in posting solutions to a course web page or distributing printed solutions to students.  For each section, there is a file containing the solutions of the preview activities, and for each activity in the text, there is a file containing the solutions for that activity.  Instructors can contact the author through his e-mail address (sundstrt@gvsu.edu) for access to the files.

%In addition, all instructor resources can be downloaded from the Prentice Hall Web site, www.prenhall.com. Select ``Browse our catalog,'' then click on ``Mathematics''; select your course and choose your text. Under ``Resources,'' on the left side, select ``instructor'' and choose the supplement you need to download. You will be required to run through a one-time registration before you can complete this process. 


\section*{Acknowledgments}
%Both the first edition and the second edition of this text have benefited greatly from the comments of students who have used text and from ideas and insights of the reviewers of the text.  I would like to thank the following reviewers:  
%$$
%\BeginTable
%    \BeginFormat
%    | p(2in) | p(2in) |
%    \EndFormat
%"Frank B\"{a}uerle, \textit{University of California, Santa Cruz} " Michael C. Berg, 
%\textit{Loyola Marymount University} "  \\+09
%%"  "  " \\ 
%" Dorothee Blum, \textit{Millersville University of Pennsylvania} " Lifeng Ding, \textit{Georgia State University} " \\+39
%%"  "  "  \\
%" Jeffrey Ehme, \\ \textit{Spelman College} " Christopher P. Grant, \textit{Brigham Young University} " \\+39
%" Joel Iiams, \textit{University of North Dakota} " Robert Jajcay, \textit{Indiana State University} " \\+39
%\EndTable
%$$
%
%$$
%\BeginTable
%    \BeginFormat
%    | p(2in) | p(2in) |
%    \EndFormat
%%" Joel Iiams, University of North Dakota " Robert Jajcay, Indiana State University " \\+09
%%"  "  "  \\
%" Corlis Johnson, \textit{Mississippi State University} " Iraj Kalantari, \textit{Western Illinois University} " \\+39
%%"  "  "  \\
%" William J. Keane, \textit{Boston College} "  Chris Meyer, \textit{Pacific Lutheran University} "  \\+39
%%"  "  "  \\
%" Aaron Montgomery, \textit{Central Washington University} " Revathi Narasimhan, \textit{Kean University} " \\+39
%%"  "  "  \\
%" Helen E. Salzburg, \textit{Rhode Island College} " Kenneth Schilling, \textit{University of Michigan, Flint} "  \\+39
%%"  "  "  \\
%" Dan Singer, \textit{Minnesota State University, Mankato} " Mary Wiest, \textit{Minnesota State University, Mankato} " \\+30
%\EndTable
% $$

%\begin{list}{}
%\item Frank B\"{a}uerle, University of California, Santa Cruz; 
%Michael C. Berg, Loyola Marymount University;
%Dorothee Blum, Millersville University of Pennsylvania;  %2
%Lifeng Ding, Georgia State University; 
%Jeffrey Ehme, Spelman College; 
%Christopher P. Grant, Brigham Young University; 
%Joel Iiams, University of North Dakota; 
%Robert Jajcay, Indiana State University;
%Corlis Johnson, Mississippi State University;
%Iraj Kalantari, Western Illinois University; 
%William J. Keane, Boston College;
%Chris Meyer, Pacific Lutheran University; 
%Aaron Montgomery, Central Washington University; 
%Revathi Narasimhan, Kean University; 
%Helen E. Salzburg, Rhode Island College;
%Kenneth Schilling, University of Michigan, Flint; 
%Dan Singer, Minnesota State University, Mankato; and 
%Mary Wiest, Minnesota State University, Mankato.   
%
%\end{list}
\vskip9pt
%\enlargethispage{\baselineskip}
I would like to express my sincere gratitude to my colleagues (and former colleagues) at Grand Valley State University who have used the text and have supported my work on this text.  This includes Professors 
Ed Aboufadel, 
David Austin, 
Will Dickinson, 
Karen Heidenreich, 
Firas Hindeleh, 
Jonathan Hodge, 
Reva Kasman, 
Darren Parker, 
Philip Pratt, 
Steve Schlicker, 
Shelly Smith, 
Jody Sorensen, 
Clark Wells,  
Pamela Wells, and 
Matt Wyneken.  Special thanks go to Professor Karen Novotny, who helped me immensely in writing the first draft of the text, Professor Matt Boelkins, who carefully read the completed text of the first edition and offered numerous helpful suggestions, and Professor Larry King of the University of Michigan--Flint, who used an early version of the manuscript and made several useful suggestions for further development of the text.  
%In addition, I would like to thank Prof. Philip Pratt and Prof. Steve Schlicker, who were the chairs of the department while I was writing the text, for their support and encouragement.

Finally, a very special thank you goes to my wife, Karen, and my daughter, Laura, for their patience, understanding, and encouragement during the long time it took to develop this text.

Comments about the text and suggestions for improving it are welcome.

\begin{flushright}
\emph{Ted Sundstrom}

sundstrt@gvsu.edu
\end{flushright}



\newpage
\thispagestyle{empty}

\endinput

%These files also will help the instructor distribute solutions of the preview activities and activities after the students have worked on them.  
