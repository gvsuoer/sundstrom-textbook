\documentclass[11pt]{article}
\usepackage{../../styles/activity}

\usepackage{xr}
\externaldocument{0-MR}

\lhead{}
%\chead{\textbf{\Large{\hspace{0pt}Beginning Activities for Section~7.4}}\\\hspace{0pt}\emph{Mathematical Reasoning: Writing and Proof}}
\bahead{7.4}
\rhead{}
\lfoot{}
\rfoot{}
\cfoot{\hspace{0pt}\scalebox{0.4}{\includegraphics{cc-by-nc-sa.eps}}}
\graphicspath{{./epsfigs/}}

\begin{document}
%\subsection*{Beginning Activity 1 (Congruence Modulo 5)}
%\begin{enumerate}
%\item $[4] = \left\{ \ldots, -11, -6, -1, 4, 9, 14, \ldots \right\}$ and 
%$[2] = \left\{ \ldots, -13, -8, -3, 2, 7, 12, \ldots \right\}$
%\item  In all cases, $s \equiv 1 \pmod 5$.
%\item In all cases, $p \equiv 3 \pmod 5$.
%\end{enumerate}
%\hbreak
%
%\noindent
\subsection*{Beginning Activity 1 (Congruence Modulo 6)}
\begin{enumerate}
\item $[3] = \left\{ \ldots, -15, -9, -3, 3, 9, 15, \ldots \right\}$ and 
$[4] = \left\{ \ldots, -14, -8, -2, 4, 10, 16, \ldots \right\}$. 

\item  In all cases, $s \equiv 1 \pmod 6$.
\item In all cases, $p \equiv 0 \pmod 6$.
\item In all cases, $q \equiv 3 \pmod 6$.
\end{enumerate}
These results with congruence modulo 6 illustrate general properties of the equivalence classes for congruence modulo $n$.  After a little more work at the start of this section, this will allow us to define an addition and multiplication of congruence classes modulo $n$ on page~404.
\hbreak

\noindent
\subsection*{Beginning Activity 2 (The Remainder When Dividing by 9)}

\begin{center}
\begin{tabular}[t]{| c | c | c | c |} \hline
     &                       &  Remainder when $n$  &  Remainder when $s \left( n \right)$ \\ 
$n$  &  $s \left( n \right)$  &  is divided by 9     &  is divided by 9  \\ \hline
498  &  21  &  3  &  3  \\ \hline
7319  &  20  &  2  &  2  \\ \hline
4672  &  19  &  1  &  1  \\ \hline
9845  &  26  &  8  &  8  \\ \hline
51381  &  18  &  0  &  0  \\ \hline
305877  &  30  &  3  &  3  \\ \hline
\end{tabular}
\end{center}
In each case, $n$ and $s(n)$ have the same remainder when divided by 9.  We will prove this result in this section and it will help justify a so-called \textbf{divisibility test} for division by 9.  That result is, 9 divides an integer $n$  if and only if 9 divides the sum of its digits.  

\eighth
\noindent
For example, 9 divides the integer 5832 since $5 + 8 + 3 + 2 = 18$ and 9 divides 18.  Also, 9 does not divide the integer 7265 since $7 + 2 + 6 + 5 = 20$ and 9 does not divide 20.
\hbreak

\end{document}
