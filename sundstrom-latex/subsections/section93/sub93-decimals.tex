We have seen examples of sets that are countably infinite, but we have not yet seen an example of an infinite set that is uncountable.  We will do so in this section.  The first example of an uncountable set will be the open interval of real numbers $( 0, 1 )$.  The proof that this interval is uncountable uses a method similar to the winning strategy for Player Two in the game of Dodge Ball from \typeu Activity~\ref*{PA:dodgeball}.  Before considering the proof, we need to state an important result about decimal expressions for real numbers.

\subsection*{Decimal Expressions for Real Numbers}
\index{decimal expression!for a real number}%

In its decimal form, any real number $a$ in the interval $( 0, 1 )$ can be written as
$a = 0.a_1 a_2 a_3 a_4 \ldots ,$ where each $a_i$ is an integer with $0 \leq a_i \leq 9$.  For example,
\[
\frac{5}{12} = 0.416666 \ldots.
\]
We often abbreviate this as $\dfrac{5}{12} = 0.41 \overline{6}$ to indicate that the $6$ is repeated.  We can also repeat a block of digits.  For example, 
$\dfrac{5}{26} = 0.19 \overline{230769}$ to indicate that the block 230769 repeats.  That is, 
\[
\frac{5}{26} = 0.19230769230769230769 \ldots.
\]
There is only one situation in which a real number can be represented as a decimal in more than one way.  A decimal that ends with an infinite string of 9's is equal to one that ends with an infinite string of 0's.  For example, $0.3199999 \ldots$ represents the same real number as 
$0.3200000 \ldots .$  Geometric series can be used to prove that a decimal that ends with an infinite string of 9's is equal to one that ends with an infinite string of 0's, but we will not do so here.

\begin{defbox}{normalized}{A decimal representation of a real number $a$ is in \textbf{normalized form}
\index{decimal expression!normalized form}%
\index{normalized form!of a decimal expression}%
\label{normalizedform}%
 provided that there is no natural number $k$ such that for all natural numbers $n$ with $n > k$, $a_n = 9$.  That is, the decimal representation of $a$ is in normalized form if and only if it does not end with an infinite string of 9's.}
\end{defbox}

\newpar
One reason the normalized form is important is the following theorem (which will not be proved here).
\begin{theorem} \label{T:normalized} Two decimal numbers in normalized form are equal if and only if they have identical digits in each decimal position.
\end{theorem}

\endinput
