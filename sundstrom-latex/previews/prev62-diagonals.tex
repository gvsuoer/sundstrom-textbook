\begin{previewactivity}[\textbf{The Number of Diagonals of a Polygon}] \label{PA:diagonals} \hfill \\
A \textbf{polygon}
\index{polygon}%
 is a closed plane figure formed by the joining of three or more straight lines. For example, a \textbf{triangle}
\index{triangle}%
 is a polygon that has three sides; a \textbf{quadrilateral}
\index{quadrilateral}%
 is a polygon that  has four sides and includes squares, rectangles, and parallelograms; a \textbf{pentagon}
\index{pentagon}%
 is a polygon that  has five sides; and an \textbf{octagon}
\index{octagon}%
 is a polygon that has eight sides. A \textbf{regular polygon}
\index{regular polygon}%
\index{polygon!regular}%
 is one that has equal-length sides and congruent interior angles.

A \textbf{diagonal of a polygon}
\index{diagonal}%
\index{polygon!diagonal}%
 is a line segment that connects two nonadjacent vertices of the polygon.  In this activity, we will assume that all polygons are \textbf{convex polygons} so that, except for the vertices, each diagonal lies inside the polygon.  For example, a triangle (3-sided polygon) has no diagonals and a rectangle has two diagonals.

\begin{enumerate}
\item How many diagonals does any quadrilateral (4-sided polygon) have?

\item Let   $D = \mathbb{N} - \left\{ {1, 2} \right\}$.  Define   
$d\x D \to \mathbb{N} \cup \left\{ 0 \right\}$ so that   $d( n )$ is the number of diagonals of a  convex polygon with  $n$  sides.   Determine the values of $d(3)$, $d(4)$, $d(5)$, $d(6)$, $d(7)$, and $d(8)$.  Arrange the results in the form of a table of values for the function $d$.
\label{PA:diagonals2}

\item Let  $f\x \mathbb{R} \to \mathbb{R}$  be defined by  
\[
f( x ) = \frac{{x\left( {x - 3} \right)}}{2}.
\]
Determine the values of $f(0)$, $f(1)$, $f(2)$, $f(3)$, $f(4)$, $f(5)$, $f(6)$, $f(7)$, 
$f(8)$, and $f(9)$.  Arrange the results in the form of a table of values for the function $f$\!.
\label{PA:diagonals3}%


\item	Compare the functions in Parts~(\ref{PA:diagonals2}) and~(\ref{PA:diagonals3}).  What are the similarities between the two functions and what are the differences?  Should these two functions be considered equal functions?  Explain.
\end{enumerate}
\end{previewactivity}
\hbreak


\endinput
