\section*{Exercises \ref{S:infinitesets}}

\begin{enumerate}
\xitem State whether each of the following is true or false.
\label{exer:sec93-1}%
\begin{enumerate}
\item If a set $A$ is countably infinite, then $A$ is infinite.
\item If a set $A$ is countably infinite, then $A$ is countable.
\item If a set $A$ is uncountable, then $A$ is not countably infinite.
\item If $A \approx \mathbb{N}_k$ for some $k \in \mathbb{N}$, then $A$ is not countable.
\end{enumerate}

\item Prove that each of the following sets is countably infinite.
\label{exer:sec93-2}%
\begin{enumerate}
\item The set $F^+$ of all natural numbers that are multiples of 5
\item The set $F$ of all integers that are multiples of 5
\end{enumerate}
\begin{multicols}{2}
\begin{enumerate} \setcounter{enumii}{2}
\item $\left\{ \left. \dfrac{1}{2^k} \right| k \in \mathbb{N} \right\}$
\item $\left\{ n \in \mathbb{Z} \mid n \geq -10 \right\}$
\yitem $\mathbb{N} - \left\{ 4, 5, 6 \right\}$
\yitem $\left\{ m \in \mathbb{Z} \mid m \equiv 2 \pmod 3 \right\}$
\end{enumerate}
\end{multicols}

\item Prove part~(\ref{T:subsetisinfinite2}) of Theorem~\ref{T:subsetisinfinite}. 
\label{exer:subsetisinfinite}%

Let $A$ and $B$ be sets.  If $A$ is infinite and $A \subseteq B$, then $B$ is infinite.



\item Complete the proof of Theorem~\ref{T:addonetocountable} by proving the following:
\label{exer:addonetocountable}%

\noindent
Let $A$ be a countably infinite set and $x \notin A$.  If $f\x \mathbb{N} \to A$ is a bijection,  then $g$ is a bijection, where
$g\x \mathbb{N} \to A \cup \left\{ x \right\}$ by
\begin{equation} \notag
g( n ) = 
\begin{cases}
x                        &\text{if $n = 1$} \\
f ( n - 1 )   &\text{if $n > 1$}.
\end{cases}
\end{equation}

\xitem Prove Theorem~\ref{T:addfinitetocountable}. 
\label{exer:addfinitetocountable}%

If $A$ is a countably infinite set and $B$ is a finite set, then $A \cup B$ is a countably infinite set.

\hint  Let $\text{card} ( B ) = n$ and use a proof by induction on $n$.  
Theorem~\ref{T:addonetocountable} is the basis step.

\xitem Complete the proof of Theorem~\ref{T:unionofcountable} by proving the following: 
\label{exer:unionofcountable}%

Let $A$ and $B$ be disjoint countably infinite sets and let $f\x \mathbb{N} \to A$ and 
$g\x \mathbb{N} \to B$ be bijections.  Define $h\x \mathbb{N} \to A \cup B$ by
\begin{equation} \notag
h( n ) = 
\begin{cases}
f \!\left( \dfrac{n+1}{2} \right)                        &\text{if $n$ is odd} \\
                                                       & \\
g \!\left( \dfrac{n}{2} \right)                          &\text{if $n$ is even}.
\end{cases}
\end{equation}
Then the function $h$ is a bijection.

\xitem Prove Theorem~\ref{T:Qiscountable}.
\label{exer:Qiscountable}%

The set $\mathbb{Q}$ of all rational numbers is countable.

\hint  Use Theorem~\ref{T:addonetocountable} and Theorem~\ref{T:unionofcountable}.

\xitem Prove that if $A$ is countably infinite and $B$ is finite, then $A - B$ is countably infinite. 
\label{exer:countinf-finite}%

\item Define $f\x \mathbb{N} \times \mathbb{N} \to \mathbb{N}$ as follows:  For each 
$( m, n ) \in \mathbb{N} \times \mathbb{N}$, 
\label{exer:NxNbijection}%
\[
f ( m, n ) = 2^{m-1} (2n - 1 ).
\] 
\begin{enumerate}
\item Prove that $f$ is an injection.  \hint  If 
$f ( m, n ) = f ( s, t )$, there are three cases to consider:  $m > s$, $m < s$, and $m = s$. Use laws of exponents to prove that the first two cases lead to a contradiction.

\item Prove that $f$ is a surjection.  \hint  You may use the fact that if $y \in \mathbb{N}$, then $y = 2^k x$, where $x$ is an odd natural number and $k$ is a non-negative integer.  This is actually a consequence of the Fundamental Theorem of Arithmetic, Theorem~\ref{T:fundtheorem}.  [See Exercise~(\ref{exer:fundtheoremcons}) in Section~\ref{S:primefactorizations}.]

\item Prove that $\mathbb{N} \times \mathbb{N} \approx \mathbb{N}$ and hence that 
$\text{card} \left( \mathbb{N} \times \mathbb{N} \right) = \aleph_0$.
\end{enumerate}

\item Use Exercise~(\ref{exer:NxNbijection}) to prove that if $A$ and $B$ are countably infinite sets, then $A \times B$ is a countably infinite set.

\item Complete the proof of Theorem~\ref{T:subsetsofN} by proving that the function $g$ defined in the proof is a bijection from $\mathbb{N}$ to $B$. 
\label{exer:subsetofN}%

\hint  To prove that $g$ is an injection, it might be easier to prove that for all 
$r, s \in \mathbb{N}$, if $r \ne s$, then $g ( r ) \ne g ( s )$.  To do this, we may assume that $r < s$ since one of the two numbers must be less than the other.  Then notice that $g ( r ) \in \left\{ g( 1 ), g( 2 ), \ldots, g( s-1 ) \right\}$.

To prove that $g$ is a surjection, let $b \in B$ and notice that for some $k \in \mathbb{N}$, there will be $k$ natural numbers in $B$ that are less than $b$.

\item Prove Corollary~\ref{C:subsetofcountable}, which states that every subset of a countable set is countable. 
\label{exer:subsetofcountable}%

\hint  Let $S$ be a countable set and assume that $A \subseteq S$.  There are two cases:  $A$ is finite or $A$ is infinite.  If $A$ is infinite, let $f\x S \to \mathbb{N}$ be a bijection and define $g\x A \to f ( A )$ by $g ( x ) = f ( x )$, for each $x \in A$.

\item Use Corollary~\ref{C:subsetofcountable} to prove that the set of all rational numbers between 0 and 1 is countably infinite. 

\item \label{exer:tworationals} Let $a, b \in \mathbb{Q}$ with $a < b$. On page~\pageref{tworationals}, we proved that $c = \dfrac{a+b}{2}$ is a rational number and that $a < c < b$, which proves that here is a rational number between any two (unequal) rational numbers.
\begin{enumerate}
\item Now let  $c_1 = \dfrac{a+b}{2}$, and define $c_2 = \dfrac{c_1 + b}{2}$.  Prove that 
$c_1 < c_2 < b$ and hence, that $a < c_1 < c_2 < b$.


\item For each $k \in \mathbb{N}$, define
\[
c_{k+1} = \frac{c_k + b}{2}.
\]
Prove that for each $k \in \mathbb{N}$, $a < c_k < c_{k+1} < b$.  Use this to explain why the set 
$\left\{ c_k \mid k \in \mathbb{N} \right\}$ is an infinite set where each element is a rational number between $a$ and $b$.
\end{enumerate}
\end{enumerate}



\subsection*{Explorations and Activities}
\setcounter{oldenumi}{\theenumi}
\begin{enumerate} \setcounter{enumi}{\theoldenumi}
\item \textbf{Another Proof that $\mathbf{\Q^+}$ Is Countable}. \label{A:Qcountable} 
For this activity, it may be helpful to use the Fundamental Theorem of Arithmetic (see Theorem~\ref{T:fundtheorem} on page~\pageref{T:fundtheorem}).  Let $\Q^+$ be the set of positive rational numbers.  Every positive rational number has a unique representation as a fraction $\dfrac{m}{n}$, where $m$ and $n$ are relatively prime natural numbers.
We will now define a function $f\x  \Q^+ \to \N$ as follows:

\eighth
\noindent
If $x \in \Q^+$ and $x = \dfrac{m}{n}$, where $m, n \in \N$, $n \ne 1$ and $\gcd(m, n) = 1$, we write
\[
\begin{aligned}
m &= p_1^{\alpha_1} p_2^{\alpha_2} \cdots p_r^{\alpha_r}, \quad \text{and} \\
n &= q_1^{\beta_1} q_2^{\beta_2} \cdots q_s^{\beta_s}, 
\end{aligned}
\]
where $p_1$, $p_2$, \ldots, $p_r$ are distinct prime numbers, $q_1$, $q_2$, \ldots, $q_s$ are distinct prime numbers, and 
$\alpha_1$, $\alpha_2$, \ldots, $\alpha_r$ and $\beta_1$, $\beta_2$, \ldots, $\beta_s$ are natural numbers.    
%In addition, for each $j$ with $1 \leq j \leq r$ and for each $k$ with 
%$1 \leq k \leq s$, $p_j \ne q_k$.  
We also write $1 = 2^0$ when $m = 1$.  We then define

\[
f(x) = p_1^{2\alpha_1} p_2^{2\alpha_2} \cdots p_r^{2\alpha_r}
q_1^{2\beta_1 -1 } q_2^{2 \beta_2 - 1} \cdots q_s^{2 \beta_s - 1}.
\]

\noindent
If $x = \dfrac{m}{1}$, then we define 
$f(x) = p_1^{2\alpha_1} p_2^{2\alpha_2} \cdots p_r^{2\alpha_r} = m^2$.  

\begin{enumerate}
\item Determine $f \!\left( \dfrac{2}{3} \right)$, $f \!\left( \dfrac{5}{6} \right)$, 
$f ( 6 )$, $f ( \dfrac{12}{25} )$, 
$f \!\left( \dfrac{375}{392} \right)$, and $f \!\left( \dfrac{2^3 \cdot 11^3}{3 \cdot 5^4} \right)$.

\item If possible, find $x \in \Q^+$ such that $f(x) = 100$.

\item If possible, find $x \in \Q^+$ such that $f(x) = 12$.

\item If possible, find $x \in \Q^+$ such that $f(x) = 2^8 \cdot 3^5 \cdot 13 \cdot 17^2$.

\item Prove that the function $f$ is an injection.

\item Prove that the function $f$ is a surjection.

\item What has been proved?
\end{enumerate}
\end{enumerate}


\hbreak


\endinput




