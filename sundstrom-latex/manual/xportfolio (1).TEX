\subsection*{Examples of Portfolio Problems}
Since a draft of a portfolio problem is expected by the end of the fourth week, and we do not start Chatper~\ref{C:proofs} until the end of the third week, I always include one problem that can be started after completing Chapter~\ref{C:intro}.  Also, many of the problems that I have included in past Portfolio Projects have been included in the exercises in the text.  I sometimes use Exercises from later in the text since they can often be done with the tools developed in Chapter~\ref{C:proofs} with perhaps a little extra guidance.

\subsubsection*{Problems that Can Be Started After Chapter~\ref{C:intro}.}

\begin{enumerate}
\item (Exercise~(\ref{exer:sec12-11}) in Section~\ref {S:direct}) Find all solutions of two quadratic equations of the form  $ax^2  + bx + c = 0$
 where  $a$, $b$, and  $c$  are real numbers, $a > 0$, and   $c < 0$.

Prove or disprove the following:
If  $a$, $b$, and  $c$  are real numbers with $a > 0$ and   $c < 0$, then at least one solution of the quadratic equation  $ax^2  + bx + c = 0$ is a positive real number.


\item (Similar to Exercise~(\ref{exer:sec12-morepythag}) in Section~\ref {S:direct})  The \textbf{Pythagorean Theorem} for right triangles states that if $a$ and $b$ are the lengths of the legs of a right triangle and $c$ is the length of the hypotenuse, then 
$a^2 + b^2 = c^2$.  For example, if the lengths of the legs of a right triangle are 4 and 7 units, then $c^2 = 4^2 + 7^2 = 63$, and the length of the hypotenuse must be $\sqrt{13}$ units (since the length must be a positive real number).

\eighth
\noindent
Prove that if $m$ is a real number the lengths of the three sides of a right triangle are $m$, $m + 7$ and $m + 8$ units, then the length of the hypotenuse must be 13 units.

\item (Part of Exercise~(\ref{exer:sec32-6}) in Section~\ref{S:moremethods}) For a right triangle, suppose that the hypotenuse has length  $c$  feet and the lengths of the sides are  $a$  feet and  $b$  feet.  If the area of the right triangle is  
$\dfrac{1}{4}c^2$, then the triangle is an isosceles triangle.

\item \begin{enumerate}
\item Is the following proposition true or false?  Justify your conclusion.

If  $x$  and  $y$  are real numbers and $xy > 0$, then  
$\dfrac{{x + y}}{2} \ge \sqrt {x{\kern 1pt} y}$.

\item If the proposition is true, write a complete proof for the proposition.  If it is false, add a reasonable condition to the hypothesis so that the new proposition is true.  Then, write a complete proof of this new proposition.

\end{enumerate}
\item Is the following proposition true or false?  Justify your conclusion.

For each integer  $n$,  $4n^2 - 6n + 3$  is an odd integer.

\item Let  $x$  be a real number. If  $0 < x < \dfrac{\pi }{2}$, then  
$\left( {\sin x + \cos x} \right) > 1$.

\underline{Note}:  This is Exercise~(\ref{exer:sec33-11}) in Section~\ref{S:contradiction}.  However, it can also be proven using a direct proof.  Many students shy away from this problem since it involves trigonometric identities.

\end{enumerate}

\subsubsection*{Problems that Can Be Started After Chapter~\ref{C:proofs}.}
\begin{enumerate}
\item Is the real number $\sqrt {12}$  a rational number or an irrational number?  Justify your conclusion.

\item (Exercise~(\ref{exer:sec32-equation17}), Section~\ref{S:moremethods})
\begin{enumerate}
\item Give examples of at least two different equations of the form  $ax^3  + bx + (b + a) = 0$   where $a$  and  $b$  are integers,  $a$  does not divide  $b$, and   $a$  is not equal to zero.

\item Find decimal approximations of all real number solutions of each of the equations from 
Part~(a).  Note:  You could use Maple or a graphing calculator to find these approximate solutions.

\item Assume that  $a$  and  $b$  are integers with  $a$  not equal to zero.  Consider the following statement:

If  $a$  does not divide  $b$, then the equation  $ax^3  + bx + (b + a) = 0$  has no solution that is a natural number.

Is this statement true or false?  Justify your conclusion.
\end{enumerate}

\item Is the following proposition true or false?  Justify your conclusion.

For all nonzero integers   $a$  and  $b$, if  $a + b \ne 7$  and  $49a + b \ne 1$, then the equation  $an^3  + bn - 7 = 0$ has no natural number solution.

\item Does the equation $n^7  - 3n^4  - 9n - 7 = 0$ have a solution that is a natural number?  Either find a natural number solution or prove that none exists.

\item Is the following proposition true or false?  Justify your conclusion.

Let  $n$  be a natural number.  If  3  does not divide  $\left( {n^2  + 2} \right)$, then  $n$  is not a prime number or  $n = 3$.


\item (Exercise~(\ref{exer:sec35-10}), Section ~\ref{S:contradiction}.) Prove or disprove the following:

There exist three consecutive natural numbers such that the cube of the largest one is equal to the sum of the cubes of the other two.

\item If  $n$  is a natural number and $m = n + 1$, then  $n$  and  $m$  are said to be consecutive natural numbers.  If  $n$  is an odd natural number and $m = n + 2$, then  $n$  and  
$m$  are said to be consecutive odd natural numbers.

Notice that  3, 5, and  7  are three consecutive odd natural numbers, all of which are prime.  Are there any others?  Either find three other consecutive odd natural numbers, all of which are prime, or prove that, except for 3, 5, and 7, every triple of consecutive odd natural numbers contains at least one composite number.

\item (Exercise~(\ref{exer:sec82-twinprimes}), Section~\ref{S:primefactorizations}) Give examples of three different pairs of prime numbers that differ by two.  Such pairs of numbers are said to be twin primes.  Calculate the product of  each of your examples of twin primes.  Is the following proposition true or false:

If  $p $ and  $q$  are twin primes other than the pair 3 and 5, then $pq + 1$ is a perfect square that is divisible by 36.

\item If  $x$, $y$, and  $z$  are natural numbers such that  $x^2  + y^2  = z^2 $
 and  $\gcd \left( {x, y, z} \right) = 1$, then exactly one of the natural numbers  $x$  and  $y$  is odd.

Note:  The greatest common divisor of three integers is the largest natural number that is a divisor of all three integers.  For example:

\begin{center}
$\gcd \left( {4, 20, 30} \right) = 2{\rm{   and   }}\gcd \left( {4, 20, 25} \right) = 1$.
\end{center}

\item \begin{enumerate}
\item Is the following proposition true or false?  Justify your conclusion.

For each integer  $n$,  if  $n$  is an odd integer, then  $n^2  \equiv 1 \pmod 8$.

\item Is the following proposition true or false?  Justify your conclusion.

For each integer  $n$, $n^2  \equiv 1 \pmod 8$  or  $n^2  \equiv 4 \pmod 8$.
\end{enumerate}

\item Is the following proposition true or false?  Justify your conclusion.

For all $a, b \in \mathbb{Z}$, if  $\left( {a^2  + b^2 } \right) \equiv 0 \pmod 3$, then  
$a \equiv 0 \pmod 4$  and  \linebreak $b \equiv 0 \pmod 3$.

\item Following are several examples of ordered triples  $\left( {x, y, z} \right)$
  where  $x$, $y$, and  $z$  are natural numbers that have no common factor except 1 and  
$x^2  + y^2  = z^2$.

\begin{multicols}{3}
$\left( {3,\,4,\,5} \right)$

$\left( {5,\,12,\,13} \right)$
	
$\left( {8,\,15,\,17} \right)$

$\left( {7,\,24,\,25} \right)$

$\left( {20,\,21,\,29} \right)$

$\left( {9,\,40,\,41} \right)$

$\left( {12,\,35,\,37} \right)$

$\left( {11,\,60,\,61} \right)$

%$\left( {28,\,45,\,53} \right)$

$\left( {33,\,56,\,65} \right)$

%$\left( {16,\,63,\,65} \right)$
\end{multicols}

Is the following statement true or false?  Justify your conclusion.

If   $x$, $y$, and  $z$  are natural numbers that have no common factor except 1 and  
$x^2  + y^2  = z^2 $, then one of  $x$ , $y$,  and  $z$  is divisible by 5.


\end{enumerate}

\subsubsection*{Problems that Can Be Started After Chapter~\ref{C:settheory}.}
\begin{enumerate}
\item (Exercise~(\ref{exer:sec43-6}), Section~\ref{S:setproperties}) Prove or disprove the following:

For any sets  $A$, $B$, and  $C$ that are subsets of a universal set  $U$, \linebreak 
$A - \left( {B \cap C} \right) = \left( {A - B} \right) \cup \left( {A - C} \right)$.

\item (Exercise~(\ref{exer:sec43-setdiff3x}), Section~\ref{S:setproperties}) Let  $A$,  $B$, and  $C$  be subsets of some universal set  $U$.  Use Venn diagrams to explore the relation between the two sets  $A - \left( {B - C} \right)$  and  
$\left( {A - B} \right) - C$. Based on these diagrams, what appears to be the relation between the sets  $\left( {A - B} \right) - C$  and   $A - \left( {B - C} \right)$?  

Formulate two propositions, each one of which states that one of these sets is (or is not) a subset of the other.  Then, justify the conclusions of these propositions.

\item (Exercise~(\ref{exer:goldbach}) in Section~\ref{S:provingset}) One of the most famous unsolved problems in mathematics is a conjecture made by Christian Goldbach in a letter to Leonhard Euler in 1742.  The conjecture made in this letter is now known as \textbf{Goldbach's Conjecture}.

State Goldbach's Conjecture and explain what it would take to prove that Goldbach's conjecture is false.  Then, prove the following:

If there exists an odd integer greater than 5 that is not the sum of three prime numbers, then Goldbach's Conjecture is false.

\item Is the following proposition true or false?  Justify your conclusion.

For any sets  $A$, $B$, and  $C$, 
$\left( {A - B} \right) \times C = \left( {A \times C} \right) - \left( {B \times C} \right)$.

If this proposition is false, you should investigate whether one of the sets is a subset of the other set.  If there is such a subset relation, you should include a proof.


\end{enumerate}


\subsubsection*{Problems that Can Be Started After Chapter~\ref{C:induction}.}
\begin{enumerate}
\item Let  $n$  be a natural number with $n \geq 3$.  A convex polygon with  $n$  sides is a polygon with  $n$  sides with the additional property that the straight line segment between any two points of the polygon lies entirely within the polygon.  So, for example, a triangle is a convex polygon with 3 sides.  In Euclidean geometry, what is the sum of the interior angles, in radians, of a triangle?

Develop a formula for the sum of the interior angles, in radians, of the interior angles of a convex polygon with  $n$  sides and prove that your formula is correct.

\item Is the following proposition true or false?  Justify your conclusion.

For each natural number $n$, 6 divides $n^3 - n$.

Note:  This proposition can be proven using induction or can be proven using cases based on congruence modulo 3.

\item (Exercise~(\ref{exer:sec52-1}), Section~\ref{S:otherinduction}) Prove the following:

For each natural number  $n$  that is greater than or equal to 3,  
$\left( {1 + \frac{1}{n}} \right)^n  < n$.

\item (Exercise~(\ref{exer:sec52-2}), Section~\ref{S:otherinduction})Make a conjecture about a formula for the product  
\[
\left( {1 - \frac{1}{4}} \right) \cdot \left( {1 - \frac{1}{9}} \right) \cdot \, \cdots \, \cdot \left( {1 - \frac{1}{{n^2 }}} \right)
\]
for all  natural numbers  $n$  with  $n \ge 2$.  Then, state a proposition and use mathematical induction to prove your proposition.

\item (Exercise~(\ref{exer:sec53-fib}), Section~\ref{S:recursion})  Prove or disprove the following:

Let  $f_1 ,\,f_2 ,\,f_3 ,\, \ldots ,\,f_m ,\, \ldots$ be the sequence of Fibonacci numbers.  Then, for all natural numbers  $n$,  $f_{5n} $  is a multiple of  5.

\item (Exercise~(\ref{exer:sec53-fib}), Section~\ref{S:recursion}) Let  
$f_1 ,\,f_2 ,\,f_3 ,\, \ldots ,\,f_m ,\, \ldots$ be the sequence of Fibonacci numbers.  Is the following proposition true or false?  Justify your conclusion.  

For each $n \in \mathbb{N}$ such that $n \not \equiv 0 \pmod 3$, $f_{n} $  is an odd natural number.



\item (Exercise~(\ref{exer:sec53-9}), Section~\ref{S:recursion})
\begin{enumerate} 
\item Compute $n!$ for the first ten natural numbers.

\item Let $a_1 = 1$, and for each natural number $k$, let
\[
a_{k + 1} = a_k + k \cdot k!.
\]
Compute $a_n$ for the first ten natural numbers.

\item Make a conjecture about a formula for $a_n$ in terms of $n$ that does not involve a summation or a recursion.

\item Prove your conjecture in Part~(c).
\end{enumerate}

\item \begin{enumerate}
\item Use mathematical induction to prove one of the following two propositions:

\begin{itemize}
\item For each natural number  $n$  that is greater than or equal to 2, 
$7^{\left( {2^n } \right)}  \equiv 1\left( {\bmod 100} \right)$�	.

\item For each natural number  $n$, $7^{4n}  \equiv 1\left( {\bmod 100} \right)$.
\end{itemize}

\item Use your proposition from Part~(a) to determine the last two digits in the decimal expansion of  $7^{331} $.  Carefully explain the procedure you used to do this.
\end{enumerate}

\item Do one of the following two problems:

\begin{itemize} 
\item For which natural numbers  $n$  do there exist natural numbers  $x$  and  $y$  such that  
$n = 4x + 5y$? 

\item For which natural numbers  $n$  do there exist non-negative integers  $x$  and  $y$  such that  $n = 4x + 5y$?
\end{itemize}
 
Use mathematical induction to prove that your conclusion is correct.

\item Is the following proposition true or false?  Justify your conclusion.

If  $a$  is any real number, then for every natural number  $n$,  
\[
\left[ {\begin{array}{*{20}c}
   1 & a  \\
   0 & 1  \\
\end{array}} \right]^n  = \left[ {\begin{array}{*{20}c}
   1 & {na}  \\
   0 & 1  \\
\end{array}} \right].
\]



\end{enumerate}

\subsubsection*{Problems that Can Be Started After Chapter~\ref{C:functions}.}
\begin{enumerate}
\item (Exercise~(\ref{exer:sec64-6}), Section~\ref{S:compositionoffunctions}). Let  $A$, $B$, and  $C$  be sets, and let  $f:A \to B$ and  $g:B \to C$ be functions.  

Prove or disprove each of the following:
\begin{itemize}
\item If the composite function  $g \circ f:A \to C$ is an injection, then the function  
$f:A \to B$  is an injection.

\item	If the composite function  $g \circ f:A \to C$ is an injection, then the function  
$g:B \to C$ is a injection.
\end{itemize}  

\item (Exercise~(\ref{exer:sec64-7}), Section~\ref{S:compositionoffunctions}). Let  $A$, $B$, and  $C$  be sets, and let  $f:A \to B$ and  $g:B \to C$ be functions.  

Prove or disprove each of the following:
\begin{itemize}
\item If the composite function  $g \circ f:A \to C$ is a surjection, then the function  
$f:A \to B$  is a surjection.

\item	If the composite function  $g \circ f:A \to C$ is a surjection, then the function  
$g:B \to C$ is surjection.
\end{itemize}  


\item \begin{enumerate}
\item Let $f: \mathbb{R} \times \mathbb{R}  \to  \mathbb{R} \times \mathbb{R}$  be defined by  
$f\left( {x,\;y} \right) = \left( {2x + y, x - y} \right)$. Is the function  $f$  an injection?  Is the function  $f$  a surjection?  Justify your conclusions.

\item Let $g: \mathbb{Z} \times \mathbb{Z} \to  \mathbb{Z} \times \mathbb{Z}$ be defined by  
$g\left( {x,\;y} \right) = \left( {2x + y, x - y} \right)$. Is the function  $g$  an injection?  Is the function  $g$  a surjection?  Justify your conclusions.
\end{enumerate}

\item Let  $M_{3, 3}$ represent the set of all  3 by 3  matrices over  $\mathbb{R}$.  \linebreak
Define  $F:M_{3, 3}  \to \mathbb{R}$  by  

\begin{center}
$F\left( {\begin{array}{*{20}c}
   a & b & c  \\
   d & e & f  \\
   g & h & i  \\
\end{array}} \right) = a^2  + e^2  + i^2  - c^2  - g^2 $
\end{center}  
for all 3 by 3 matrices  $\left( {\begin{array}{*{20}c}
   a & b & c  \\
   d & e & f  \\
   g & h & i  \\
\end{array}} \right)$
 in  $M_{3, 3} $.

Is the function  $F$  an injection?  Is the function  $F$  a surjection? Justify your conclusions.

\item Let  $M_{3,3}$ represent the set of all  3 by 3  matrices over  $\mathbb{R}$.  Define  
$D :M_{3, 3}  \to \mathbb{R}$  by  
\[
D \left( {\begin{array}{*{20}c}
   a & b & c  \\
   d & e & f  \\
   g & h & i  \\
\end{array}} \right) = aei - afh - bdi + bfg + cdh - ceg
\]
for all 3 by 3 matrices  $\left( {\begin{array}{*{20}c}
   a & b & c  \\
   d & e & f  \\
   g & h & i  \\
\end{array}} \right)$ in  $M_{3,3} $.

Is the function $D$ an injection?  Is the function $D$ a surjection? Justify your conclusions.




\end{enumerate}



