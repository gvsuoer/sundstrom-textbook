\hbreak

\begin{previewactivity}[Sets of Ordered Pairs] \label{PA:functionasordered2} \hfill \\
In Preview Activity~\ref{PA:functionasordered}, we saw that any function  $f\x A \to B$ can be thought of as a set of ordered pairs that is a subset of  $A \times B$.  This subset is
\[
f = \left\{ { {( {a, f( a )} ) } \mid a \in A} \right\} \quad \text{or} 
\quad f = \left\{ {( {a, b} ) \in A \times B   \mid b = f( a )} \right\}\!.
\]

\noindent
On the other hand, if we started with  $A = \left\{ {1, 2, 3} \right\}$, 
$B = \left\{ {a, b} \right\}$, and 
\[
G = \left\{ {( {1, a} ), ( {2, a} ), ( {3, b} )} \right\} \subseteq A \times B ,
\]
then we could think of  $G$  as a function from  $A$  to  $B$  with  $G( 1 ) = a$, $G( 2 ) = a$, and $G( 3 ) = b.$  The idea is to use the first coordinate of each ordered pair as the input, and the second coordinate as the output.  However, not every subset of  $A \times B$ can be used to define a function from  $A$  to  $B$.

\begin{enumerate}
\item Let  $f = \left\{ {( {1, a} ), ( {2, a} ), ( {3, a} ), ( {1, b} )} \right\}$. Could this set of ordered pairs be used to define a function from  $A$  to  $B$?  Explain.

\item Let $g = \left\{ {( {1, a} ), ( {2, b} ), ( {3, a} )} \right\}$.  Could this set of ordered pairs be used to define a function from  $A$  to  $B$?  Explain.

\item Let $h = \left\{ {( {1, a} ), ( {2, b} )} \right\}$.  Could this set of ordered pairs be used to define a function from  $A$  to  $B$?  Explain.
\end{enumerate}
\end{previewactivity}
\hbreak
%
\begin{previewactivity}[A Composition of Two Specific Functions] \label{PA:compositionoftwo} \hfill \\
Let  $A = \left\{ {a, b, c, d} \right\}$ and  let  $B = \left\{ {p, q, r, s} \right\}$.

\begin{enumerate}
\item Construct an example of a function  $f\x A \to B$  that is a bijection.  Draw an arrow diagram for this function.

\item On your arrow diagram, draw an arrow from each element of  $B$  back to its corresponding element in  $A$.  Explain why this defines a function  from  $B$  to  $A$.  \label{PA:compositionoftwo2}

\item If the name of the function in Part~(\ref{PA:compositionoftwo2}) is  $g$, so that  
$g\x B \to A$, what are  
$g( p )$, $g( q )$, $g( r )$, and $g( s )$?

\item Construct a table of values for each of the functions $g \circ f\x A \to A$  and  
$f \circ g\x B \to B$. What do you observe about these tables of values?
\label{PA:compositionoftwo4}

%\begin{center}
%\begin{tabular}{c | c  p{0.5in}  c | c   }
%$x$  &  $( {g \circ f} )( x )$  &  & $y$ & 
%$( {f \circ g} )( y )$ \\ \cline{1-2} \cline{4-5}
%$a$  &  &  &  $p$  &  \\ \cline{1-2} \cline{4-5}
%$b$  &  &  &  $q$  &  \\ \cline{1-2} \cline{4-5}
%$c$  &  &  &  $r$  &  \\ \cline{1-2} \cline{4-5}
%$d$  &  &  &  $s$  &  \\ \cline{1-2} \cline{4-5}
%\end{tabular}
%\end{center}

%\item What do you observe about the tables of values in Part~(\ref{PA:compositionoftwo4})?
\end{enumerate}
\end{previewactivity}
\hbreak
%
%\begin{previewactivity}[Cubes and Cube Roots] \label{PA:cubes} \hfill
%
%Let  $f:\mathbb{R} \to \mathbb{R}$ be defined by  $f( x ) = x^3 $.	
%Let  $g:\mathbb{R} \to \mathbb{R}$ be defined by $g( x ) = \sqrt[3]{x}$.
%
%Complete the following tables:
%
%\begin{center}
%\begin{tabular}{c | c  p{0.5in}  c | c   }
%$x$  &  $f( x )$  &  & $x$ & $g( x )$  \\ \cline{1-2} \cline{4-5}
%0  &  &  &  0  &  \\ \cline{1-2} \cline{4-5}
%1  &  &  &  1  &  \\ \cline{1-2} \cline{4-5}
%2  &  &  &  8  &  \\ \cline{1-2} \cline{4-5}
%3  &  &  &  27  &  \\ \cline{1-2} \cline{4-5}
%$-1$  &  &  &  $-1$  &  \\ \cline{1-2} \cline{4-5}
%$-3$  &  &  &  $-27$  &  \\ \cline{1-2} \cline{4-5}
%\end{tabular}
%\end{center}
%%
%\begin{enumerate}
%\item What is happening?  In particular, how are the ordered pairs of the functions  $f$  and  $g$  related?
%
%\item Solve the equation   $( {2t - 1} )^3  = 20$.  Explain how the cube root function, $g$, was used in solving this equation.
%
%\item Solve the equation   $\sqrt[3]{{t - 3}} = 2$.  Explain how the cubing function, $f$,  was used in solving this equation.
%
%\item For each  $x \in \mathbb{R}$, determine 
%$( {g \circ f} ) ( x )$ and  
%$( {f \circ g} ) ( x )$.
%\end{enumerate}
%\end{previewactivity} 
%\hbreak


\endinput
