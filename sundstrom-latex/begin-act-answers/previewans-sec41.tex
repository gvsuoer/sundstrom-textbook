\documentclass[11pt]{article}
\usepackage{../../styles/activity}

\usepackage{xr}
\externaldocument{0-MR}

\lhead{}
%\chead{\textbf{\Large{\hspace{0pt}Beginning Activities for Section~4.1}}\\\prevhead}
\bahead{4.1}
\rhead{}
\lfoot{}
\rfoot{}
\cfoot{\hspace{0pt}\scalebox{0.4}{\includegraphics{cc-by-nc-sa.eps}}}

\begin{document}

\subsection*{Beginning Activity 1 \textbf{(Exploring Statements of the Form \mathversion{bold} $\left( \forall n \in \N \right)\left( P(n) \right)$)}}
Notice the experimentation in all parts of this beginning activity.  This is a good practice to use to explore whether a proposition appears to be true.  In this section, we will study a method of proof that can be used to prove the two propositions in this beginning activity.

\eighth \noindent
The following table shows that $P(1)$ through $P(7)$ are true.
\begin{center}
\begin{tabular}[t]{| c | c | c |} \hline
$n$  &  $\left( {5^n - 1} \right)$  &  Does 4 divide $\left( {5^n - 1} \right)$? \\ \hline
1  &  4  &  yes  \\ \hline
2  &  24  &  yes \\ \hline
3  &  124  &  yes  \\ \hline
4  &  624  &  yes  \\ \hline
5  &  3124  &  yes \\ \hline
6  &  15,624  &  yes  \\ \hline
7  &  78,124  &  yes  \\ \hline
\end{tabular}
\end{center}
%Based on the work summarized in this table, it seems the following is a true proposition:
%\begin{center}
%For each natural number  $n$, 4  divides $5^n - 1$.
%\end{center}
The following table shows that $Q(1)$ through $Q(7)$ are true.
\begin{center}
\begin{tabular}[t]{| c | c | c |} \hline
$n$  &  $1^2  + 2^2  + \, \cdots \, + n^2 $  &  $\dfrac{{n(n + 1)(2n + 1)}}{6}$ \\ \hline
1  &  1  &  1  \\ \hline
2  &  5  &  5 \\ \hline
3  &  14  &  14  \\ \hline
4  &  30  &  30  \\ \hline
5  &  55  &  55 \\ \hline
6  &  91  &  91  \\ \hline
7  &  140  &  140  \\ \hline
\end{tabular}
\end{center}

\hbreak

\newpage
\subsection*{Beginning Activity 2 \textbf{(A Property of the Natural Numbers)}}
The concept of an inductive set will provide the logical foundation for proofs by induction, which will be studied in this section.  Again, notice that after the definition was given, we explored examples of sets that are inductive and examples that are not inductive.  We also wrote what it means to say that a subset of the integers is not an inductive set.
\begin{enumerate}
  \item A set $T$ that is a subset of $\Z$ is not an inductive set provided that there exists an integer $k$ such that $k \in T$ and $k+1 \notin T$
\item \begin{enumerate}
\item The set  $A = \left\{ {1, 2, 3,  \ldots , 20} \right\}$  is not inductive since  
$20 \in A$, but $21 \notin A$.

\item The set of natural numbers, $\mathbb{N}$, is inductive since the natural numbers are closed with respect to addition.


\item The set  $B = \left\{ {\left. {n \in \mathbb{N}} \right| n \geq 5} \right\}$
 is inductive.  If  $k \in \mathbb{Z}$ and  $k \geq 5$, then  
$\left( {k + 1} \right) \geq 5$.

\item The set  $S = \left\{ {\left. {n \in \mathbb{N}} \right| n \geq  - 3} \right\}$
 is inductive.  If  $k \in \mathbb{Z}$ and  $k \geq  - 3$, then  
$\left( {k + 1} \right) \geq  - 3$.

\item The set  $R = \left\{ {n \in \mathbb{Z}\left| {n \leq 100} \right.} \right\}$
 is not inductive since $100 \in R$ but $101 \notin R$.

\item The set of integers, $\mathbb{Z}$, is inductive since  the set of integers is closed with respect to addition.
\item The set of all odd integers is not an inductive set.  For example, $1$ is in the set of all odd integers but 2 is not in the set of all odd integers.
\end{enumerate}


\item Now assume that  $T \subseteq \mathbb{N}$ and assume that  $1 \in T$ and that  $T$ is inductive.
\begin{enumerate}
\item Since $1 \in T$ and $T$ is inductive,   $2 \in T$.

\item Since $2 \in T$ and $T$ is inductive,  $3 \in T$.

\item Since $3 \in T$ and $T$ is inductive,   $4 \in T$.

\item Since $4 \in T$ and $T$ is inductive, $5 \in T$.  This in turn guarantees that  $6 \in T$, which guarantees that  $7 \in T$.  We can continue this process to argue that any succeeding natural number is in  $T$.  So, $100 \in T$.

\item The argument in Part (d) suggests that  $T = \mathbb{N}$.

\end{enumerate}
\end{enumerate}
\hbreak



\end{document}

