In this section, we will describe several infinite sets and define the cardinal number for so-called countable sets.  Most of our examples will be subsets of some of our standard number systems such as $\mathbb{N}$, $\mathbb{Z}$, and $\mathbb{Q}$.  

%We will see that there are infinite sets that may appear to be different in size, and that there exist infinite sets that are not equivalent.

\subsection*{Infinite Sets}
In \typeu Activity~\ref*{PA:introtoinfinite}, we saw how to use Corollary~\ref{C:propersubsets} to prove that a set is infinite.  This corollary implies that if $A$ is a finite set, then $A$ is not equivalent to any of its proper subsets.  By writing the contrapositive of this conditional statement, we can restate Corollary~\ref{C:propersubsets} in the following form:

\vskip9pt
\noindent
\textbf{Corollary~\ref{C:propersubsets}}.
\emph{If a set $A$ is equivalent to one of its proper subsets, then $A$ is infinite.}

%\begin{theorem} \label{T:infinitesets}
%The set $A$ is an infinite set if and only if $A$ is equivalent to one of its proper subsets.
%\end{theorem}

%\begin{example} 
\newpar
In \typeu Activity~\ref*{PA:introtoinfinite}, we used Corollary~\ref{C:propersubsets} to prove that
\begin{itemize}
\item The set of natural numbers, $\mathbb{N}$, is an infinite set.

\item The open interval $( 0, 1 )$ is an infinite set.
\end{itemize}
%\end{example}
Although Corollary~\ref{C:propersubsets} provides one way to prove that a set is infinite, it is sometimes more convenient to use a proof by contradiction to prove that a set is infinite.  The idea is to use results from Section~\ref{S:finitesets} about finite sets to help obtain a contradiction.  This is illustrated in the next theorem.
\begin{theorem}\label{T:subsetisinfinite}
Let $A$ and $B$ be sets.
\begin{enumerate}
\item If $A$ is infinite and $A \approx B$, then $B$ is infinite.  \label{T:subsetisinfinite1}
\item If $A$ is infinite and $A \subseteq B$, then $B$ is infinite.  \label{T:subsetisinfinite2}
\end{enumerate}
\end{theorem}
%
\begin{myproof}
We will prove part~(\ref{T:subsetisinfinite1}).  The proof of part~(\ref{T:subsetisinfinite2}) is exercise~(\ref{exer:subsetisinfinite}) on page~\pageref{exer:subsetisinfinite}.

To prove part~(\ref{T:subsetisinfinite1}), we use a proof by contradiction and assume that $A$ is an infinite set, $A \approx B$, and $B$ is not infinite.  That is, $B$ is a finite set.  Since $A \approx B$ and $B$ is finite, Theorem~\ref{T:equivfinitesets} on page~\pageref{T:equivfinitesets} implies that $A$ is a finite set.  This is a contradiction to the assumption that $A$ is infinite.  We have therefore proved that if $A$ is infinite and $A \approx B$, then $B$ is infinite.
\end{myproof}

%\begin{activity}[Proof of Theorem~\ref{T:subsetisinfinite}] \label{A:subsetisfinite}
%Write a proof for both parts of Theorem~\ref{T:subsetisinfinite}.  For both parts, use a proof by contradiction.  For each proof, a theorem in Section~\ref{S:finitesets} will provide a contradiction.
%\end{activity}
\hbreak
%
%\begin{example}[Examples of Infinite Sets] \label{E:infinitesets} \hfill
%\begin{itemize}
%\item Theorem~\ref{T:subsetisinfinite} allows us to conclude that our standard number systems are infinite sets since they all have $\mathbb{N}$ as a subset.  So $\mathbb{Z}$, $\mathbb{Q}$, and $\mathbb{R}$ are all infinite sets.  In addition, the set of all positive rational numbers, 
%$\mathbb{Q}^+$, and the set of all positive real numbers, $\mathbb{R}^+$, are infinite sets.
%
%\item Let $D^+$ be the set of all odd natural numbers.  In Part~(\ref{PA:introtoinfinite2}) of Beginning Activity~\ref{PA:introtoinfinite}, we proved that $D^+ \approx \mathbb{N}$.  Therefore, by Theorem~\ref{T:subsetisinfinite}, $D^+$ is an infinite set.  In a similar manner, the set $E^+$ of all even natural numbers is an infinite set.  To do this, we can use the function $g\x \mathbb{N} \to E^+$ defined by $g ( x ) = 2x$ for all $x \in \mathbb{N}$.
%\end{itemize}
%\end{example}
\begin{prog}[\textbf{Examples of Infinite Sets}] \label{E:infinitesets} \hfill
\begin{enumerate}
\item In \typeu Activity~\ref*{PA:introtoinfinite}, we used Corollary~\ref{C:propersubsets} to prove that $\N$ is an infinite set.  Now use this and Theorem~\ref{T:subsetisinfinite} to explain why our standard number systems $( \Z, \Q, \text{ and } \R )$ are infinite sets.  Also, explain why the set of all positive rational numbers, 
$\Q^+$, and the set of all positive real numbers, $\R^+$, are infinite sets.

\item Let $D^+$ be the set of all odd natural numbers.  In Part~(\ref{PA:introtoinfinite2}) of \typeu Activity~\ref*{PA:introtoinfinite}, we proved that $D^+ \approx \mathbb{N}$.  Use 
Theorem~\ref{T:subsetisinfinite} to explain why $D^+$ is an infinite set.  

\item Prove that the set $E^+$ of all even natural numbers is an infinite set.
\end{enumerate}
\end{prog}
\hbreak

\endinput

