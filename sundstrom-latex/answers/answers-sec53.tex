\section*{Section \ref{S:setproperties}}
\renewcommand{\labelenumi}{(\textbf{\alph{enumi}})}

\begin{list}{\bf{\ref{exer:sec43-1}.}}
\item \begin{list}{\bf{(a)}}
\item Let  $x \in ( A^c )^c$.  Then  $x \notin A^c$, which means $x \in A$.  Hence, 
$( A^c )^c \subseteq A$.  Now let $y \in A$.  Then, 
$y \notin A^c$ and hence, $y \in \left( A^c \right)^c$.  Therefore, 
$A \subseteq \left( A^c \right)^c$.
\end{list}
\end{list}

\begin{list}{}
\item \begin{list}{\bf{(c)}}
\item Let $x \in U$.  Then $x \notin \emptyset$ and so $x \in \emptyset^c$.  Therefore, $U \subseteq \emptyset^c$.  Also, since every set we deal with is a subset of the universal set, 
 $\emptyset^c \subseteq U$.
\end{list}
\end{list}

%\begin{list}{\bf{\ref{exer:sec43-2}.}}
%\item We still need to prove that if $B^c \subseteq A^c$, then $A \subseteq B$.
%\end{list}

\begin{list}{\bf{\ref{exer:distributive}.}}
\item To prove that $A \cap ( B \cup C ) \subseteq ( A \cap B ) 
\cup ( A \cap C )$, we let $x \in A \cap ( B \cup C )$.  Then $x \in A$ and 
$x \in B \cup C$.  So we will use two cases: 
(1) $x \in B$; (2) $x \in C$.

In Case~(1), $x \in A \cap B$ and, hence, $x \in ( A \cap B ) \cup ( A \cap C )$.  In Case~(2), 
$x \in  A \cap C$ and, hence, $x \in ( A \cap B ) \cup ( A \cap C )$.  This proves that $A \cap ( B \cup C ) \subseteq ( A \cap B ) \cup ( A \cap C )$.

To prove that $\left( A \cap B \right) \cup \left( A \cap C \right) \subseteq 
A \cap \left( B \cup C \right)$,  
let $y \in \left( A \cap B \right) \cup \left( A \cap C \right)$.  Then, $y \in A \cap B$ or 
$y \in A \cap C$.  If $y \in A \cap B$, then $y \in A$ and $y \in B$.  Therefore, $y \in A$ and 
$y \in B \cup C$.  So, we may conclude that $y \in A \cap \left( B \cup C \right)$.  In a similar manner, we can prove that if $y \in A \cap C$, then $y \in A \cap \left( B \cup C \right)$.  This proves that $\left( A \cap B \right) \cup \left( A \cap C \right) \subseteq 
A \cap \left( B \cup C \right)$, and hence that $A \cap \left( B \cup C \right) = \left( A \cap B \right) \cup \left( A \cap C \right)$. 

\end{list}


%\begin{list}{}
%\item 
%\[
%\begin{aligned}
%( A - B ) \cap ( A - C ) &= ( A \cap B^c ) \cap ( A \cap C^c ) \\
%  &= ( A \cap A ) \cap ( B^c \cap C^c ) \\
%  &= A \cap ( B \cup C )^c \\
%  &= A - ( B \cup C ).
%\end{aligned}
%\]
%\end{list}



\begin{list}{\bf{\ref{exer:sec43-5}.}}
\item \begin{list}{\bf{(a)}}
\item $A - ( {B \cup C} ) = ( {A - B} ) \cap ( {A - C} )$.
\end{list}
\end{list}

\begin{list}{}
\item \begin{list}{\bf{(c)}}
\item Using the algebra of sets, we obtain
\begin{align*}
( A - B ) \cap ( A - C ) &= ( A \cap B^c ) \cap ( A \cap C^c ) \\
  &= ( A \cap A ) \cap ( B^c \cap C^c ) \\
  &= A \cap ( B \cup C )^c \\
  &= A - ( B \cup C ).
\end{align*}
\end{list}
\end{list}


\begin{list}{\bf{\ref{exer53:exer6}.}}
\item \begin{enumerate}
\item Using the algebra of sets, we see that
\begin{align*}
(A - C) \cap (B - C) &= (A \cap C^c) \cap (B \cap C^c) \\
                     &= (A \cap B) \cap (C^c \cap C^c) \\
                     &= (A \cap B) \cap C^c \\
                     &= (A \cap B) - C.
\end{align*}
\end{enumerate}
\end{list}


\begin{list}{\bf{\ref{exer:sec43-7}.}}
\item \begin{list}{\bf{(a)}}
\item Use a proof by contradiction.  Assume the sets are not disjoint and let 
$x \in A \cap ( B - A )$.  Then $x \in A$ and $x \in B - A$, which implies that 
$x \notin A$.
\end{list}
\end{list}
\hbreak
\renewcommand{\labelenumi}{\textbf{\arabic{enumi}.}}

\endinput

