\subsection*{Relatively Prime Integers}
In \typeu Activity~\ref*{PA:exporingexamples}, we constructed several examples of integers  $a$, $b$, and  $c$  such that  $a \mid \left( {bc} \right)$ but  $a$  does not divide  $b$  and  $a$  does not divide  $c$.  For each example, we observed that  $\gcd( {a, b} ) \ne 1$
  and  $\gcd( {a, c} ) \ne 1$.  

We also constructed several examples where  $a \mid \left( {bc} \right)$ and  
$\gcd( {a, b} ) = 1$.  In all of these cases, we noted that  $a$  divides  $c$.  Integers whose greatest common divisor is equal to 1 are given a special name.

\begin{defbox}{relativelyprime}{Two nonzero integers  $a$  and  $b$  are \textbf{relatively prime}
\index{relatively prime integers}%
 provided that  $\gcd( {a, b} ) = 1$.}
\end{defbox}
%
\begin{prog}[\textbf{Relatively Prime Integers}] \label{prog:relativelyprime} \hfill
\begin{enumerate}
\item Construct at least three different examples where  $p$  is a prime number, $a \in \mathbb{Z}$\,, and  $p \mid a$.  In each example, what is  $\gcd( {a, p} )$?  Based on these examples, formulate a conjecture about $\gcd( {a, p} )$ when $p \mid a$.

\item Construct at least three different examples where  $p$  is a prime number, $a \in \mathbb{Z}$\,, and  $p$  does not divide  $a$.  In each example, what is  
$\gcd( {a, p} )$?  Based on these examples, formulate a conjecture about $\gcd( {a, p} )$ when 
$p$ does not divide $a$.

\item Give at least three different examples of integers  $a$  and  $b$  where  $a$  is not prime,  $b$  is not prime, and  $\gcd( {a, b} ) = 1$, or explain why it is not possible to construct such examples.
\end{enumerate}
\end{prog}
\hbreak
%
\begin{theorem} \label{T:relativelyprime}
Let  $a$  and  $b$  be nonzero integers, and let  $p$  be a  prime number.
\begin{enumerate}
\item If  $a$  and  $b$  are relatively prime, then there exist integers  $m$  and  $n$  such that  $am + bn = 1$.  That is,  1  can be written as linear combination of  $a$  and  $b$. \label{T:relativelyprime1}

\item If  $p \mid a$, then  $\gcd( {a, p} ) = p$. \label{P:relativelyprime2}

\item If  $p$  does not divide  $a$, then  $\gcd( {a, p} ) = 1$. \label{T:relativelyprime3}
\end{enumerate}
\end{theorem}

Part~(\ref{T:relativelyprime1}) of Theorem~\ref{T:relativelyprime} is actually a corollary of Theorem~\ref{T:gcddivideslincombs}.  Parts~(\ref{P:relativelyprime2}) and~(\ref{T:relativelyprime3}) could have been the conjectures you formulated in 
Progress Check~\ref{prog:relativelyprime}.  The proofs are included in Exercise~(\ref{exer:sec82-1}).
\hbreak
%
Given nonzero integers  $a$  and  $b$, we have seen that it is possible to use the Euclidean Algorithm to write their greatest common divisor as a linear combination of  $a$  and  $b$.  We have also seen that this can sometimes be a tedious, time-consuming process, which is why people have programmed computers to do this.  Fortunately, in many proofs of number theory results, we do not actually have to construct this linear combination since simply knowing that it exists can be useful in proving results.  This will be illustrated in the proof of Theorem~\ref{T:relativelyprimeprop}, which is based on work in \typeu Activity~\ref*{PA:exporingexamples}.

\begin{theorem}\label{T:relativelyprimeprop}
Let  $a$, $b$ be nonzero integers and let $c$ be an integer.  If  $a$  and  $b$  are relatively prime  and  $a \mid \left( {bc} \right)$, then  $a \mid c$.
\end{theorem}

\setcounter{equation}{0}
The explorations in \typeu Activity~\ref*{PA:exporingexamples} were related to this theorem.  
We will first explore the forward-backward process for the proof.
The goal is to prove that  $a \mid c$.  A standard way to do this is to prove that there exists an integer  $q$  such that
%
\begin{equation}\label{eq:relprime1}
c = aq.
\end{equation}
%
Since we are given that  $a \mid \left( {bc} \right)$, there exists an integer  $k$  such that
\begin{equation}\label{eq:relprime2}
bc = ak.
\end{equation}
It may seem tempting to divide both sides of equation~(\ref{eq:relprime2})  by  $b$, but if we do so, we run into problems with the fact that the integers are not closed under division.  Instead, we look at the other part of the hypothesis, which is that  $a$  and  $b$  are relatively prime.  This means that   $\gcd( {a, b} ) = 1$.  How can we use this?  This means that  $a$  and  $b$  have no common factors except for  1.  In light of equation~(\ref{eq:relprime2}), it seems reasonable that any factor of  $a$ must also be a factor of  $c$.  But how do we formalize this?

One conclusion that we can use is that since  $\gcd( {a, b} ) = 1$, by Theorem~\ref{T:relativelyprime}, there exist integers  $m$  and  $n$  such that
\begin{equation} \label{eq:relprime3}
am + bn = 1.
\end{equation}

We may consider solving equation~(\ref{eq:relprime3}) for  $b$  and substituting this into equation~(\ref{eq:relprime2}).  The problem, again, is that in order to solve equation~(\ref{eq:relprime3}) for  $b$, we need to divide by  $n$.  

Before doing anything else, we should look at the goal in equation~(\ref{eq:relprime1}).  We need to introduce  $c$  into equation~(\ref{eq:relprime3}).  One way to do this is to multiply both sides of equation~(\ref{eq:relprime3}) by  $c$.  (This keeps us in the system of integers since the integers are closed under multiplication.)  This gives
\begin{align} \notag
  \left( {am + bn} \right)c &= 1 \cdot c \\ \label{eq:relprime4}
                  acm + bcn &= c. \\ \notag
\end{align} 
Notice that the left side of equation~(\ref{eq:relprime4}) contains a term,  $bcn$, that contains  $bc$.  This means that we can use equation~(\ref{eq:relprime2}) and substitute  
$bc = ak$  in equation~(\ref{eq:relprime4}).  After doing this, we can factor the left side of the equation to prove that  $a \mid c$.
%
\hbreak
%
\begin{prog}[\textbf{Completing the Proof of Theorem~\ref{T:relativelyprimeprop}}] 
\label{prog:relativelyprimeprop} \hfill \\
Write a complete proof of Theorem~\ref{T:relativelyprimeprop}.
\end{prog}
\hbreak
%
\begin{corollary}\label{C:primedivides} \hfill
\begin{enumerate}
\item Let  $a, b \in \mathbb{Z}$, and let  $p$  be a prime number.  If  
$p \mid \left( {ab} \right)$, then  $p \mid a$  or  $p \mid b$.  \label{C:primedivides1}

\item Let  $p$  be a prime number, let  $n \in \mathbb{N}$, and let  
$a_1 ,a_2 , \ldots , a_n  \in \mathbb{Z}$.  If  \linebreak
$p \mid \left( {a_1 a_2  \cdots a_n } \right)$, then there exists a natural number $k$ 
 with  $1 \leq k \leq n$ such that  $p \mid a_k $.  \label{C:primedivides2}
\end{enumerate}
\end{corollary}

Part~(\ref{C:primedivides1}) of Corollary~\ref{C:primedivides} is a corollary of Theorem~\ref{T:relativelyprimeprop}.  Part~(\ref{C:primedivides2}) is proved using mathematical induction.  The basis step is the case where  $n = 1$, and Part~(\ref{C:primedivides1}) is the case where  
$n = 2$.  The proofs of these two results are included in Exercises~(\ref{exer:sec82-2}) 
and~(\ref{exer:sec82-3}).
%
\hbreak
%
\subsection*{Historical Note}

Part~(\ref{C:primedivides1}) of Corollary~\ref{C:primedivides} is known as  \textbf{Euclid's Lemma}.
\index{Euclid's Lemma}%
  Most people associate geometry with \textit{Euclid's Elements}, but these books  also contain many basic results in number theory.  Many of the results that are contained in this section appeared in \textit{Euclid's Elements}.
\index{Euclid's Elements@\emph{Euclid's Elements}}%
\hbreak

\endinput
