\section*{Section \ref{S:reviewproofs}}

\begin{list}{\bf{~\ref{exer:sec32-9}.}}
\item \begin{description}
\item[~~(b)] Assuming 4 divides $a$, there exist an integer $n$ such that $a = 4n$.  Using this, we see that $b^3 = 16n^2$. So, $b^3$ is even and hence $b$ is even and there exists an integer $m$ such that $b = 2m$.  This implies that
\begin{align*}
8m^3 &= 16n^2 \\
m^3 &= 2n^2
\end{align*}
Hence, $m^3$ is even and so $m$ is even.  We can now use the fact that $b = 2m$ to conclude that 4 divides $b$.
\end{description}
\end{list}

\begin{list}{\bf{\ref{exer:sec32-equation}.}}
\item Prove the contrapositive.  That way, you can assume that the equation $ax^3 + bx + \left( b + a \right) = 0$ has a solution that is an natural number.  You will then have to prove that $a$ divides $b$.
%Let $a$ and $b$ be integers with $a \ne 0$.  We will prove the contrapositive.  So assume that the equation $ax^3 + bx + \left( b + a \right) = 0$ has a solution that is an natural number.  Let $n$ be a natural number that is a solution of this equation.  Then
%\[
%\begin{aligned}
%an^3 + bn + \left( b + a \right) &= 0 \\
%an^3 + a &= -bn - b \\
%a \left(n + 1\right) \left(n^2 - n + 1 \right) &= -b \left( n + 1 \right) \\
%a \left( n^2 - n + 1 \right) &= -b \\
%\end{aligned}
%\]
%The last equation can be used to conclude that $a \mid b$, and this completes the proof of the contrapositive.
\end{list}
\hbreak

\endinput
