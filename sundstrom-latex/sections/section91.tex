\section{Finite Sets} \label{S:finitesets}
%\markboth{Chapter~\ref{C:topicsinsets}. Topics in Set Theory}{\ref{S:finitesets}. Finite Sets}
\setcounter{previewactivity}{0}
%
\begin{previewactivity}[\textbf{Equivalent Sets, Part 1}]\label{PA:equivalentsets} \hfill
\begin{enumerate}
\item Let $A$ and $B$ be sets and let $f$ be a function from $A$ to $B$.  $\left(f:A \to B \right)$.  Carefully complete each of the following using appropriate quantifiers:  (If necessary, review the material in Section~\ref{S:typesoffunctions}.)
\begin{enumerate}
\item The function $f$ is an injection provided that $\ldots .$
\item The function $f$ is not an injection provided that $\ldots .$
\item The function $f$ is a surjection provided that $\ldots .$
\item The function $f$ is not a surjection provided that $\ldots .$
\item The function $f$ is a bijection provided that $\ldots .$
\end{enumerate}
\end{enumerate}

\begin{defbox}{equivsets}{Let $A$ and $B$ be sets.  The set $A$ is \textbf{equivalent} to the set $B$ 
\index{equivalent sets}%
provided that there exists a bijection from the set $A$ onto the set $B$.  In this case, we write 
$A \approx B$.  
\label{sym:AequivB}%

\vskip6pt
When $A \approx B$, we also say that the set $A$ is in \textbf{one-to-one correspondence} 
\index{one-to-one correspondence}%
with the set $B$ and that the set $A$ has the same \textbf{cardinality} 
\index{cardinality}%
as the set $B$.
}
\end{defbox}
\noindent
\note  When $A$ is not equivalent to $B$, we write $A \not \approx B$.

%\enlargethispage{\baselineskip}
\setcounter{oldenumi}{\theenumi}
\begin{enumerate} \setcounter{enumi}{\theoldenumi}
%\begin{enumerate} \addtocounter{enumi}{1}
\item For each of the following, use the definition of equivalent sets to determine if the first set is equivalent to the second set.

\begin{enumerate}
\item $A = \left\{ 1, 2, 3 \right\}$ and $B = \left\{ a, b, c \right\}$  

\item $C = \left\{ 1, 2 \right\}$ and $B = \left\{ a, b, c \right\}$  

\item $X = \left\{ 1, 2, 3, \ldots, 10 \right\}$   and 
$Y = \left\{ 57, 58, 59, \ldots, 66 \right\}$
\end{enumerate}  

\item Let $D^+$ be the set of all odd natural numbers.   Prove that the function \linebreak
$f\x \mathbb{N} \to D^+$ defined by 
$f \left( x \right) = 2x - 1$, for all $x \in \mathbb{N}$,  is a bijection and hence that $\mathbb{N} \approx D^+$. 
\label{PA:equivalentsets5}%

%\item Let $r$ be an real number with $r > 0$.  Let $\left( 0, 1 \right)$ and 
%$\left( 0, r \right)$ be the open intervals from 0 to 1 and 0 to $r$, respectively, and let  
%$g: \left( 0, 1 \right) \to \left( 0, r \right)$ by $g \left( x \right) = rx$, for all $x \in \mathbb{R}$.  Prove that the function $g$ is a bijection and hence that 
%$\left( 0, 1 \right) \approx \left( 0, r \right)$. \label{PA:equivalentsets6}

\item Let $\R^+$ be the set of all positive real numbers.  Prove that the function 
\linebreak
$g\x \R \to \R^+$ defined by $g (x ) = e^x$, for all $x \in \R$ is a bijection and hence, that 
$\R \approx \R^+$.
\end{enumerate}
\end{previewactivity}
\hbreak

\endinput

\begin{previewactivity}[\textbf{Equivalent Sets, Part 2}]\label{PA:equivsets2} \hfill
\begin{enumerate}
\item Review Theorem~\ref{T:compositefunctions} in Section~\ref{S:compositionoffunctions}, Theorem~\ref{T:inversenotation} in Section~\ref{S:inversefunctions}, and 
Exercise~(\ref{exer:finversebijection}) in Section~\ref{S:inversefunctions}.

%\item Review the definitions of a reflexive relation on a set, a symmetric relation, a transitive relation, and an equivalence relation on a set in Section~\ref{S:equivrelations}.

\item Prove each part of the following theorem.
\begin{theorem} \label{T:equivsets} Let $A$, $B$, and $C$ be sets.

\begin{enumerate}
\item For each set $A$, $A \approx A$.  \label{T:equivsets1}

\item For all sets $A$ and $B$, if $A \approx B$, then 
$B \approx A$.  \label{T:equivsets2}

\item For all sets $A$, $B$, and $C$, if $A \approx B$ and 
$B \approx C$, then $A \approx C$.  \label{T:equivsets3}
\end{enumerate}
\end{theorem}
\end{enumerate}
%\textbf{Technical Note}:   The three properties we proved in this activity are very similar to the concepts of reflexive, symmetric, and transitive relations.  However, we do not consider equivalence of sets to be an equivalence relation on a set $U$ since an equivalence relation requires an underlying (universal) set 
%$U$.  In this case, our elements would be the sets $A$, $B$, and $C$, and these would then have to subsets of some universal set $W$ (elements of the power set of $W$).  For equivalence of sets, we are not requiring that the sets $A$, $B$, and $C$ be subsets of the same universal set.  So we do not use the term relation in regards to the equivalence of sets.  However, if $A$ and $B$ are sets and $A \equiv B$, then we often say that $A$ and $B$ are \textbf{equivalent sets}.
\end{previewactivity}
\hbreak

\endinput

%
\subsection*{Equivalent Sets}

In \typeu Activity~\ref*{PA:equivalentsets}, we introduced the concept of equivalent sets.  The motivation for this definition was to have a formal method for determining whether or not two sets ``have the same number of elements.''  This idea was described in terms of a one-to-one correspondence (a bijection) from one set onto the other set.  This idea may seem simple for finite sets, but as we will see, this idea has surprising consequences when we deal with infinite sets.  (We will soon provide precise definitions for finite and infinite sets.)

\newpar
\textbf{Technical Note}:   The three properties we proved in Theorem~\ref{T:equivsets} in \typeu Activity~\ref*{PA:equivsets2} are very similar to the concepts of reflexive, symmetric, and transitive relations.  However, we do not consider equivalence of sets to be an equivalence relation on a set $U$ since an equivalence relation requires an underlying (universal) set $U$.  In this case, our elements would be the sets $A$, $B$, and $C$, and these would then have to be subsets of some universal set $W$ (elements of the power set of $W$).  For equivalence of sets, we are not requiring that the sets $A$, $B$, and $C$ be subsets of the same universal set.  So we do not use the term relation in regards to the equivalence of sets.  However, if $A$ and $B$ are sets and 
$A \approx B$, then we often say that $A$ and $B$ are \textbf{equivalent sets}.

\hbreak

\begin{prog}[\textbf{Examples of Equivalent Sets}]\label{prog:equivsets} \hfill \\
We will use the definition of equivalent sets from \typeu Activity~\ref*{PA:equivalentsets} in all parts of this progress check.  It is no longer sufficient to say that two sets are equivalent by simply saying that the two sets have the same number of elements.

\begin{enumerate}
  \item Let $A = \left\{ 1, 2, 3, \ldots, 99, 100 \right\}$ and let 
$B = \left\{ 351, 352, 353, \ldots, 449, 450 \right\}$.  Define $f\x A \to B$ by $f(x) = x + 350$, for each $x$ in $A$.  Prove that $f$ is a bijection from the set $A$ to the set $B$ and hence, $A \approx B$.

\item Let $E$ be the set of all even integers and let $D$ be the set of all odd integers.  Prove that $E \approx D$ by proving that $F\x E \to D$, where $F \left( x \right) = x + 1$, for all $x \in E$, is a bijection.

\item Let $\left( 0, 1 \right)$ be the open interval of real numbers between 0 and 1.  Similarly, if $b \in \mathbb{R}$ with $b > 0$, let $\left( 0, b \right)$ be the open interval of real numbers between 0 and $b$. \label{A:equivsets4}

Prove that the function $f\x \left( 0, 1 \right) \to \left( 0, b \right)$ by $f \left( x \right) = bx$, for all 
$x \in \left( 0, 1 \right)$, is a bijection and hence 
$\left( 0, 1 \right) \approx \left( 0, b \right)$.
\end{enumerate}
\end{prog}
\hbreak
%
In Part~(\ref{A:equivsets4}) of Progress Check~\ref{prog:equivsets}, notice that if $b > 1$, then 
$\left( 0, 1 \right)$ is a proper subset of $\left( 0, b \right)$ and 
$\left( 0, 1 \right) \approx \left( 0, b \right)$.

Also, in Part~(\ref{PA:equivalentsets5}) of \typeu Activity~\ref*{PA:equivalentsets}, we proved that the set $D$ of all odd natural numbers is equivalent to $\mathbb{N}$, and we know that $D$ is a proper subset of $\mathbb{N}$.  

These results may seem a bit strange, but they are logical consequences of the definition of equivalent sets.  Although we have not defined the terms yet, we will see that one thing that will distinguish an infinite set from a finite set is that an infinite set can be equivalent to one of its proper subsets, whereas a finite set cannot be equivalent to one of its proper subsets.  

\endinput

\subsection*{Finite Sets}
In Section~\ref{S:setoperations}, we defined the \textbf{cardinality}
\index{cardinality!finite set}%
 of a finite set $A$, denoted by $\card (A)$, to be the number of elements in the set $A$.  Now that we know about functions and bijections, we can define this concept more formally and more rigorously.  First, for each $k \in \mathbb{N}$, we define $\mathbb{N}_k$ to be the set of all natural numbers between 1 and $k$, inclusive.  That is,
\label{sym:firstk}
\[
\mathbb{N}_k = \left\{ 1, 2, \ldots, k \right\}\!.
\]
We will use the concept of \textbf{equivalent sets}
\index{equivalent sets}%
 introduced in \typeu Activity~\ref*{PA:equivalentsets} to define a finite set.

\begin{defbox}{cardinalityfinite}{A set $A$ is a \textbf{finite set}
\index{finite set}%
\index{finite set!properties of|(}%
 provided that $A = \emptyset$ or there exists a natural number $k$ such that 
$A \approx \mathbb{N}_k$.  A set is an \textbf{infinite set}
\index{infinite set}%
 provided that it is not a finite set.

If $A \approx \mathbb{N}_k$, we say that the set $A$ has \textbf{cardinality}
\index{cardinality}%
 $\boldsymbol{k}$ (or \textbf{cardinal number}
\index{cardinal number}%
 $\boldsymbol{k}$), and we write  $\text{card} \left( A \right) = k$.  \label{sym:cardk} 

In addition, we say that the empty set has \textbf{cardinality 0} (or \textbf{cardinal number 0}), and we write $\text{card} \left( \emptyset \right) = 0$.}
\end{defbox}
%
Notice that by this definition, the empty set is a finite set.  In addition, for each 
$k \in \mathbb{N}$, the identity function on $\mathbb{N}_k$ is a bijection and hence, by definition, the set $\mathbb{N}_k$ is a finite set with cardinality $k$.

\begin{theorem}\label{T:equivfinitesets}
Any set equivalent to a finite nonempty set $A$ is a finite set and has the same cardinality as 
$A$.
\end{theorem}
%
\begin{myproof}
Suppose that $A$ is a finite nonempty set, $B$ is a set, and $A \approx B$.  Since $A$ is a finite set, there exists a $k \in \mathbb{N}$ such that $A \approx \mathbb{N}_k$.  We also have assumed that $A \approx B$ and so by part~(b) of Theorem~\ref{T:equivsets} (in \typeu Activity~\ref*{PA:equivsets2}), we can conclude that 
$B \approx A$.  Since $A \approx \mathbb{N}_k$, we can use part~(c) of Theorem~\ref{T:equivsets} to conclude that $B \approx \mathbb{N}_k$.  Thus, $B$ is finite and has the same cardinality as $A$.
\end{myproof}
%\hbreak

It may seem that we have done a lot of work to prove an ``obvious'' result in 
Theorem~\ref{T:equivfinitesets}.  The same may be true of the remaining results in this section, which give further results about finite sets.  One of the goals is to make sure that the concept of cardinality for a finite set corresponds to our intuitive notion of the number of elements in the set.  Another important goal is to lay the groundwork for a more rigorous and mathematical treatment of infinite sets than we have encountered before.  Along the way, we will see the mathematical distinction between finite and infinite sets.

The following two lemmas will be used to prove the theorem that states that every subset of a finite set is finite.

\begin{lemma} \label{L:addone}
If $A$ is a finite set and $x \notin A$, then $A \cup \left\{ x \right\}$ is a finite set and  
$\text{card} \!\left( A \cup \left\{ x \right\} \right) = \text{card}( A ) + 1$.
\end{lemma}
%
\begin{myproof}
Let $A$ be a finite set and assume $\text{card}( A ) = k$, where 
$k = 0$ or $k \in \mathbb{N}$.  Assume $x \notin A$.

If $A = \emptyset$, then $\text{card}( A ) = 0$ and 
$A \cup \left\{ x \right\} = \left\{ x \right\}$, which is equivalent to $\mathbb{N}_1$.  Thus, 
$A \cup \left\{ x \right\}$ is finite with cardinality 1, which equals 
$\text{card}( A ) + 1$.

If $A \ne \emptyset$, then $A \approx \mathbb{N}_k$, for some $k \in \mathbb{N}$.  This means that $\text{card}( A ) = k$, and there exists a bijection $f\x A \to \mathbb{N}_k$.  We will now use this bijection to define a function $g\x A \cup \left\{ x \right\} \to \mathbb{N}_{k+1}$ and then prove that the function $g$ is a bijection.  We define $g\x A \cup \left\{ x \right\} \to \mathbb{N}_{k+1}$ as follows:  For each 
$t \in A \cup \left\{ x \right\}$,
\[
g \left( t \right) = \left\{ \begin{gathered}
  f \left( t \right) \text{  if  }t \in A \hfill \\
  k + 1 \text{  if  }t = x. \hfill \\ 
\end{gathered}  \right.
\]
To prove that $g$ is an injection, we let $x_1, x_2 \in A \cup \left\{ x \right\}$ and assume 
$x_1 \ne x_2$.
\begin{itemize}
\item If $x_1, x_2 \in A$, then since $f$ is a bijection, 
$f( x_1 ) \ne f ( x_2 )$, and this implies that 
$g( x_1 ) \ne g( x_2 )$.

\item If $x_1 = x$, then since $x_2 \ne x_1$, we conclude that $x_2 \ne x$ and hence 
$x_2 \in A$.  So $g( x_1 ) = k + 1$, and since $f( x_2 ) \in \mathbb{N}_k$ and 
$g( x_2 ) = f ( x_2 )$, we can conclude that $g( x_1 ) \ne g( x_2 )$.

\item The case where $x_2 = x$ is handled similarly to the previous case.
\end{itemize}

\noindent
This proves that the function $g$ is an injection.  The proof that $g$ is a surjection is 
Exercise~(\ref{exer:addonesurjection}).  Since $g$ is a bijection, we conclude that $A \cup \left\{ x \right\} \approx \mathbb{N}_{k+1}$, and
\[
\text{card} \!\left(A \cup \left\{ x \right\} \right) = k + 1.
\]
Since $\text{card} \left(A \right) = k$, we have proved that
$\text{card} \!\left(A \cup \left\{ x \right\} \right) = \text{card}( A ) + 1.$
\end{myproof}
%\hbreak
%
%Lemma~\ref{L:addone} implies that adding one element to a finite set increases its cardinality by %one.  It is also true that removing one element from a finite set reduces the cardinality by one.  %The proof of Corollary~\ref{C:removeone} is left as Exercise~(\ref{exer:sec92corollary}).

%\begin{corollary} \label{C:removeone}
%If $A$ is a finite set and $x \in A$, then $A - \left\{ x \right\}$ is a finite set and  
%$\text{card} \left( A - \left\{ x \right\} \right) = \text{card} \left( A \right) - 1$.
%\end{corollary}
%\hbreak
%
\begin{lemma} \label{L:subsetsofNk}
For each natural number $m$, if $A \subseteq \mathbb{N}_m$, then $A$ is a finite set and 
$\text{card} \left( A \right) \leq m$.
\end{lemma}
%
\begin{myproof}
We will use a proof using induction on $m$.  For each $m \in \mathbb{N}$, let 
$P( m )$ be, ``If $A \subseteq \mathbb{N}_m$, then $A$ is finite and 
$\text{card}( A ) \leq m$.''

We first prove that $P( 1 )$ is true.  If $A \subseteq \mathbb{N}_1$, then 
$A = \emptyset$ or $A = \left\{ 1 \right\}$, both of which are finite and have cardinality less than or equal to the cardinality of $\mathbb{N}_1$.  This proves that $P( 1 )$ is true.

%\vskip6pt
For the inductive step, let $k \in \mathbb{N}$ and assume that $P( k )$ is true.  That is, assume that if $B \subseteq \mathbb{N}_k$, then $B$ is a finite set and 
$\text{card}( B ) \leq k$.  We need to prove that 
$P( k+1 )$ is true.

So assume that $A$ is a subset of $\mathbb{N}_{k+1}$.  Then $A - \left\{ k+1 \right\}$ is a subset of $\mathbb{N}_k$.  Since $P( k )$ is true, $A - \left\{ k+1 \right\}$ is a finite set and 
\[
\text{card} \!\left( A - \left\{ k+1 \right\} \right) \leq k.
\]
There are two cases to consider:  Either $k+1 \in A$ or $k+1 \not \in A$.

\vskip6pt
If $k+1 \not \in A$, then $A = A - \left\{ k+1 \right\}$.  Hence, $A$ is finite and
\[
\text{card} ( A ) \leq k < k+1.
\]

If $k+1 \in A$, then $A = ( A - \left\{ k+1 \right\} ) \cup \left\{ k+1 \right\}$. Hence, by Lemma~\ref{L:addone}, $A$ is a finite set and
\[
\text{card} ( A ) = 
\text{card} \!\left(  A - \left\{ k+1 \right\}  \right) + 1.
\]
Since $\text{card} \!\left( A - \left\{ k+1 \right\} \right) \leq k$, we can conclude that 
$\text{card} ( A ) \leq k + 1$. 

This means that we have proved the inductive step.  Hence, by mathematical induction, for each 
$m \in \mathbb{N}$, if $A \subseteq \mathbb{N}_m$, then $A$ is finite and 
$\text{card} ( A ) \leq m$.
\end{myproof}
%\hbreak
The preceding two lemmas were proved to aid in the proof of the following theorem.

\begin{theorem} \label{T:finitesubsets}
If $S$ is a finite set and $A$ is a subset of $S$, then $A$ is a finite set and 
$\text{card} ( A ) \leq \text{card} ( S )$.
\end{theorem}
%
\begin{myproof}
Let $S$ be a finite set and assume that $A$ is a subset of $S$.  If $A = \emptyset$, then $A$ is a finite set and $\text{card} ( A ) \leq \text{card} ( S )$.  So we assume that $A \ne \emptyset$.  

Since $S$ is finite, there exists a bijection $f\x S \to \mathbb{N}_k$ for some $k \in \mathbb{N}$.  In this case, $\text{card} ( S ) = k$.  We need to show that $A$ is equivalent to a finite set.  To do this, we define 
$g\x A \to f ( A )$ by
\begin{center}
$g ( x ) = f ( x )$ for each $x \in A$.
\end{center}
Since $f$ is an injection, we conclude that $g$ is an injection.  Now let 
$y \in f ( A )$.  Then there exists an $a \in A$ such that $f ( a ) = y$.  But by the definition of $g$, this means that $g ( a ) = y$, and hence $g$ is a surjection.  This proves that $g$ is a bijection.

Hence, we have proved that $A \approx f ( A )$.  Since $f ( A )$ is a subset of $\mathbb{N}_k$, we use Lemma~\ref{L:subsetsofNk} to conclude that $f ( A )$ is finite and $\text{card} ( f ( A ) ) \leq k$.  In addition, by 
Theorem~\ref{T:equivfinitesets}, $A$ is a finite set and 
$\text{card} ( A ) = \text{card} ( f ( A ) )$.  This proves that $A$ is a finite set and $\text{card} ( A ) \leq \text{card} ( S )$.
\end{myproof}
\hbreak

Lemma~\ref{L:addone} implies that adding one element to a finite set increases its cardinality by 1.  It is also true that removing one element from a finite nonempty set reduces the cardinality by 1.  The proof of Corollary~\ref{C:removeone} is 
Exercise~(\ref{exer:sec92corollary}).

\begin{corollary} \label{C:removeone}
If $A$ is a finite set and $x \in A$, then $A - \left\{ x \right\}$ is a finite set and  
$\text{card} \!\left( A - \left\{ x \right\} \right) = \text{card} ( A ) - 1$.
\end{corollary}
%
The next corollary will be used in the next section to provide a mathematical distinction between finite and infinite sets.
\begin{corollary} \label{C:propersubsets}
A finite set is not equivalent to any of its proper subsets.
\end{corollary}
%
\begin{myproof}
Let $B$ be a finite set and assume that $A$ is a proper subset of $B$.  Since $A$ is a proper subset of $B$, there exists an element $x$ in $B - A$.  This means that $A$ is a subset of 
$B - \left\{ x \right\}$.  Hence, by Theorem~\ref{T:finitesubsets},
\[
\text{card} ( A ) \leq \text{card} \!\left( B - \left\{ x \right\} \right).
\]
Also, by Corollary~\ref{C:removeone}
\[
\text{card} \!\left( B - \left\{ x \right\} \right) = \text{card} ( B ) - 1.
\]
Hence, we may conclude that $\text{card} ( A ) \leq \text{card} ( B ) - 1$ and that 
\[
\text{card} ( A ) < \text{card} ( B ).
\]
Theorem~\ref{T:equivfinitesets}
implies that $B \not \approx A$.  This proves that a finite set is not equivalent to any of its proper subsets.
\end{myproof}
\hbreak

\endinput

\subsection*{The Pigeonhole Principle}

The last property of finite sets that we will consider in this section is often called the 
\textbf{Pigeonhole Principle}.
\index{Pigeonhole Principle}%
  The ``pigeonhole'' version of this property says,  ``If $m$ pigeons go into $r$ pigeonholes and $m > r$, then at least one pigeonhole has more than one pigeon.''

In this situation, we can think of the set of pigeons as being equivalent to a set $P$ with cardinality $m$ and the set of pigeonholes as being equivalent to a set $H$ with cardinality 
$r$.  We can then define a function $f\x P \to H$ that maps each pigeon to its pigeonhole.  The Pigeonhole Principle states that this function is not an injection.  (It is not one-to-one since there are at least two pigeons ``mapped'' to the same pigeonhole.)

\begin{theorem} [\textbf{The Pigeonhole Principle}] \label{T:pigeonhole}
Let $A$ and $B$ be finite sets.  If $\text{card} ( A ) > \text{card} ( B )$, then any function $f\x A \to B$ is not an injection.
\end{theorem}
%
\begin{myproof}
Let $A$ and $B$ be finite sets. We will prove the contrapositive of the theorem, which is, 
if there exists a function $f\x A \to B$ that is an injection, then 
$\text{card} ( A ) \leq \text{card} ( B )$.

So assume that $f\x A \to B$ is an injection.  As in Theorem~\ref{T:finitesubsets}, we define a function $g\x A \to f ( A )$ by 
\begin{center}
$g ( x ) = f ( x )$ for each $x \in A$.
\end{center}
As we saw in Theorem~\ref{T:finitesubsets}, the function $g$ is a bijection.  But then 
$A \approx f ( A )$ and $f ( A ) \subseteq B$.  Hence, 
\begin{center}
$\text{card} ( A ) = \text{card} \!\left( f ( A ) \right)$ and 
$\text{card} \!\left( f ( A ) \right) \leq \text{card} ( B )$.
\end{center}
Hence, $\text{card} ( A ) \leq \text{card} ( B )$, and this proves the contrapositive.  Hence, if $\text{card} ( A ) > \text{card} ( B )$, then  any function $f\x A \to B$  is not an injection.
\end{myproof}
The Pigeonhole Principle has many applications in the branch of mathematics called ``combinatorics.''  Some of these will be explored in the exercises.
\hbreak
\index{finite set!properties of|)}%

\endinput


\endinput




\hbreak

\endinput

\begin{example}
Let $\mathcal{F}$ be the set of all functions from $\mathbb{N}$ to the set 
$\left\{ 0, 1 \right\}$.  This set is often denoted by $\left\{ 0, 1 \right\}^\mathbb{N}$.  We will prove that $\mathcal{F} \approx \mathcal{P} ( \mathbb{N} )$, the power set of 
$\mathbb{N}$.

To prove that $\mathcal{F} \approx \mathcal{P} ( \mathbb{N} )$, we will construct a function $G\x \mathcal{F} \to \mathcal{P} ( \mathbb{N} )$ and then prove that $G$ is a bijection.  Notice that the domain of the function $G$ will be a set of functions.  So, the inputs for $G$ will themselves be functions with domain $\mathbb{N}$ and codomain 
$\left\{ 0, 1 \right\}$.

So, we define $G\x \mathcal{F} \to \mathcal{P} ( \mathbb{N} )$ as follows:
\[
\text{For } f \in \mathcal{F}, \quad G ( f ) = \left\{x \in \mathbb{N} \mid 
f ( x ) = 1 \right\}.
\]
Notice that $G ( f ) \subseteq \mathbb{N}$ and hence 
$G ( f ) \in \mathcal{P} ( \mathbb{N} )$.

To prove that the function $G$ is an injection, we let $f_1, f_2 \in \mathcal{F}$ and suppose that 
$f_1 \ne f_2$.  Since $f_1\x  \mathbb{N} \to \left\{ 0, 1 \right\}$ and 
$f_2\x  \mathbb{N} \to \left\{ 0, 1 \right\}$, this means that 
\begin{center}
there exists an $m \in \mathbb{N}$ such that $f_1 ( m ) \ne f_2 ( m )$.
\end{center}
Since the codomain of both $f_1$ and $f_2$ is $\left\{ 0, 1 \right\}$, we may assume that 
$f_1 ( m ) = 1$ and $f_2 ( m ) = 0$.  (The proof for the case where 
$f_1 ( m ) = 0$ and $f_2 ( m ) = 1$ is similar.)  But then
\[
m \in \left\{x \in \mathbb{N} \mid f_1 ( x ) = 1 \right\} \text{ and } 
m \not \in \left\{x \in \mathbb{N} \mid f_2 ( x ) = 1 \right\}.
\]
That is, $m \in G ( f_1 )$ and $m \not \in G ( f_2 )$.  Hence, 
$G ( f_1 ) \ne G ( f_2 )$.  This proves that $G$ is an injection.

To show that $G$ is a surjection, we let $B \in \mathcal{P} ( \mathbb{N} )$.  So, 
$B \subseteq \mathbb{N}$.  We need to find a function $f \in \mathcal{F}$ such that 
$G ( f ) = B$.  We will use the \textbf{characteristic function} of the set $B$.  This was introduced in Section~\ref{S:moreaboutfunctions} on page~\pageref{charfunction}, and is defined as follows:  $\chi _B \x \mathbb{N} \to \left\{ {0, 1} \right\}$ by
\[
\chi _B ( x ) = \left\{ \begin{gathered}
  1\text{  if  }x \in B \hfill \\
  0\text{  if  }x \notin B \hfill \\ 
\end{gathered}  \right.
\]
Therefore, 
\[
G ( \chi_B ) = \left\{x \in \mathbb{N} \mid \chi_B ( x ) = 1 \right\} = B.
\]
This proves that the function $G$ is a surjection, and hence, $G$ is a bijection.  Therefore, we have proved that $\mathcal{F} \approx \mathcal{P} ( \mathbb{N} )$.
\end{example}
\hbreak
%


Hence,
$\text{card} ( A ) = k$.

Also, $\left\{ x \right\} \approx \left\{ k + 1 \right\}$.  In addition, 
$A \cap \left\{ x \right\} = \emptyset$ and 
$\mathbb{N}_k \cap \left\{ k + 1 \right\} = \emptyset$.  Hence, by Theorem~\ref{T:equivalentsets},
\[
A \cup \left\{ x \right\} \approx \mathbb{N}_k \cup \left\{ k + 1 \right\} = \mathbb{N}_{k+1}.
\]
This proves that $A \cup \left\{ x \right\}$ is a finite set with cardinality $k + 1$, and hence 
$\text{card} ( A \cup \left\{ x \right\} ) = \text{card} ( A ) + 1$.



We will now state a theorem that will prove useful in our study of equivalent sets.  Its proof will be outlined in Activity~\ref{A:theoremequivsets}

\begin{theorem} \label{T:equivalentsets}
Let $A$, $B$, $C$, and $D$ be sets with $A \approx B$ and $C \approx D$.  Then:
\begin{enumerate}
\item $A \times C \approx B \times D$. \label{T:equivalentsets1}

\item If $A$ and $C$ are disjoint and $B$ and $D$ are disjoint, then 
$A \cup C \approx B \cup D$. \label{T:equivalentsets2}
\end{enumerate}
\end{theorem}

\begin{activity}[Proof of Theorem~\ref{T:equivalentsets}] \label{A:theoremequivsets}

Let $A$, $B$, $C$, and $D$ be sets with $A \approx B$ and $C \approx D$.  Since $A \approx B$ and 
$C \approx D$, there exist bijections $f\x A \to B$ and $g\x C \to D$.

\begin{enumerate}
\item To prove that $A \times C \approx B \times D$, prove that the following function is a bijection:  $h\x A \times C \to B \times D$ where
\[
h ( a, c ) = ( f ( a ), g ( c ) )
\]
for all $( a, c ) \in A \times C$.

\item Now assume that $A \cap C = \emptyset$ and $B \cap D = \emptyset$.  To prove that 
$A \cup C \approx B \cup D$, prove that the following function is a bijection:  
$k\x A \cup C \to B \cup D$ where
\[
k ( x ) = \left\{ \begin{gathered}
  f ( x ) \text{  if  }x \in A \hfill \\
  g ( x ) \text{  if  }x \notin B \hfill \\ 
\end{gathered}  \right.
\]
for all $( a, c ) \in A \times C$.
\end{enumerate}
\end{activity}
\hbreak

