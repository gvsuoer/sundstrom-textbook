\documentclass[11pt]{article}
\usepackage{c://pctex/activity}

\lhead{}
\chead{\textbf{\large{Exercise 13 -- Section 5.3\\Comparison to Properties of the Real Numbers}}}
\rhead{}
\lfoot{\emph{Mathematical Reasoning: Writing and Proof, Third Ed.} \\Ted Sundstrom}
\cfoot{}
\rfoot{\copyright \the\year\, by Pearson Education, Inc.\\}


\begin{document}
\begin{itemize}
\item Both addition and multiplication of the real numbers are commutative and associative.  Both union and intersection of sets are commutative and associative.

\item In the real numbers, there is one distributive property:  multiplication is distributive over addition.  However, for sets, there are two distributive properties:  intersection is distributive over union and union is distributive over intersection.

\item The number zero is an additive identity for addition of the real numbers.  The empty set can be considered an identity for the operation of union since  $A \cup \emptyset  = A$ 
for each set  $A$.

\item The number one is a multiplicative identity for addition of the real numbers.  The universal set  $U$  can be considered an identity for the operation of intersection since  $A \cap U = A$
for each set  $A$.
\end{itemize}





\end{document}
