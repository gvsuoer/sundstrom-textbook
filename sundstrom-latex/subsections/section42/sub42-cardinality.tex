\subsection*{The Cardinality of a Power Set}
In Beginning Activity~\ref{PA:subsetsofaset4}, we observed that a set with one element has two subsets, a set with two elements has four subsets, and a set with three elements has eight subsets.  We also studied a way to use the eight subsets of a set with three elements to create the 16 subsets of a set with four elements.  This work suggests that the following proposition is true.

\begin{proposition} \label{P:inductivestepforsubsets}
Let $A$ and $B$ be subsets of some universal set.  If  $A = B \cup \left\{ x \right\}$, where  $x \notin B$, then any subset of  $A$  is either a subset of  $B$  or a set of the form   
$C \cup \left\{ x \right\}$, where  $C$  is a subset of  $B$.
\end{proposition}
%
\begin{myproof}
Let $A$ and $B$ be subsets of some universal set, and assume that  $A = B \cup \left\{ x \right\}$ where  $x \notin B$.  Let  $Y$  be a subset of  $A$.  We need to show that  $Y$  is a subset of  $B$  or that   $Y = C \cup \left\{ x \right\}$, where  $C$   is some subset of  $B$.  There are two cases to consider:  (1)  $x$  is not an element of  $Y$\!, and (2)  $x$  is an element of  $Y$\!.
\vskip6pt
\noindent
\textit{Case 1}:   Assume that  $x \notin Y$\!.  Let  $y \in Y$.  Then  $y \in A$  and  
$y \ne x$.  Since  
\[
A = B \cup \left\{ x \right\}\!,
\]
this means that  $y$  must be in  $B$.  Therefore,  $Y \subseteq B$\!.
\vskip6pt
\noindent
\textit{Case 2}:  Assume that  $x \in Y$\!. In this case, let  $C = Y - \left\{ x \right\}$.  Then every element of  $C$  is an element of  $B$. Hence, we can conclude that  $C \subseteq B$  and that  $Y = C \cup \left\{ x \right\}$.
\vskip10pt
\noindent
Cases (1) and (2) show that if  $Y \subseteq A$, then  $Y \subseteq B$  or  
$Y = C \cup \left\{ x \right\}$,  where  \linebreak
$C \subseteq B$.
\end{myproof}
%\hbreak
%
\noindent
The power set of  $A$, $\mathcal{P}( A )$, is the set of all subsets of  $A$.  So 
\[
Y \in \mathcal{P}\left( A \right)\text{ if and only if }Y \subseteq A.
\]
%Proposition~\ref{P:inductivestepforsubsets} states that  if $A = B \cup \left\{ x \right\}$ and  $x \notin B$, then $Y \subseteq A$ if and only if  $Y \subseteq B$ or that 
%$Y = C \cup \left\{ x \right\}$, where  $C$   is some subset of  $B$.

Using power sets and assuming that $A = B \cup \left\{ x \right\}$ and  $x \notin B$, 
Proposition~\ref{P:inductivestepforsubsets} can be written as follows:  
$Y \in \mathcal{P}( A )$ if and only if  
$Y \in \mathcal{P}( B )$ or that   $Y = C \:\cup\: \left\{ x \right\}$, where  
$C \in \mathcal{P}( B )$.  This gives us the following corollary of 
Proposition~\ref{P:inductivestepforsubsets}.
%
\begin{corollary}\label{C:inductivestepforsubsets}
Let $A$ and $B$ be subsets of some universal set. If  $x \notin B$ and 
$A = B \cup \left\{ x \right\}$, then  $\mathcal{P}(A) = \mathcal{P}(B) \cup \left\{ {C \cup \left\{ x \right\} \left| {C \in \mathcal{P}(B)} \right.} \right\}$.
\end{corollary}
\hbreak

We can now use Proposition~\ref{P:inductivestepforsubsets} (or its corollary) and mathematical induction to prove Theorem~\ref{P:powersetcardinality}.

\begin{theorem}\label{P:powersetcardinality}
Let  $n$  be a nonnegative integer and let  $A$  be a subset of some universal set.  If  $A$  is a finite set with  $n$  elements, then  $A$  has  $2^n $ subsets.  That is,  if  $\left| A \right| = n$, then  $\left| {\mathcal{P}( A )} \right| = 2^n $.
\end{theorem}

\noindent
The proof of this theorem will be done in Activity~\ref{A:powersetcardinality}.


\begin{activity}[The Cardinality of a Power Set] \label{A:powersetcardinality}
\index{power set}%
\index{power set!cardinality}%
 \hfill \\
For each nonnegative integer $n$, let $P ( n )$ be, ``If   a finite set has exactly  
$n$  elements, then  that set  has exactly  $2^n $ subsets.''
\begin{enumerate}
\item Verify that $P ( 0 )$ is true.  (This is the basis step for the induction proof.)

\item Verify that $P( 1 )$ and $P( 2 )$ are true.

\item Now assume that  $k$  is a nonnegative integer and assume that $P( k )$is true.  That is, assume that if a set has  $k$  elements, then that set has  $2^k $  subsets.  (This is the inductive assumption for the induction proof.)

Let  $A$  be a subset of the universal set with  $\left| A \right| = k + 1$, and let  
$x \in A$.  Then the set  $B = A - \left\{ x \right\}$ has  $k$  elements.

Now use the inductive assumption to determine how many subsets  $B$  has.  Then use  Proposition~\ref{P:inductivestepforsubsets} to prove that  $A$  has twice as many subsets as  $B$.  This should help complete the inductive step for the induction proof.
\end{enumerate}
\end{activity}
\hbreak

\endinput
