\begin{previewactivity}[\textbf{Some Work with Congruence Modulo \emph{n}}]\label{PA:congruencereview} \hfill 
\begin{enumerate}
\item Let $n$ be a natural number and let $a$ and $b$ be integers.
\begin{enumerate}
\item Write the definition of ``$a$ is congruent to $b$ modulo $n$,'' which is written 
$a \equiv b \pmod n$.

\item Use the definition of ``divides'' to complete the following:

\begin{list}{}
\item When we write $a \equiv b \pmod n$, we may conclude that there exists an integer $k$ such that \ldots.
\end{list}
\end{enumerate}
\end{enumerate}
We will now explore what happens when we multiply several pairs of integers where the first one is congruent to 3 modulo 6 and the second is congruent to 5 modulo 6.  We can use set builder notation and the roster method to specify the set 
$A$ of all integers that are congruent to 3 modulo 6 as follows:
\[
A = \left\{ a \in \Z \mid \mod{a}{3}{6} \right\} = \{ \ldots, -15, -9, -3, 3, 9, 15, 21, \ldots \}.
\]
\setcounter{oldenumi}{\theenumi}
\begin{enumerate} \setcounter{enumi}{\theoldenumi}
\item Use the roster method to specify the set $B$ of all integers that are congruent to 5 modulo 6.
\[
B = \left\{ b \in \Z \mid \mod{b}{5}{6} \right\} = \cdots \hspace{36pt}.
\]
\end{enumerate}
Notice that $15 \in A$ and $11 \in B$ and that $15 + 11 = 26$.  Also notice that $\mod{26}{2}{6}$ and that 2 is the smallest positive integer that is congruent to 26 (mod 6).
\setcounter{oldenumi}{\theenumi}
\begin{enumerate} \setcounter{enumi}{\theoldenumi}
\item Now choose at least four other pairs of integers $a$ and $b$ where $a \in A$ and $b \in B$.  For each pair, calculate $(a + b)$ and then determine the smallest positive integer $r$ for which 
$\mod{(a + b)}{r}{6}$.  \note The integer $r$ will satisfy the inequalities $0 \leq r < 6$.

\item Prove that for all integers $a$ and $b$, if $\mod{a}{3}{6}$ and $\mod{b}{5}{6}$, then 
$\mod{(a + b)}{2}{6}$.
\end{enumerate}
\end{previewactivity}
\hbreak

\endinput



\item 
\begin{enumerate}
\item Write the reflexive, symmetric, and transitive properties of congruence?  See 
Theorem~\ref{T:modprops} on page~\pageref{T:modprops}.

\item Explain why $17 \equiv 5 \pmod 6$.  If $x$ is an integer and  $x \equiv 17 \pmod 6$, then what conclusion can be made about the integer $x$ using the transitive property of congruence?

\item Find an integer $r$ such that $16^2 \equiv r \pmod 6$ and $0 \leq r < 5$.  Is there more than one such integer?  Find an integer $s$ such that \linebreak
$5^3 \equiv s \pmod 6$ and 
$0 \leq x < 5$.  Is there more than one such integer?
\end{enumerate}
