\section*{Section \ref{S:gcd} The Greatest Common Divisor}

\begin{enumerate}
\item \begin{multicols}{2}
\begin{enumerate}
\item $\gcd \left( {21, 28} \right) = 7$

\item $\gcd \left( { - 21, 28} \right) = 7$

\item $\gcd \left( {58, 63} \right) = 1$

\item $\gcd \left( {0, 12} \right) = 12$

\item $\gcd \left( {110, 215} \right) = 5$

\item $\gcd \left( {110, -215} \right) = 12$
\end{enumerate}
\end{multicols}



\item \begin{enumerate}
\item If $k \mid a$ and $k \mid \left( a + 1 \right)$, then by a result in 
Exercise~(\ref{exer:3truefalse}) in Section~\ref{S:directproof}, 
$k \mid \left[ \left( a + 1 \right) - a \right]$.  Hence, $k \mid 1$.

\item Let $d = \gcd \left( a, a + 1 \right)$.  Then, $d \mid a$ and 
$d \mid \left( a + 1 \right)$.  Hence, $d \mid 1$ and so $d = 1$.
\end{enumerate}



\item \begin{enumerate}
\item If $k \mid a$ and $k \mid \left( a + 2 \right)$, then by a result in 
Exercise~(\ref{exer:3truefalse}) in Section~\ref{S:directproof}, 
$k \mid \left[ \left( a + 2 \right) - a \right]$.  Hence, $k \mid 2$.

\item Let $d = \gcd \left( a, a + 2 \right)$.  Then, $d \mid a$ and $d \mid \left( a + 2 \right)$.  Hence, $d \mid 2$ and so $d = 1$ or $d = 2$.  In addition, it can be shown that if $a$ is odd, then $d = 1$ and if $a$ is even, then $d = 2$.
\end{enumerate}


\item \begin{enumerate}
\item If $b \in \mathbb{Z}$ and $b \ne 0$, then $\left| b \right|$ is the largest positive divisor of $b$.  Hence, $\gcd \left( 0, b \right) = \left|b \right|$.

\item The integers $b$ and $-b$ have the same divisors.  Therefore, \\
$\gcd \left( a, -b \right) = \gcd \left( a, b \right)$.
\end{enumerate}



\item \begin{enumerate}
\item $\gcd \left( {36, 60} \right) = 12$, and 
$12 = 36 \cdot 2 + 60 \cdot \left( { - 1} \right)$.

\item $\gcd \left( {901, 935} \right) = 17$, and 
$17 = 901 \cdot 27 + 935 \cdot \left( { - 26} \right)$.

\item $\gcd \left( 72, 714 \right) = 6$, and  
$6 = 72 \cdot 10 + 714 \cdot \left( -1 \right) $.

\item $\gcd \left( 12628, 21361 \right) = 41$, and 
$41 = 12628 \cdot 181 + 21361 \cdot \left( 107 \right)$.

\item $\gcd \left( -36, -60 \right) = 12$, and  
$12 = -36 \cdot (-2) + (-60) \cdot 1 $.

\item $\gcd \left( 901, -935 \right) = 17$, and  
$17 = 901 \cdot 27 + (-935) \cdot 26 $.
\end{enumerate}



\item \begin{enumerate}
\item One possibility is $u = -3$ and $v = 2$.  In this case, $9u + 14v = 1$.  We then multiply both sides of this equation by 10 to obtain
\[
9 \cdot (-30) + 14 \cdot 20 = 10.
\]
So we can use $x = -30$ and $y = 20$.

\item This is not possible.  If we could find such integers $x$ and $y$, we would then have 
$9x + 15y = 10$.  However, 3 divides the left side of the equation and 3 does not divide 10.  This is a contradiction.

\item First write $9 \cdot (-3) + 15 \cdot 2 = 3$.  Mutliply both sides of this equation by 1054 to obtain
\[
9 \cdot (-3162) + 15 \cdot 2108 = 3162.
\]
So we can use $x = -3162$ and $y = 2108$.
\end{enumerate}




\item \begin{enumerate}
\item $11 \left( -3 \right) + 17 \cdot 2 = 1$.

\item $\dfrac{m}{11} + \dfrac{n}{17} = \dfrac{17m + 11n}{187}$.

\item Multiply both sides of the equation in Part~(a) to obtain \\
$11 \left( -30 \right) + 17 \cdot 20 = 10$.  Then, divide both sides of this equation by 
$11 \cdot 17 = 187$ to obtain
\[
\begin{aligned}
\frac{11 \left( -30 \right) + 17 \cdot 20}{187} &= \frac{10}{187} \\
                                                & \\
                 \frac{-30}{17} + \frac{20}{11} &= \frac{10}{187}. \\
\end{aligned}
\]
\end{enumerate}

\end{enumerate}
\hbreak
\endinput
