\section*{Section \ref{S:diophantine}}
\subsection*{Progress Check~\ref{prog:lineardioph}}
\begin{enumerate} \setcounter{enumi}{1}
\item $x = 2 + 3k$ \quad and \quad $y = 0 - 2k$,
where $k$ can be any integer.  Again, this does not prove that these are the only solutions.
\end{enumerate}


\subsection*{Progress Check~\ref{prog:prevact2}}
One of the Diophantine equations in \typeu Activity~2 was $3x + 5y = 11$.  We were able to write the solutions of this Diophantine equation in the form
\[
x = 2 + 5k \quad \text{and} \quad y = 1 - 3k,
\]
where $k$ is an integer.  Notice that $x = 2$ and $y = 1$ is a solution of this equation.  If we consider this equation to be in the form $ax + by = c$, then we see that $a = 3$, $b = 5$, and 
$c = 11$.  Solutions for this equation can be written in the form
\[
x = 2 + bk \quad \text{and} \quad y = 1 - ak,
\]
where $k$ is an integer.

The other equation was $4x + 6y = 16$.  So in this case, $a = 4$, $b = 6$, and $c = 16$.  Also notice that $d = \gcd ( 4, 6 ) = 2$.  We note that $x = 4$ and $y = 0$ is one solution of this Diophantine equation and solutions can be written in the form
\[
x = 4 + 3k \quad \text{and} \quad y = 0 - 2k,
\]
where $k$ is an integer.  Using the values of $a$, $b$, and $d$ given above, we see that the solutions can be written in the form
\[
x = 2 + \frac{b}{d} k \quad \text{and} \quad y = 0 - \frac{a}{d},
\]
where $k$ is an integer. 


\subsection*{Progress Check~\ref{prog:lindiophequations}}
\begin{enumerate}
\item Since 21 does not divide 40, Theorem~\ref{T:lindioph2} tells us that the Diophantine equation $63x + 336y = 40$ has no solutions.  
Remember that this means there is no ordered pair of integers $( x, y )$ such that 
$63x + 336y = 40$.  However, if we allow $x$ and $y$ to be real numbers, then there are real number solutions.  In fact, we can graph the straight line whose equation is 
$63x + 336y = 40$ in the Cartesian plane.  From the fact that there is no pair of integers $x, y$ such that $63x + 336y = 40$, we can conclude that there is no point on the graph of this line in which both coordinates are integers.

\item To write formulas that will generate all the solutions, we first need to find one solution for $144x + 225y = 27$.  This can sometimes be done by trial and error, but there is a systematic way to find a solution.  The first step is to use the Euclidean Algorithm in reverse to write $\gcd ( {144, 225} )$ as a linear combination of 144 and 225.  See 
Section~\ref{S:gcd} to review how to do this.  The result from using the Euclidean Algorithm in reverse for this situation is
\[
144 \cdot 11 + 225 \cdot ( {-7} ) = 9.
\]
If we multiply both sides of this equation by 3, we obtain
\[
144 \cdot 33 + 225 \cdot ( {-21} ) = 27.
\]
This means that $x_0 = 33, y_0 = -21$ is a solution of the linear Diophantine equation 
$144x + 225y = 27$.  We can now use Theorem~\ref{T:lindioph2} to conclude that all solutions of this Diophantine equation can be written in the form
\[
x = 33 + \frac{225}{9} k \qquad
y = -21 -\frac{144}{9} k,
\]
where $k \in \mathbb{Z}$.  Simplifying, we see that all solutions can be written in the form
\[
x = 33 + 25 k \qquad
y = -21 -16 k,
\]
where $k \in \mathbb{Z}$.
\vskip10pt

We can check this general solution as follows:  Let $k \in \mathbb{Z}$.  Then
\[
\begin{aligned}
144x + 225y &= 144 ( {33 + 25k} ) + 225 ( {-21 - 16k} ) \\
            &= ( {4752 + 3600k} ) + ( {-4725 - 3600k} ) \\
            &= 27. \\
\end{aligned}
\]


\end{enumerate}
\hbreak


\endinput
