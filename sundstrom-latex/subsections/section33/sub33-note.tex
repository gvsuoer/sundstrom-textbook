\subsection*{Important Note}
A proof by contradiction is often used to prove a conditional statement \linebreak
$P \to Q$  when a direct proof has not been found and it is relatively easy to form the negation of the proposition.  The advantage of a proof by contradiction is that we have an additional assumption with which to work (since we assume not only  $P$ but also  $\mynot  Q$).  The disadvantage is that there is no well-defined goal to work toward.  The goal is simply to obtain some contradiction.  There usually is no way of telling beforehand what that contradiction will be, so we have to stay alert for a possible absurdity.  Thus, when we set up a know-show table for a proof by contradiction, we really only work with the know portion of the table.  %This was illustrated in Example~\ref{E:contradiction} and Proposition~\ref{P:contradiction}.
\hbreak

\begin{prog}[\textbf{Exploration and a Proof by Contradiction}]\label{pr:exploreproof} \hfill \\
Consider the following proposition:
\begin{list}{}
  \item For each integer $n$, if  $n \equiv 2 \pmod 4$, then  
$n\not  \equiv 3 \pmod 6$.
\end{list}
\begin{enumerate}
  \item Determine at least five different integers that are congruent to  2  modulo  4, 
\label{pr:exploreproof1}%
 and determine at least five different integers that are congruent to  3  modulo  6.  Are there any integers that are in both of these lists?

  \item For this proposition, why does it seem reasonable to try a proof by contradiction?

  \item For this proposition, state clearly the assumptions that need to be made at the beginning of a proof by contradiction, and then use a proof by contradiction to prove this proposition.
\end{enumerate}
\end{prog}
\hbreak

\endinput
