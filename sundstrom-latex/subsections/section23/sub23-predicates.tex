\subsection*{Variables and Open Sentences}
As we have seen in the \typel activities, not all mathematical sentences are statements.  This is often true if the sentence contains a variable.  The following terminology is useful in working with sentences and statements.
%
\begin{defbox}{D:predicate}{An \textbf{open sentence}  
\index{open sentence}%
 is a sentence  $P( {x_1 ,x_2 , \ldots ,x_n })$ involving variables $x_1 ,x_2 , \ldots ,x_n $ with the property that when specific values from the universal set are assigned to $x_1 ,x_2 , \ldots ,x_n $, then the resulting sentence is either true or false.  That is, the resulting sentence is a statement.  An open sentence is also called a \textbf{predicate}
\index{predicate}%
 or a \textbf{propositional function}.}
\index{propositional function}%
\end{defbox}
%
\noindent
\textbf{Notation:}  One reason an open sentence is sometimes called a propositional function is the fact that we use function notation $P( {x_1 ,x_2 , \ldots ,x_n })$ for an open sentence in  $n$  variables.  When there is only one variable, such as  $x$, we write  $P( x )$, which is read ``$P$ of  $x$.''  In this notation,  $x$  represents  an arbitrary element of the universal set, and  $P(x)$ represents a sentence.   When we substitute a specific element of the universal set for  $x$, the resulting sentence becomes a statement.  This is illustrated in the next example. 
%\hbreak
%
\begin{example}[\textbf{Open Sentences}] \label{E:exam21} \hfill \\
If the universal set is  $\mathbb{R}$, then  the sentence  ``$x^2  - 3x - 10 = 0$'' is an open sentence involving the one variable  $x$.  
\begin{itemize}
  \item If we substitute  $x = 2$, we obtain the false statement 
``$2^2  - 3 \cdot 2 - 10 = 0$.''

  \item If we substitute  $x = 5$, we obtain the true statement  
``$5^2  - 3 \cdot 5 - 10 = 0$.''

\end{itemize}
In this example, we can let  $P( x )$ be the predicate ``$x^2  - 3x - 10 = 0$
'' and then say that  $P( 2 )$ is false and $P( 5 )$ is true.

Using similar notation, we can let  $Q( {x, y} )$ be the predicate  ``$x + 2y = 7$.''  This predicate involves two variables.  Then,

\begin{itemize}
\item $Q( {1, 1})$ is false since  ``$1 + 2 \cdot 1 = 7$'' is false; and  

\item $Q( {3, 2} )$ is true since  ``$3 + 2 \cdot 2 = 7$'' is true.
\end{itemize}
\end{example}
\hbreak
%
\begin{prog}[\textbf{Working with Open Sentences}]\label{pr:predicates} \hfill
\begin{enumerate}
  \item Assume the universal set for all variables is  $\Z$ and let  $P( x )$ be the predicate  ``$x^2  \leq 4$.'' 
  \begin{enumerate}
    \item Find two values of  $x$  for which  $P( x )$ is false.
    \item Find two values of  $x$  for which  $P( x )$ is true.
    \item Use the roster method to specify the set of all $x$  for which  $P( x )$ is true.
  \end{enumerate}

\item Assume the universal set for all variables is  $\mathbb{Z}$, and let  
$R( {x,\;y,\;z} )$ be the predicate  ``$x^2  + y^2  = z^2 $.'' 
  \begin{enumerate}
    \item Find two different examples for which  $R( {x, y, z})$ is false.
    \item Find two different examples for which  $R( {x, y, z})$ is true.
  \end{enumerate}

%\item Assume the universal set for the variable  $x$  is  $\mathbb{R}$  and the universal set for the variable  $y$  is the set of all real numbers that are greater than or equal to  2.  Let  $Q\left( {x,\;y} \right)$ be the predicate  ``$y = x^2  + 2$.''
%  \begin{enumerate}
%    \item Find two different examples for which  $Q\left( {x,\;y} \right)$ is false.
%    \item Find two different examples for which  $Q\left( {x,\;y} \right)$ is true.
%    \item Use a graph in the coordinate plane to describe the set of all ordered pairs  $\left( {x,\;y} \right)$ for which  $Q\left( {x,\;y} \right)$ is true.
%  \end{enumerate}
\end{enumerate}
\end{prog}
\hbreak
Without using the term, Example~\ref{E:exam21} and Progress Check~\ref{pr:predicates} (and \typeu Activity~\ref*{PA:variable}) dealt with a concept called the truth set of a predicate.  
%
\begin{defbox}{D:truthset}{The \textbf{truth set of an open sentence with one variable}
\index{truth set}%
 is the collection of objects in the universal set that can be substituted for the variable to make the predicate a true statement.}
\end{defbox}
%
%\begin{flushleft}
%\fbox{\parbox{5in}{\begin{definition} \label{D:truthset}
%The \textbf{truth set of a predicate with one variable} is the collection of objects in the universal set that can be substituted for the variable to make the predicate a true statement.
%\end{definition}}}
%\end{flushleft}
%
%\begin{flushleft}

\newpar
One part of elementary mathematics consists of learning how to solve equations.  In more formal terms, the process of solving an equation is a way to determine the truth set for the equation, which is an open sentence.  In this case, we often call the truth set the \textbf{solution set}.
\index{solution set}%
Following are three examples of truth sets.
\begin{itemize}
  \item If the universal set is $\R$, then the truth set of the equation $3x - 8 = 10$ is the set $\{ 6 \}$.
  \item If the universal set is  $\mathbb{R}$, then the truth set of the equation ``$x^2  - 3x - 10 = 0$''  is  $\left\{ { - 2, 5} \right\}$.
  \item If the universal set is  $\mathbb{N}$, then the truth set of the open sentence  
``$\sqrt {n}  \in \mathbb{N}$''  is  $\left\{ {1, 4, 9, 16, \ldots } \right\}$.
\end{itemize}
\hbreak


\endinput
