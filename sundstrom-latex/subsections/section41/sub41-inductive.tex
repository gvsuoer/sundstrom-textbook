\subsection*{Inductive Sets}
The two open sentences in \typeu Activity~\ref*{PA:exploringstatements} appeared to be true for all values of  $n$  in the set of natural numbers, $\mathbb{N}$.  That is, the examples in this \typel activity provided evidence that the following two statements are true.  

\begin{itemize}
\item For each natural number $n$,  4  divides  $\left( {5^n  - 1} \right)$. \label{conjecture1}
\item For each natural number  $n$, $1^2  + 2^2  + \, \cdots \, + n^2  = \dfrac{{n(n + 1)(2n + 1)}}{6}$. \label{conjecture2}
\end{itemize}
One way of proving statements of this form uses the concept of an inductive set introduced in \typeu Activity~\ref*{PA:propertyofN}.  The idea is to prove that if one natural number makes the open sentence true, then the next one also makes the open sentence true.  This is how we handle the phrase ``and so on'' when dealing with the natural numbers.
%
In \typeu Activity~\ref*{PA:propertyofN}, we saw that the number systems $\N$  and  $\Z$ and other sets are inductive.  What we are trying to do is somehow distinguish  $\N$ from the other inductive sets.  The way to do this was suggested in Part~(\ref{PA:propertyofN6}) of \typeu Activity~\ref*{PA:propertyofN}.  Although we will not prove it, the following statement should seem true.

\begin{description}
\item [Statement 1:] For each subset  $T$ of $\N$,  if  $1 \in T$  and  $T$  is inductive, then  $T = \mathbb{N}$. \label{inductivestatement1}
\end{description}
Notice that the integers,  $\mathbb{Z}$, and the set  
$S = \left\{ {n \in \mathbb{Z} \mid n \geq  - 3} \right\}$ both contain  1  and both are inductive, but they both contain numbers other than natural numbers.  For example, the following statement is false:

\begin{description}
\item [Statement 2:] For each subset $T$ of $\Z$, if  $1 \in T$  and  $T$  is inductive, then  $T = \mathbb{Z}$. \label{inductivestatement2}
\end{description}
The set  $S = \left\{ {n \in \mathbb{Z} \mid n \geq  - 3} \right\} = \left\{ { -3, -2, -1,0,1,2,3, \ldots \: } \right\}$ is a counterexample that shows that this statement is false.
%

\begin{prog}[\textbf{Inductive Sets}] \label{prog:inductivesets} \hfill \\
Suppose that $T$ is an inductive subset of the integers.  Which of the following statements are true, which are false, and for which ones is it not possible to tell?

\begin{multicols}{2}
\begin{enumerate}
\item $1 \in T$ and $5 \in T$\!.

\item If $1 \in T$, then $5 \in T$\!.

\item If $5 \notin T$\!, then $2 \notin T$\!.

\item For each integer $k$, if $k \in T$\!, then $k + 7 \in T$.

\item For each integer $k$, $k \notin T$ or $k + 1 \in T$.

\item There exists an integer $k$ such that $k \in T$ and $k + 1 \notin T$\!.

\item For each integer $k$, if $k + 1 \in T$\!, then $k \in T$\!.

\item For each integer $k$, if $k + 1 \notin T$\!, then $k \notin T$\!.
\end{enumerate}
\end{multicols}
\end{prog}
\hbreak

\endinput
