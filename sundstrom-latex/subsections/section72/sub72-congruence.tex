\subsection*{Congruence Modulo $\boldsymbol{n}$}
One of the important equivalence relations we will study in detail is that of congruence
\index{congruence}%
 modulo  $n$.  We reviewed this relation in \typeu Activity~\ref*{PA:reviewofcongruence}.

%Let  $n \in \mathbb{N}$.  For  $a, b \in \mathbb{Z}$, we have defined  $a$  to be congruent to  $b$  modulo  $n$, denoted by  $a \equiv b \pmod n$, as follows:
%\begin{center}
%$a \equiv b \pmod n$  provided that  $n \mid \left( {a - b} \right)$.
%\end{center}
%This is equivalent to saying that  $a \equiv b \pmod n$  provided that there exists an integer  $k$  such that  $a - b = nk$.

Theorem~\ref{T:modprops} on page~\pageref{T:modprops} tells us that congruence modulo $n$ is an equivalence relation on  $\mathbb{Z}$.  Recall that by the Division Algorithm, if  $a \in \mathbb{Z}$, then there exist unique integers  $q$  and  $r$  such that
\[
a = nq + r \text{  and  }0 \leq r < n.
\]
Theorem~\ref{T:congtorem} and Corollary~\ref{C:congtorem} then tell us that  
$a \equiv r \pmod n$.  That is,  $a$  is congruent modulo  $n$ to its remainder $r$ when it is divided by  $n$.  When we use the term ``remainder'' in this context, we always mean the remainder $r$ with $0 \leq r < n$ that is guaranteed by the Division Algorithm.  We can use this idea to prove the following theorem.
%\hbreak
%
\begin{theorem} \label{T:congruence-remainder}
Let  $n \in \mathbb{N}$ and let  $a, b \in \mathbb{Z}$.  Then 
$a \equiv b \pmod n$  if and only if  $a$  and  $b$  have the same remainder when divided by  $n$.
\end{theorem}
%
\begin{myproof}
Let  $n \in \mathbb{N}$ and let  $a, b \in \mathbb{Z}$.  We will first prove that if  $a$  and  $b$  have the same remainder when divided by  $n$, then $a \equiv b \pmod n$.  So assume that  $a$  and  $b$  have the same remainder when divided by  $n$, and let  $r$  be this common remainder.  Then, by Theorem~\ref{T:congtorem},
\[
a \equiv r \pmod n \text{  and  }b \equiv r \pmod n\!.
\]
Since congruence modulo  $n$  is an equivalence relation, it is a symmetric relation.  Hence, since  $b \equiv r \pmod n$, we can conclude that  
$r \equiv b \pmod n$.  Combining this with the fact that  
$a \equiv r \pmod n$, we now have 
\[
a \equiv r \pmod n\text{  and  }r \equiv b \pmod n\!.
\]
We can now use the transitive property to conclude that  
$a \equiv b \pmod n$.  This proves that if $a$  and  $b$  have the same remainder when divided by  $n$, then  \linebreak
$a \equiv b \pmod n$.
\vskip6pt

We will now prove that if  $a \equiv b \pmod n$, then  $a$  and  $b$  have the same remainder when divided by  $n$.  Assume that  $a \equiv b \pmod n$, and let  $r$  be the least nonnegative remainder when  $b$   is divided by  $n$.  Then $0 \leq r < n$ and, by Theorem~\ref{T:congtorem},
\[
b \equiv r \pmod n\!.
\]
Now, using the facts that  $a \equiv b \pmod n$ and   
$b \equiv r \pmod n$, we can use the transitive property to conclude that
\[
a \equiv r \pmod n\!.
\]
This means that there exists an integer  $q$   such that  $a - r = nq$ or that
\[
a = nq + r.
\]
Since we already know that  $0 \leq r < n$, the last equation tells us that  $r$  is the least nonnegative remainder when  $a$  is divided by  $n$.  Hence we have proven that if  
$a \equiv b \pmod n$, then  $a$  and  $b$  have the same remainder when divided by  $n$.
\end{myproof}
\hbreak

\endinput
