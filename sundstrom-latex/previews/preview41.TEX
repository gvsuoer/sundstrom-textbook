\begin{previewactivity}[A Conjecture about Divisibility] \label{PA:dividesconjecture} \hfill \\
In Section~\ref{S:predicates}, we defined the \textbf{truth set}
\index{truth set}%
 of a predicate 
$P( x )$ to be the collection of objects in the universal set that make the predicate a true statement when substituted for  $x$.  

Assume the universal set is  $\mathbb{N}$, and let  $n$  be a natural number.  Consider the following predicate:
\[
4 \text{ divides }  \left( {5^n  - 1} \right)\!.
\]
\begin{enumerate}
\item Does this predicate become a true statement when  $n = 1$?  That is, is  1  in the truth set of this predicate?  

\item Does this predicate become a true statement when  $n = 2$?  That is, is  2  in the truth set of this predicate?

\item Choose at least four more natural numbers and determine whether the predicate is true or false for each of your choices.

\item Based on this work, do you think the following proposition is true or false?  Explain.
\begin{center}
For each natural number  $n$, 4  divides  $\left( {5^n  - 1} \right)$.
\end{center}

\end{enumerate}
\end{previewactivity}
\hbreak
\newpage
%
\begin{previewactivity}[Exploring a Summation] \label{PA:exploresum} \hfill \\
\setcounter{equation}{0}
Let  $n$  be a natural number.  Let $P( n )$ be the following open sentence:
\begin{equation} \label{eq:PAexplore} \notag
1^2  + 2^2  + \, \cdots \, + n^2  = \frac{{n(n + 1)(2n + 1)}}{6}.
\end{equation}
The expression on the left side of the previous equation is the sum of the squares of the first  $n$  natural numbers.  So when  $n = 1$, the left side of equation~(\ref{eq:PAexplore}) is  
$1^2 $.  When  $n = 2$, the left side of equation~(\ref{eq:PAexplore}) is  $1^2  + 2^2 $.

\begin{enumerate}
\item Does $P ( n )$ become a true statement when
\begin{itemize}
\item  $n = 1$?  That is, is  1  in the truth set of $P ( n )$?

\item $n = 2$?  That is, is  2  in the truth set of $P ( n )$?

\item $n = 3$?  That is, is  3  in the truth set of $P ( n )$?
\end{itemize}

\item Choose at least four more natural numbers and determine whether the open sentence is true or false for each of your choices.  A table with the columns $n$, $1^2  + 2^2  + \, \cdots \, + n^2$, and $\dfrac{{n(n + 1)(2n + 1)}}{6}$ may help you organize your work.

\item Based on this work, do you think the following proposition is true or false?  Explain.
\begin{list}{}
\item For each natural number  $n$,  $1^2  + 2^2  + \, \cdots \, + n^2  = \dfrac{{n(n + 1)(2n + 1)}}{6}$.
\end{list}

\end{enumerate}
\end{previewactivity}
\hbreak
%
\begin{previewactivity}[A Property of the Natural Numbers] \label{PA:propertyofN} \hfill \\
Intuitively, the natural numbers begin with the number  1, and then there is 2, then 3, then 4, and so on.  Does this process of ``starting with 1'' and ``adding 1 repeatedly'' result in all the natural numbers?  We will use the concept of an inductive set to explore this idea in this activity.

\begin{defbox}{D:inductiveset}{A set  $T$  that is a subset of  $\mathbb{Z}$ is an 
\textbf{inductive set}
\index{inductive set}%
 provided that for each integer $k$, if $k \in T$, then  $k + 1 \in T$.}
\end{defbox}

%Consider the following property for a set   $T$  that is a subset of  $\mathbb{Z}$, the set of all integers.
%\begin{center}
%\textbf{Property I:} For every  $k \in \mathbb{Z}$, if  $k \in T$, then  $k + 1 \in T$.
%\end{center}

\begin{enumerate}
\item Which of the following sets are inductive sets?  Do not worry about formal proofs, but if a set is  not inductive, be sure to provide a specific counterexample that proves it is not inductive.

\begin{multicols}{2}
\begin{enumerate}
\item $A = \left\{ {1,2,3, \ldots ,20} \right\}$

\item The set of natural numbers, $\mathbb{N}$

\item $B = \left\{ { {n \in \mathbb{N}} \mid n \geq 5} \right\}$
 
\item $S = \left\{ { {n \in \mathbb{Z}} \mid n \geq  - 3} \right\}$
 
\item $R = \left\{ { {n \in \mathbb{Z}} \mid n \leq  100} \right\}$

\item The set of integers, $\mathbb{Z}$
\end{enumerate}
\end{multicols}

\item Now assume that  $T \subseteq \mathbb{N}$ and assume that  $1 \in T$ and that  $T$ is an inductive set. \label{PA:propertyofN6}

\begin{multicols}{2}
\begin{enumerate}
  \item Is  $2 \in T$?  Explain.
  \item Is  $3 \in T$?  Explain.
  \item Is  $4 \in T$?  Explain.
  \item Is  $100 \in T$?  Explain.
  \item Do you think that  $T = \mathbb{N}$?  Explain.
\end{enumerate}
\end{multicols}

\end{enumerate}
\end{previewactivity}
\hbreak
\endinput

