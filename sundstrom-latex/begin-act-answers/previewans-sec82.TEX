\documentclass[11pt]{article}
\usepackage{../../styles/activity}

\usepackage{xr}
\externaldocument{0-MR}

\lhead{}
%\chead{\textbf{\Large{\hspace{0pt}Beginning Activities for Section~8.2}}\\\hspace{0pt}\emph{Mathematical Reasoning: Writing and Proof}}
\bahead{8.2}
\rhead{}
\lfoot{}
\rfoot{}
\cfoot{\hspace{0pt}\scalebox{0.4}{\includegraphics{cc-by-nc-sa.eps}}}
\graphicspath{{./epsfigs/}}


\begin{document}


\noindent
\subsection*{Beginning Activity 1 (Exploring Examples where 
$\boldsymbol{a}$ divides $\boldsymbol{b \cdot c}$)}
\begin{enumerate}
\item In all of the examples where  $a \mid \left( {bc} \right)$ but  $a$  does not divide  $b$  and  $a$  does not divide  $c$, we should see that  $\gcd \left( {a, b} \right) \ne 1$  and   $\gcd \left( {a, c} \right) \ne 1$.

\item In all of the examples where $\gcd \left( {a, b} \right) = 1$  and  
$a \mid \left( {bc} \right)$, we should see that  $a$  divides  $c$.

\item Conjecture:  Let  $a, b, c \in \mathbb{Z}$.  If  $\gcd \left( {a, b} \right) = 1$
  and  $a \mid \left( {bc} \right)$, then  $a$  divides  $c$.
\end{enumerate}
We will prove this result in this section.  (Theorem 8.12).  If $a$, $b$, and $c$ are non-zero integers and $a$ divides $bc$, it is tempting to conclude that $a$ divides $b$ or $a$ divides $c$.  The examples in part 1 of this beginning activity show that this cannot be done.  However, if we add the condition that $\gcd(a, b) = 1$, then we can conclude that $a$ divides $c$.
\hbreak


\noindent
\subsection*{Beginning Activity 2 (Prime Factorizations)}
\begin{enumerate} \setcounter{enumi}{1}
\item
\begin{tabular}[t]{p{2in} p{2in}}
\[
\begin{aligned}
40 &= 2 \cdot 20 \\ 
   &= 2 \cdot 2 \cdot 2 \ \cdot 5 \\
\end{aligned}
\]
&  
\[
\begin{aligned}
40 &= 5 \cdot 8 \\
   &= 5 \cdot 2 \cdot 2 \cdot 2 \\
\end{aligned}
\]
\end{tabular} 

\item In Part (2), the same prime number factors were obtained but in a different order.  Since multiplication of natural numbers is commutative and associative, we can say that we have the same factorization.

\item
\begin{tabular}[t]{p{2in} p{2in}}
\[
\begin{aligned}
150 &= 3 \cdot 50 \\ 
   &= 3 \cdot 2 \cdot 5 \ \cdot 5 \\
\end{aligned}
\]
&  
\[
\begin{aligned}
150 &= 5 \cdot 30 \\
   &= 5 \cdot 2 \cdot 3 \cdot 5 \\
\end{aligned}
\]
\end{tabular} 
\end{enumerate}
\hbreak

\end{document}


\subsection*{Beginning Activity 1 (Linear Combinations of Integers)}
\begin{multicols}{2}
\begin{enumerate}
\item $\gcd \left( {20,\;12} \right) = 4$.	
\item One possibility:  $4 = 20\left( { - 1} \right) + 12 \cdot 2$.
\end{enumerate}
\end{multicols}

\begin{enumerate} \setcounter{enumi}{2}
\item Some examples are:
\begin{tabular}[t]{| c | c | c | c |} \hline
$x$  &  $y$  &  $ax + by$  &  Does $d$ divide $ax + by$ \\ \hline
1  &  1  &  32  &  yes \\ \hline
1  &  $-1$  &  8  &  yes  \\ \hline
2  &  2  &  64  &  yes  \\ \hline
2  &  $-3$  &  4  &  yes  \\ \hline
$-2$  &  $-5$  &  $-100$  &  yes  \\ \hline
\end{tabular}

\item Some examples are:
\begin{tabular}[t]{| c | c | c | c |} \hline
$x$  &  $y$  &  $ax + by$  &  Does $d$ divide $ax + by$ \\ \hline
1  &  1  &  15  &  yes \\ \hline
1  &  $-1$  &  27  &  yes  \\ \hline
2  &  2  &  30  &  yes  \\ \hline
2  &  $-3$  &  60  &  yes  \\ \hline
$-2$  &  $-5$  &  $-12$  &  yes  \\ \hline
\end{tabular}

\textbf{Proposition 4.15}  \emph{Let a, b, and  t  be integers with $t \ne 0$.  If  t  divides  a  and  t  divides  b, then for all integers  x  and  y,  t  divides  
\text{(}ax + by\text{)}.}

\begin{myproof}
Let $a$, $b$, and  $t$  be integers with $t \ne 0$, and assume that $t$  divides  $a$  and  $t$  divides  $b$.  We will prove that for all integers  $x$  and  $y$,  $t$  divides  $ax + by$.

So, let  $x \in \mathbb{Z}$ and let  $y \in Z$.  Since  $t$  divides  $a$, there exists an integer  $m$  such that
\begin{equation} \label{eqans-821}
a = tm.
\end{equation}
In addition,  $t$  divides  $b$.  Therefore, there exists an integer  $n$  such that
\begin{equation} \label{eqans-822}
b = tn.
\end{equation}
Now, if  $x$  and  $y$  are any integers, we can use Equations~(\ref{eqans-821}) 
and~(\ref{eqans-822})  and obtain:
\[
\begin{aligned}
  ax + by &= \left( {tm} \right)x + \left( {tn} \right)y \\ 
          &= t \left( {mx + ny} \right). \\ 
\end{aligned} 
\]
By the closure properties of the integers,  $mx + ny$ is an integer, and hence this last equation proves that  $t$  divides  $ax + by$.  Hence, we have proven that if  $t$  divides  $a$  and  $d$  divides  $b$, then for all integers  $x$  and  $y$,  $t$  divides  $ax + by$.
\end{myproof}
\end{enumerate}
\hbreak

\newpage
