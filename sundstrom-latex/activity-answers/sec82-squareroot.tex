\documentclass[11pt]{article}
\usepackage{c://pctex/activity}

\lhead{}
\chead{\textbf{\large{Exercise 19 -- Section 8.2\\Square Roots and Irrational Numbers}}}
\rhead{}
\lfoot{\emph{Mathematical Reasoning: Writing and Proof, Third Ed.} \\Ted Sundstrom}
\cfoot{}
\rfoot{\copyright \the\year\, by Pearson Education, Inc.\\}


\begin{document}
\begin{enumerate}
\item If $n$ is a composite number, the according to the Fundamental Theorem of Arithmetic, $n$ has a unique factorization as a product of prime numbers.  By grouping the equal prime numbers in this factorization together, we can use exponents and say that there exist prime numbers 
$p_1, p_2, \ldots, p_r$ and natural numbers 
$\alpha_1, \alpha_2, \ldots, \alpha_r$ such that
\begin{equation}\label{A:squarerootsirrational1}
n = p_1^{\alpha_1} p_2^{\alpha_2} \cdots p_r^{\alpha_r} \notag.
\end{equation} 
Notice that if $r = 1$ and $\alpha_1 = 1$, then $n = P_1^{\alpha_1} = p_1$ and so $n$ is a prime number.  This means that any natural number can be written in the form given in 
equation~(\ref{A:squarerootsirrational1}).


\item  The numbers 36, 100, and 1764 are perfect squares since $36 = 6^2$, $400 = 20^2$, and
$15876 = 126^2$.  The prime factorizations of these numbers are:
\begin{align*}
36 &= 2^2 3^2  &  400 &= 2^4 5^2  &  15876 &= 2^2 3^4 7^2 
\end{align*}
In each prime factorization, notice that every exponent is even.



\item \textbf{Proposition}. \emph{Let $n$ be a natural number with 
$n = p_1^{\alpha_1} p_2^{\alpha_2} \cdots p_r^{\alpha_r}$, where $p_1, p_2, \ldots, p_r$ are prime numbers and $\alpha_1, \alpha_2, \ldots, \alpha_r$ are natural numbers.  The natural number $n$ is a perfect square if and only if for each natural number $k$ with $1 \leq k \leq r$, $\alpha_k$ is even}.

\begin{myproof}
Let $n$ be a natural number and assume 
\begin{equation}\label{nfactor}
n = p_1^{\alpha_1} p_2^{\alpha_2} \cdots p_r^{\alpha_r},
\end{equation} where $p_1, p_2, \ldots, p_r$ are prime numbers and $\alpha_1, \alpha_2, \ldots, \alpha_r$ are natural numbers.

We first assume that $n$ is a perfect square.  This means that there exists a natural number $m$ such that $n = m^2$.  We now write the prime factorization of $m$.  So there exist prime numbers $q_1, q_2, \ldots q_s$ and natural numbers $\beta_1, \beta_2, \ldots \beta_s$ such that 
$m = q_1^{\beta_1} q_2^{\beta_2} \cdots q_s^{\beta_s}$.  Then,
\begin{equation}\label{mfactor}
n = m^2 = q_1^{2 \beta_1} q_2^{2 \beta_2} \cdots q_s^{2\beta_s}.
\end{equation}
Since prime factorizations are unique, by comparing equations~(\ref{nfactor}) 
and~(\ref{mfactor}), we can conclude that $r = s$ and for each natural number $k$ with 
$1 \leq k \leq r$, $p_k = q_k$ and $\alpha_k = 2 \beta_k$.  This implies that in the factorization in~(\ref{nfactor}), $\alpha_k$ is even for each $k$ with $1 \leq k \leq r$.

We now assume that for each natural number $k$ with $1 \leq k \leq r$, $\alpha_k$ is even.  Hence, for each $k$, there exists a natural number $\beta_k$ such that 
$\alpha_k = 2 \beta_k$.  So we can write
\begin{align*}
n &= p_1^{2 \beta_1} p_2^{2 \beta_2} \cdots p_r^{2 \beta_r} \\
  &= \left( p_1^{\beta_1} p_2^{\beta_2} \cdots p_r^{\beta_r} \right)^2
\end{align*}
The last equation shows that $n$ is a perfect square.  Hence, we have proven that the natural number $n$ is a perfect square if and only if for each natural number $k$ with $1 \leq k \leq r$, $\alpha_k$ is even.
\end{myproof}

\newpage
\item \textbf{Proposition}. \emph{For each natural number $n$, if $n$ is not a perfect square, then $\sqrt{n}$ is an irrational number}.

\begin{myproof}
We will use a proof by contradiction.  So we assume that $n$ is a natural number, that $n$ is not a perfect square, and that $\sqrt{n}$ is a rational number.  This means that there exist integers $a$ and $b$ with $b > 0$ such that
\[
\sqrt{n} = \frac{a}{b}.
\]
If we square both sides of this equation and then multiply both sides by $b^2$, we obtain
\begin{equation}\label{contra}
b^2 n = a^2
\end{equation}
We now consider the primes factorizations of $b^2 n$ and $a^2$.  
\begin{itemize}
\item The exponents of the prime factors in the prime factorization of $a^2$ will all be even.
\item There will be at least one odd exponent in the prime factorization of $n$ since $n$ is not a perfect square.  The exponents of the prime factors in the prime factorization of $b^2$ will all be even.  Hence, there will be at least one odd exponent in the prime factorization of 
$b^2 n$.
\end{itemize}
This is a contradiction to the uniqueness of the prime factorization of $b^2 n = a^2$.  Hence, we have proven that if $n$ is not a perfect square, then $\sqrt{n}$ is an irrational number.
\end{myproof}

\end{enumerate}

\end{document}
