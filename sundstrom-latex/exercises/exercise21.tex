\section*{Exercises for Section~\ref{S:logop}}
\begin{enumerate}
\xitem \label{exer:sec22-1}%
Suppose that Daisy says, ``If it does not rain, then I will play golf.''  Later in the day you come to know that it did rain but Daisy still played golf.  Was Daisy's statement true or false?  Support your conclusion. 

%\pagebreak
\xitem\label{exer:sec22-2}%
Suppose that  $P$  and  $Q$  are statements for which  $P \to Q$ is true and  for which 
$\mynot Q$ is true.  What conclusion (if any) can be made about the truth value of each of the following statements?
%$$
%\BeginTable
%\BeginFormat
%| p(1.3in) | p(1.3in) | p(1.3in) |
%\EndFormat
%" \textbf{(a)} $P$ " \textbf{(b)} $P \wedge Q$ " \textbf{(c)} $P \vee Q$ " \\
%\EndTable
%$$
\begin{multicols}{3}
\begin{enumerate}
    \item $P$
    \item $P \wedge Q$
    \item $P \vee Q$
\end{enumerate}
\end{multicols}

\item \label{exer:sec22-3}%
Suppose that  $P$  and  $Q$  are statements for which  $P \to Q$ is false. 
  What conclusion (if any) can be made about the truth value of each of the following statements?

\begin{multicols}{3}
\begin{enumerate}
    \item $\mynot  P \to Q$
    \item $Q \to P$
    \item $P \vee Q$
\end{enumerate}
\end{multicols}


\item \label{exer:statements4}%
Suppose that $P$ and $Q$ are statements for which $Q$ is false and $\mynot P \to Q$ is true (and it is not known if $R$ is true or false).  What conclusion (if any) can be made about the truth value of each of the following statements? 

\begin{multicols}{2}
\begin{enumerate}
\item $\mynot Q \to P$

\item $P$

\yitem $P \wedge R$

\item $R \to \mynot P$
\end{enumerate}
\end{multicols}


%\item Suppose that $P$ and $Q$ are statements for which $Q$ is false and $\mynot P \to Q$ is true (and it is not known if $R$ is true or false).  What conclusion (if any) can be made about the truth value of each of the following statements?
%
%\begin{multicols}{2}
%\begin{enumerate}
%\item $\mynot Q \to P$
%
%\item $P$
%
%\item $P \wedge R$
%
%\item $R \to \mynot P$
%\end{enumerate}
%\end{multicols}



\xitem \label{exer:sec22-5}%
Construct a truth table for each of the following statements: 
\begin{multicols}{2}
  \begin{enumerate}
    \item $P \to Q$
    \item $Q \to P$
    \item $\mynot  P \to \mynot  Q$
    \item $\mynot  Q \to \mynot  P$
  \end{enumerate}
  \end{multicols}
  Do any of these statements have the same truth table?

\item Construct a truth table for each of the following statements: \label{exer:sec22-4}
  \begin{multicols}{2}
  \begin{enumerate}
    \item $P \vee \mynot  Q$
    \item $\mynot  \left( {P \vee Q} \right)$
    \item $\mynot  P \vee \mynot  Q$
    \item $\mynot  P \wedge \mynot  Q$
  \end{enumerate}
  \end{multicols}
  Do any of these statements have the same truth table?


\xitem \label{exer:sec22-6}%
Construct truth tables for  $P \wedge \left( {Q \vee R} \right)$  and  $\left( {P \wedge Q} \right) \vee \left( {P \wedge R} \right)$.  What do you observe? 
%
\item \label{exer:sec22-7}%
Suppose each of the following statements is true.  
  \begin{itemize}
    \item Laura is in the seventh grade.
    \item Laura got an A on the mathematics test or Sarah got an A on the mathematics test.
    \item If Sarah got an A on the mathematics test, then Laura is not in the seventh grade.
  \end{itemize}
If possible, determine the truth value of each of the following statements.  Carefully explain your reasoning.
  \begin{enumerate}
    \item Laura got an A on the mathematics test.
    \item Sarah got an A on the mathematics test.
    \item Either Laura or Sarah did not get an A on the mathematics test.
  \end{enumerate}
%
\item Let  $P$  stand for  ``the integer  $x$  is even,'' and let  $Q$  stand for ``$x^2$  is even.''  Express the conditional statement  $P \to Q$ in English using
  \begin{enumerate}
    \item The ``if then'' form of the conditional statement
    \item The word ``implies''
    \yitem The ``only if'' form of the conditional statement
    \yitem The phrase ``is necessary for''
    \item The phrase ``is sufficient for''
  \end{enumerate}  \label{exer:sec22-8}%
%
\item \label{exer:sec22-9}%
Repeat Exercise~(\ref{exer:sec22-8}) for the conditional statement  $Q \to P$. 



\xitem For statements $P$ and $Q$, use truth tables to determine if each of the following statements is a tautology, a contradiction, or neither. \label{exer:tautology-contra}
\begin{multicols}{2}
\begin{enumerate}
  \item $\mynot Q \vee (P \to Q)$.
  \item $Q \wedge (P \wedge \mynot Q)$.
  \item $(Q \wedge P) \wedge (P \to \mynot Q)$.
  \item $\mynot Q \to (P \wedge \mynot P)$.
\end{enumerate}
\end{multicols}



\item For statements $P$, $Q$, and $R$:
\begin{enumerate}
  \item Show that 
$\left[ (P \to Q) \wedge P \right] \to Q$ is a tautology.  \note In symbolic logic, this is an important logical argument form called \textbf{modus ponens}.
\index{modus ponens}%
  \item Show that $\left[ (P \to Q) \wedge (Q \to R) \right] \to (P \to R)$ is a tautology.  \note In symbolic logic, this is an important logical argument form called \textbf{syllogism}.
\index{syllogism}%
\end{enumerate}
\end{enumerate}



\subsection*{Explorations and Activities}
\setcounter{oldenumi}{\theenumi}
\begin{enumerate} \setcounter{enumi}{\theoldenumi}
  \item \textbf{Working with Conditional Statements}.  Complete the following table: \label{exer:working-conditional}
$$
\BeginTable
\BeginFormat
|l|c|c|c|
\EndFormat
  \_
 | \textbf{English Form}  |  \textbf{Hypothesis}  |  \textbf{Conclusion} |  \textbf{Symbolic Form}  | \\+22  \_
 | If $P$, then $Q$.         |  $P$  |  $Q$  |  $P \to Q$  | \\ \_1
|  $Q$ only if $P$.          |  $Q$  |  $P$  |  $Q \to P$  | \\ \_1
|  $P$ is necessary for $Q$. |       |       |             | \\ \_1
|  $P$ is sufficient for $Q$. |      |       |             | \\ \_1
|  $Q$ is necessary for $P$.  |       |       |            |  \\ \_1
|  $P$ implies $Q$.           |       |       |            |  \\ \_1
|  $P$ only if $Q$.           |       |       |            |  \\ \_1
|  $P$ if $Q$.                |       |       |             | \\ \_1
|  If $Q$ then $P$.           |       |       |             | \\ \_1
|  If  $\neg  Q$, then $\neg  P$. |  |   |             | \\ \_1
|  If $P$, then $Q \wedge R$. |       |       |            | \\ \_1
|  If $P \vee Q$, then $R$.   |       |       |            | \\ \hline
\EndTable
$$


\item \textbf{Working with Truth Values of Statements}.  Suppose that $P$ and $Q$ are true statements, that $U$ and $V$ are false statements, and that $W$ is a statement and it is not known if $W$ is true or false.
\label{exer:working-truth}

\newpar
Which of the following statements are true, which are false, and for which statements is it not possible to determine if it is true or false?  Justify your conclusions.
\begin{multicols}{2}
\begin{enumerate}
  \item $(P \vee Q) \vee (U \wedge W)$
  \item $P \wedge (Q \to W)$
  \item $P \wedge (W \to Q)$
  \item $W \to (P \wedge U)$
  \item $W \to (P \wedge \mynot U)$
  \item $(\mynot P \vee \mynot U) \wedge (Q \vee \mynot V)$
  \item $(P \wedge \mynot V) \wedge (U \vee W)$
  \item $(P \vee \mynot Q) \to (U \wedge W)$
  \item $(P \vee W) \to (U \wedge W)$
  \item $(U \wedge \mynot V) \to (P \wedge W)$
\end{enumerate}
\end{multicols}

\end{enumerate}
\hbreak

\endinput
