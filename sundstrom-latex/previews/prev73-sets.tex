\begin{previewactivity}[\textbf{Sets Associated with a Relation}] \label{PA:setswithrelation} \hfill \\
As was indicated in Section~\ref{S:equivrelations}, an equivalence relation on a set $A$ is a relation with a certain combination of properties (reflexive, symmetric, and transitive) that allow us to sort the elements of the set into certain classes.  This is done by means of certain subsets of $A$ that are associated with the elements of the set $A$.  This will be illustrated with the following example.  Let 
$A = \left\{ {a, b, c, d, e} \right\}$, and  let $R$ be the relation on the set $A$ defined as follows:  %For each $x \in A$, $x \mathrel{R} x$ and
\begin{align*}
a &\mathrel{R} a  &b &\mathrel{R} b  &c &\mathrel{R} c  &d &\mathrel{R} d   &e &\mathrel{R} e \\
a &\mathrel{R} b  &b &\mathrel{R} a  &b &\mathrel{R} e  &e &\mathrel{R} b    \\
a &\mathrel{R} e  &e &\mathrel{R} a  &c &\mathrel{R} d  &d &\mathrel{R} c 
\end{align*}
For each $y \in A$, define the subset $R[y]$ of $A$ as follows:
\[
R[y] = \left\{ x \in A \mid x \mathrel{R} y \right\}.
\]
That is, $R[y]$ consists of those elements in $A$ such that $x \mathrel{R} y$.  For example, using $y = a$, we see that $a \mathrel{R} a $, $b \mathrel{R} a$, and $e \mathrel{R} a$,  and so $R[a] = \{ a, b, e \}$.
\begin{enumerate}
  \item Determine $R[b]$, $R[c]$, $R[d]$, and $R[e]$.
  \item Draw a directed graph for the relation $R$ and explain why $R$ is an equivalence relation on $A$.
  \item Which of the sets  $R[ a ]$, $R[ b ]$, $R[ c ]$, $R[ d ]$,  and $R[ e ]$ are equal?
  \item Which of the sets  $R[ a ]$, $R[ b ]$, $R[ c ]$, $R[ d ]$,  and $R[ e ]$ are disjoint?
\end{enumerate}
As we will see in this section, the relationships between these sets are typical for an equivalence relation.  The following example will show how different this can be for a relation that is not an equivalence relation.

\noindent
Let  $A = \left\{ {a, b, c, d, e} \right\}$, and  let $S$ be the relation on the set $A$ defined as follows:
\begin{align*}
b &\mathrel{S} b  &c &\mathrel{S} c  &d &\mathrel{S} d   &e &\mathrel{S} e \\
a &\mathrel{S} b  &a &\mathrel{S} d  &b &\mathrel{S} c      \\
c &\mathrel{S} d  &d &\mathrel{S} c   
\end{align*}
%
%
%\[
%S = \left\{ { ( {b, b} ), ( {c, c} ), ( {d, d} ), ( {e, e} ), ( {a, b} ), ( {b, c} ), ( {a, d} ), ( {c, d} ), ( {d, c} )} \right\}.
%\]

\setcounter{oldenumi}{\theenumi}
\begin{enumerate} \setcounter{enumi}{\theoldenumi} 
\item Draw a digraph that represents the relation  $S$  on  $A$.  Explain why  $S$  is not an equivalence relation on  $A$.
\end{enumerate}
For each $y \in A$, define the subset $S[y]$ of $A$ as follows:
\[
S[ y ] = \left\{ { {x \in A } \mid x \mathrel{S} y} \right\} = \left\{ { {x \in A } \mid \left( {x, y} \right) \in S} \right\} .
\]
For example, using $y = b$, we see that  $S[b] = \{ a, b \}$ since $(a, b) \in S$ and $(b, b) \in S$.  In addition, we see that $S[a] = \emptyset$ since there is no $x \in A$ such that $(x, a) \in S$.
\setcounter{oldenumi}{\theenumi}
\begin{enumerate} \setcounter{enumi}{\theoldenumi} 
\item Determine  $S[ c ]$, $S[ d ]$, and $S[ e ]$.
\item Which of the sets  $S[ a ]$, $S[ b ]$, $S[ c ]$, $S[ d ]$, and $S[ e ]$ are equal?
\item Which of the sets  $S[ b ]$, $S[ c ]$, $S[ d ]$, and $S[ e ]$ are disjoint?
\end{enumerate}

\end{previewactivity}
\hbreak

\endinput
