\section*{Section \ref{S:moremethods}}
\renewcommand{\labelenumi}{(\textbf{\alph{enumi}})}

\begin{list}{\bf{\ref{exer:ncubed}.}}
\item \begin{list}{\bf{(a)}}
\item Let  $n$  be an even integer.  Since  $n$  is even, there exists an integer  $k$ such that
$n = 2k$. Now use this to prove that $n^3$   must be even.
\end{list}
\end{list}
%
\begin{list}{}
\item \begin{list}{\bf{(b)}}
\item Prove the contrapositive.
\end{list}
\end{list}
%
\begin{list}{}
\item \begin{list}{\bf{(c)}}
\item Explain why Parts (a) and (b) prove this.
\end{list}
\end{list}
%
\begin{list}{}
\item \begin{list}{\bf{(d)}}
\item Explain why Parts (a) and (b) prove this.
\end{list}
\end{list}

\vskip9pt
\begin{list}{\bf{\ref{exer:sec32-2}.}}
\item \begin{list}{\bf{(a)}}
\item The contrapositive is,  For all integers $a$  and  $b$, if  $\mod{ab}{0}{6}$
, then  $\mod{a}{0}{6}$  or   $\mod{b}{0}{6}$.
\end{list}
\end{list}

\begin{list}{\bf{3}.}
\item \begin{enumerate}
\item The contrapositive is:  For all positive real numbers $a$ and $b$,  if $a = b$, then 
$\sqrt{ab} = \dfrac{a + b}{2}$.

\item The statement is true.  If $a = b$, then $\dfrac{a + b}{2} = \dfrac{2a}{2} = a$, and \\
$\sqrt{ab} = \sqrt{a^2} = a$.  This proves the contrapositive.
\end{enumerate}
\end{list}


\begin{list}{\bf{\ref{exer:sec32-4}.}}
\item \begin{enumerate}  
\item True.  If $a \equiv 2 \pmod 5$, then there exists an integer $k$ such that $a - 2 = 5k$.  
Then, 
\[
a^2 - 4  = \left( 2 + 5k \right)^2 - 4  = 20k + 25k^2.  
\]
This means that $a^2 - 4 = 5 \left( 4k + 5k^2 \right)$, and hence, $a^2 \equiv 4 \pmod 5$.

\item False.  A counterexample is $a = 3$ since $\mod{3^2}{4}{5}$ and $\notmod{3}{2}{5}$.

\item False.  Part~(b) shows this is false.
\end{enumerate}
\end{list}


\begin{list}{\bf{\ref{exer:sec32-congmod7}.}}
%\item One of the two conditional statements is true and one is false.
\item \begin{enumerate}
\item For each integer $a$, if $a \equiv 3 \pmod 7$, then $(a^2 + 5a) \equiv 3 \pmod 7$, and for each integer $a$, if $(a^2 + 5a) \equiv 3 \pmod 7$, then $a \equiv 3 \pmod 7$.

\item For each integer $a$, if $a \equiv 3 \pmod 7$, then $(a^2 + 5a) \equiv 3 \pmod 7$ is true.  To prove this, if $a \equiv 3 \pmod 7$, then there exists an integer $k$ such that 
$a = 3 + 7k$.  We can then prove that
\[
\left( a^2 + 5a \right) - 3 = 21 + 77k + 49k^2 = 7(3 + 11k + 7k^2).
\]
This shows that $(a^2 + 5a) \equiv 3 \pmod 7$.

For each integer $a$, if $(a^2 + 5a) \equiv 3 \pmod 7$, then $a \equiv 3 \pmod 7$ is false.  A counterexample is $a = 6$.  When $a = 6$, $a^2 + 5a = 66$ and 
$66 \equiv 3 \pmod 7$ and $6 \not \equiv 3 \pmod 7$.

\item Since one of the two conditional statements in Part~(b) is false, the given proposition is false.
\end{enumerate}

\end{list}

\begin{list}{\bf{\ref{exer:sec32-6}.}}
\item Prove both of the conditional statements:  (1) If the area of the right triangle is 
$c^2/4$, then the right triangle is an isosceles triangle.  (2)  If the right triangle is an isosceles triange, then the area of the right triangle is $c^2/4$.
\end{list}


\begin{list}{\bf{\ref{exer:sec32-rational}.}}
\item The statement is true.  It is easier to prove the contrapositive, which is:
\begin{list}{}
\item For each positive real number $x$, if $\sqrt{x}$ is rational, then $x$ is rational.
\end{list}
Let $x$ be a positive real number.  If there exist positive integers $m$ and $n$ such that $\sqrt{x} = \dfrac{m}{n}$, then 
$x = \dfrac{m^2}{n^2}$.
\end{list}


\begin{list}{\bf{\ref{exer:sec32-8}.}}
\item Remember that there are two conditional statements associated with this biconditional statement.  Be willing to consider the contrapositive of one of these conditional statements.
\end{list}


\begin{list}{\bf{\ref{exer:IVT}.}}
\item Define an appropriate function and use the Intermediate Value Theorem.
\end{list}


\begin{list}{\bf{\ref{exer:sec32-16}.}}
\item \begin{list}{\bf{(b)}}
\item Since 4 divides $a$, there exist an integer $n$ such that $a = 4n$.  Using this, we see that $b^3 = 16n^2$. This means that $b^3$ is even and hence by Exercise~(1), $b$ is even.   So there exists an integer $m$ such that $b = 2m$.  Use this to prove that $m^3$ must be even and hence by Exercise~(1), $m$ is even.
\end{list}
\end{list}

\begin{list}{\bf{\ref{exer:sec32-equation17}.}}
\item It may be necessary to factor a sum  of cubes.  Recall that 
\[
u^3 + v^3 = ( u + v ) ( u^2 - uv + v^2 ).
\]
\end{list}

\hbreak
\renewcommand{\labelenumi}{\textbf{\arabic{enumi}.}}

\endinput
