\section*{Section \ref{S:uncountablesets} }

\subsection*{Progress Check~\ref{prog:diagonal}}
Player Two has a winning strategy.  On the $k$th turn, whatever symbol Player One puts in the 
$k$th position of the $k$th row, Player Two must put the other symbol in the $k$th position of his or her row.  This guarantees that the row of symbols produced by Player Two will be different than any of the rows produced by Player One.

This is the same idea used in Cantor's Diagonal Argument.  Once we have a ``list'' of real numbers in normalized form, we create a real number that is not in the list by making sure that its 
$k$th decimal place is different than the $k$th decimal place for the $k$th number in the list.  The one complication is that we must make sure that our new real number does not have a decimal expression that ends in all 9's.  This was done by using only 3's and 5's.


\subsection*{Progress Check~\ref{prog:openintervals}}
\begin{enumerate}
  \item \begin{myproof}
In order to find a bijection $f\x  ( 0, 1 ) \to ( a, b )$, we will use the linear function through the points $( 0, a )$ and $( 1, b )$.  The slope is 
$( b - a)$ and the $y$-intercept is $( 0, a )$. So define 
$f\x  ( 0, 1 ) \to ( a, b )$ by
\begin{center}
$f ( x ) = ( b - a )x + a$, for each $x \in ( 0, 1 )$.
\end{center}
Now, if $x, t \in ( 0, 1 )$ and $f ( x ) = f ( t )$, then 
\[
( b - a )x + a = ( b - a )t + a.
\]
This implies that $( b - a )x = ( b - a )t$, and since $b - a \ne 0$, we can conclude that $x = t$.  Therefore, $f$ is an injection.

To prove that $f$ is a surjection, we let $y \in ( a, b )$.  If 
$x = \dfrac{y - a}{b - a}$, then
\[
\begin{aligned}
f ( x ) &= f \!\left( \frac{y - a}{b - a} \right) \\
                   &= ( b - a ) \!\left( \frac{y - a}{b - a} \right) + a \\
                   &= ( y - a ) + a \\
                   &= y
\end{aligned}
\]
This proves that $f$ is a surjection.  Hence, $f$ is a bijection and 
$( 0, 1 ) \approx ( a, b )$.  Therefore, $( a, b )$ is uncountable and has cardinality $\boldsymbol{c}$.
\end{myproof}

\item Now, if $a, b, c, d$ are real numbers with $a < b$ and $c < d$, then we know that

\begin{center}
$( a, b ) \approx ( 0, 1 )$ and $( c, d ) \approx ( 0, 1 )$.
\end{center}

\noindent
Since $\approx$ is an equivalence relation,we can conclude that 
$( a, b ) \approx ( c, d )$.
\end{enumerate}
\hbreak

\endinput
