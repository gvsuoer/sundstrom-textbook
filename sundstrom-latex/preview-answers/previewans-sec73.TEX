\documentclass[11pt]{article}
\usepackage{../../styles/activity}

\usepackage{xr}
\externaldocument{0-MR}

\lhead{}
%\chead{\textbf{\Large{\hspace{0pt}Beginning Activities for Section~7.3}}\\\hspace{0pt}\emph{Mathematical Reasoning: Writing and Proof}}
\bahead{7.3}
\rhead{}
\lfoot{}
\rfoot{}
\cfoot{\hspace{0pt}\scalebox{0.4}{\includegraphics{cc-by-nc-sa.eps}}}
\graphicspath{{./epsfigs/}}

\begin{document}
\subsection*{Beginning Activity 1 (Sets Associated with a Relation)}
\begin{enumerate}
  \item \begin{multicols}{4}
  $R[b] = \{ a, b, e \}$

  $R[c] = \{ c, d \}$

  $R[d] = \{ c, d \}$

  $R[e] = \{ a, b, e \}$.
\end{multicols}
\item %The directed graph has loops at each of the five vertices, an arrow from $a$ to $b$, an arrow from $a$ to $d$, an arrow from $b$ to $c$, an arrow from $c$ to $d$, and an arrow from $d$ to $c$.
By inspection of the directed graph, we see that the relation  $R$  is reflexive, symmetric, and transitive.  Therefore,  $R$  is an equivalence relation on  $A$.

\item It can be seen that $R[a] = R[b] = R[e]$ and $R[c] = R[d]$.

\item We also see that $R[a] \cap R[c] = \emptyset$.  Because of the set equalities in~(3), we also know that 
$R[a] \cap R[d] = \emptyset$, $R[b] \cap R[c] = \emptyset$, $R[b] \cap R[d] = \emptyset$, $R[e] \cap R[c] = \emptyset$, and 
$R[e] \cap R[d] = \emptyset$.

\item The directed graph has loops at vertices $b$, $d$, and $e$, an arrow from $a$ to $b$, an arrow from $b$ to $c$, an arrow from $a$ to $d$, an arrow from $c$ to $d$, and an arrow from $d$ to $c$.

By inspection, we see that the relation  $S$  is not reflexive, not symmetric, and not transitive.  Therefore,  $S$  is not an equivalence relation on  $A$.

%\item \begin{enumerate}
%\item \begin{multicols}{3}
%$R\left[ a \right] = \left\{ {a, b} \right\}$
%
%$R\left[ b \right] = \left\{ {a, b} \right\}$
%
%$R\left[ c \right] = \left\{ {c, d} \right\}$
%
%$R\left[ d \right] = \left\{ {c, d} \right\}$
%
%$R\left[ e \right] = \left\{ e \right\}$
%\end{multicols}
%
%\item $R\left[ a \right] = R\left[ b \right]\text{ and }R\left[ c \right] = R\left[ d \right]$.
%
%\item $R\left[ a \right]$ and $R\left[ c \right]$ are disjoint, $R\left[ a \right]$ and 
%$R\left[ e \right]$ are disjoint, and $R\left[ c \right]$ and $R\left[ e \right]$ are disjoint.
%\end{enumerate}

\item \begin{multicols}{4}
%$S\left[ a \right] = \emptyset $

$S\left[ b \right] = \left\{ {a, b} \right\}$

$S\left[ c \right] = \left\{ {b, c, d} \right\}$

$S\left[ d \right] = \left\{ {a, c, d} \right\}$

$S\left[ e \right] = \left\{ e \right\}$
\end{multicols}

\item None of the sets are equal.

\item $S\left[ a \right]$ and each of the other sets are disjoint. $S\left[ e \right]$ and each of the other sets are disjoint.

\end{enumerate}
\hbreak

\noindent
\subsection*{Beginning Activity 2 (Congruence Modulo 3)}
\begin{enumerate}
\item \begin{multicols}{2}
$C\left[ 0 \right] = \left\{ { \ldots ,  - 9,  - 6,  - 3, 0, 3, 6, 9,  \ldots } \right\}$	

$C\left[ 1 \right] = \left\{ { \ldots ,  - 8,  - 5,  - 2, 1, 4, 7, 10,  \ldots } \right\}$

$C\left[ 2 \right] = \left\{ { \ldots ,  - 7,  - 4,  - 1, 2, 5, 8, 11,  \ldots } \right\}$	

$C\left[ 3 \right] = \left\{ { \ldots ,  - 9,  - 6,  - 3, 0, 3, 6, 9,  \ldots } \right\}$
\end{multicols}

\item \begin{enumerate}
\item The intersection of any two of the sets, $C\left[ 0 \right], C\left[ 1 \right], C\left[ 2 \right]$, is the empty set.

\item Since $734 = 244 \cdot 3 + 2$, we know that  $734 \equiv 2 \pmod 3$, and  
$734 \in C\left[ 2 \right]$.

\item Since $79 = 26 \cdot 3 + 1$, we know that $79 \equiv 1 \pmod 3$, and  
$79 \in C\left[ 1 \right]$.

Since $ - 79 = \left( { - 27} \right) \cdot 3 + 2$,  we know that$ - 79 \equiv 2 \pmod 3$, and  $ - 79 \in C\left[ 2 \right]$.

\item Since each integer is congruent to  0, 1, or  2  modulo 3, we know that each integer is in precisely one of the three sets  $C\left[ 0 \right]$, $C\left[ 1 \right]$, or 
$C\left[ 2 \right]$.  Hence, we may conclude that 
$C\left[ 0 \right] \cup C\left[ 1 \right] \cup C\left[ 2 \right] = \mathbb{Z}$.

\item $C[3] = C[0]$.
\item $C[4] = C[1]$.
\end{enumerate}

\end{enumerate}
\hbreak



\end{document}
