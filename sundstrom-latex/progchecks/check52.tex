\section*{Section~\ref{S:provingset}}
\subsection*{Progress Check~\ref{prog:setequality}}
$A = \left\{ x \in \mathbb{Z} \mid x \text{  is a multiple of  9} \right\}$ and  
$B = \left\{ x \in \mathbb{Z} \mid  x \text{  is a multiple of  3} \right\}$.
\begin{enumerate}
\item The set $A$ is a subset of $B$.  To prove this, we let $x \in A$.  Then there exists an integer $m$ such that $x = 9m$, which can be written as
\[
x = 3 \left( 3m \right)\!.
\]
Since $3m \in \Z$, the last equation proves that $x$ is a multiple of 3 and so $x \in B$.  Therefore, $A \subseteq B$.

\item The set $A$ is not equal to the set $B$.  We note that $3 \in B$ but $3 \notin A$.  Therefore, $B \not \subseteq A$ and, hence, $A \ne B$.
\end{enumerate}



\subsection*{Progress Check \ref{prog:usingchoose}}
$$
\BeginTable
\def\C{\JustCenter}
\BeginFormat
|p(0.4in)|p(2in)|p(1.8in)|
\EndFormat
\_
  | \textbf{Step}  |  \textbf{Know}  |  \textbf{Reason}  |    \\+02 \_
  | $P$     |  $A \subseteq B$    |  Hypothesis | \\ \_1
  | $P1$    |   Let  $x \in B^c $.  |  Choose an arbitrary element of  $B^c$. |    \\ \_1
  | $P2$  |  If  $x \in A$, then  $x \in B$.  |  Definition of ``subset'' | \\  \_1
  | $P3$  |  If $x \notin B$, then $x \notin A$.  |  Contrapositive | \\ \_1
  | $P4$  |  If $x \in B^c$, then $x \in A^c$.  |  Step $P3$ and definition of ``complement'' | \\ \_1
  | $Q2$  |  The element $x$ is in $A^c$.  | Steps $P1$ and $P4$ | \\ \_1
  | $Q1$    |   Every element of  $B^c $  is an element of  $A^c $.   |  The choose-an-element method with Steps $P1$ and $Q2$.  |  \\  \_1  
  | $Q$     |  $B^c \subseteq A^c$                     |  Definition of ``subset''  | \\ \_
%  \textbf{Step}  |  \textbf{Show}  |  \textbf{Reason}     \\ \hline
\EndTable
$$



\subsection*{Progress Check~\ref{prog:setequality2}}
\begin{myproof}
Let  $A$  and  $B$  be subsets of some universal set.  We will prove that $A - B = A \cap B^c$ by proving that each set is a subset of the other set.  We will first prove that $A - B \subseteq A \cap B^c$.  Let 
$x \in A - B$.  We then know that $x \in A$ and $x \notin B$.  However, $x \notin B$ implies that $x \in B^c$.  Hence, $x \in A$ and $x \in B^c$, which means that $x \in A \cap B^c$.  This proves that 
$A - B \subseteq A \cap B^c$.

To prove that $A \cap B^c \subseteq A - B$, we let $y \in A \cap B^c$.  This means that $y \in A$ and 
$y \in B^c$, and hence, $y \in A$ and $y \notin B$.  Therefore, $y \in A - B$ and this proves that $A \cap B^c \subseteq A - B$.  Since we have proved that each set is a subset of the other set, we have proved that 
$A - B = A \cap B^c$.
\end{myproof}



\subsection*{Progress Check~\ref{prog:disjointsets}}
\begin{myproof}
Let $A = \{ x \in \Z \mid \mod{x}{3}{12} \}$ and  $B = \{ y \in \Z \mid \mod{y}{2}{8} \}$.  We will use a proof by contradiction to prove that $A \cap B = \emptyset$.  So we assume that $A \cap B \ne \emptyset$ and let $x \in A \cap B$.  We can then conclude that $\mod{x}{3}{12}$ and that $\mod{x}{2}{8}$.  This means that there exist integers $m$ and $n$ such that
\[
x = 3 + 12m \quad \text{and} \quad x = 2 + 8n.
\]
By equating these two expressions for $x$, we obtain $3 + 12m = 2 + 8n$, and this equation can be rewritten as $1 = 8n - 12m$.  This is a contradiction since 1 is an odd integer and $8n - 12m$ is an even integer.  We have therefore proved that $A \cap B = \emptyset$.
\end{myproof}
\hbreak

\endinput
