\section*{Section \ref{S:mathinduction}}
\renewcommand{\labelenumi}{(\textbf{\alph{enumi}})}

\begin{list}{\bf{\ref{exer:sec51-1}.}}
\item The sets in Parts (a) and (b) are inductive.
\end{list}

\begin{list}{\bf{\ref{exer:sec51-2}.}}
\item A finite nonempty set is not inductive (why?) but the empty set is inductive (why?).
\end{list}

\begin{list}{\bf{\ref{exer:sec51-3}.}}
\item \begin{list}{\bf{(a)}}
\item For each  $n \in \mathbb{N}$, let  $P( n )$ be, 
$2 + 5 + 8 +  \cdots  + (3n - 1) = \dfrac{{n( {3n + 1} )}}{2}$.  When we use $n = 1$, the summation on the left side of the equation is 2, and the right side is $\dfrac{1(3 \cdot 1 + 1)}{2} = 2$.  Therefore, $P( 1 )$  is true.   For the inductive step, let $k \in \mathbb{N}$ and assume that 
$P \left( k \right)$ is true.  Then,
\[
  2 + 5 + 8 + \cdots  + \left( {3k - 1} \right) = \frac{{k \left( {3k + 1} \right)}}{2}. 
\]
We now add $3 \left( k + 1 \right) - 1$ to both sides of this equation.  This gives
\begin{align*}
  2 + 5 + 8 + \cdots  + (3k - 1) & \\
+ 3 \left( {k + 1} \right)-1 &= 
           \frac{{k \left( {3k + 1} \right)}}{2} + \left( 3 \left( k + 1 \right) - 1 \right) \\   
   &= \frac{{k \left( {3k + 1} \right)}}{2} + \left( {3k + 2} \right)
\end{align*}
If we now combine the terms on the right side of the equation into a single fraction, we obtain
\begin{align*} 
 2 + 5 +  \cdots  + \left( {3k - 1} \right) + 3 \left( {k + 1} \right)-1  &= \frac{k \left( 3k + 1 \right) + 6k+4}{2} \\
   &= \frac{ 3k^2 + 7k + 4}{2} \\
   &= \frac{ \left( k + 1 \right) \left( 3k + 4 \right)}{2} \\
   &= \frac{ \left( k + 1 \right) \left( 3 \left( k + 1 \right) + 1 \right)}{2}
\end{align*}
This proves that if  $P\left( k \right)$ is true, then $P\left( {k + 1} \right)$ is true.






%The key to the inductive step is that if  
%$P( k )$ is true, then 
%%\begin{align*}
%%  2 + 5 + 8 +  \cdots  + (3k - 1) + \left[3 ( {k + 1} ) - 1 \right]
%%   &= ( {2 + 5 +   \cdots  + (3k - 1)} ) + ( 3k + 2 )  \\
%%   &= \frac{{3k( {k + 1} )}}{2} + ( {3k + 2} ). \\ 
%%\end{align*}
%
%\begin{equation} \notag
%\begin{split}
%2 + 5 + 8 +  \cdots  + (3k - 1) + \\
%\left[3 ( {k + 1} ) - 1 \right]
% &= ( {2 + 5 +   \cdots  + (3k - 1)} ) + ( 3k + 2 )  \\
% &= \frac{{3k( {k + 1} )}}{2} + ( {3k + 2} ). \\ 
%\end{split}
%\end{equation}
%
%Now use algebra to show that the last expression can be rewritten as 
%$\dfrac{(k + 1)(3k + 4)}{2}$ and then explain why this completes the proof that if  
%$P( k )$ is true, then 
%$P( {k + 1} )$ is true.
\end{list}
\end{list}

\begin{list}{\bf{\ref{exer:sec51-5}.}}
\item The conjecture is that  for each  $n \in \mathbb{N}$,  $1 + 3 + 5 + \cdots + (2n - 1) = n^2$.
%$\sum\limits_{j = 1}^n {( {2j - 1} )}  = n^2 $.  
The key to the inductive step is that
%\begin{align*}
% \sum\limits_{j = 1}^{k + 1} {( {2j - 1} )}  &= \sum\limits_{j = 1}^k {( {2j - 1} )}  + \left[ {2( {k + 1} ) - 1} \right] \\ 
%   &= \sum\limits_{j = 1}^k {( {2j - 1} )}  + \left[ {2k + 1} \right]. \\ 
%\end{align*}
\begin{align*}
 1 + 3 + 5 + \cdots + (2k - 1) & \\
+ (2(k+1) - 1)  &= \left[1 + 3 + 5 + \cdots + (2k - 1) \right] + (2(k+1) - 1)\\ 
   &= k^2 + (2k + 1) \\
   &= (k + 1)^2 
\end{align*}
\end{list}

\begin{list}{\bf{\ref{exer:mod3conjecture}.}}
\item \begin{enumerate} \setcounter{enumi}{4}
\item  For each natural number $n$, $4^n \equiv 1 \pmod 3$.

\item For each natural number $n$, let $P(n)$ be, ``$4^n \equiv 1 \pmod 3$.''  Since 
$4^1 \equiv 1 \pmod 3$, we see that $P(1)$ is true.  Now let $k \in \N$ and assume that 
$P(k)$ is true.  That is, assume that
\[
4^k \equiv 1 \pmod 3.
\]
Multiplying both sides of this congruence by 4 gives
\[
4^{k+1} \equiv 4 \pmod 3.
\]
However, $4 \equiv 1 \pmod 3$ and so by using the transitive property of congruence, we see that 
$4^{k+1} \equiv 1 \pmod 3$.  This proves that if $P(k)$ is true, then $P(k+1)$ is true.
\end{enumerate}
\end{list}

\begin{list}{\bf{\ref{exer:sec51-6}.}}
\item \begin{list}{\bf{(a)}}
\item The key to the inductive step is that if $4^k = 1 + 3m$, then 
$4^k \cdot 4 = 4 ( 1 + 3m )$, which implies that
\[
4^{k+1} - 1 = 3 (1 + 4m ).
\]
\end{list}
\end{list}


\begin{list}{\bf{\ref{exer:sec51-cong}.}}
\item Let $k$ be a natural number.  If $a^k \equiv b^k \pmod n$, then since we are also assuming that $a \equiv b \pmod n$, we can use Part~(2) of Theorem~\ref{T:propsofcong} to conclude that 
$a \cdot a^k \equiv b \cdot b^k \pmod n$.
\end{list}

\begin{list}{\bf{\ref{exer:sec51-13}.}}
\item  Three consecutive natural numbers may be represent by $n$, $n + 1$, and $n + 2$, where $n$ is a natural number.  For the inductive step, think before you try to do a lot of algebra.    You should be able to complete a proof of the inductive step by expanding the cube of only one expression. 
\end{list}



\hbreak
\renewcommand{\labelenumi}{\textbf{\arabic{enumi}.}}


\endinput

\begin{list}{\ref{exer:sec51-11}.}
\item For the inductive step, the following trigonometric identities are useful:
\begin{list}{}
\item $\cos ( \alpha + \beta ) = \cos \alpha \cos \beta - \sin \alpha \sin \beta$.
\item $\sin ( \alpha + \beta ) = \sin \alpha \cos \beta + \cos \alpha \sin \beta$.
\end{list}
\end{list}

