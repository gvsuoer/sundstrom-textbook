\begin{previewactivity}[\textbf{Congruence Modulo 3}]\label{PA:congruencemodulo3} \hfill \\
%\enlargethispage{\baselineskip}
An important equivalence relation that we have studied is congruence modulo $n$ on the integers.  We can also define subsets of the integers based on congruence modulo $n$.  We will illustrate this with congruence modulo 3.  For example, we can define $C[0]$ to be the set of all integers $a$ that are congruent to 0 modulo 3.  That is,
\[
C[ 0 ] = \left\{ { {a \in \mathbb{Z} } \mid a \equiv 0 \pmod 3} \right\}\!.
\]
Since an integer $a$ is congruent to 0 modulo 3 if and only if 3 divides $a$, we can use the roster method to specify this set as follows:
\[
C[0] = \left\{  \ldots, -9, -6, -3, 0, 3, 6, 9, \ldots \right\}.
\]

\begin{enumerate}
\item Use the roster method to specify each of the following sets:

\begin{enumerate}
%\item The set  $C[ 0 ]$ of all integers  $a$  that are congruent to 0 modulo 3.   That is, 
%\label{PA:congruencemodulo3-1}%
%$C[ 0 ] = \left\{ { {a \in \mathbb{Z} } \mid a \equiv 0 \pmod 3} \right\}\!.$
%
\item The set  $C[ 1 ]$ of all integers  $a$  that are congruent to 1 modulo 3.   That is, 
\label{PA:congruencemodulo3-2}%
$C[ 1 ] = \left\{ { {a \in \mathbb{Z} } \mid a \equiv 1 \pmod 3} \right\}\!.$
%
\item The set  $C[ 2 ]$ of all integers  $a$  that are congruent to 2 modulo 3.   That is, 
\label{PA:congruencemodulo3-3}%
$C[ 2 ] = \left\{ { {a \in \mathbb{Z} } \mid a \equiv 2 \pmod 3} \right\}\!.$
%
\item The set  $C[ 3 ]$ of all integers  $a$  that are congruent to 3 modulo 3.   That is, 
$C[ 3 ] = \left\{ { {a \in \mathbb{Z} } \mid a \equiv 3 \pmod 3} \right\}\!.$
\end{enumerate}

\item Now consider the three sets, $C[ 0 ]$, $C[ 1 ]$, and $C[ 2 ]$. 
%in Parts~(\ref{PA:congruencemodulo3-1}), (\ref{PA:congruencemodulo3-2}), 
%and~(\ref{PA:congruencemodulo3-3}).
\begin{enumerate}
  \item Determine the intersection of any two of these sets.  That is,  determine  
  $C[ 0 ] \cap C[ 1 ]$, $C[ 0 ] \cap C[ 2 ]$,        and  $C[ 1 ] \cap C[ 2 ]$. 

  \item Let  $n = 734$.  What is the remainder when  $n$ is divided by 3?  Which of the three sets, if any,  contains  $n = 734$?  
\label{PA:congruencemodulo3-5b}%

  \item Repeat Part~(\ref{PA:congruencemodulo3-5b}) for  $n = 79$ and for $n=-79$.

  %\item Repeat Part~(\ref{PA:congruencemodulo3-5b}) for  $n =  - 79$.

  \item Do you think that  
        $C[ 0 ] \cup C[ 1 ] \cup C[ 2 ] = \mathbb{Z}$?  Explain.
  \item Is the set $C[3]$ equal to one of the sets $C[0], C[1]$, or $C[2]$?
  \item We can also define $C[4] = \left\{ { {a \in \mathbb{Z} } \mid a \equiv 4 \pmod 3} \right\}\!.$  Is this set equal to any of the previous sets we have studied in this part?  Explain.
\end{enumerate}

\end{enumerate}
\end{previewactivity}
\hbreak

\endinput
