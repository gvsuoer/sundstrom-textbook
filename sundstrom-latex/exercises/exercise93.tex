%\newpage
\section*{Exercises \ref{S:uncountablesets}}

\begin{enumerate}
\item Use an appropriate bijection to prove that each of the following sets has cardinality $\boldsymbol{c}$. 
\label{exer:sec94-1}%
\begin{multicols}{2}
\begin{enumerate}
\yitem $( 0, \infty )$
\yitem $( a, \infty )$, for any $a \in \mathbb{R}$
\item $\mathbb{R} - \left\{ 0 \right\}$
\item $\mathbb{R} - \left\{ a \right\}$, for any $a \in \mathbb{R}$
\end{enumerate}
\end{multicols}

\xitem Is the set of irrational numbers countable or uncountable? Prove that your answer is correct. 
\label{exer:irrationaluncount}%

\xitem Prove that if $A$ is uncountable and $A \subseteq B$, then $B$ is uncountable.
\label{exer:supersetofuncount}%

\xitem Do two uncountable sets always have the same cardinality?  Justify your conclusion. \label{exer93:uncountablesets}


\item Let  $C$  be the set of all infinite sequences, each of whose entries is the digit 0 or the digit 1.  For example,
\[
\begin{aligned}
\left( 1, 0, 1, 0, 1, 0, 1, 0, \ldots \right) &\in C; \\
\left( 0, 1, 0, 1, 1, 0, 1, 1, 1, 0, 1, 1, 1, 1, \ldots \right) &\in C; \\
\left( 2, 1, 0, 1, 1, 0, 1, 1, 1, 0, 1, 1, 1, 1, \ldots \right) &\notin C.
\end{aligned}
\]
Is the set  $C$  a countable set or an uncountable set?  Justify your conclusion.


\item The goal of this exercise is to use the Cantor-Schr\"{o}der-Bernstein Theorem to prove that the cardinality of the closed interval $[ 0, 1 ]$ is $\boldsymbol{c}$.
\label{exer:closedinterval}%

\begin{enumerate}
\item Find an injection $f\x  ( 0, 1 ) \to [ 0, 1 ]$.

\item Find an injection $h\x  [ 0, 1 ] \to ( -1, 2 )$.

\item Use the fact that $( -1, 2 ) \approx ( 0, 1 )$ to prove that there exists an injection 
$g\x  [ 0, 1 ] \to ( 0, 1 )$.  (It is only necessary to prove that the injection $g$ exists.  It is not necessary to determine a specific formula for 
$g ( x )$.)  

\note  Instead of doing Part~(b) as stated, another approach is to find an injection 
$k \x [0, 1] \to (0, 1)$.  Then, it is possible to skip Part~(c) and go directly to Part~(d).

\item Use the Cantor-Schr\"{o}der-Bernstein Theorem to conclude that \linebreak
$[ 0, 1 ] \approx ( 0, 1 )$ and hence that the cardinality of 
$[ 0, 1 ]$ is $\boldsymbol{c}$.
\end{enumerate}

\item In Exercise~(\ref{exer:closedinterval}), we proved that the closed interval $[ 0, 1 ]$ is uncountable and has cardinality $\boldsymbol{c}$.  Now let $a, b \in \mathbb{R}$ with $a < b$.  Prove that 
$[ a, b ] \approx [ 0, 1 ]$ and hence that $[ a, b ]$ is uncountable and has cardinality $\boldsymbol{c}$.



\item Is the set of all finite subsets of $\N$ countable or uncountable? Let $F$ be the set of all finite subsets of $\N$.  Determine the cardinality of the set $F$.

\eighth
\noindent
Consider defining a function $f:F \to \N$ that produces the following.
\begin{itemize}
  \item If $A = \{1, 2, 6 \}$, then $f(A) = 2^1 3^2 5^6$.
  \item If $B = \{3, 6 \}$, then $f(B) = 2^3 3^6$.
  \item If $C = \left\{ m_1, m_2, m_3, m_4 \right\}$ with $m_1 < m_2 < m_3 < m_4$, then 
           $f(C) = 2^{m_1} 3^{m_2} 5^{m_3} 7^{m_4}$.
\end{itemize}
It might be helpful to use the Fundamental Theorem of Arithmetic on page~\pageref{T:fundtheorem} and to denote the set of all primes as  $P = \left\{ p_1, p_2, p_3, p_4, \ldots \right\}$ with $p_1 < p_2 < p_3 < p_4 \cdots$.  Using the sets $A$, $B$, and $C$ defined above, we would then write
\[
f(A) = p_1^1 p_2^2 p_3^6, \quad f(B) = p_1^3 p_2^6, \quad \text{and} \quad f(C) = p_1^{m_1} p_2^{m_2} p_3^{m_3} p_4^{m_4}.
\]

\item In Exercise~(\ref{exer:irrationaluncount}), we showed that the set of irrational numbers is uncountable.  However, we still do not know the cardinality of the set of irrational numbers.  Notice that we can use $\Q^c$ to stand for the set of irrational numbers.

\begin{enumerate}
\item Construct a function $f\x \Q^c \to \R$ that is an injection.
\end{enumerate}
We know that any real number $a$ can be represented in decimal form as follows:
\[
a = A.a_1 a_2 a_3 a_4 \cdots a_n \cdots ,
\]
where $A$ is an integer and the decimal part $( 0.a_1 a_2 a_3 a_4 \cdots )$ is in normalized form.  (See page~\pageref{normalizedform}.)  We also know that the real number $a$ is an irrational number if and only $a$ has an infinite non-repeating decimal expansion.  We now associate with $a$ the real number
\setcounter{equation}{0}
\begin{equation}\label{exer:irrationalsuncount}
A.a_1 0 a_2 1 1 a_3 0 0 0 a_4 1 1 1 1 a_5 0 0 0 0 0 a_6 1 1 1 1 1 1 \cdots .
\end{equation}
Notice that to construct the real number in (\ref{exer:irrationalsuncount}), we started with the decimal expansion of $a$, inserted a 0 to the right of the first digit after the decimal point, inserted two 1's to the right of the second digit to the right of the decimal point, inserted three 0's to the right of the third digit to the right of the decimal point, and so on.

\begin{enumerate} \setcounter{enumii}{1}
\item Explain why the real number in (\ref{exer:irrationalsuncount}) is an irrational number.
\item Use these ideas to construct a function $g\x \R \to \Q^c$ that is an injection.
\item What can we now conclude by using the Cantor-Schr\"{o}der-Bernstein Theorem?
\end{enumerate}

\item Let $J$ be the unit open interval.  That is, $J = \left\{ x \in \R \mid 0 < x < 1 \right\}$ and let $S = \left\{ (x, y) \in \R \times \R \mid 0 < x < 1 \text{ and } 0 < y < 1 \right\}$.  We call $S$ the unit open square.  We will now define a function $f$ from $S$ to $J$.  Let 
$(a, b) \in S$ and write the decimal expansions of $a$ and $b$ in normalized form as
\begin{align*}
a &= 0.a_1 a_2 a_3 a_4 \cdots a_n \cdots \\
b &= 0.b_1 b_2 b_3 b_4 \cdots b_n \cdots .
\end{align*}
We then define $f(a, b) = 0.a_1 b_1 a_2 b_2 a_3 b_3 a_4 b_4 \cdots a_n b_n \cdots$.
\begin{enumerate}
\item Determine the values of $f ( 0.3, 0.625 )$, $f \!\left( \dfrac{1}{3}, \dfrac{1}{4} \right)$, and $f \!\left( \dfrac{1}{6}, \dfrac{5}{6} \right)$.

\item If possible, find $(x, y) \in S$ such that $f ( x, y ) = 0.2345$.

\item If possible, find $(x, y) \in S$ such that $f ( x, y ) = \dfrac{1}{3}$.

\item If possible, find $(x, y) \in S$ such that $f ( x, y ) = \dfrac{1}{2}$.

\item Explain why the function $f\x S \to J$ is an injection but is not a surjection.

\item Use the Cantor-Schr\"{o}der-Bernstein Theorem to prove that the cardinality of the unit open square $S$ is equal to $\boldsymbol{c}$. If this result seems surprising, you are in good company.  In a letter written in 1877 to the mathematician Richard Dedekind describing this result that he had discovered, Georg Cantor wrote, ``I see it but I do not believe it.''
\end{enumerate}
 
\end{enumerate}

\subsection*{Explorations and Activities}
\setcounter{oldenumi}{\theenumi}
\begin{enumerate} \setcounter{enumi}{\theoldenumi}
\item \textbf{The Closed Interval $\boldsymbol{ [ 0, 1 ]}$}.  \label{A:closedinterval}  In Exercise~(\ref{exer:closedinterval}), the Cantor-Schr\"{o}der-Bernstein Theorem was used to prove that the closed interval $[0, 1]$ has cardinality $\boldsymbol{c}$.  
%We have seen that $\text{card} \!\left( ( 0, 1 ) \right) = \boldsymbol{c}$ and 
%$\text{card} ( \mathbb{R} ) = \boldsymbol{c}$.  It would seem reasonable to expect that if we add the endpoints to the open interval $( 0, 1 )$, we would not change the cardinality.  That is, it might be reasonable to expect that 
%$\text{card} \!\left( [ 0, 1 ] \right) = \boldsymbol{c}$.  We have, in fact, indicated that the Cantor-Schr\"{o}der-Bernstein Theorem (Theorem~\ref{T:bernstein}) can be used to prove that this is true.  
This may seem a bit unsatisfactory since we have not proved the 
Cantor-Schr\"{o}der-Bernstein Theorem.  In this activity, we will prove that 
$\text{card} \!\left( [ 0, 1 ] \right) = \boldsymbol{c}$ by using appropriate bijections.

\begin{enumerate}
\item Let $f\x  [ 0, 1 ] \to [ 0, 1 )$ by
\begin{equation} \notag
f ( x ) = 
\begin{cases}
\dfrac{1}{n+1}         &\text{if $x=\dfrac{1}{n}$ for some $n \in \mathbb{N}$} \\
 %                     &                      \\
x        &\text{otherwise}.
\end{cases}
\end{equation}
\label{A:closedinterval1}%
%
\begin{enumerate}
\item Determine $f ( 0 )$, $f ( 1 )$, $f \!\left( \dfrac{1}{2} \right)$, 
$f \!\left( \dfrac{1}{3} \right)$, $f \!\left( \dfrac{1}{4} \right)$, and 
$f \!\left( \dfrac{1}{5} \right)$. 
\label{A:closedinterval1a}%
\item Sketch a graph of the function $f$.  \hint  Start with the graph of $y = x$ for 
$0 \leq x \leq 1$.  Remove the point $( 1, 1 )$ and replace it with the point 
$\!\left( 1, \dfrac{1}{2} \right)$.  Next, remove the point 
$\!\left( \dfrac{1}{2}, \dfrac{1}{2} \right)$ and replace it with the point 
$\!\left( \dfrac{1}{2}, \dfrac{1}{3} \right)$.  Continue this process of removing points on the graph of $y = x$ and replacing them with the points determined from the information in 
Part~(\ref{A:closedinterval1a}).  Stop after repeating this four or five times so that pattern of this process becomes apparent.

\item Explain why the function $f$ is a bijection.
\item Prove that $[ 0, 1 ] \approx [ 0, 1 )$.
\end{enumerate}
%
\item Let $g\x  [ 0, 1 ) \to ( 0, 1 )$ by
\begin{equation} \notag
g ( x ) = 
\begin{cases}
\dfrac{1}{2}           &\text{if $x=0$} \\
%                      &                      \\
\dfrac{1}{n+1}         &\text{if $x=\dfrac{1}{n}$ for some $n \in \mathbb{N}$} \\
%                      &                      \\
x        &\text{otherwise}.
\end{cases}
\end{equation}
\begin{enumerate}
\item Follow the procedure suggested in Part~(\ref{A:closedinterval1}) to sketch a graph of $g$.
\item Explain why the function $g$ is a bijection.
\item Prove that $[ 0, 1 ) \approx ( 0, 1 )$.
\end{enumerate}

\item Prove that $[ 0, 1 ]$ and $[ 0, 1 )$ are both uncountable and have cardinality $\boldsymbol{c}$.
\end{enumerate}
\end{enumerate}

\hbreak
\endinput
