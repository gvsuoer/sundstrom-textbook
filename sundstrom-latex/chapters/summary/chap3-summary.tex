\section{Chapter \ref{C:proofs} Summary} \label{Su:proofs}

\subsection*{Important Definitions}
\begin{multicols}{2}
\begin{itemize}
\item Divides, divisor, page~\pageref*{divides}
\item Factor, multiple, page~\pageref*{divides}
\item Proof, page~\pageref*{proof}
\item Undefined term, page~\pageref*{undefined}
\item Axiom, page~\pageref*{axiom}
\item Definition, page~\pageref*{definition}
\item Conjecture, page~\pageref*{conjecture}
\item Theorem, page~\pageref*{theorem}
\item Proposition, page~\pageref*{proposition}
\item Lemma, page~\pageref*{lemma}
\item Corollary, page~\pageref*{corollary}
\item Congruence modulo $n$, page~\pageref*{congruence}
\item Tautology, page~\pageref*{D:tautology}
\item Contradiction, page~\pageref*{D:tautology}
\item Absolute value, page~\pageref*{D:absvalue}
\end{itemize}
\end{multicols}
\hbreak



\subsection*{Important Theorems and Results about Even and Odd Integers} \label{SS:evenodd}
%\begin{theorem}[Properties of Even and Odd Integers] \label{T:evenodd} \hfill
\index{even integer!properties}%
\index{odd integer!properties}%

%
\begin{itemize}
\item \textbf{Exercise~(\ref{exer:nextint}), Section~\ref{S:direct}} \\
\emph{
If $m$ is an even integer, then $m+1$ is an odd integer. \\ 
If $m$ is an odd integer, then $m+1$ is an even integer}.

\item \textbf{Exercise~(\ref{exer:integeradd}), Section~\ref{S:direct}} \\
\emph{
If $x$ is an even integer and $y$ is an even integer, then $x+y$ is an even integer.  \\
If $x$ is an even integer and $y$ is an odd integer, then $x+y$ is an odd integer. \\
If $x$ is an odd integer and $y$ is an odd integer, then $x+y$ is an even integer}.

\item \textbf{Exercise~(\ref{exer:integermult}),  Section~\ref{S:direct}}.  \emph{If $x$ is an even integer and $y$ is an integer, then $ x \cdot y $ is an even integer}.

\item \textbf{Theorem~\ref{T:xyodd}}.  \emph{If $x$ is an odd integer and $y$ is an odd integer, then $ x \cdot y $ is an odd integer}.

\item \textbf{Theorem~\ref{T:n2odd}}. \emph{The integer $n$ is an even integer if and only if $n^2$ is an even integer}.  \\
\textbf{\typeu Activity \ref*{PA:biconditional} in Section~\ref{S:moremethods}}. \emph{The integer $n$ is an odd integer if and only if $n^2$ is an odd integer}.
\end{itemize}
\hbreak


\subsection*{Important Theorems and Results about Divisors} \label{SS:divisors}
\index{divisor!properties}%
\noindent
\begin{itemize}
\item \textbf{Theorem \ref{T:transdivide}}. \emph{For all integers $a$, $b$, and $c$ with 
$a \ne 0$, if $ a \mid b $ and $ b \mid c $, then $ a \mid c $}.

\item \textbf{Exercise~(\ref{exer:3truefalse}), Section \ref{S:directproof}}.  \emph{For all integers $a$, $b$, and $c$ with $a \ne 0$,  \\
If  $a \mid b$ and $a \mid c$, then $a \mid \left(b+c \right) $. \\
If  $a \mid b$ and $a \mid c$, then $a \mid \left(b-c \right) $}.

\item \textbf{Exercise~(\ref{exer:divprod}), Section \ref{S:directproof}}. \emph{For all integers $a$, $b$, and $c$ with $a \ne 0$, if $a \mid b$, then $a \mid \left( bc \right)$}.

\item \textbf{Exercise~(\ref{exer:diveach}), Section \ref{S:directproof}}.  \emph{For all nonzero integers $a$ and $b$, if $a \mid b$ and $b \mid a$, then $a = \pm b$}.
\end{itemize}
\hbreak


\subsection*{The Division Algorithm}
\emph{Let  $a$ and $b$ be integers with $b>0$.  Then there exist unique integers $q$ and $r$ such that}
\[
a=bq+r  \text{ and }  0 \leq r < b.
\]
\hbreak



\subsection*{Important Theorems and Results about Congruence} \label{SS:congruence}
%Let $n$ be a natural number and let $a$, $b$, and $c$ be integers.

\begin{itemize}
\item \textbf{Theorem~\ref{T:propsofcong}}.
\emph{Let $a, b, c \in \Z$ and let $n \in \N$.  
If $a \equiv b \pmod n$ and $c \equiv d \pmod n$, then
\begin{list}{}
  \item $\left( {a + c} \right) \equiv \left( {b + d} \right) \pmod n$.  
  \item $ac \equiv bd \pmod n$.  
  \item For each $m \in \mathbb{N}$, $a^m  \equiv b^m \pmod n$. 
\end{list}}

\item \textbf{Theorem~\ref{T:modprops}}. \emph{For all integers $a$, $b$, and $c$},
\begin{list}{}
\item \textbf{Reflexive Property}. $a \equiv a \pmod n$.
\index{congruence!reflexive property}%

\textbf{Symmetric Property}. \emph{If  $a \equiv b \pmod n$, then  $b \equiv a \pmod n$}.
\index{congruence!symmetric property}%

\textbf{Transitive Property}. \emph{If  $a \equiv b \pmod n$ and $b \equiv c \pmod n$, then  \\
$a \equiv c \pmod n$}.
\index{congruence!transitive property}%
\end{list}



\item \textbf{Theorem~\ref{T:congtorem}}.   \emph{Let $a \in \Z$ and let $n \in \N$.
If  $a = nq + r\text{  and  }0 \leq r < n$ for some integers  $q$  and  $r$, then  $a \equiv r \pmod n$}.


\item \textbf{Corollary~\ref{C:congtorem}}.  \emph{Each integer is congruent, modulo $n$, to precisely one of the integers $0,1,2, \ldots ,n - 1$.  That is, for each integer $a$, there exists a unique integer 
$r$ such that}
\[
a \equiv r \pmod n \quad \text{and} \quad 0 \leq r < n.
\]
\end{itemize}
\hbreak


%\subsection*{A Comparison of Direct Proofs, Proofs Using the Contrapositive, and Proofs by Contradiction} \label{SS:proofcompare}
%
%Following are descriptions of three of the most common methods of proving a conditional statement.
%
%\vskip10pt
%\noindent
%\textbf{{Direct Proof of} {\mathversion{bold} $P \to Q$}}
%\index{direct proof}%
%\index{proof!direct}%
%
%\begin{itemize}
%\item \textbf{When is it indicated}?  This type of proof is often used when the hypothesis and the conclusion are both stated in  a ``positive'' manner.  That is, no negations are evident in the hypothesis and conclusion.  That is, no negations are evident in the hypothesis and conclusion.
%
%\item \textbf{Description of the process}.  Assume that  $P$  is true and use this to conclude that  $Q$  is true.  That is, we use the forward-backward method and work forward from  $P$  and backward from  $Q$.
%
%\item \textbf{Why the process makes sense}.  We know that the conditional statement  $P \to Q$  is automatically true when the hypothesis is false.  Therefore, because our goal is to prove that  $P \to Q$  is true, there is nothing to do in the case that  $P$  is false.  Consequently, we may assume that  $P$  is true.  Then, in order for  $P \to Q$  to be true,  the conclusion  $Q$  must also be true.  (When  $P$  is true, but  $Q$ is false, $P \to Q$  is false.)  Thus, we must use our assumption that  $P$  is true to show that  $Q$  is also true.
%\end{itemize}
%
%
%\noindent
%\textbf{{Proof of} {\mathversion{bold} $P \to Q$} Using the Contrapositive}
%\index{direct proof}%
%\index{proof!direct}%
%
%\begin{itemize}
%\item \textbf{When is it indicated}?  This type of proof is often used when both the hypothesis and the conclusion are stated in the form of negations.  This often works well if the conclusion contains the operator ``or'';  that is, if the conclusion is in the form of a disjunction.  In this case, the negation will be a conjunction.
%
%\item \textbf{Description of the process}.  We prove the logically equivalent statement  p
%$\mynot  Q \to \mynot  P$.  The forward-backward method is used to prove  
%$\mynot  Q \to \mynot  P$.  That is, we work forward from   $\mynot  Q$  and backward from  
%$\mynot P$.  
%
%\item \textbf{Why the process makes sense}.  When we prove  $\mynot  Q \to \mynot  P$, we are also proving  $P \to Q$  because these two statements are logically equivalent.  When we prove the contrapositive of   $P \to Q$, we are doing a direct proof of  $\mynot  Q \to \mynot P$.  So we assume  $\mynot  Q$  because, when doing a direct proof, we assume the hypothesis, and  $\mynot  Q$  is the hypothesis of the contrapositive.  We must show  $\mynot  P$  because it is the conclusion of the contrapositive.
%\end{itemize}
%
%
%\noindent
%\textbf{{Proof of} {\mathversion{bold} $P \to Q$} Using a Proof by Contradiction}
%\index{direct proof}%
%\index{proof!direct}%
%
%\begin{itemize}
%\item \textbf{When is it indicated}?  This type of proof is often used when the  conclusion is stated in the form of a negation, but the hypothesis is not.  This often works well if the conclusion contains the operator ``or'';  that is, if the conclusion is in the form of a disjunction.  In this case, the negation will be a conjunction.
%
%\item \textbf{Description of the process}.  Assume  $P$  and $\mynot Q$   and work forward from these two assumptions until a contradiction is obtained.
%
%\item \textbf{Why the process makes sense}.  The statement  $P \to Q$
%  is either true or false.  In a proof by contradiction, we show that it is true by eliminating the only other possibility (that it is false).  We show that  $P \to Q$
%  cannot be false by assuming it is false and reaching a contradiction.  Since we assume that  $P \to Q$
%  is false, and the only way for a conditional statement to be false is for its hypothesis to be true and its conclusion to be false,  we assume that  $P$ is true and that  $Q$  is false (or, equivalently, that  $\mynot  Q$
%  is true).  When we reach a contradiction, we know that our original assumption that  $P \to Q$
%  is false is incorrect.  Hence,  $P \to Q$
%  cannot be false, and so it must be true.
%\end{itemize}
%
%
%Please keep in mind that these descriptions summarize three of the most common methods of proving a conditional statement of the form  $P \to Q$.  As was indicated in 
%Section~\ref{S:moremethods}, other methods can sometimes be used.  Quite often, these other methods involve the use of a logical equivalency.  For example, in order to prove a statement of the form
%
%\[
%P \to \left( {Q \vee R} \right),
%\]
%
%it is sometimes possible to use the logical equivalency
%
%\[\left[ {P \to \left( {Q \vee R} \right)} \right] \equiv \left[ {\left( {P \wedge \mynot  Q} \right) \to R} \right].
%\]
%
%We would then prove the statement
%\[
%\left( {P \wedge \mynot Q} \right) \to R.
%\]
%
%Because of the logical equivalency, by proving one statement, we have also proven the other statement.
%\hbreak
%
%\subsection*{Other Types of Proofs}
%\noindent
%\textbf{Constructive Proof} \hfill \\ 
%\index{constructive proof}%
%\index{proof!constructive}%
%This is a technique that is often used to prove a so-called \textbf{existence theorem.}
%\index{existence theorem}%
%  The objective of an existence theorem is to prove that a certain mathematical object exists.  That is, the goal is usually to prove a statement of the form  
%\begin{center}
%There exists an $x$  such that  $P( x )$.
%\end{center}
%For a constructive proof of such a proposition, we actually name, describe, or explain how to construct  some object in the universe that makes  $P( x )$ true.
%
%\vskip6pt
%\noindent
%\textbf{Nonconstructive Proof}  \hfill \\
%This is another type of proof that is often used to prove an existence theorem is the so-called \textbf{nonconstructive proof.}
%%\index{existence theorem}%
%\index{proof!non-constructive}%
%  For this type of proof, we make an argument that an object  in the universal set that makes  
%$P\left( x \right)$ true must exist but we never construct or name the object that makes  
%$P\left( x \right)$  true.
%
%\vskip6pt
%\noindent
%\textbf{Proof Using Cases} \hfill \\
%\index{cases, proof using}%
%\index{proof!using cases}%
%When we are trying to prove a proposition or a theorem, we often run into the problem that there does not seem to be enough information to proceed.  In this situation, we will sometimes use cases to provide additional assumptions for the forward process of the proof.  When this is done, the  original proposition is divided into a number of separate cases that are proven independently of each other.  The cases must be chosen so that they exhaust all possibilities for the hypothesis of the original proposition.  This method of case analysis is justified by the logical equivalency  $\left( {P \vee Q} \right) \to R \equiv \left( {P \to R} \right) \wedge \left( {Q \to R} \right)$, which was established in Beginning Activity~\ref{PA:logicalequiv} in Section~\ref{S:cases}.
%\hbreak


\endinput
