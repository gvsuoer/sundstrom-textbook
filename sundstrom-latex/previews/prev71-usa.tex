\begin{previewactivity}[\textbf{The United States of America}] \label{PA:USA} \hfill \\
Recall from Section~\ref{S:cartesian} that the \textbf{Cartesian product}
\index{Cartesian product}%
 of two sets  $A$  and  $B$, written  $A \times B$, is the set of all ordered pairs  
$\left( {a,b} \right)$,  where  $a \in A$  and  $b \in B$.  That is,
$A \times B = \left\{ {\left( {a,b} \right)  \mid a \in A\text{ and }b \in B} \right\}$.

\noindent
Let  $A$  be the set of all states in the United States and let
\[
R = \left\{ { {\left( {x, y} \right) \in A \times A } \mid x \text{  and  }y 
\text{  have a land border in common}} \right\}\!.
\]
For example, since California and Oregon have a land border, we can say that 
$(\text{California, Oregon}) \in R$ and $(\text{Oregon, California}) \in R$.  Also, since California and Michigan do not share a land border, $\text{(California, Michigan)} \notin R$ and 
$(\text{Michigan, California}) \notin R$.
\begin{enumerate}
\item Use the roster method  to specify the elements in each of the following sets:
\begin{enumerate}
\item $B = \left\{ {y \in A\left| {\left( {\text{Michigan, }y} \right) \in R} \right.} \right\}$

\item $C = \left\{ {x \in A\left| {\left( {x,\text{Michigan}} \right) \in R} \right.} \right\}$

\item $D = \left\{ {y \in A\left| {\left( {\text{Wisconsin, }y} \right) \in R} \right.} \right\}$

\end{enumerate}

\item Find two different examples of two ordered pairs,  $\left( {x, y} \right)$ and 
$\left( {y, z} \right)$ such that  $\left( {x, y} \right) \in R$,  
$\left( {y, z} \right) \in R$,  but  $\left( {x, z} \right)\not  \in R$, or explain why no such example exists.  Based on this, is the following conditional statement true or false?
\begin{center}
For all $x, y, z \in A$, if $(x, y) \in R$ and $(y, z) \in R$, then $(x, z) \in R$.
\end{center}

\item Is the following conditional statement true or false?  Explain.
\begin{center}
For all $x, y \in A$, if $(x, y) \in R$, then $(y, x) \in R$.
\end{center}


%\item Are the following statements true or false?  Justify your conclusions.
%\begin{enumerate}
%\item For all $x, y \in A$, if $(x, y) \in R$, then $(y, x) \in R$.
%\item For all $x, y, z \in A$, if $(x, y) \in R$ and $(y, z) \in R$, then $(x, z) \in R$.
%\end{enumerate}

%\item In Section~\ref{S:inversefunctions}, we learned how to represent a function as a set of ordered pairs, and we learned under what conditions a set of ordered pairs can be used to define a function. Can the set  $R$  be used to define a function from the set  $A$  to the set  $A$?  Explain.
\end{enumerate}
\end{previewactivity}
\hbreak

\endinput
