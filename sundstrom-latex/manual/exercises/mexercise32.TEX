\section*{Section \ref{S:moremethods} More Methods of Proof}

\begin{enumerate}
\item \begin{enumerate}
\item If there exists an integer $k$ such that $n = 2k$, then $n^3 = 8k^3 = 2 \left( 4k^3 \right)$.

\item Prove the contrapositve:  If $n$ is odd, then $n^3$ is odd.  In this case, if there exists an integer $k$ such that $n = 2k + 1$, then
\[
\begin{aligned}
n^3 &= 8k^3 + 12k^2 + 6k + 1 \\
    &= 2 \left( 4k^3 + 6k^2 + 3k) + 1\right).
\end{aligned}
\]

\item The two conditional statements in Part~(a) prove this biconditional statement.

\item The two conditional statements whose conjuction is equivalent to this biconditional statement are the contrapositives of the statements in Parts~(a) and~(b).
\end{enumerate}

\item \begin{enumerate}
\item The contrapositive is:  Let  $a$  and  $b$  be integers.  If  
$ab \equiv 0\left( {\bmod 6} \right)$, then  $a \equiv 0\left( {\bmod 6} \right)$  or   
$b \equiv 0\left( {\bmod 6} \right)$.

\item The proposition is false.  A counterexample is $a = 3$ and $b = 2$.
\end{enumerate}

\item \begin{enumerate}
\item The contrapositive is:  For all positive real numbers $a$ and $b$,  if $a = b$, then 
$\sqrt{ab} = \dfrac{a + b}{2}$.

\item The statement is true.  If $a = b$, then $\dfrac{a + b}{2} = \dfrac{2a}{2} = a$, and \\
$\sqrt{ab} = \sqrt{a^2} = a$.
\end{enumerate}



\item \begin{enumerate}
\item True.  If $a \equiv 2 \pmod 5$, then there exists an integer $k$ such that $a - 2 = 5k$.  
Then, 
\[
a^2 = \left( 2 + 5k \right)^2 = 4 + 20k + 25k^2.  
\]
This means that $a^2 - 4 = 5 \left( 4k + 5k^2 \right)$, and hence, $a^2 \equiv 4 \pmod 5$.

\item False.  A counterexample is $a = 3$.

\item False.  Part~(b) shows this is false.
\end{enumerate}


\item The proposition is true.  The contrapositive, which is:  For all integers $a$ and $b$, if $a$ is odd and $b$ is odd, then $ab$ is odd.  This is true by Theorem~1.8 in Section~1.2.


\item \begin{enumerate}
\item For each integer $a$, if $a \equiv 3 \pmod 7$, then $(a^2 + 5a) \equiv 3 \pmod 7$, and if $(a^2 + 5a) \equiv 3 \pmod 7$, then $a \equiv 3 \pmod 7$.

\item For each integer $a$, if $a \equiv 3 \pmod 7$, then $(a^2 + 5a) \equiv 3 \pmod 7$ is true.  To prove this, if $a \equiv 3 \pmod 7$, then there exists an integer $k$ such that 
$a = 3 + 7k$.  We can then prove that
\[
a^2 + 5a = 24 + 77k + 49k^2 = 3 + 7(3 + 11k + 7k^2).
\]
This shows that $(a^2 + 5a) \equiv 3 \pmod 7$.

For each integer $a$,if $(a^2 + 5a) \equiv 3 \pmod 7$, then $a \equiv 3 \pmod 7$ is false.  A counterexample is $a = 6$.  When $a = 6$, $a^2 + 5a = 66$ and 
$66 \equiv 3 \pmod 7$ and $6 \not \equiv 3 \pmod 7$.

\item Since one of the two conditional statements in Part~(b) is false, the given proposition is false.
\end{enumerate}



\item \begin{enumerate}
\item For each integer $a$, if $a \equiv 2 \pmod 8$, then $(a^2 + 4a) \equiv 4 \pmod 8$, and if $(a^2 + 4a) \equiv 4 \pmod 8$, then $a \equiv 2 \pmod 8$.

\item For each integer $a$, if $a \equiv 2 \pmod 8$, then $(a^2 + 4a) \equiv 4 \pmod 8$ is true.  To prove this, if $a \equiv 2 \pmod 8$, then there exists an integer $k$ such that 
$a = 2 + 8k$.  We can then prove that
\[
a^2 + 4a = 4 + 32k + 64k^2 = 4 + 8(4k + 8k^2).
\]
This shows that $(a^2 + 4a) \equiv 2 \pmod 8$.

For each integer $a$,if $(a^2 + 4a) \equiv 4 \pmod 8$, then $a \equiv 2 \pmod 8$ is false.  A counterexample is $a = 6$.  When $a = 6$, $a^2 + 4a = 60$.  We note that $60 \equiv 4 \pmod 8$ and $6 \not \equiv 2 \pmod 8$.

\item Since one of the two conditional statements in Part~(b) is false, the given proposition is false.
\end{enumerate}


\item %\begin{multicols}{2}
\begin{enumerate}
\item  $A = \dfrac{1}{2}ab$.  An isosceles triangle is a triangle in which two sides have equal length.

\item $c^2 = a^2 + b^2$
\end{enumerate}
%\end{multicols}

\begin{enumerate} \setcounter{enumii}{2}
%\item An isosceles triangle is a triangle in which two sides have equal length.

\item If the triangle is an isosceles triangle, then $a = b$.  Consequently, 
$A = \dfrac{1}{2}a^2$.  Using the Pythagorean Theorem, we see that
\[
c^2 = a^2 + a^2 = 2a^2.
\]
Hence, $a^2 = \dfrac{1}{2}c^2$, and we obtain $A = \dfrac{1}{2}a^2 = \dfrac{1}{4}c^2$.

Conversely, if the area of this isosceles triangle is $A = \dfrac{1}{4}c^2$, then since the area is also $\dfrac{1}{2}ab$, we see that
\[
\begin{aligned}
\frac{1}{4}c^2 &= \frac{1}{2}ab \\
c^2 &= 2ab \\
\end{aligned}
\]
We now use the Pythagoren Theorem to conlcude that $a^2 + b^2 = 2ab$.  This equation can be rewrttten as follows:
\[
\begin{aligned}
a^2 - 2ab + b^2 &= 0 \\
\left( a - b \right)^2 &= 0. \\
\end{aligned}
\]
The last equation implies that $a = b$ and hence the right triangle is an isosceles triangle.
\end{enumerate}

\item  The statement is true.  It is easier to prove the contrapositive, which is:

\begin{list}{}
\item For each positive real number $x$, if $\sqrt{x}$ is rational, then $x$ is rational.
\end{list}
If there exist positive integers $m$ and $n$ such that $\sqrt{x} = \dfrac{m}{n}$, then 
$x = \dfrac{m^2}{n^2}$.



\item The statement is true.  One way to prove this is to prove the following two conditional statements:
\begin{itemize}
\item If $n$ is even, then 4 divides $n^2$.
\item If 4 divides $n^2$, then $n$ is even.
\end{itemize}
Use a direct proof for the first conditional statement and prove the contrapositive of the second condtional statement.


\item If $a$ is an integer and $a^2 - 1$ is even, then $a^2$ is odd.  This implies that $a$ is odd and hence, there exists an integer $k$ such that $a = 2k + 1$.  So
\begin{align*}
a^2 - 1 &= (2k + 1)^2 - 1 \\
        &= 4(k^2 + k),
\end{align*}
and this proves that 4 divides $a^2 - 1$.


\item From the Pythagorean Theorem, $a^2 + m^2 = (m + 1)^2$, and this implies that 
$a^2 + m^2 = m^2 + 2m + 1$.  From this, we see that $a^2 = 2m + 1$.  Therefore, $a^2$ is odd and hence, $a$ is odd.


\item If $p, q \in \Q$ with $p < q$, then let $x = \dfrac{p + q}{2}$.  Since $\Q$ is closed under addition and division by nonzero numbers, $x \in \Q$.  In addition,
\[
2p < p + q  < 2q,
\]
and hence, $p < \dfrac{p + q}{2} < q$.  This proves that there exists a rational number $x$ such that $p < x < q$.


\item \begin{enumerate}
\item True. If $x = 2$ and $y = -1$, then $4x + 6y = 2$.

\item False.  For all integers $x$ and $y$, 3 divides $6x + 15y$.

\item True.  If $x = 4$ and $y = -1$, then $6x + 15y = 9$.
\end{enumerate}


\item Let $f(x) = x^3 - 4x$. Then $f(4) = 0$ and $f(5) = 25$.  By the Intermediate Value Theorem, there exists a real number $x$ with $4 < x < 5$ such that $f(x) = 7$. 


\item Let $y_{\text{max}}$ be the largest of $y_1$, $y_2$, $y_3$, and $y_4$.  For each $i$ with $1 \leq i \leq 4$, $y_{\text{max}} \geq y_i$.  Hence,
\[
\overline y \leq \frac{4 y_{\text{max}}}{4},
\]
or $\overline y \leq y_{\text{max}}$.  Since $y_{\text{max}} = y_i$ for some $i$ with 
$1 \leq i \leq 4$, this proves that there exists a $y_i$ with $1 \leq i \leq 4$ such that 
$y_i \geq \overline y$.



\item  We are given that $a$ and $b$ are natural numbers and that $a^2 = b^3$.
\begin{enumerate}
\item Assume that $a$ is even.  Then there exists an integer $k$ such that $a = 2k$.  So,
\[
4k^2 = b^3.
\]
Hence, $b^3$ is even and by Exercise~(1), $b$ is even. So, there exists an integer $m$ such that $b = 2m$.  Since $a^2 = b^3$, we see that $4k^2 = 8m^3$.  Hence, $k^2$ is even.  By 
Theorem~\ref{T:n2isodd}, $k$ is even.  So, there exists an integer $q$ such that $k = 2q$ and since $a = 2k$, we see that $a = 4q$.  Hence, 4 divides $a$.

\item Since 4 divides $a$, there exist an integer $n$ such that $a = 4n$.  Using this, we see that $b^3 = 16n^2$. So, $b^3$ is even and hence $b$ is even and there exists an integer $m$ such that $b = 2m$.  This implies that
\[
\begin{aligned}
8m^3 &= 16n^2 \\
m^3 &= 2n^2 \\
\end{aligned}
\]
Hence, $m^3$ is even and so by Exercise~(1), $m$ is even.  Since $b = 2m$, we see that 4 divides $b$.

\item If 4 divides $b$, then since $a^2 = b^3$, we conclude that $a^2$ is even.  But this implies that $a$ is even (Theorem~\ref{T:n2isodd}).  Hence, by Part~(a), 4 divides $a$.  So, there exist integers $s$ and $t$ such that $b = 4s$ and $a = 4t$.  Hence,
\[
\begin{aligned}
16t^2 &= 64s^3 \\
t^2 &= 4s^3 \\
\end{aligned}
\]
So, $t^2$ is even and hence $t$ must be even.  Since $a = 4t$, we conclude that 8 divides $a$.

\item Use Parts~(a), (b), and~(c).

\item $a = 8$, $b = 4$.
\end{enumerate}

\item Let $a$ and $b$ be integers with $a \ne 0$.  We will prove the contrapositive.  So assume that the equation $ax^3 + bx + \left( b + a \right) = 0$ has a solution that is an natural number.  Let $n$ be a natural number that is a solution of this equation.  Then
\[
\begin{aligned}
an^3 + bn + \left( b + a \right) &= 0 \\
an^3 + a &= -bn - b \\
a \left(n + 1\right) \left(n^2 - n + 1 \right) &= -b \left( n + 1 \right) \\
a \left( n^2 - n + 1 \right) &= -b \\
\end{aligned}
\]
The last equation can be used to conclude that $a \mid b$, and this completes the proof of the contrapositive.
\end{enumerate}


\subsection*{Evaluation of Proofs}
\setcounter{oldenumi}{\theenumi}
\begin{enumerate} \setcounter{enumi}{\theoldenumi}
\item \begin{enumerate}
\item The proposition is true, but the proof is not a valid proof.  Basically, the proof begins with the conclusion of the proposition and proceeds to prove the hypothesis.  This is essentially a proof of the converse of the proposition.

In addition, the assumptions for the proposition are not stated at the beginning of the proof and there is no indication of what will be proven.  In particular, the variable $m$ is not defined in the body of the proof.  
Following is a valid proof of this proposition.

\begin{myproof}
Let $m$ be an odd integer.  We will prove that $m + 6$ is an odd integer.  Since $m$ is odd, there exists an integer $n$ such that
\[
m = 2n + 1.
\]
By adding 6 to both sides of this equation, we obain 
\[
\begin{aligned}
m +6 &= (2n + 1) + 6 \\
     &= (2n + 6) + 1 \\
     &= 2(n + 3) + 1.
\end{aligned}
\]
By the closure properties of the integers, $(n + 3)$ is an integer, and hence, the last equation implies that $m + 6$ is an odd integer.  This proves that if $m$ is an odd integer, then $m+6$ is an odd integer.
\end{myproof}


\item The proposition is true, but the proof is not a valid proof.  Basically, the proof begins with the conclusion of the proposition and proceeds to prove the hypothesis.  This is essentially a proof of the converse of the proposition.

In addition, the assumptions for the proposition are not stated at the beginning of the proof and there is no indication of what will be proven.  In particular, the variables $m$ and $n$ are not defined in the body of the proof.

It is important to realize that a direct proof of this proposition is really not possible.  It is better to prove the contrapositive of this proposition as follows:

\begin{myproof}
We will prove the contrapositive of this proposition.  So we will prove,

\begin{list}{}
\item For all integers $m$ and $n$, if $m$ is an odd integer and $n$ is an odd integer, then $mn$ is an odd integer.
\end{list}
\vskip6pt
\noindent
So we let $m$ and $n$ be odd integers.  We will prove that $mn$ is an odd integer.  Since $m$ and $n$ are odd, there exist integers $p$ and $q$ such that $m = 2p + 1$ and $n = 2q + 1$.  Using these equations, we see that
\begin{align*}
mn &= (2p + 1)(2q + 1) \\
   &= 4pq + 2p + 2q + 1 \\
   &= 2(2pq + p + q) + 1.
\end{align*}
Using the closure properties of the integers, we know that $(2pq + p + q)$ is an integer and, hence, the last equation implies that $mn$ is odd.  This proves the contrapositive of the proposition.  Therefore, we have proven that for all integers $m$ and $n$, if $mn$ is an even integer, then $m$ is even or $n$ is even.
\end{myproof}
\end{enumerate}
\end{enumerate}



\subsection*{Explorations and Activities}
\setcounter{oldenumi}{\theenumi}
\begin{enumerate} \setcounter{enumi}{\theoldenumi}
\item \begin{enumerate}
\item If we use the symbolic form  
$\left( {\mynot  Q \wedge \mynot  R} \right) \to \mynot  P$  as a model for this proposition, then  $P$  is, ``3 divides the product  $a \cdot b$'',  $Q$  is, ``3  divides  $a$'', and  $R$ is, 
``3 divides  $b$.''  

\item A  symbolic form for the contrapositive of   
$\left( {\mynot  Q \wedge \mynot  R} \right) \to \mynot  P$ is  $P \to \left( {Q \vee R} \right)$.

\item For all integers $a$  and  $b$, if  3  divides the product  $a \cdot b$, then  3  divides  $a$  or  3  divides  $b$.

\item \begin{enumerate}
  \item When  $a = 5$, we see that  $2 \cdot 3 + 5\left( { - 1} \right) = 1$.
  \item When  $a = 2$, we see that  $1 \cdot 3 + 2\left( { - 1} \right) = 1$.
  \item When $a =  - 2$, we see that  $1 \cdot 3 + \left( { - 2} \right)\left( 1 \right) = 1$.
\end{enumerate}

\item
For all integers $a$  and  $b$, if  3  divides the product  $a \cdot b$, then  3  divides  $a$  or  3  divides  $b$.

\begin{myproof}
Let  $a$  and  $b$  be integers.  We will prove this proposition by proving that if  3  divides the product  $a \cdot b$ and  3  does not divide  $a$, then  3  divides  $b$.

So, assume that   3  divides the product  $a \cdot b$  and  3  does not divide  $a$.  Since  3  divides  $a \cdot b$, there exists an integer  $m$  such that
\setcounter{equation}{0}
\begin{equation} \label{eq:act312a}
a b = 3m,
\end{equation}
%
and since  3  does not divide  $a$, there exist integers  $x$  and  $y$  such that
\begin{equation} \label{eq:act312b}
3x + ay = 1.
\end{equation}

If we multiply equation~(\ref{eq:act312b}) by  $b$, we obtain
\begin{equation} \label{eq:act312c}
3bx + aby = b.
\end{equation}

We can now use equation~(\ref{eq:act312a}) to substitute for  $a b$ in 
Equation~(\ref{eq:act312c}).  This gives:
\[
3bx + 3my = b.
\]
Factoring a  3  from the left side of this equation gives 
\[
3\left( {bx + my} \right) = b.
\]
By the closure properties of the integers, we know that  $bx + my$ is an integer, and hence we have proven that  3 divides  $b$.  So, we have proven that if  3  divides the product  
$a \cdot b$, then  3  divides  $a$  or  3  divides  $b$.
\end{myproof}
\end{enumerate}

\end{enumerate}


\hbreak

\endinput
