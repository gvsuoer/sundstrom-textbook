\subsection*{Introduction to Relations}
In Section~\ref{S:introfunctions}, we introduced the formal definition of a function from one set to another set.  The notion of a function can be thought of as one way of relating the elements of one set with those of another set (or the same set).  A function is a special type of \textbf{relation} in the sense that each element of the first set, the domain,  is ``related'' to exactly one element of the second set, the codomain. 
%In addition, in Section~\ref{S:inversefunctions}, we learned that a function $f:A \to B$ can be thought of as a set of ordered pairs that is a special type of subset of  $A \times B$.  
%We generalize this idea to make a formal definition of a relation from the set  $A$  to the set  $B$. 

This idea of relating the elements of one set to those of another set using ordered pairs is not restricted to functions.  For example, we may say that one integer, $a$, is related to another integer, $b$, provided that  $a$  is congruent to  $b$  modulo 3.  Notice that this relation of congruence modulo 3 provides a way of relating one integer to another integer.  However, in this case, an integer  $a$  is related to more than one other integer.  For example, since
\[
5 \equiv 5 \pmod 3\!, \quad 5 \equiv 2 \pmod 3\!, \quad \text{and} \quad 5 \equiv -1 \pmod 3\!,
\]
\noindent
we can say that 5 is related to 5, 5 is related to 2, and 5 is related to $-1$.  Notice that, as with functions, each relation of the form  
$a \equiv b \pmod 3$ involves two integers  $a$  and  $b$ and hence involves an ordered pair  
$\left( {a, b} \right)$, which is an element of  $\Z \times \Z$.  

\begin{defbox}{relation}{Let $A$ and $B$ be sets.  A \textbf{relation \emph{R} from the set}
\index{relation}  $\boldsymbol{A}$  \textbf{to the set}  $\boldsymbol{B}$  is a subset of  $A \times B$.  That is,  $R$ is a collection of ordered pairs where the first coordinate of each ordered pair is an element of $A$, and the second coordinate of each ordered pair is an element of  $B$.

\newpar
A relation from the set $A$ to the set $A$ is called a  \textbf{relation on the set \emph{A}}.  So a relation on the set $A$ is a subset of  $A \times A$.}
\end{defbox}

In Section~\ref{S:introfunctions}, we defined the domain and range of a function.  We make similar definitions for a relation.

\begin{defbox}{domrangeofrelation}{If  $R$ is a relation from the set  $A$  to the set  $B$, then the subset of  $A$  consisting of all the first coordinates of the ordered pairs in  $R$  is called the \textbf{domain}
\index{domain!of a relation}%
\index{relation!domain}%
 of  $R$.  The subset of  $B$  consisting of all the second coordinates of the ordered pairs in  $R$  is called the \textbf{range}
\index{range!of a relation}%
\index{relation!range}%
 of  $R$.  

\newpar
We use the notation $\text{dom}( R )$ for the domain of $R$ and  $\text{range}( R )$ for the range of $R$.  So using set builder notation, \label{sym:domrel}
\begin{align*}
  \text{dom} ( R )   &= \left\{ { {u \in A } \mid \left( {u, y} \right) \in R\text{ for at least one }y \in B} \right\}  \\ 
  \text{range} ( R ) &= \left\{ { {v \in B } \mid \left( {x, v} \right) \in R\text{ for at least one }x \in A} \right\}\!.  \label{sym:rangerel} \\ 
\end{align*}
}
\end{defbox}
%
\begin{example}[\textbf{Domain and Range}]\label{exam:relation} \hfill \\
A relation was studied in each of the \typel activities for this section.
For \typeu Activity~\ref*{PA:eqn2variables}, the set $S = \left\{ (x, y) \in \R \times \R \mid 4x^2 + y^2 = 16 \right\}\!$
 is a subset of $\R \times \R$ and, hence, $S$ is a relation on $\R$.  In Problem~(\ref{PA:eqn2variables3}) of 
\typeu Activity~\ref*{PA:eqn2variables}, we actually determined the domain and range of this relation.
\begin{align*}
\text{dom}(S) &= A = \left\{ x \in \R \mid \text{ there exists a } y \in \R \text{ such that } 4x^2 + y^2 = 16 \right\} \\
\text{range}(S) &= B = \left\{ y \in \R \mid \text{ there exists an } x \in \R \text{ such that } 4x^2 + y^2 = 16 \right\}
\end{align*}
So from the results in \typeu Activity~\ref*{PA:eqn2variables}, we can say that the domain of the relation $S$ is the closed interval $\left[ -2, 2 \right]$ and the range of $S$ is the closed interval $\left[ -4, 4 \right]$.
\end{example}
\hbreak
%
\begin{prog}[\textbf{Examples of Relations}]\label{A:relationexamples} \hfill
\begin{enumerate}
\item Let  
$T = \left\{ {\left( {x, y} \right) \in \mathbb{R} \times \mathbb{R}   \mid x^2  + y^2  = 64} \right\}$. 
\begin{enumerate}
  \item Explain why  $T$ is a relation on  $\mathbb{R}$.

  \item Find all values of  $x$  such that  $\left( {x, 4} \right) \in T$.  Find all values of $x$  such that  
$\left( {x, 9} \right) \in T$\!.

  \item What is the domain of the relation  $T$?  What is the range of  $T$?

  \item Since  $T$  is a relation on  $\mathbb{R}$, its elements can be graphed in the coordinate plane.  Describe the graph of the relation  $T$\!.
\end{enumerate}
\item From \typeu Activity~\ref*{PA:USA}, $A$  is the set of all states in the United States, and
\[
R = \left\{ { {\left( {x, y} \right) \in A \times A } \mid x\text{  and  }y\text{  have a land border in common}} \right\}\!.
\]
\begin{enumerate}
  \item Explain why  $R$  is a relation on  $A$.

  \item What is the domain of the relation  $R$?  What is the range of the relation  $R$?

  \item Are the following statements true or false?  Justify your conclusions.
  \begin{enumerate}
  \item For all $x, y \in A$, if $(x, y) \in R$, then $(y, x) \in R$.
  \item For all $x, y, z \in A$, if $(x, y) \in R$ and $(y, z) \in R$, then $(x, z) \in R$.
  \end{enumerate}
\end{enumerate}
\end{enumerate}
\end{prog}
\hbreak

\endinput
