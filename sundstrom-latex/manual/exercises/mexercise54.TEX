\section*{Section \ref{S:cartesian} Cartesian Products}

STOPPED HERE

\begin{enumerate}
\item \begin{enumerate}
\item $A \times B = \left\{ {\left( {1, a} \right), \left( {1, b} \right), \left( {1, c} \right), \left( {1, d} \right), \left( {2, a} \right), \left( {2, b} \right), \left( {2, c} \right), \left( {2, d} \right)} \right\}$.

\item $B \times A = \left\{ {\left( {a, 1} \right), \left( {b, 1} \right), \left( {c, 1} \right), \left( {d, 1} \right), \left( {a, 2} \right), \left( {b, 2} \right), \left( {c, 2} \right), \left( {d, 2} \right)} \right\}$.

\item $A \times C = \left\{ { \left( 1, 1 \right), \left( {1, a} \right), \left( {1, b} \right), \left( 2, 1 \right), \left( {2, a} \right), \left( {2, b} \right)} \right\}$.

\item $A^2 = \left\{ \left( 1, 1 \right), \left( 1, 2 \right), \left( 2, 1 \right), \left( 2, 2 \right) \right\}$.

\item $A \times \left( {B \cap C} \right) = \left\{ {\left( {1, a} \right), \left( {1, b} \right), \left( {2, a} \right), \left( {2, b} \right)} \right\}$.

\item $\left( A \times B \right) \cap \left( A \times C \right) = 
\left\{ {\left( {1, a} \right), \left( {1, b} \right), \left( {2, a} \right), \left( {2, b} \right)} \right\}$.

\item $A \times \emptyset = \emptyset$.

\item $B \times \left\{ 2 \right\} = \left\{ \left( a, 2 \right), \left( b, 2 \right), 
\left( c, 2 \right), \left( d, 2 \right) \right\}$.
\end{enumerate}

\item \begin{enumerate}
\item $\left[ 0, 2 \right] \times \left[ 1, 3 \right] = \left\{ \left( x, y \right) \in \mathbb{R} \times \mathbb{R} \mid 0 \leq x \leq 2, 1 \leq y \leq 3 \right\}$.

\item $\left( 0, 2 \right) \times \left[ 1, 3 \right] = \left\{ \left( x, y \right) \in \mathbb{R} \times \mathbb{R} \mid 0 < x < 2, 1 \leq y \leq 3 \right\}$.

\item $\left[ 2, 3 \right] \times \left\{ 1 \right\} = \left\{ \left( x, 1 \right) \in \mathbb{R} \times \mathbb{R} \mid 2 \leq x \leq 3 \right\}$.

\item $\left\{ 1 \right\} \times \left[ 2, 3 \right] = \left\{ \left( 1, y \right) \in \mathbb{R} \times \mathbb{R} \mid 2 \leq y \leq 3 \right\}$.

\item $\mathbb{R} \times \left( 2, 4 \right) = \left\{ \left( x, y \right) \in \mathbb{R} \times \mathbb{R} \mid 2 < y < 4 \right\}$.

\item $\left( 2, 4 \right) \times \mathbb{R} = \left\{ \left( x, y \right) \in \mathbb{R} \times \mathbb{R} \mid 2 < x < 4 \right\}$.

\item $\mathbb{R} \times \left\{ -1 \right\} = \left\{ \left( x, -1 \right) \mid x \in \mathbb{R} \right\}$.

\item $\left\{ -1 \right\} \times \left[ 1, +\infty \right] = \left\{ \left( -1, y \right) \in \mathbb{R} \times \mathbb{R} \mid y \geq 1 \right\}$.
\end{enumerate}

\item Let $u \in A \times \left( {B \cap C} \right)$. Then, there exists $x \in A$ and there exists  $y \in B \cap C$ such that  $u = \left( {x, y} \right)$.  Since  $y \in B \cap C$, we know that  $y \in B$ and  $y \in C$.  So,  we have:
\begin{list}{}
\item $u = \left( x, y \right) \in A \times B$ and  $u = \left( x, y \right) \in A \times C$.
\end{list}
Hence, $u \in \left( A \times B \right) \cap \left( A \times C \right)$ and this proves that 
$A \times \left( {B \cap C} \right) \subseteq \left( A \times B \right) \cap \left( A \times C \right)$.

Now let $v \in \left( A \times B \right) \cap \left( A \times C \right)$.  Then, 
$v \in A \times B$ and $v \in A \times C$.  So, there exists an $s$ in $A$ and a $t$ in $B$ such that $v = \left( s, t \right)$.  But, since $v$ is also in $A \times C$, we see that $t$ must also be in $C$.  Thus, $t \in B \cap C$ and so, $v \in A \times \left( B \cap C \right)$.  This proves that $\left( A \times B \right) \cap \left( A \times C \right) \subseteq A \times \left( B \cap C \right)$.

\item Let  $u \in \left( {A \cup B} \right) \times C$. Then, there exists $x \in A \cup B$ and there exists  $y \in C$ such that  $u = \left( {x, y} \right)$.  Since  $x \in A \cup B$, we know that  $x \in A$  or  $x \in B$.  In the case where $x \in A$, we see that $u \in A \times C$, and in the case where $x \in B$, we see that $u \in B \times C$.  Hence, 
$u \in \left( A \times C \right) \cup \left( B \times C \right)$.  This proves that 
$\left( {A \cup B} \right) \times C \subseteq \left( {A \times C} \right) \cup \left( {B \times C} \right)$.

Now let $v \in \left( {A \times C} \right) \cup \left( {B \times C} \right)$.  Then, 
$v \in A \times C$ or $v \in B \times C$.  In the case where $v \in A \times C$, there exist 
$s \in A$, $t \in C$ such that $v = \left( s, t \right)$.  However, since $s \in A$, we know that $s \in A \cup B$.  Hence, $v \in \left( A \cup B \right) \times C$.  In a similar manner, we can show that if $v \in B \times C$, then $v \in \left( A \cup B \right) \times C$.  This proves that 
$\left( {A \times C} \right) \cup \left( {B \times C} \right) \subseteq \left( {A \cup B} \right) \times C$.

\item Let $u \in A \times \left( {B - C} \right)$. Then, there exists $x \in A$ and there exists  $y \in B - C$ such that  $u = \left( {x, y} \right)$.  Since  $y \in B - C$, we know that  
$y \in B$ and  $y \notin C$.  So,  we have:
\begin{list}{}
\item $u = \left( x, y \right) \in A \times B$ and  $u = \left( x, y \right) \notin A \times C$.
\end{list}
Hence, $u \in \left( A \times B \right) - \left( A \times C \right)$ and this proves that 
$A \times \left( {B - C} \right) \subseteq \left( A \times B \right) - \left( A \times C \right)$.

Now let $v \in \left( A \times B \right) - \left( A \times C \right)$.  Then, 
$v \in A \times B$ and $v \notin A \times C$.  So, there exists an $s$ in $A$ and a $t$ in $B$ such that $v = \left( s, t \right)$.  But, since $v$ is not in $A \times C$, we see that $s$ is not in $A$ or $t$ is not in $C$.  But we already know that $s \in A$.  Therefore, $t \notin C$.  Thus, $t \in B - C$ and so, $v \in A \times \left( B - C \right)$.  This proves that 
$\left( A \times B \right) - \left( A \times C \right) \subseteq A \times \left( B - C \right)$.

\item Assume $T \subseteq A$, and let $u \in T \times B$.  Then, there exists $t \in T$ and there exists $b \in B$ such that $u = \left( t, b \right)$.  Since $T \subseteq A$, we see that 
$t \in A$ and hence, $u \in A \times B$.  Therefore, $T \times B \subseteq A \times B$.

\item \begin{enumerate}
\item $A \times B = \left\{ \left( 1, 2 \right) \right\}$ and 
$B \times A = \left\{ \left( 2, 1 \right) \right\}$.  Since 
$\left( 1, 2 \right) \ne \left( 2, 1 \right)$, we see that $A \times B \ne B \times A$.

\item $\left( A \times B \right) \times C = \left\{ \left( \left( 1, 2 \right), 3 \right) \right\}$ and
$A \times \left( B \times C \right) = \left\{ \left( 1, \left( 2, 3 \right) \right) \right\}$.
\end{enumerate} 

\item Assume $A = B$.  Then $A \times B = A \times A = B \times A$.  Now assume that 
$A \times B = B \times A$, and let $x \in A$.  Since $B \ne \emptyset$, there exists a $b$ in $B$.  This means that $\left( x, b \right) \in A \times B$, and hence 
$\left( x, b \right) \in B \times A$.  This means that $x \in B$ (and $b \in A$).  Therefore, 
$A \subseteq B$.  We prove that $B \subseteq A$ in a similar manner.

\item Assume that $A \times B = A \times C$ and that $A \ne \emptyset$.  Let $x \in B$.  Since 
$A \ne \emptyset$, there exists an element $a$ in $A$.  Then, 
$\left( a, x \right) \in A \times B$, and hence, $\left( a, x \right) \in A \times C$.  Therefore, 
$x \in C$ and hence, $B \subseteq C$.  The fact that $C \subseteq B$ can be proven in a similar manner.
\end{enumerate}



\subsection*{Explorations and Activities}
\setcounter{oldenumi}{\theenumi}
\begin{enumerate} \setcounter{enumi}{\theoldenumi}
\item \begin{enumerate}
\item Notice that $\left( {3, 5} \right) = \left\{ {\left\{ 3 \right\},\left\{ {3, 5} \right\}} \right\}$ and that  $\left( {5, 3} \right) = \left\{ {\left\{ 5 \right\},\left\{ {5, 3} \right\}} \right\}$.  Each of these ordered pairs is a set whose elements are sets.  In particular,
\begin{center}
$\left\{ 3 \right\} \in \left( {3, 5} \right)$ and 
$\left\{ 3 \right\} \notin \left( {5, 3} \right)$.

$\left\{ 5 \right\} \notin \left( {3, 5} \right)$ and 
$\left\{ 5 \right\} \in \left( {5, 3} \right)$.
\end{center}
Hence, as sets, $\left( {3, 5} \right) \ne \left( {5, 3} \right)$.

\item Let  $A$  and  $B$  be sets and let  $a,c \in A$  and  $b,d \in B$.  We will prove that  $\left( {a,b} \right) = \left( {c,d} \right)$  if and only if  $a = c$ and $b = d$.

First, assume that $a = c$ and $b = d$.  Then, $\left\{ a \right\} = \left\{ c \right\}$ and 
$\left\{ a, b \right\} = \left\{ c, d \right\}$.  This means that 
$\left\{ {\left\{ a \right\},\left\{ {a, b} \right\}} \right\} = 
\left\{ {\left\{ c \right\},\left\{ {c, d} \right\}} \right\}$ and hence, 
$\left( {a,b} \right) = \left( {c,d} \right)$.

Now assume that $\left( {a,b} \right) = \left( {c,d} \right)$.  Then, 
$\left\{ {\left\{ a \right\},\left\{ {a, b} \right\}} \right\} = 
\left\{ {\left\{ c \right\},\left\{ {c, d} \right\}} \right\}$, and hence
\[
\left\{ a \right\} \in \left\{ {\left\{ c \right\},\left\{ {c, d} \right\}} \right\}.
\]
In the case where $c = d$, then  $a$ must be equal to $c$. In addition,
\[
\left\{ a, b \right\} \in \left\{ {\left\{ c \right\},\left\{ {c, d} \right\}} \right\},
\]
and hence, $a = b = c = d$.  

In the case where $c \ne d$, then 
$\left\{ a \right\} \in \left\{ {\left\{ c \right\},\left\{ {c, d} \right\}} \right\}$ implies that $\left\{ a \right\} = \left\{ c \right\}$ and hence, $a = c$.  Now, $a \ne b$ since if 
$a = b$, then the set $\left\{ {\left\{ a \right\},\left\{ {a, b} \right\}} \right\}$ would only contain one set as an element and the equal set 
$\left\{ {\left\{ c \right\},\left\{ {c, d} \right\}} \right\}$ would contain two sets as elements since $c \ne d$.  Because the two ordered pairs are equal, we conclude that 
$\left\{ a, b \right\} = \left\{ c, d \right\}$.  Since we have already proven that $a = c$, we may conclude that $b = d$.  This completes the proof.

\item Let $A$, $B$, and $C$ be sets and let $x \in A$, $y \in B$, and $z \in C$.  The \textbf{ordered triple} $(x, y, z)$ is defined to be the set $\left\{ \left\{ x \right\}, \left\{ x, y \right\}, \left\{ x, y, z \right\} \right\}$.  That is,
\[
\left( x, y, z \right) = 
\left\{ \left\{ x \right\}, \left\{ x, y \right\}, \left\{ x, y, z \right\} \right\}.
\]




\end{enumerate}


\end{enumerate}
\hbreak
\endinput
