\subsection*{The Euclidean Algorithm}
The example in Progress Check~\ref{prog:lemma81} illustrates the main idea of the 
\textbf{Euclidean Algorithm}
\index{Euclidean Algorithm}%
 for finding $\gcd( {a, b} )$, which is explained in the proof of the following theorem.


%\begin{prog}[Illustrations of Lemma~\ref{L:gcdanddivalgo}] \label{prog:lemma81} \hfill \\
%We completed several examples illustrating Lemma~\ref{L:gcdanddivalgo} in Beginning Activity~\ref{PA:gcdanddivalgo}.  For another example, let  $c = 54$  and  $d = 8$.  We can verify that  $\gcd( {54, 8} ) = 2$.
%
%Now, if we use the Division Algorithm for dividing 54 by 8, we obtain  $54 = 6 \cdot 8 + 6$.  So the remainder   $\left( r \right)$ when  54  is divided by  8  is 6.  Notice that  
%$\gcd( {8, 6} ) = 2$.  Hence
%\[
%\gcd( {54, 8} ) = \gcd( {8, 6} ).
%\]
%The key to finding the greatest common divisor (in more complicated cases) is to use the Division Algorithm again, this time with  8  and  $r$.  (In this case,  $r = 6$.)  We now find integers  $q_2 \text{ and }r_2 $ such that
%\[
%\begin{aligned}
%  8 &= r \cdot q_2  + r_2  \\
%  8 &= 6q_2  + r_2 . \\ 
%\end{aligned} 
%\]
%(The result should be  $r_2  = 2$.)  Notice that  
%$\gcd( {6, 2}) = 2 = \gcd( {54, 8} )$.
%
%This example illustrates the main idea of the \textbf{Euclidean Algorithm}
%\index{Euclidean Algorithm}%
% for finding $\gcd( {a, b} )$, which is explained in the proof of the following theorem.
%\end{prog}
%\hbreak
%
\setcounter{equation}{0}

\begin{theorem}\label{T:euclidalgo}
Let  $a$  and  $b$  be integers with $a \ne 0$  and  $b > 0$.  Then  
$\gcd( {a, b} )$ is the only natural number  $d$  such that
\renewcommand{\labelenumi}{(\textbf{\alph{enumi}})}
\renewcommand{\theenumi}{({\alph{enumi}})}
\begin{enumerate}
\item \label{T:euclidalgo1} $d$ divides  $a$ and  $d$ divides  $b$, and  %\label{T:euclidalgo2}
\item \label{T:euclidalgo2}if  $k$  is an integer that divides both $a$  and  $b$, then  $k$  divides  $d$.  
\end{enumerate}
\end{theorem}
%

\begin{myproof}
Let  $a$  and  $b$  be integers with $a \ne 0$ and  $b > 0$, and let  
$d = \gcd( {a, b} )$.  By the Division Algorithm, there exist integers  $q_1 $ and  $r_1 $  such that
\begin{equation} \label{eq:euclidalgo1}
a = b \cdot q_1  + r_1 \text{,  and  }0 \leq r_1  < b.
\end{equation}
If  $r_1  = 0$, then equation~(\ref{eq:euclidalgo1}) implies that  $b$  divides  $a$.  Hence,  $b = d = \gcd( {a, b} )$ and  this number satisfies  Conditions~\ref{T:euclidalgo1} and~\ref{T:euclidalgo2}.

If  $r_1  > 0$, then by Lemma~\ref{L:gcdanddivalgo},  
$\gcd( {a, b} ) = \gcd( {b, r_1 } )$.  We use the Division Algorithm again to obtain integers  $q_2 $  and  $r_2 $  such that
\begin{equation} \label{eq:euclidalgo2}
b = r_1  \cdot q_2  + r_2 \text{,  and  }0 \leq r_2  < r_1.
\end{equation}
If  $r_2  = 0$, then equation~(\ref{eq:euclidalgo2}) implies that  $r_1 $ divides  $b$.  This means that  $r_1  = \gcd( {b, r_1 } )$.  But we have already seen that  
$\gcd( {a, b} ) = \gcd( {b, r_1 } )$.  Hence, 
$r_1  = \gcd( {a, b} )$.  In addition, if  $k$  is an integer that divides both  $a$  and  $b$, then, using equation~(\ref{eq:euclidalgo1}), we see that  $r_1  = a - b \cdot q_1 $ and, hence  $k$  divides  $r_1 $.  This shows that  $r_1  = \gcd( {a, b} )$  satisfies Conditions~\ref{T:euclidalgo1} and \ref{T:euclidalgo2}.

If  $r_2  > 0$, then by Lemma~\ref{L:gcdanddivalgo},  
$\gcd( {b, r_1 } ) = \gcd( {r_1 , r_2 } )$.  But we have already seen that  $\gcd( {a, b} ) = \gcd( {b, r_1 } )$.  Hence, 
$\gcd( {a, b} ) = \gcd( {r_1 , r_2 } )$.  We now continue to apply the Division Algorithm to produce a sequence of pairs of  integers (all of which have the same greatest common divisor).  This is summarized in the following table:

%\begin{center}
%\begin{tabular}[h]{| l | l | l | l |}
%  \hline
%  \textbf{Original}  &  \textbf{Equation from}  &  \textbf{Inequality from}  &  \textbf{New}    \\
%  \textbf{Pair}     &  \textbf{Division Algorithm}  &  \textbf{Division Algorithm}  &  \textbf{Pair}  \\ \hline
%$\left( {a, b} \right)$  &  	$a = b \cdot q_1  + r_1 $  &  $0 \leq r_1  < b$  &  	$\left( {b, r_1 } \right)$  \\ \hline
%$\left( {b, r_1 } \right)$  &  $b = r_1  \cdot q_2  + r_2 $  &  $0 \leq r_2  < r_1 $  &
%	$\left( {r_1 , r_2 } \right)$  \\ \hline
%$\left( {r_1 , r_2 } \right)$  &  $r_1  = r_2  \cdot q_3  + r_3 $  &  $0 \leq r_3  < r_2 $
%  &	$\left( {r_2 , r_3 } \right)$  \\ \hline
%$\left( {r_2 , r_3 } \right)$  &  $r_2  = r_3  \cdot q_4  + r_4 $  &  $0 \leq r_4  < r_3 $
%  &	$\left( {r_3 , r_4 } \right)$  \\ \hline
%$\left( {r_3 , r_4 } \right)$  &  $r_3  = r_4  \cdot q_5  + r_5 $  &  $0 \leq r_5  < r_4 $
%  &	$\left( {r_4 , r_5 } \right)$  \\ \hline
%$ \vdots $  &  $ \vdots $  &  $ \vdots $  &  $ \vdots $  \\ \hline
%\end{tabular}
%\end{center}
$$
\BeginTable
\BeginFormat
|  l  |  l  |  l  |  l  |
\EndFormat
\_
| \textbf{Original}  |  \textbf{Equation from}  |  \textbf{Inequality from}  |  \textbf{New} | \\+20 
| \textbf{Pair}     |  \textbf{Division Algorithm}  |  \textbf{Division Algorithm}  |  \textbf{Pair} | \\+02 \_
| $\left( {a, b} \right)$  |  	$a = b \cdot q_1  + r_1 $  |  $0 \leq r_1  < b$  |  	$\left( {b, r_1 } \right)$ | \\ \hline
| $\left( {b, r_1 } \right)$  |  $b = r_1  \cdot q_2  + r_2 $  |  $0 \leq r_2  < r_1 $  |
	$\left( {r_1 , r_2 } \right)$ | \\ \_
| $\left( {r_1 , r_2 } \right)$  |  $r_1  = r_2  \cdot q_3  + r_3 $  |  $0 \leq r_3  < r_2 $
  |	$\left( {r_2 , r_3 } \right)$ | \\ \_
| $\left( {r_2 , r_3 } \right)$  |  $r_2  = r_3  \cdot q_4  + r_4 $  |  $0 \leq r_4  < r_3 $
  |	$\left( {r_3 , r_4 } \right)$ | \\ \_
| $\left( {r_3 , r_4 } \right)$  |  $r_3  = r_4  \cdot q_5  + r_5 $  |  $0 \leq r_5  < r_4 $
  |	$\left( {r_4 , r_5 } \right)$ | \\ \_
| $ \vdots $  |  $ \vdots $  |  $ \vdots $  |  $ \vdots $ | \\ \_
\EndTable
$$
From the inequalities in the third column of this table, we have a strictly decreasing sequence of nonnegative integers  $\left( {b > r_1  > r_2  > r_3  > r_4  \cdots } \right)$.  Consequently, a term in this sequence must eventually be equal to zero.  Let  $p$  be the smallest natural number such that  $r_{p + 1}  = 0$.  This means that the last two rows in the preceding table will be
$$
\BeginTable
\BeginFormat
|  l  |  l  |  l  |  l  |
\EndFormat
\_
| \textbf{Original}  |  \textbf{Equation from}  |  \textbf{Inequality from}  |  \textbf{New} | \\+20 
| \textbf{Pair}     |  \textbf{Division Algorithm}  |  \textbf{Division Algorithm}  |  \textbf{Pair} | \\+02 \_
| $\left( {r_{p - 2} , r_{p - 1} } \right)$  |  $r_{p - 2}  = r_{p - 1}  \cdot q_p  + r_p $  |
  $0 \leq r_p  < r_{p - 1} $  |  $\left( {r_{p - 1} , r_p } \right)$ | \\+22 \_
| $\left( {r_{p - 1} , r_p } \right)$  |  $r_{p - 1}  = r_p  \cdot q_{p + 1}  + 0$  |  | | 
\\+22 \_
\EndTable
$$
%\begin{center}
%\begin{tabular}[h]{| l | l | l | l |}
%  \hline
%  \textbf{Original}  &  \textbf{Equation from}  &  \textbf{Inequality from}  &  \textbf{New}    \\
%  \textbf{Pair}     &  \textbf{Division Algorithm}  &  \textbf{Division Algorithm}  &  \textbf{Pair}  \\ \hline
%$\left( {r_{p - 2} , r_{p - 1} } \right)$  &  $r_{p - 2}  = r_{p - 1}  \cdot q_p  + r_p $  &
%  $0 \leq r_p  < r_{p - 1} $  &  $\left( {r_{p - 1} , r_p } \right)$  \\ \hline
%$\left( {r_{p - 1} , r_p } \right)$  &  $r_{p - 1}  = r_p  \cdot q_{p + 1}  + 0$  &  &  \\ \hline
%\end{tabular}
%\end{center}
Remember that this table was constructed by repeated use of Lemma~\ref{L:gcdanddivalgo} and that the greatest common divisor of each pair of integers produced equals 
$\gcd( {a, b} )$.  Also, the last row in the table indicates that  $r_p $  divides  $r_{p - 1} $.  This means that  $\gcd( {r_{p - 1} , r_p } ) = r_p $ and hence  
$r_p  = \gcd( {a, b} )$.

This proves that  $r_p  = \gcd( {a, b} )$ satisfies Condition~\ref{T:euclidalgo1} of this theorem.  Now assume that  $k$  is an integer such that  $k$  divides  $a$  and  $k$  divides  $b$.  We proceed through the table row by row.  First,  since  $r_1  = a - b \cdot q$, we see that  
\begin{center}
$k$  must divide  $r_1 $.
\end{center}
The second row tells us that  $r_2  = b - r_1  \cdot q_2 $.  Since  $k$  divides  $b$  and  $k$ divides  $r_1 $, we conclude that  
\begin{center}
$k$  divides  $r_2 $.
\end{center}
Continuing with each row, we see that  $k$  divides each of the remainders  
$r_1$, $r_2$, $r_3$, $\ldots, r_p$.  This means that  
$r_p  = \gcd( {a, b} )$ satisfies Condition~\ref{T:euclidalgo2} of the theorem.  
\end{myproof}
%

%\renewcommand{\theenumi}{{\arabic{enumi}}}

\hrule
%
\begin{prog}[\textbf{Using the Euclidean Algorithm}]\label{prog:usingeuclid} \hfill
\begin{enumerate}
\item Use the Euclidean Algorithm to determine  $\gcd( {180, 126} )$.  Notice that we have deleted the third column (Inequality from Division Algorithm) from the following table.  It is not needed in the computations.

\begin{center}
\begin{tabular}[h]{| l | l | l | l |}
  \hline
  \textbf{Original}  &  \textbf{Equation from}    &  \textbf{New}    \\
  \textbf{Pair}     &  \textbf{Division Algorithm} &  \textbf{Pair}  \\ \hline
$\left( {180, 126} \right)$  &  	$180 = 126 \cdot 1  + 54 $  &  $\left( {126, 54} \right)$  \\ \hline
$\left( {126, 54} \right)$  &  $126 = $   &      \\ \hline
                            &             &	  \\ \hline
\end{tabular}
\end{center}
Consequently,  $\gcd( {180, 126} ) = \underline{\qquad \qquad}$.

\item Use the Euclidean Algorithm to determine $\gcd( {4208, 288} )$.
\begin{center}
\begin{tabular}[h]{| l | l | l | l |}
  \hline
  \textbf{Original}  &  \textbf{Equation from}    &  \textbf{New}    \\
  \textbf{Pair}     &  \textbf{Division Algorithm} &  \textbf{Pair}  \\ \hline
$\left( {4208, 288} \right)$  &  	$4208 = 288 \cdot 14  + 176 $  &  $\left( {288, \qquad} \right)$  \\ \hline
                            &             &	  \\ \hline
\end{tabular}
\end{center}
Consequently,  $\gcd( {4208, 288} ) = \underline{\qquad \qquad}$.
\end{enumerate}
\end{prog}
\hbreak

\subsection*{Some Remarks about Theorem~\ref{T:euclidalgo}}
Theorem~\ref{T:euclidalgo} was proven with the assumptions that  $a, b \in \mathbb{Z}$  with  
$a \ne 0$  and  $b > 0$.  A more general version of this theorem can be proven with  
$a, b \in \mathbb{Z}$ and  $b \ne 0$.  This can be proven using Theorem~\ref{T:euclidalgo} and the results in the following lemma.

\begin{lemma} \label{L:moregcd}
Let $a, b \in \mathbb{Z}$ with  $b \ne 0$.  Then

\begin{enumerate}
\item $\gcd( {0, b} ) = \left| b \right|$.

\item If  $\gcd( {a, b} ) = d$, then $\gcd( {a,  - b} ) = d$.

\end{enumerate}
\end{lemma}

\noindent
The  proofs of these results are in Exercise~(\ref{exer:sec81-props}).  An application of this result is given in the next example.

\begin{example}[\textbf{Using the Euclidean Algorithm}]\label{E:euclidalgo} \hfill \\
Let  $a = 234$  and  $b =  - 42$.  We will use the Euclidean Algorithm to determine  
$\gcd( {234, 42} )$.

\begin{center}
\begin{tabular}[h]{| c | l | l | l |}
  \hline
\textbf{Step}  &  \textbf{Original}  &  \textbf{Equation from}    &  \textbf{New}    \\
 &  \textbf{Pair}     &  \textbf{Division Algorithm} &  \textbf{Pair}  \\ \hline
1 & $\left( {234, 42} \right)$  &  $234 = 42 \cdot 5 + 24$ &  $\left( {42, 24} \right)$  \\ \hline
2 & $\left( {42, 24} \right)$  &	$42 = 24 \cdot 1 + 18$  &  $\left( {24, 18} \right)$  \\ \hline
3 & $\left( {24, 18} \right)$  &	$24 = 18 \cdot 1 + 6$   &  $\left( {18, 6} \right)$   \\ \hline
4 & $\left( {18, 6} \right)$   &	$18 = 6 \cdot 3$        &                             \\ \hline
\end{tabular}
\end{center}

\noindent
So  $\gcd( {234, 42} ) = 6$ and hence $\gcd( {234,  - 42} ) = 6$.
\end{example}
\hbreak

\endinput
