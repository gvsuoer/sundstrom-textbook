\begin{previewactivity}[\textbf{Thinking about a Proof}]\label{PA:thinking} \hfill \\
Consider the following proposition:  
\begin{flushleft}
\textbf{Proposition.}  If  $x$  and  $y$  are odd integers, then $ x \cdot y$  is an odd integer.
\end{flushleft}
Think about how you might go about proving this proposition. A \textbf{direct proof}
\index{direct proof}%
\index{proof!direct}%
 of a conditional statement is a demonstration that the conclusion of the conditional statement follows logically from the hypothesis of the conditional statement.  Definitions and previously proven propositions are used to justify each step in the proof.    To help get started in proving this proposition, answer the following questions:
\begin{enumerate}
  \item The proposition is a conditional statement.  What is the hypothesis of this conditional statement?  What is the conclusion of this conditional statement?
  \item If  $x = 2$ and  $y = 3$, then  $x \cdot y = 6$, and 6 is an even integer.  Does this example prove that the proposition is false?  Explain.
  \item If  $x = 5$ and  $y = 3$, then  $x \cdot y = 15$.  Does this example prove that the proposition is true?  Explain.
\end{enumerate}
In order to prove this proposition, we need to prove that whenever both $x$ and $y$ are odd integers, 
$x \cdot y$ is an odd integer.  Since we cannot explore all possible pairs of integer values for $x$ and $y$, we will use the definition of an odd integer to help us construct a proof.  

\begin{enumerate} \setcounter{enumi}{3}
  \item To start a proof of this proposition, we will assume that the hypothesis of the conditional statement is true.  So in this case, we assume that both $x$ and $y$ are odd integers.  We can then use the definition of an odd integer to conclude that there exists an integer $m$ such that $x = 2m + 1$.  Now use the definition of an odd integer to make a conclusion about the integer $y$.
\label{PA:prev12-Q4}

\note The definition of an odd integer says that a certain other integer exists.  This definition may be applied to both $x$ and $y$.  However, do not use the same letter in both cases.  To do so would imply that 
$x = y$ and we have not made that assumption.  To be more specific, if $x = 2m + 1$ and $y = 2m +1$, then 
$x = y$.
  \item We need to prove that if the hypothesis is true, then the conclusion is true.  So, in this case, we need to prove that $x \cdot y$ is an odd integer.  At this point, we usually ask ourselves a so-called 
\textbf{backward question}.  In this case, we ask, ``Under what conditions can we conclude that $x \cdot y$ is an odd integer?''  Use the definition of an odd integer to answer this question.% and be careful to use a different letter for the new integer than was used in Part~(\ref{PA:prev12-Q4}).
\end{enumerate}
\hbreak
\end{previewactivity}


\endinput
