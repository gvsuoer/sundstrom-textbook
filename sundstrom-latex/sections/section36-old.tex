\section{Constructive Proofs} \label{S:constructive}
%\markboth{Chapter~\ref{C:proofs}. Constructing Proofs}{\ref{S:constructive}. Constructive Proofs}
\setcounter{previewactivity}{0}
%
\begin{previewactivity}[Systems of Linear Equations] \label{PA:linearsystems} \hfill

If $x$ and $y$ represent real numbers, solve each of the following systems of two linear equations in two unknowns:
\begin{align} 
  \textbf{1.  } 3x+5y &= 1   &  \textbf{2.  } 2x-5y &=1  &  \textbf{3.  } 2x-5y &=1 \notag  \\
                7x-y  &= 15  &               -4x+10y &= 2 &              -4x+10y &=-2 \notag  \\
\notag
\end{align}
\hbreak
\end{previewactivity}
%
\begin{previewactivity}[Intersections of Circles] \label{PA:circles} \hfill

Let $a, b \in \mathbb{R} $ and let $r$ be a positive real number.  Then the graph of an equation of the form 
\[
\left( {x-a} \right)^2 + \left( {y-b} \right)^2 = r^2
\]
is a circle with center at the point $ \left( a,b \right)$ and radius $r$.
\begin{enumerate}
\item Using an appropriate pair of coordinate axes, carefully draw the graphs of the circles given by the following two equations. 
\[
\begin{aligned}
                       x^2  + y^2  &= 4 \\ 
  \left( {x - 3} \right)^2  + y^2  &= 4 \\ 
\end{aligned}
\]
Then solve the system of two equations in two unknowns to find all points of intersection of the two circles.

\item Using an appropriate pair of coordinate axes, carefully draw the graphs of the circles given by the following two equations.  
\[
\begin{aligned}
                       x^2  + y^2  &= 4 \\ 
  \left( {x - 5} \right)^2  + y^2  &= 4 \\ 
\end{aligned}
\]
Then solve the system of two equations in two unknowns to find all points of intersection of the two circles.
\end{enumerate}
\hbreak

\end{previewactivity}

\endinput

  

%
Preview Activity~\ref{PA:linearsystems} was intended to be a review of methods of solving a system of two linear equations in two variables.  The first system of equations has a unique solution $\left( {x = 2,y =  - 1} \right)$, the second system of equations has no solution, and the third system of equations has infinitely many solutions.  These solutions can be expressed in parametric form as
\[
x = t,y = \frac{2}{5}t - \frac{1}{5}.
\]
We are going to focus mainly on systems of two linear equations in two variables similar to the first system of equations in Preview Activity~\ref{PA:linearsystems}.

For a more general situation, we are going to let  $a$, $b$, $c$, $d$, $r$, and $s$  be real numbers and assume that  $ad - bc \ne 0$.  The question we are going to investigate is this:  Under these conditions, is it possible to find a solution for  $x$  and  $y$  for the following system of equations?
%
\setcounter{equation}{0}
\begin{align}
  ax + by &= r \notag \\
  cx + dy &= s 
  \label{eq:linear}
\end{align} 
\hbreak
%
\begin{activity}[A General System of Linear Equations] \label{A:linearsystem} \hfill
\begin{enumerate}
\item For the system of equations given in Equation~(\ref{eq:linear}) (with $ad - bc \ne 0$), multiply the first equation by  $d$  and multiply the second equation by  $b$.

\item For the two equations obtained in  Part (1), subtract the second equation from the first equation and then solve the resulting equation for  $x$.  What assumption was needed in order to solve this equation for  $x$?
\end{enumerate}
\end{activity}
%
This activity was intended to show that we can find a solution for  $x$.  We can do similar work to find a solution for  $y$.  This will be done in Theorem~\ref{T:linearsystem}.
\vskip10pt
\noindent
\underline{Note}:  In linear algebra class, we say that the system of linear equations \emph{has a solution} 
$\left( x, y \right)$.  That is, the solution is an ordered pair of real numbers.
\hbreak
%
\begin{theorem} \label{T:linearsystem}
If  $a,b,c,d,r,\text{and }s$ are real numbers with  $ad - bc \ne 0$, then for the system of equations
\[
\begin{aligned}
  ax + by & = r  \\
  cx + dy & = s,  \\ 
\end{aligned}
\]
there exists an ordered pair $\left( x, y \right)$ that makes both equations simultaneously true.
\end{theorem}
%
\setcounter{equation}{0}
\begin{myproof}
Let  $a,b,c,d,r,\text{and }s$ be real numbers and assume that  $ad - bc \ne 0$.  Now consider the following system of equations:
\begin{align}
ax + by &= r \label{eq:sys1} \\
cx + dy &= s.  \label{eq:sys2}
\end{align}
%
We will prove that there exists an ordered pair $\left( x, y \right)$ such that both equations are simultaneously true by actually constructing the values for $x$ and $y$.  
In Activity~\ref{A:linearsystem}, we found a value for $x$ that can be written as
\begin{equation}
x = \frac{{dr - bs}}{{ad - bc}}. \label{eq:sys3}
\end{equation}
%
We will now find a value for $y$ in a manner similar to the one used in 
Activity~\ref{A:linearsystem}.  By multiplying Equation~(\ref{eq:sys2}) by $a$ and Equation~(\ref{eq:sys1}) by $c$, and then subtracting the two equations, we obtain
\begin{equation}
\left( {ad - bc} \right)y = as - cr. \label{eq:sys4}
\end{equation}
%
From the hypothesis, $\left( {ad - bc} \right) \ne 0$.  Hence, dividing both sides of 
Equation~(\ref{eq:sys4}) by $\left( {ad - bc} \right)$, we see that
%
\begin{equation}
y = \frac{{as - cr}}{{ad - bc}}. \label{eq:sys5}
\end{equation}
%
We now claim that the ordered pair 
$\left( x, y \right) = \left(\dfrac{{dr - bs}}{{ad - bc}}, \dfrac{{as - cr}}{{ad - bc}} \right)$ simultaneously satisfies both equations.

To verify this claim, we substitute the values of $x$ and $y$ given in Equations~(\ref{eq:sys3}) and~(\ref{eq:sys5}) into the left side of Equation~(\ref{eq:sys1}).  This gives
\[
\begin{aligned}
  ax + by &= a \left( {\frac{{dr - bs}}{{ad - bc}}} \right) + b\text{ }\left( {\frac{{as - cr}}{{ad - bc}}} \right) \\ 
   &= \frac{{a(dr - bs) + b(as - cr)}}{{ad - bc}} \\ 
   &= \frac{{adr - abs + abs - bcr}}{{ad - bc}} \\ 
   &= \frac{{adr - bcr}}{{ad - bc}} \\ 
   &= \frac{{r(ad - bc)}}{{ad - bc}} \\ 
   &= r. \\ 
\end{aligned} 
\]
%
This shows that the values for $x$ and $y$ given in Equations~(\ref{eq:sys3}) and~(\ref{eq:sys5}) satisfy Equation~(\ref{eq:sys1}).  A similar substitution shows that these values for $x$ and $y$ satisfy Equation~(\ref{eq:sys2}).  This completes the proof that there exists an ordered pair that makes both Equations~(\ref{eq:sys1}) and~(\ref{eq:sys2}) true simultaneously when 
$\left( {ad - bc} \right) \ne 0$.
\end{myproof}
\hbreak
%
\subsection*{Constructive Proofs}
The proof given for Theorem~\ref{T:linearsystem} is often called a \textbf{constructive proof.}
\index{constructive proof}%
\index{proof!constructive}%
  This is a technique that is often used to prove a so-called \textbf{existence theorem.}
\index{existence theorem}%
  The simplest form of an existence theorem is  
\begin{center}
There exists an $x$  such that  $P\left( x \right)$.
\end{center}
The symbolic form is  $\left( {\exists x} \right)\left( {P\left( x \right)} \right)$.  For a constructive proof of such a proposition, we actually name, describe, or explain how to construct  some object in the universe that makes  $P\left( x \right)$ true.  This is what we did in 
Theorem~\ref{T:linearsystem}.

The proposition in Theorem~\ref{T:linearsystem} has two variables ($x$ and $y$), but it is basically stating that there exists a solution of the given system of equations.  In this case, we named an object in the universe (a value for  $x$  and a value for  $y$)  that was an actual solution of the system of equations.  Notice that although we discovered the solution essentially using the backward method, we did not claim that we had found a solution until we actually checked the solution.

\subsection*{Nonconstructive Proofs}
Another type of proof that is often used to prove an existence theorem is the so-called \textbf{nonconstructive proof.}
\index{existence theorem}%
\index{proof!Non-constructive}%
  For this type of proof, we make an argument that an object  in the universal set that makes  
$P\left( x \right)$ true must exist but we never construct or name the object that makes  
$P\left( x \right)$  true.  The advantage of a constructive proof over a nonconstructive proof is that the constructive proof will yield a procedure or algorithm for obtaining the desired object.

The proof of the \textbf{Intermediate Value Theorem}
\index{Intermediate Value Theorem}%
 from calculus is an example of an nonconstructive proof.  The Intermediate Value Theorem can be stated as follows:
\begin{list}{}
\item If  $f$  is a continuous function on the closed interval  $\left[ {a,b} \right]$ and if  $q$  is any real number strictly between  $f\left( a \right)$  and  $f\left( b \right)$, then there exists a number  $c$  in the interval  $\left( {a,b} \right)$ such that  $f\left( c \right) = q$.
\end{list}
\vskip10pt
%
The Intermediate Value Theorem can be used to prove that a solution to some equations must exist.  This is shown in the next example.
\hbreak
%
\begin{example}
Let  $x$  represent a real number.  We will use the Intermediate Value Theorem to prove that the equation  $x^3  - x + 1 = 0$ has a real number solution.

To investigate solutions of the equation  $x^3  - x + 1 = 0$, we will use the function
\[
f\left( x \right) = x^3  - x + 1.
\]
Notice that  $f\left( { - 2} \right) =  - 5$  and that  $f\left( 0 \right) = 1$.  Since 
$f \left( -2 \right) < 0$ and $f \left( 0 \right) > 0$, the Intermediate Value Theorem tells us that there is a real number  $c$  between  $-2$ and  $0$  such that  $f\left( c \right) = 0$.  This means that there exists a real number $c$ between $-2$ and $0$ such that
\[
c^3  - c + 1 = 0,
\]
and hence  $c$  is a real number solution of the equation  $x^3  - x + 1 = 0$.  This proves that the equation  $x^3  - x + 1 = 0$  has at least one real number solution.


Notice that this proof does not tell us how to find the exact value of  $c$.  It does, however, suggest a method for approximating the value of  $c$.  This can be done by finding smaller and smaller intervals  $\left[ {a,\;b} \right]$  such that  $f\left( a \right)$  and  $f\left( b \right)$  have opposite signs.
\end{example}
\hbreak
%
\begin{activity}[Intersections of Circles] \label{A:circles}
In Preview Activity~\ref{PA:circles}, we investigated the points of intersection of two different pairs of circles.  For the first pair, there were two points of intersection, and for the second pair, there were no points of intersection.  That is, the circles did not intersect.  The two examples from this preview activity are related to the following proposition:
\begin{list}{}
\item \textbf{Proposition:}
Let  $a$  be a nonzero real number and let  $r$  be a positive real number.  If  $\left| a \right| < 2r$, then there exist two points of intersection for the two circles whose equations are    $x^2  + y^2  = r^2 $  and  $\left( {x - a} \right)^2  + y^2  = r^2 $.
\end{list}
\vskip10pt
\noindent
Prove this proposition by using a constructive proof and stating the points of intersection. \vskip10pt
\noindent
\underline{Hint}:  Try to proceed in this general case as you did in the specific cases in Preview Activity~\ref{PA:circles}.  Remember that  $\sqrt {a^2 }  = \left| a \right|$.
\end{activity}
\hbreak

\endinput





