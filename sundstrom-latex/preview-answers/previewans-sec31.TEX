\documentclass[11pt]{article}
\usepackage{../../styles/activity}

\usepackage{xr}
\externaldocument{0-MR}

\lhead{}
%\chead{\textbf{\Large{\hspace{0pt}Preview Activities for Section~3.1}}\\\prevhead}
\bahead{3.1}
\rhead{}
\lfoot{}
\rfoot{}
\cfoot{\hspace{0pt}\scalebox{0.4}{\includegraphics{cc-by-nc-sa.eps}}}

\begin{document}

\subsection*{Beginning Activity 1 (Definition of Divides, Divisor, Multiple)}
In Section~1.2, we introduced the definition of even and odd integers and discussed a process that mathematicians use when encountering a new definition.  Namely, 
when we encounter a new definition, we should do the following:
\begin{itemize}
  \item Write out some specific examples of the definition.
  \item Write out some specific non-examples of the definition.
  \item Write a carefully worded negation of the definition.
\end{itemize}
Please note how the items in this beginning activity led us through this process for the definition of divides.

\begin{enumerate} 
  \item The integer 4 divides 32 since $32 = 4 \cdot 8$.  The integer 8 divides $-96$ since 
          $-96 = 8 \cdot (-12)$.
  \item For example:  3 does not divide 4; 5 does not divide 17; 2 does not divide $-7$.
  \item The integer 10 divides 0 since $0 = 10 \cdot 0$.

  \item The integer  $m$  does not divide the integer  $n$  means:  
          $\left( {\forall q \in \mathbb{Z}} \right)\left( {n \ne m \cdot q} \right)$.

\item The integer  $m$  does not divide the integer  $n$  means that for all  
$q \in \mathbb{Z}$, $n \ne m \cdot q$.
\addtocounter{enumi}{1}
\item Each example in part (6) should be an example where the hypothesis of the conjecture is true and the conclusion of the conjecture is true, that is, the examples should indicate that the conjecture is true.
\end{enumerate}

\noindent
\textbf{A Note about the Conjecture}
This conjecture could have been be stated as follows:

\vskip6pt
For all integers $a$, $b$, and  $c$, if $a$ divides $b$ and $b$ divides $c$, then $a$ divides $c$.
\vskip6pt

\noindent
Sometimes, to vary the way we write things, we do not explicitly use the quantifiers.  The conjecture written in the text is an acceptable way to write the conjecture.  Using the word ``let'' is a method that mathematicians accept as equivalent to saying it is true for all integers.  
The idea is the we can let $a$, $b$, and  $c$  be any integers.

You will be asked to formulate conjectures in this and other courses.  The message here is to be very careful in how you formulate your conjecture, and that you should do so by formulating the conjecture as a statement or proposition that can be proven true or false.

By doing this, we can also easily formulate the negation of the conjecture in case the conjecture happens to be false.


%\setcounter{oldenumi}{7}
\begin{enumerate} \setcounter{enumi}{7} %\setcounter{enumi}{\theoldenumi}
  \item We would assume that $a, b$, and $c$ are integers, $a \ne 0$, $b \ne 0$, and that $a$ divides $b$ and 
        $b$ divides $c$.
  \item There exist integers $s$ and $t$ such that $b = a \cdot s$ and $c = b \cdot t$.
  \item We would be trying to prove that $a$ divides $c$.
  \item We could prove that there exists an integer $q$ such that  $c = a \cdot q$.
\end{enumerate}

\hbreak
\newpage



\subsection*{Beginning Activity 2 (Calendars and Clocks)}
This somewhat unusual beginning activity is intended to provide some context for the definition of congruent integers given later in this section.  The concept of congruence is used to describe cylces in the world of integers.  We are all familiar with how the names of the days of the week repeat every seven days.  This idea is the basis for an important definition in mathematics.
\begin{enumerate}
\item \begin{multicols}{3} 
\begin{enumerate}
\item Friday

\item Friday

\item Friday
\end{enumerate}
\end{multicols}

\begin{enumerate}
\setcounter{enumii}{3}
\item Some examples are: 31, 38, 45, 52, 59, 66, 73.

\addtocounter{enumii}{1}
\item All the numbers from Part (e) are multiples of 7.
\end{enumerate}

%\newpage
\item \begin{multicols}{2} 
\begin{enumerate}

\item 9:00

\item 9:00
\end{enumerate}
\end{multicols}

\begin{enumerate}
\setcounter{enumii}{2}
\item Some examples are: 28, 40, 52, 64, 76, 88.

\addtocounter{enumii}{1}
\item All the numbers from Part (d) are multiples of 12.
\end{enumerate}

%\newpage
\item \begin{multicols}{2} 
\begin{enumerate}

\item Friday

\item Friday
\end{enumerate}
\end{multicols}

\begin{enumerate}
\setcounter{enumii}{2}
\item Some examples are: -32, -25, -18, -11, -4, 3, 10, 17, 24, 31, 38.

\addtocounter{enumii}{1}
\item All the numbers from Part (d) are multiples of 7.
\end{enumerate}

\end{enumerate}
\hbreak

\end{document}

