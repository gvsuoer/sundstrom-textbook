\begin{previewactivity}[Proof by Contradiction]\label{PA:contradicton} \hfill
\index{proof!by contradiction}%

\begin{defbox}{D:tautology}{A \textbf{tautology}
\index{tautology}%
 is a propositional expression that yields a true statement regardless of what statements replace its variables. A \textbf{contradiction}
\index{contradiction}%
 is a propositional expression that yields a false statement regardless of what statements replace its variables.}
\end{defbox}

That is, a tautology is necessarily true in all circumstances, and a contradiction is necessarily false in all circumstances.

\begin{enumerate}
\item Use truth tables to explain why  $\left( {P \vee \mynot  P} \right)$ is a tautology and $\left( {P \wedge \mynot  P} \right)$ is a contradiction.
%\item Use a truth table to explain why  $\left( {P \wedge \mynot  P} \right)$ is a contradiction.
\end{enumerate}

Another method of proof that is frequently used in mathematics is a 
\textbf{proof by contradiction.}
\index{proof!by contradiction|(}%
  This method is based on the fact that a statement  $Q$  can only be true or false (and not both).  The idea is to prove that the statement $Q$  is true by showing that it cannot be false.  This is done by assuming that  $Q$  is false and proving that this leads to a contradiction.  (The contradiction often has the form  
$\left( {R \wedge \mynot  R} \right)$, where  $R$  is some statement.)  When this happens, we can conclude that the assumption that the statement  $Q$  is false is incorrect and hence $Q$  cannot be false.  Since it cannot be false, then $Q$  must be true.

A logical basis for the contradiction method of proof is the tautology
\[
\left[ {\mynot  Q \to \left( {R \wedge \mynot  R} \right)} \right] \to Q.
\]

\begin{enumerate}
\setcounter{enumi}{1}
  \item Use a truth table to show that 
$\left[ {\mynot  Q \to \left( {R \wedge \mynot  R} \right)} \right] \to Q$  is a tautology.

  \item Explain why this tautology shows that if the assumption that  $Q$  is false leads to a contradiction, then you have proven that  $Q$  is true.
\end{enumerate}
\end{previewactivity}
\hbreak
%
\begin{previewactivity}[Proof by Contradiction {[}continued{]} ]\label{PA:contradiction2} \hfill \\
The idea of a proof by contradiction is to assume that the statement we want to prove is false and reach a contradiction based on this assumption.  When we try to prove  the conditional statement, \lq\lq If $P$ then $Q$\rq\rq~using a proof by contradiction, we must assume that  $P \to Q$ is false and show that this leads to a contradiction.  Recall that
\begin{center}
\begin{tabular}{r  l  l}
  $P \to Q$  &  is logically equivalent to  &  $\mynot  P \vee Q$, \\
\end{tabular}
\end{center}
and that
\begin{center}
\begin{tabular}{r  l  l}
  $\mynot  \left( {P \to Q} \right)$ &  is logically equivalent to  &  $P \wedge \mynot  Q$. \\
\end{tabular}
\end{center}
The preceding logical equivalencies show that when you assume that  \mbox{$P \to Q$} is false, you are assuming that  $P$  is true and  $Q$  is false.  If you can prove that this leads to a contradiction, then you have shown that $\mynot  \left( {P \to Q} \right)$ is false and hence that  $P \to Q$ is true.
\begin{enumerate}
\item Give an example to show that the following statement is false.
\label{PA:contradiction2-1}%
  \begin{center}
   For all real numbers  $x$  and  $y$, if  $x \ne y$, then  $\dfrac{x}{y} + \dfrac{y}{x} > 2.$
  \end{center}
\item Instead of working with the statement in~(\ref{PA:contradiction2-1}), we will work with a related statement that is obtained by adding conditions to the hypothesis.

For all real numbers  $x$  and  $y$, if  
\[
  \text{if } x \ne y,\textnormal{ } x > 0, \text{ and }y > 0, \text{ then } \frac{x}{y} + \frac{y}{x} > 2.
\]
What assumptions need to be made for a proof by contradiction for this statement?  To do this, we need to negate the entire statement, including the quantifier.  Recall that the negation of a statement with a universal quantifier is a statement that contains an existential quantifier.  With this in mind, carefully write down all assumptions made at the beginning of a proof by contradiction.
\end{enumerate}
\end{previewactivity}
\index{proof!by contradiction|)}%
\hbreak
%
\begin{previewactivity} [Rational Numbers]\label{PA:rational} \hfill
%In Exercise~(\ref{exer:rational}) from Section~\ref{S:moremethods}, we defined a rational number as follows:
%\begin{defbox}{D:rational}{A real number  $r$  is a \textbf{rational number}
%\index{rational numbers}%
% provided that there exist integers  $m$  and  $n$ , with  $n \ne 0$, such that  
%$r = \dfrac{m}{n}$.  A real number  $r$  is an \textbf{irrational number}
%\index{irrational numbers}%
% if it is not a rational number.}
%\end{defbox}
\begin{enumerate}
%  \item Write the complete definition of a rational number.  See 
%Exercise~(\ref{exer:sec32-rational}) from Section~\ref{S:moremethods} on 
%page~\pageref{exer:sec32-rational}.

  \item See Exercise~(\ref{exer:sec32-rational}) from Section~\ref{S:moremethods} on 
page~\pageref{exer:sec32-rational} for a complete definition of a rational number.  Give examples of at least five different rational numbers.

  \item Are integers rational numbers?  Explain.

  \item Are any of the rational numbers $\dfrac{2}{3}$, $\dfrac{4}{6}$, $\dfrac{-15}{12}$, and $\dfrac{20}{-16}$ equal? %are rational numbers.  %Are any of these rational numbers equal?
\label{q:3}%

  \item Are any of the rational numbers $\dfrac{-5}{4}$, $\dfrac{-10}{8}$, $\dfrac{18}{27}$, and $\dfrac{-8}{-12}$ equal?%  Are any of these rational numbers equal? 
\label{q:4}%

  \item What does it mean to say that a real number $r$ is an irrational number?  Explain by writing a precise negation of the definition of a rational number. 

\end{enumerate}
Questions~(\ref{q:3}) and~(\ref{q:4}) were included to illustrate the fact that a rational number can be written as a fraction in ``lowest terms'' with a positive denominator.  This means that any rational number can be written as a quotient $\dfrac{m}{n}$, where $m$ and $n$ are integers, $ n > 0 $, and $m$ and $n$ have no common factor greater than 1.  This fact will be used in a proof by contradiction that the square root of 2 is an irrational number.  (Theorem~\ref{T:squareroot2})
\end{previewactivity}
\hbreak

\endinput
