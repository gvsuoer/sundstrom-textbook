\documentclass[11pt]{article}
\usepackage{../../styles/activity}

\usepackage{xr}
\externaldocument{0-MR}

\lhead{}
%\chead{\textbf{\Large{\hspace{0pt}Beginning Activities for Section~6.3}}\\\hspace{0pt}\emph{Mathematical Reasoning: Writing and Proof}}
\bahead{6.3}
\rhead{}
\lfoot{}
\rfoot{}
\cfoot{\hspace{0pt}\scalebox{0.4}{\includegraphics{cc-by-nc-sa.eps}}}

\begin{document}

\subsection*{Beginning Activity 1 (Functions with Finite Domains)}
Notice that for a function with a finite domain, we can specify the ``rule'' for the function simply by giving the output for each input.  In this situation, each function has the set $A = \{1, 2, 3 \}$ as its domain.  So for the function $f$, for example, all we have to do is specify $f(1)$, $f(2)$, and $f(3)$.  In this beginning activity, we actually did this in two ways.  One was the arrow diagram for $f$ and the other just listed $f(1) = a$, $f(2) = b$, and $f(3) = c$.
\begin{enumerate}
\item The function  $f$  satisfies the stated property.  The functions  $g$  and  $h$  do not.

\item The property in (2) is the contrapositive of the property in (1).  So the function  $f$  satisfies the stated property.  The functions  $g$  and  $h$  do not.

\item $\text{range}\left( f \right) = \left\{ {a, b, c} \right\}$, 
	$\text{range}\left( g \right) = \left\{ {a, b} \right\}$, 
	$\text{range}\left( h \right) = \left\{ {s, t} \right\}$.

\item For the function  $h$, the range is equal to the codomain.  For the functions  $f$  and  
$g$, the range is not equal to the codomain.

%\item The function  $f$  satisfies the stated property.  The functions  $g$  and  $h$  do not.  




\item The function  $h$  satisfies the stated property.  The functions  $f$  and  $g$  do not.
Note:  The property in (5) is asking if the codomain of the function is a subset of the range.  Since the range is always a subset of the codomain, if a function satisfies this property, then its range is equal to its codomain.  This is why the answer to (5) is the same as the answer to (4).
\end{enumerate}
\hbreak

\newpage
\noindent
\subsection*{Beginning Activity 2 (Statements Involving Functions)}
\begin{enumerate}
\item \begin{enumerate}
\item The contrapositive is:  For all $x, y \in A$, if 
$f \left( x \right) = f \left( y \right)$, then $x = y$.
\item The negation is:  There exist $x, y \in A$ such that $x \ne y$ and 
$f \left( x \right) = f \left( y \right)$.
\end{enumerate}

\item The negation is:  There exists a $y$ in  $B$ such that for all $x$ in $A$, 
$f \left( x \right) \ne y$.

\item \begin{enumerate}
\item For all $a, b \in \R$, if $g \left( a \right) = g \left( b \right)$, then $a = b$.

\noindent
\textbf{\emph{Proof}}.  We let $a, b \in \R$, and we assume that 
$g \left( a \right) = g \left( b \right)$.  This means that
\[
5a + 3 = 5b + 3.
\]
By subtracting 3 from both sides of this equation and then dividing both sides by 5, we obtain
\begin{align*}
5a &= 5b \\
 a &= b 
\end{align*}
This proves that for all $a, b \in \R$, if $g \left( a \right) = g \left( b \right)$, then 
$a = b$.

\item For all $b \in \R$, there exists an $a \in \R$ such that $g \left( a \right) = b$.

\noindent
\textbf{\emph{Proof}}.  We let $b \in \R$.  We need to find an $a \in \R$ such that 
$g \left( a \right) = b$.  In order for this to happen, we need $5a + 3 = b$.    Solving this equation for $a$ gives $a = \dfrac{b-3}{5}$.  We then see that $a \in \R$ and
\begin{align*}
g \left( a \right) &=g \left( \frac{b-3}{5} \right) \\
                   &= 5 \left( \frac{b-3}{5} \right) + 3 \\
                   &= \left( b - 3 \right) + 3 \\
                   &= b
\end{align*}
This proves that for all $b \in \R$, there exists an $a \in \R$ such that 
$g \left( a \right) = b$.
\end{enumerate}
\end{enumerate}
\hbreak
%\begin{enumerate}
%\item One possibility is:  $f\left( a \right) = s$, $f\left( b \right) = t$, 
%$f\left( c \right) = u$, $f\left( d \right) = v$.
%
%\item This is not possible since such a function can have at most  3  different outputs, and the set  $C$ contains  4  elements.
%
%\item The function in Part (1) satisfies this condition.
%
%\item It is not possible to create such a function.  In order to do so, we would need four different outputs, and the set  $\left\{ {s, t, u} \right\}$  contains only  3  elements.
%\end{enumerate}





\end{document}

\noindent
\subsection*{Beginning Activity 3 (A Function of Two Variables)}
\begin{enumerate}
\item \begin{multicols}{3}
$f\left( {3, 2} \right) =  - 3$	

$f\left( {3,  - 2} \right) = 9$	

$f\left( {0, 2} \right) =  - 6$

$f\left( {2, 0} \right) = 2$	

$f\left( { - 6, 0} \right) =  - 6$
\end{multicols}

\item $f\left( {m, 0} \right) = m$ and $f\left( {0, n} \right) =  - 3n$.

\item Since  $f\left( {m, 0} \right) = m$ for all  $m \in \mathbb{Z}$, we can conclude that 
$\text{range}\left( f \right) = \mathbb{Z}$.

\item The statement is true.  If  $y \in \mathbb{Z}$, then  
$\left( {y, 0} \right) \in \mathbb{Z} \times \mathbb{Z}$  and  $f\left( {y, 0} \right) = y$.
\end{enumerate}

The statements in (5) and (6) are contrapositives of each other.  They are both false.  This is proven by the facts that  $f\left( {0, 2} \right) =  - 6$ and   
$f\left( { - 6, 0} \right) =  - 6$, but  $\left( {0, 2} \right) \ne \left( { - 6, 0} \right)$.
\hbreak
