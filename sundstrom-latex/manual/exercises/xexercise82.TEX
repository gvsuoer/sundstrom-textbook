\section*{Section \ref{S:primefactorizations} Prime Numbers and Prime Factorizations}

\begin{enumerate}
\item \begin{enumerate}
\item If $p$ is prime, then the only positive divisors of $p$ are 1 and $p$.  Since $p \mid a$, $p$ must be the greatest positive integer that divides both $a$ and $p$.  Therefore, 
$\gcd \left( a, p \right) = p$.

\item If the prime $p$ does not divide $a$, then 1 is the only positive common divisor of $a$ and 
$p$.  Hence, $\gcd \left( a, p \right) = 1$.
\end{enumerate}

\item Either $p$ divides $a$ or $p$ does not divide $a$.  In the case where $p \mid a$, we see that the conclusion $\left( p \mid a \text{ or } p \mid b \right)$ is true.  In the case where 
$p$ does not divide $a$, by Exercise~(1), $\gcd \left( a , p \right) = 1$.  Hence, $p$ and $a$ are relatively prime and so by Theorem~\ref{T:relativelyprimeprop}, $p \mid b$.

\item Let $P \left( n \right)$ be, ``If $p \mid \left( {a_1 a_2  \cdots a_n } \right)$, then there exists a natural number $k$  with  $1 \leq k \leq n$ such that  $p \mid a_k $.''

The statement $P \left( 1 \right)$ is true since if $p \mid a_1$, then $p \mid a_1$ is true.  So, let $m \in \mathbb{N}$ and assume that $P \left( m \right)$ is true.  Now let 
$a_1, a_2, \ldots a_m, a_{m+1}$ be integers and assume that 
$p \mid \left( {a_1 a_2  \cdots a_m a_{m+1} } \right)$.  We can rewrite this as
\[
p \mid \left( a_1 a_2  \cdots a_m \right) a_{m+1} 
\]
and use Part~(1) to conclude that $p \mid \left( a_1 a_2  \cdots a_m \right)$ or $p \mid a_{m+1}$.  In the case where $p \mid \left( a_1 a_2  \cdots a_m \right)$, we use the assumption that 
$P \left( m \right)$ is true to conclude that there exists a natural number $k$  with  
$1 \leq j \leq m$ such that  $p \mid a_j$.  Thus, we may may conclude that there exists a natural number $k$  with  $1 \leq k \leq m +1$  such that  $p \mid a_k$.  This proves that if 
$P \left( m \right)$ is true, then $P \left( m + 1 \right)$ is true.

\item \begin{enumerate}
\item Let $d = \gcd \left( a, b \right)$. If 1 is a linear combination of $a$ and $b$, then by 
Theorem~\ref{T:gcddivideslincombs}, $d \mid 1$ and hence, $d = 1$.

\item In the same manner, if 2 is a linear combination of $a$ and $b$, then $d \mid 2$ and hence, 
$d = 1$ or $d = 2$.
\end{enumerate}

\item \begin{enumerate}
\item If $a \in \mathbb{Z}$, then $\left( a + 1 \right) - a = 1$.  By Exercise~(4), 
$\gcd \left( a, a + 1 \right) = 1$.

\item Similarly, $\left( a + 2 \right) - a = 2$.  By Exercise~(4), 
$\gcd \left( a, a + 2 \right) = 1$ or $\gcd \left( a, a + 2 \right) = 1$.
\end{enumerate}

\item \begin{enumerate}
\item If $a \in \mathbb{Z}$, then $\left( a + 3 \right) - a = 3$.  By Theorem~\ref{T:gcddivideslincombs}, 
$\gcd \left( a, a + 3 \right) \mid 3$.  Hence, $\gcd \left( a, a + 3 \right) = 1$ or 
$\gcd \left( a, a + 3 \right) = 3$.

\item If $a \in \mathbb{Z}$, then $\left( a + 4 \right) - a = 4$.  By Theorem~\ref{T:gcddivideslincombs}, 
$\gcd \left( a, a + 4 \right) \mid 4$.  Hence, $\gcd \left( a, a + 4 \right) = 1$,  
$\gcd \left( a, a + 4 \right) = 2$, or $\gcd \left( a, a + 4 \right) = 4$.
\end{enumerate}

\item \begin{enumerate}
\item $\gcd \left( 16, 28 \right) = 4$.  Also, $\dfrac{16}{4} = 4$, $\dfrac{28}{4} = 7$, and 
$\gcd \left( 4, 7 \right) = 1$.

\item $\gcd \left( 10, 45 \right) = 5$.  Also, $\dfrac{10}{5} = 2$, $\dfrac{45}{5} = 9$, and 
$\gcd \left( 2, 9 \right) = 1$.

\item If $d = \gcd \left( a, b \right)$, then there exist integers $x$ and $y$ such that 
$ax + by = d$.  If we divide both sides of this equation by $d$, we obtain
\[
\frac{a}{d} x + \frac{b}{d} y = 1.
\]
Since $d \mid a$ and $d \mid b$, we see that $\dfrac{a}{d}$ and $\dfrac{b}{d}$ are integers.  Hence, the previous equation shows that 1 can be written as a linear combination of 
$\dfrac{a}{d}$ and $\dfrac{b}{d}$ and hence, 
$\gcd \left( \dfrac{a}{d}, \dfrac{b}{d} \right) = 1$.
\end{enumerate}

\item \begin{enumerate}
\item The statement is false.  A counterexample is $a = 4$, $b = 6$, and $c = 12$.  In this case, 
$a \mid c$, $b \mid c$, but $ab$ does not divide $c$.

\item Assume that $a \mid c$, $b \mid c$, and $\gcd \left( a, b \right) = 1$.  Then, there exist integers $m, n, x, y$ such that
\[
c = am, c = bn, \text{ and }, ax + by = 1.
\]
We now multiply both sides of the linear combination of $a$ and $b$ by $c$ to obtain
\[
acx + bcy = c.
\]
Now, substitute $c = bn$ for the $c$ in $acx$ and substitute $c = am$ for the $c$ in $bcy$.  This gives
\[
abnx + bamy = c,
\]
and hence, $ab \left( nx + my \right) = c$.  This proves that $\left( ab \right) \mid c$.
\end{enumerate}

\item If $n$ is an odd integer, then we have already proven that $8 \mid \left( n^2 - 1 \right)$.  Also, since 3 does not divide $n$, then in $\mathbb{Z}_3$, $\left[ n \right] = \left[ 1 \right]$ or $\left[ n \right] = \left[ 2 \right]$.  In both cases, it can be verified that 
$\left[ n^2 \right] = \left[ 1 \right]$, which implies that $3 \left( n^2 -1 \right)$.  Since 
$\gcd \left( 3, 8 \right)$, Exercise~(8b) implies that $3 \cdot 8$ divides 
$\left( n^2 - 1 \right)$.

\item \begin{enumerate}
\item If $\gcd \left( {a, b} \right) = 1$ and  $\gcd \left( {a, c} \right) = 1$, there exist integers $m, n, x, y$ such that $am + bn = 1$ and $ax + cy = 1$.  From this, we see that
\[
\begin{aligned}
          \left( am + bn \right) \left( ax + cy \right) &= 1 \\
                              a^2 mx +amcy +abnx + bcny &= 1 \\
a \left( amx + mcy + bnx \right) + c \left( bny \right) &=1. \\
\end{aligned}
\]
This proves that $\gcd \left( a, bc \right) = 1$.

\item Let $P \left( n \right)$ be, ``If  $\gcd \left( {a, b_i } \right) = 1$ for all  
$i \in \mathbb{N}$ with  $1 \leq i \leq n$, then  
$\gcd \left( {a, b_1 b_2  \cdots b_n } \right) = 1$''.  $P \left( 1 \right)$ is true, and by 
Part~(a), $P \left( 2 \right)$ is true.

Now let $k \in \mathbb{N}$ and assume that $P \left( k \right)$ is true.  To prove that 
$P \left( k + 1 \right)$ is true, we let $b_1, b_2, \ldots, b_k, b_{k+1}$ be integers and assume that \\
$\gcd \left( {a, b_i } \right) = 1$ for all  $i \in \mathbb{N}$ with  $1 \leq i \leq k+1$.  We now write
\[
b_1 b_2  \cdots b_k b_{k+1} = \left( b_1 b_2  \cdots b_k \right) b_{k+1}.
\] 
Since we have assume that $P \left( k \right)$ is true, we may conclude that 
$\gcd \left( {a, b_1 b_2  \cdots b_k } \right) = 1$.  We then use Part~(a) to conclude that \\
$\gcd \left( \left( {a, b_1 b_2  \cdots b_k } \right)b_{k+1} \right) = 1$ and hence that \\
$\gcd \left( a, b_1 b_2  \cdots b_k b_{k+1} \right) = 1$.  Therefore, if $P \left( k \right)$ is true, then $P \left( k + 1 \right)$ is true.
\end{enumerate}

\item The statement is true.  If   $\gcd \left( {a, b} \right) = 1$  and  
$c \mid \left( {a + b} \right)$, then there exist integers $x$ and $y$ such that 
$ax + by = 1$ and there exists an integer $m$ such that $a + b = cm$.  So, we can write 
$b = cm - a$ and substitute this into the other equation.  This gives
\[
\begin{aligned}
              ax + \left( cm - a \right) y &= 1 \\
a \left( x - y \right) + c \left( my \right) &= 1. \\
\end{aligned}
\]
This proves that $\gcd \left( a, c \right) = 1$.  A similar proof shows that 
$\gcd \left( b, c \right) = 1$.

\item Since $-12 \left( 5n + 2 \right) + 5 \left( 12n + 5 \right) = 1$, by Exercise~(4), 
$\gcd \left( 5n + 2, 12n + 5 \right) = 1$ for every natural number $n$.

\item Write the prime factorization of $y$ as $y = p_1 p_2 \ldots p_r$ with 
$p_1 \leq p_2 \leq \cdots \leq p_r$.  If 2 does not divide $y$, then
$y = 2^0 y$.

If 2 divides $y$ and $p_r = 2$, then $y = 2^r$.  If 2 divides $y$ and $p_r \ne 2$, then there exists a natural number $k$ with $1 \leq k < r$ such that $p_k = 2$ and $p_{k+1}>2$.  In this case,
\[
y = 2^k \left( p_{k+1} p_{k+2} \cdots p_r \right).
\]

\item \begin{enumerate}
\item 3, 7, 11, 19, and 23 are all prime numbers that are congruent to 3 modulo 4.

\item Use a proof by contradiction.  Assume that there are only finitely many primes that are congruent to 3 modulo 4.  Let $3, p_1, p_2, \ldots, p_m$ be the list of all the primes that are congruent to 3 modulo 4.  Order these primes so that $3 < p_1 < p_2 < \cdots p_m$.

If $m$ is odd, then let $M = 3 p_1 p_2 \cdots p_m + 2$.  Then prove that 
$3 p_1 p_2 \cdots p_m \equiv 1 \pmod 4$ and hence that $M \equiv 3 \pmod 4$.  Now notice that 3 does not divide $M$ and that for each $j$ with $1 \leq j \leq m$, $p_j$ does not divide 
$M$.  So if $M$ is not prime, then all of its prime factors must be congruent to 1 modulo 4.  But this is impossible since any product of prime numbers, each of which is congruent to 1 modulo 4, must be congruent to 1 modulo 4.  Thus $M$ is a prime number that is congruent to 3 modulo 4.  This is a contradiction to the assumption that we have already listed all the prime numbers congruent to 3 modulo 4.

Now assume that $m$ is even, and let $M =3 p_1 p_2 \cdots p_m + 4$.  Use a similar argument to the previous one that $M$ must be a prime number that is congruent to 3 modulo 4.
\end{enumerate}

\item \begin{enumerate}
\item Since $2 \mid \left( n + 1 \right)!$, $2 \mid \left[\left( n + 1 \right)! + 2 \right]$.

\item Since $n \geq 2$, $3 \mid \left( n + 1 \right)!$, and hence, 
$3 \mid \left[\left( n + 1 \right)! + 3 \right]$.

\item Since $2 \leq k \leq \left( n + 1 \right)$, $k \mid \left( n + 1 \right)!$, and hence, 
$k \mid \left[\left( n + 1 \right)! + k \right]$.

\item $\left( n + 1 \right)! + 2$, $\left( n + 1 \right)! + 3$, \ldots,  
$\left( n + 1 \right)! + \left( n + 1 \right)$ is a list of $n$ consecutive composite numbers.
\end{enumerate}

\item \begin{enumerate}
\item The primes 2 and 5 are the only pair of primes that differ by 3.  If $p$ is a prime greater than 2, then $p$ is odd.  Therefore, $p + 3$ is even and hence is not prime.

\item The primes 3, 5, and 7 are the only triplet of primes.  Let $p$ be an odd number greater than 3.  If $p \equiv 0 \pmod 3$, then $3 \mid p$ and $p$ is not prime.  If $p \equiv 1 \pmod 3$, then $p + 2 \equiv 0 \pmod 3$.  In this case, $3 \mid \left( p + 2 \right)$ and hence, $p + 2$ is not prime.  If $p \equiv 2 \pmod 3$, then $p + 4 \equiv 0 \pmod 3$.  In this case, 
$3 \mid \left( p + 4 \right)$ and hence, $p + 4$ is not prime.  This proves that except for 3, 5, 7, any 3 consecutive odd natural numbers must contain a composite number.
\end{enumerate}

\item If $\gcd \left( a, n \right) = 1$, then there exist integers $r$ and $s$ such that 
$as + nr = 1$.  This implies that $as \equiv 1 \pmod n$.  Multiply both sides of this congruence by $b$.  This gives $a \left( sb \right) \equiv b \pmod n$, which proves there exists an 
$x \in \mathbb{Z}$ such that $ax \equiv b \pmod n$.

\item The key to the proof is to first prove that the even number between the two twin primes is a multiple of 6.  Let $m$ and $n$ be twin primes with $m > 3$.  Then, there exist an even integer 
$k$ with $k >4$ such that $m = k - 1$ and $n = k + 1$.  Look at six cases for $k$ based on congruence modulo 6.

If $k \equiv 1 \pmod 6$ or  $k \equiv 3 \pmod 6$ or $k \equiv 5 \pmod 6$, then $k$ is odd, which contradicts the fact that $k$ is even.  Therefore, $k \not \equiv 1 \pmod 6$.

If $k \equiv 2 \pmod 6$, then $k + 1 \equiv 3 \pmod 6$, which implies that 3 divides $n = k + 1$.  This contradicts the assumption that $n$ is prime.  Therefore, $k \not \equiv 2 \pmod 6$.

If $k \equiv 4 \pmod 6$, then $k - 1 \equiv 3 \pmod 6$, which implies that 3 divides $m = k - 1$.  This contradicts the assumption that $m$ is prime.  Therefore, $k \not \equiv 2 \pmod 6$.

This proves that $k$ must be congruent to 0 modulo 6. Hence, there exists an integer $q$ such that 
$k = 6q$.  Then, $m = 6q - 1$ and $n = 6q + 1$.  Consequently,
\[
\begin{aligned}
mn + 1 &= \left( 36q^2 -1 \right) + 1 \\
       &= 36q^2 \\
       &= \left( 6q \right)^2 \\
\end{aligned}
\]
This proves that $mn + 1$ is a perfect square that is divisible by 36.
\end{enumerate}
\hbreak
\endinput


