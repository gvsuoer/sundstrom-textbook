\subsection*{Forms of Quantified Statements in English}
There are many ways to write statements involving quantifiers in English.  In some cases, the quantifiers are not apparent, and this often happens with conditional statements.  The following examples illustrate these points.  Each example contains a quantified statement written in symbolic form followed by several ways to write the statement in English.
\begin{enumerate}
  \item $\left( {\forall x \in \mathbb{R}} \right)\left( {x^2  > 0} \right)$.
  \begin{itemize}
    %\item For any real number  $x$,  $x^2  > 0$.
    \item For each real number  $x$, $x^2  > 0$.
    \item The square of every real number is greater than 0.
    \item The square of a real number is greater than 0.
    \item If  $x \in \mathbb{R}$, then  $x^2  > 0$.
  \end{itemize}
In the second to the last example, the quantifier is not stated explicitly.  Care must be taken when reading this because it really does say the same thing as the previous examples.
The last example illustrates the fact that conditional statements often contain a ``hidden'' universal quantifier.  

If the universal set is  $\R$, then the truth set of the open sentence  $x^2  > 0$ is the set of all nonzero real numbers.  That is, the truth set is
\[
\left\{ {x \in \mathbb{R}} \mid x \ne 0 \right\}.
\]
So the preceding statements are false.  For the conditional statement, the example using  
$x = 0$ produces a true hypothesis and a false conclusion.  This is a \textbf{counterexample}
\index{counterexample}%
\label{D:counterexample}%
 that shows that the statement with a universal quantifier is false.

%\pagebreak
\item $\left( {\exists x \in \mathbb{R}} \right)\left( {x^2  = 5} \right)$.
  \begin{itemize}
    \item There exists a real number  $x$  such that  $x^2  = 5$.
    \item $x^2  = 5$ for some real number $x$.
    \item There is a real number whose square equals 5.
  \end{itemize}

The second example is usually not used since it is not considered good writing practice to start a sentence with a mathematical symbol. 

If the universal set is  $\R$, then the truth set of the predicate  ``$x^2  = 5$''  is  
$\left\{ { - \sqrt 5 ,\;\sqrt 5 } \right\}$.  So these are all true statements.
\end{enumerate}
\hbreak

\endinput

