\section*{Section \ref{S:moreaboutfunctions} More about Functions}

\begin{enumerate}
\item \begin{enumerate}
\item $f(0) = 4$, $f(1) = 0$, $f(2) = 3$, $f(3) = 3$, $f(4) = 0$

\item $g(0) = 4$, $g(1) = 0$, $g(2) = 3$, $g(3) = 3$, $g(4) = 0$

\item The function $f$ is equal to the function $g$.
\end{enumerate}



\item \begin{enumerate}
\item $f(0) = 4$, $f(1) = 5$, $f(2) = 2$, $f(3) = 1$, $f(4) = 2$, $f(5) = 5$

\item $g(0) = 4$, $g(1) = 4$, $g(2) = 0$, $g(3) = 3$, $g(4) = 5$, $g(5)= 0$

\item The function $f$ is not equal to the function $g$.
\end{enumerate}



%\item \begin{enumerate}
%\item $f( 0 ) = 0$, $f( 1 ) = 1, f( 2 ) = 1$, 
%$f( 3 ) = 1$, $f( 4 ) = 1.$
%
%\addtocounter{enumii}{1}
%\item $f = \left\{ {( {0, 0} ), ( {1, 1} ), ( {2, 1} )( {3, 1} ), ( {4, 1} )} \right\}$.
%
%\item The function $f$ is not a constant function.
%\end{enumerate}
%
%\item \begin{enumerate}
%\item $g( 0 ) = 0$, $g( 1 ) = 1, g( 2 ) = 2$, 
%$g( 3 ) = 3$, $g( 4 ) = 1.$
%
%\addtocounter{enumii}{1}
%\item $g = \left\{ {( {0, 0} ), ( {1, 1} ), ( {2, 2} )( {3, 3} ), ( {4, 4} )} \right\}$.
%
%\item The function $g$ is equal to the indentity function on $\mathbb{Z}_5$.
%\end{enumerate}


\item \begin{enumerate}
\item $f(2) = 9$, $f(-2) = 9$, $f(3) = 14$, $f(\sqrt{2} = 7$

\item $g(0) = 5$, $g(2) = 9$, $g(-2) = 9$, $g(3) = 14$, $g(\sqrt{2}) = 7$

\item The function $f$ is not equal to the function $g$ since they do not have the same domain.

\item The function $h$ is equal to the function $f$ since if $x \ne 0$, then 
$\dfrac{x^2 + 5x}{x} = x^2 + 5$.

\end{enumerate}

\item \begin{enumerate}
\item $\left\langle {a_n } \right\rangle $  where  $a_n  = \dfrac{1}{{n^2 }}$  for each  
$n \in \mathbb{N}$.  The domain is  $\mathbb{N}$.

\item $\left\langle {a_n } \right\rangle $  where  $a_n  = \dfrac{1}{{3^n }}$  for each  
$n \in \mathbb{N}$.  The domain is  $\mathbb{N}$.

\item $\left\langle {a_n } \right\rangle $  where  $a_n  = ( -1 )^{n}$  for each  $n \in \mathbb{N} \cup {0}$.  The domain is  $\mathbb{N} \cup {0}$.

\item $\left\langle {a_n } \right\rangle $  where  $a_n  = \cos ( {n\pi } )$  for each  $n \in \mathbb{N} \cup {0}$.  The domain is  $\mathbb{N} \cup {0}$.  This is equal to the sequence in Part (c).
\end{enumerate}



\item \begin{enumerate}
\item
\begin{multicols}{3}
$p_1 ( 1, x ) = 1$

$p_1 ( 2, x ) = 2$

$p_1 ( 1, y ) = 1$

$p_1 ( 2, y ) = 2$

$p_1 ( 1, z ) = 1$

$p_1 ( 2, z ) = 2$
\end{multicols}

\item
\begin{multicols}{3}
$p_2 ( 1, x ) = x$

$p_2 ( 2, x ) = x$

$p_2 ( 1, y ) = y$

$p_2 ( 2, y ) = y$

$p_2 ( 1, z ) = z$

$p_2 ( 2, z ) = z$
\end{multicols}
\item $\text{range} ( p_1 ) = A$ and $\text{range} ( p_2 ) = B$

\item The statement is false.  For example, $ ( 1, x ) \ne ( 1, y )$ and \\
$p_1 ( 1, x ) = p_1 ( 1, y )$
\end{enumerate}


\item Let $P ( n )$ be, ``A convex polygon with $n$ sides has $\dfrac{n ( n-3 )}{2}$ sides.''  Then, $P ( 3 )$ is true since a triangle has no diagonals and hence $d ( 3 ) = 0$.
  
Let  $k \in D$  and assume that $P ( k )$ is true.  So, we assume that a convex polygon with  $k$  sides has  $\dfrac{{k( {k - 3} )}}{2}$  diagonals.   Now let  $Q$  be convex polygon with  $( {k + 1} )$ sides.  Let  $v$  be one of the 
$( {k + 1} )$ vertices of $P$  and let  $u$  and  $w$  be the two vertices adjacent to  $v$.  By drawing the line segment from  $u$  to  $w$ and omitting the vertex  $v$, we form a convex polygon with  $k$  sides.  Since we are assuming that $P ( k )$ is true, this polygon has $\dfrac{{k( {k - 3} )}}{2}$  diagonals.  These are also diagonals of the convex polygon $Q$.  The other diagonals of $Q$ are formed by connecting the vertex  $v$ to the 
$k - 1$ vertices of $Q$ that are not adjacent to $v$.  So, the total number of diagonals of $Q$ is
\[
\begin{aligned}
\frac{{k( {k - 3} )}}{2} + ( k - 1 ) &= 
\frac{k ( k - 3 ) + 2 ( k - 1 )}{2} \\
  &= \frac{k^2 - 3k + 2k - 2}{2} \\
  &= \frac{k^2 - k - 2}{2} \\
  &= \frac{( k + 1 ) ( k - 2 )}{2} \\
  &= \frac{( k + 1 ) \left[ ( k + 1 ) - 3 \right]}{2}. \\
\end{aligned}
\]
This proves that if $P ( k )$ is true, then $P ( k + 1 )$ is true.



\item \begin{enumerate}
\item $f( { - 3, 4} ) = 9$, $f( { - 2, - 7} ) =  - 23$

\item $\left\{ { {( {m, n} ) \in \mathbb{Z} \times \mathbb{Z} } \mid m = 4 - 3n} \right\}$
\end{enumerate}


\item \begin{enumerate}
\item $g ( 3, 5 ) = ( 6, -2 )$, \qquad
$g ( -1, 4 ) = ( -2, -5 )$.

\item $( 0, 0 )$ is the only preimage of $( 0, 0 )$.

\item The set of  preimages of $( 8, -3 )$ is $\left\{ ( 4, 7 ) \right\}$. 

\item The set of  preimages of $( 1, 1 )$ is $\emptyset$ since the first coordinate of $f(m, n)$ is always an even integer.

\item Part~(d) shows that the statement is false since there does not exist an 
$( m, n ) \in \mathbb{Z} \times \mathbb{Z}$ such that 
$f ( m, n ) = ( 1, 1 )$.
\end{enumerate}



\item \begin{enumerate}
\item $\det \left[ {\begin{array}{*{20}c}
   3 & 5  \\
   4 & 1  \\
\end{array} } \right] =  - 17, \det \left[ {\begin{array}{*{20}c}
   1 & 0  \\
   0 & 7  \\
 \end{array} } \right] = 7, \text{and det}\left[ {\begin{array}{*{20}c}
   3 & { - 2}  \\
   5 & 0  \\
 \end{array} } \right] = 10$

\item The process of finding the determinant of a 2 by 2 matrix over $\mathbb{R}$ can be thought of as a function from $\mathcal{M}_{2, 2}$ to $\mathbb{R}$.  So, the domain is 
$\mathcal{M}_{2, 2}$ and the codomain is $\mathbb{R}$.  That is, 
$\det : \mathcal{M}_{2, 2} \to \mathbb{R}$ where
\[
\det \left[ {\begin{array}{*{20}c}
   a & b  \\
   c & d  \\
\end{array} } \right] = ad - bc.
\]
\end{enumerate}

\item \begin{enumerate}
\item $\left[ {\begin{array}{*{20}c}
   3 & 5  \\
   4 & 1  \\
\end{array} } \right]^T =  \left[ {\begin{array}{*{20}c}
   3 & 4  \\
   5 & 1  \\
\end{array} } \right]$, \qquad
$\left[ {\begin{array}{*{20}c}
   1 & 0  \\
   0 & 7  \\
\end{array} } \right]^T =  \left[ {\begin{array}{*{20}c}
   1 & 0  \\
   0 & 7  \\
\end{array} } \right]$, \\
$\left[ {\begin{array}{*{20}c}
   3 & -2  \\
   5 & 0  \\
\end{array} } \right]^T =  \left[ {\begin{array}{*{20}c}
   3 & 5  \\
   -2 & 0  \\
\end{array} } \right]$

\item The process of finding the transpose of a 2 by 2 matrix over $\mathbb{R}$ can be thought of as a function from $\mathcal{M}_{2, 2}$ to $\mathcal{M}_{2, 2}$.  So, the domain is 
$\mathcal{M}_{2, 2}$ and the codomain is $\mathcal{M}_{2, 2}$.  That is, 
$\text{tran} : \mathcal{M}_{2, 2} \to \mathcal{M}_{2, 2}$ where
\[
\text{tran}\left[ {\begin{array}{*{20}c}
   a & b  \\
   c & d  \\
\end{array} } \right] = \left[ {\begin{array}{*{20}c}
   a & b  \\
   c & d  \\
\end{array} } \right]^T = \left[ {\begin{array}{*{20}c}
   a & c  \\
   b & d  \\
\end{array} } \right].
\]
\end{enumerate}
\end{enumerate}




\subsection*{Explorations and Activities}
\setcounter{oldenumi}{\theenumi}
\begin{enumerate} \setcounter{enumi}{\theoldenumi}
\item \begin{enumerate}
\item \begin{enumerate}
\item For each  $f \in C_{\left[ {a, b} \right]} $, we can associate one real number given by 
$ dx$.  So we can define  $I:C_{\left[ {a, b} \right]}  \to \mathbb{R}$
by
\[
I\left( f \right) =  \int_a^b {f\left( x \right) dx}, \text{ for each }  
f \in C_{\left[ {a, b} \right]}.
\]

\item $I \left( f \right) =\int_0^2 {\left( x^2+1 \right)}  dx = \left. {\left( {\dfrac{1}{3}x^3  + x} \right)} \right|_0^2  = \dfrac{{14}}{3}$.

\item $I\left( g \right) = \int_0^2 \text{sin} \left( \pi x \right) dx = 
\left. {\left( {\dfrac{{ - 1}}{\pi }\cos \left( {\pi x} \right)} \right)} \right|_0^2  = 0$
\end{enumerate}

\item We determine that
\begin{align*}
\int {\left( {x^2  + 1} \right) dx} &= \dfrac{1}{3}x^3  + x + c  \text{ where } c 
\text{ is a real number and that} \\
\int {\cos \left( 2x \right) dx} &= \dfrac{1}{2}\sin \left( 2x \right) + c 
\text{ where } c  \text{ is a real number}.
\end{align*}
 

\item This does not define a function from  $C_{\left[ {0, 1} \right]} $ to  $T$  since  
$A\left( f \right)$ represents infinitely many functions that are antiderivatives of  $f$.

\item If  $f$  is continuous on the interval  $\left[ {a, b} \right]$, then by the Fundamental Theorem of Calculus,  the function  $g$  (where  
$g\left( x \right) =  \int_a^x {f\left( t \right) dt}$ is one particular antiderivative of  $f$.  That is, each input  $f \in C_{\left[ {a, b} \right]} $  has one output  $g$.  In this case,  $g'\left( x \right) = f\left( x \right)$  and  $g\left( a \right) = 0$.




\end{enumerate}
\end{enumerate}
\hbreak

\endinput
