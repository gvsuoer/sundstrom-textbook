\subsection*{Countably Infinite Sets}
In Section~\ref{S:finitesets}, we used the set $\mathbb{N}_k$ as the standard set with cardinality $k$ in the sense that a set is finite if and only if it is equivalent to 
$\mathbb{N}_k$.  In a similar manner, we will use some infinite sets as standard sets for certain infinite cardinal numbers.  The first set we will use is $\mathbb{N}$.  

We will formally define what it means to say the elements of a set can be ``counted'' using the natural numbers.  The elements of a finite set can be ``counted'' by defining a bijection (one-to-one correspondence) between the set and $\mathbb{N}_k$ for some natural number $k$.  We will be able to ``count'' the elements of an infinite set if we can define a one-to-one correspondence between the set and $\mathbb{N}$.

\begin{defbox}{aleph0}{The \textbf{cardinality of} $\boldsymbol{\mathbb{N}}$
\index{cardinality}%
\index{cardinality!natural numbers}%
\index{cardinality!$\aleph_0$}%
%\index{$\aleph_0$}%
 is denoted by $\aleph_0$.  
\label{sym:aleph0}%
 The symbol $\aleph$ is the first letter of the Hebrew alphabet, \textbf{aleph}.
\index{aleph}%
  The subscript 0 is often read as ``naught'' (or sometimes as ``zero'' or ``null'').  So we write
\[
\text{card}( \N ) = \aleph_0
\]
and say that the cardinality of $\mathbb{N}$ is ``aleph naught.''}
\end{defbox}
%
\begin{defbox}{countinfinite}{A set $A$ is \textbf{countably infinite}
\index{countably infinite set}%
 provided that 
$A \approx \mathbb{N}$.  In this case, we write
\[
\text{card}( A ) = \aleph_0.
\]
A set that is countably infinite is sometimes called a \textbf{denumerable}
\index{denumerable set}%
 set.  A set is 
\textbf{countable}
\index{countable set}%
 provided that it is finite or countably infinite.  An infinite set that is not countably infinite is called an \textbf{uncountable set}.
\index{uncountable set}%
}
\end{defbox}
%
\begin{prog}[\textbf{Examples of Countably Infinite Sets}]\label{prog:countablyinfinitesets} \hfill
\begin{enumerate}
\item In \typeu Activity~\ref*{PA:equivalentsets} from Section~\ref{S:finitesets}, we proved that 
$\mathbb{N} \approx D^+$, where $D^+$ is the set of all odd natural numbers.  Explain why 
$\text{card}( D^+ ) = \aleph_0$.

\item Use a result from Progress Check~\ref{E:infinitesets} to explain why 
$\text{card}( E^+ ) = \aleph_0$.

\item At this point, if we wish to prove a set $S$ is countably infinite, we must find a bijection between the set $S$ and some set that is known to be countably infinite.

Let $S$ be the set of all natural numbers that are perfect squares.  Define a function 
\[
f\x S \to \mathbb{\N}
\]
that can be used to prove that $S \approx \mathbb{N}$ and, hence, that 
$\text{card}( S ) = \aleph_0$.
\end{enumerate}
\end{prog}\hbreak
%
The fact that the set of integers is a countably infinite set is important enough to be called a theorem.  The function we will use to establish that $\N \approx \Z$ was explored in \typeu Activity~\ref*{PA:functionNtoZ}.

\begin{theorem}\label{T:ZequivtoN}
The set $\Z$ of integers is countably infinite, and so 
$\text{card}( \Z ) = \aleph_0$.
\end{theorem}
%
\begin{myproof}
To prove that $\N \approx \Z$, we will use the following function:
$f\x \N \to \Z$, where
%
\begin{equation} \notag
f ( n ) = 
\begin{cases}
\dfrac{n}{2}         &\text{if $n$ is even} \\
                      &                      \\
\dfrac{1-n}{2}       &\text{if $n$ is odd}.
\end{cases}
\end{equation}
%
From our work in \typeu Activity~\ref*{PA:functionNtoZ}, it appears that if $n$ is an even natural number, then $f ( n ) > 0$, and if $n$ is an odd natural number, then 
$f ( n ) \leq 0$.  So it seems reasonable to use cases to prove that $f$ is a surjection and that $f$ is an injection.
To prove that $f$ is a surjection, we let $y \in \mathbb{Z}$.
\begin{itemize}
\item If $y > 0$, then $2y \in \mathbb{N}$, and
\[
f ( 2y ) = \frac{2y}{2} = y.
\]
\item If $y \leq 0$, then $-2y \geq 0$ and $1 - 2y$ is an odd natural number.  Hence,
\[
f ( 1 - 2y ) = \frac{1 - (1 - 2y )}{2} = \frac{2y}{2}=y.
\]
\end{itemize}
These two cases prove that if $y \in \mathbb{Z}$, then there exists an $n \in \mathbb{N}$ such that \linebreak
$f ( n ) = y$.  Hence, $f$ is a surjection.

\vskip6pt
To prove that $f$ is an injection, we let $m, n \in \mathbb{N}$ and assume that  
$f ( m ) = f ( n )$.  First note that if one of $m$ and $n$ is odd and the other is even, then one of $f ( m )$ and $f ( n )$ is positive and the other is less than or equal to 0.  So if $f ( m ) = f ( n )$, then both $m$ and $n$ must be even or both $m$ and $n$ must be odd.
\begin{itemize}
\item If both $m$ and $n$ are even, then
\begin{center}
$f ( m ) = f ( n )$ implies that $\dfrac{m}{2} = \dfrac{n}{2}$
\end{center}
and hence that $m = n$.

\item If both $m$ and $n$ are odd, then
\begin{center}
$f ( m ) = f ( n )$ implies that $\dfrac{1-m}{2} = \dfrac{1-n}{2}$.
\end{center}
From this, we conclude that $1 - m$ = $1 - n$ and hence that $m = n$.  This proves that if 
$f ( m ) = f ( n )$, then $m = n$ and hence that $f$ is an injection.
\end{itemize}

Since $f$ is both a surjection and an injection, we see that $f$ is a bijection and, therefore, 
$\mathbb{N} \approx \mathbb{Z}$.  Hence, $\mathbb{Z}$ is countably infinite and
$\text{card} ( \Z ) = \aleph_0$.
\end{myproof}

The result in Theorem~\ref{T:ZequivtoN} can seem a bit surprising.  It exhibits one of the distinctions between finite and infinite sets.  If we add elements to a finite set, we will increase its size in the sense that the new set will have a greater cardinality than the old set.  However, with infinite sets, we can add elements and the new set may still have the same cardinality as the original set.  For example, there is a one-to-one correspondence between the elements of the sets $\N$ and $\Z$.  We say that these sets have the same cardinality.

Following is a summary of some of the main examples dealing with the cardinality of sets that we have explored.

\begin{itemize}
\item The sets $\mathbb{N}_k$, where $k \in \mathbb{N}$, are examples of sets that are countable and finite.
\item The sets $\mathbb{N}$, $\mathbb{Z}$, the set of all odd natural numbers, and the set of all even natural numbers are examples of sets that are countable and countably infinite.
\item We have not yet proved that any set is uncountable.
\end{itemize}

%

\endinput
