\section*{Section~\ref{S:moreaboutfunctions}}

\subsection*{Progress Check~\ref{pr:congfunctions}}
\begin{enumerate}
\item $f(0) = 0$, $f(1) = 1$, $f(2) = 1$, $f(3) = 1$, $f(4) = 1$.
\item $g(0) = 0$, $g(1) = 1$, $g(2) = 2$, $g(3) = 3$, $g(4) = 4$.
\end{enumerate}

\subsection*{Progress Check~\ref{pr:equalfunc}}
$I_{\Z_5} \ne f$ and $I_{\Z_5} = g$.


\subsection*{Progress Check~\ref{pr:average}}
\begin{multicols}{3}
\begin{enumerate}
\item 3.5
\item 4.02
\item $\dfrac{\pi + \sqrt{2}}{4}$
\end{enumerate}
\end{multicols}

\begin{enumerate} \setcounter{enumi}{3}
\item The process of finding the average of a finite set of real numbers can be thought of as a function from $\mathcal{F} ( \R )$ to $\R$.  So the domain is 
$\mathcal{F} ( \R )$, the codomain is $\R$, and we can define a function 
$\text{avg}: \mathcal{F} ( \R ) \to \R$ as follows:  If 
$A \in \mathscr{F} ( \R )$ and $A = \left\{ a_1, a_2, \ldots, a_n \right\}$, then 
$\text{avg} ( A ) = \dfrac{a_1 + a_2 + \cdots + a_n}{n}$.
\end{enumerate}


\subsection*{Progress Check~\ref{pr:sequences}}
\begin{enumerate}
\item The sixth term is $\dfrac{1}{18}$ and the tenth term is $\dfrac{1}{30}$.
\item The sixth term is $\dfrac{1}{36}$ and the tenth term is $\dfrac{1}{100}$.
\item The sixth term is $1$ and the tenth term is $1$.
\end{enumerate}



\subsection*{Progress Check~\ref{pr:function-two}}
\begin{enumerate}
\item $g ( 0, 3 ) = -3$; $g ( 3, -2 ) = 11$; 
$g ( -3, -2 ) = 11$; $g ( 7, -1 ) = 50$.

\item $\left\{ (m, n) \in \Z \times \Z \mid n = m^2 \right\}$
\item $\left\{ (m, n) \in \Z \times \Z \mid n = m^2 - 5 \right\}$
\end{enumerate}
\hbreak


\endinput

