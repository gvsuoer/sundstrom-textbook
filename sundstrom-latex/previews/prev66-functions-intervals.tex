\begin{previewactivity}[\textbf{Functions and Intervals}] \label{PA:functionsandint} \hfill \\
Let $g\x \mathbb{R} \to \mathbb{R}$ be defined by $g ( x ) = x^2$, for each 
$x \in \mathbb{R}$.

\begin{enumerate}
\item We will first determine where $g$ maps the closed interval $\left[ 1, 2 \right]$.  (Recall that 
$[1, 2] = \left\{ x \in \R \mid 1 \leq x \leq 2 \right\}$.)  That is, we will describe, in simpler terms, the set 
$\left\{ g ( x ) \mid x \in \left[ 1, 2 \right] \right\}$.  This is the set of all images of the real numbers in the closed interval $\left[ 1, 2 \right]$.

\begin{enumerate}
\item Draw a graph of the function $g$ using $-3 \leq x \leq 3$.

\item On the graph, draw the vertical lines $x = 1$ and $x = 2$ from the $x$-axis to the graph.   Label the points $P \!\left(1, f ( 1 ) \right)$ and 
$Q \!\left(2, f ( 2 ) \right)$ on the graph.

\item Now draw horizontal lines from the points $P$ and $Q$ to the $y$-axis.  Use this information from the graph to describe the set 
$\left\{ g ( x ) \mid x \in \left[ 1, 2 \right] \right\}$ in simpler terms.  Use interval notation or set builder notation.
\end{enumerate}

\item We will now determine all real numbers that $g$ maps into the closed interval 
$\left[ 1, 4 \right]$.  That is, we will describe the set 
$\left\{ x \in \mathbb{R} \mid g ( x ) \in \left[ 1, 4 \right] \right\}$ in simpler terms.  This is the set of all preimages of the real numbers in the closed interval $\left[ 1, 4 \right]$.

\begin{enumerate}
\item Draw a graph of the function $g$ using $-3 \leq x \leq 3$.

\item On the graph, draw the horizontal lines $y = 1$ and $y = 4$ from the $y$-axis to the graph.   Label all points where these two lines intersect the graph.

\item Now draw vertical lines from the points in Part~(2) to the $x$-axis, and then use the resulting information  to describe the set \linebreak
$\left\{ x \in \mathbb{R} \mid g ( x ) \in \left[ 1, 4 \right] \right\}$ in simpler terms.  (You will need to describe this set as a union of two intervals.  Use interval notation or set builder notation.)
\end{enumerate}
\end{enumerate}
\end{previewactivity}
\hbreak

\endinput
