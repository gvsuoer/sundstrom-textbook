\section*{Section~\ref{S:contradiction} Proof by Contradiction}
Plan at least one and one-half class periods for this section.  


\subsection*{Main Topics}
Proof by contradiction and the logic that justifies the method of proof by contradiction.  After studying this section, it is a good idea to have the students read the comparison of direct proofs, proof using the contrapositive, and proofs by contradiction that starts on page~\pageref{SS:proofcompare} in the summary for Chapter~\ref{C:proofs}.

\subsection*{The Preview Activities}
\subsubsection*{Preview Activity~\ref{PA:contradicton} (Proof by Contradiction)} 
The purpose of this preview activity is to provide a logical basis for the method of proof by contradiction.  A different rationale for the method of proof by contradiction is given in 
Exercise~(\ref{exer:sec33-1}).

\subsubsection*{Preview Activity~\ref{PA:contradiction2} (Proof by Contradiction (continued))}  
In this preview activity, students are asked state the assumptions that need to be made for a  proof by contradiciton.  They may have difficulty with this.  The example in this preview activity is discussed further in Example~\ref{E:contradiction}, and a complete proof is given in Proposition~\ref{P:contradiction}.   Progress Check~\ref{pr:start-con} can be done as a classroom activity after this preview activity and Preview Activity~\ref{PA:rational} are discussed in class.  

\subsubsection*{Preview Activity~\ref{PA:rational} (Rational Numbers)}
A version of the ``classic'' proof that the square root of 2 is irrational is given in 
Theorem~\ref{T:squareroot2}.  In order to understand this proof, most students need to review rational numbers and some properties of rational numbers.  This is done in this preview activity.  In particular, students need to understand that any rational number can be written as a quotient $\dfrac{m}{n}$, where $m$ and $n$ are integers, $ n > 0 $, and $m$ and $n$ have no common factor greater than 1.
\hbreak

%\subsection*{The Activities}
%There are four activities in this section.  Do not plan to complete all of them in class.  Activity~\ref{A:contradiction} should be completed and discussed in class since it is directly related to Preview Activity~\ref{PA:contradicton}.  If they are not completed in class, Activities~\ref{A:exploreproof} and~\ref{A:lineareq} should be assigned along with the exercises.  Activity~\ref{A:quadratic} may be too long to do in class, but it is an excellent activity to use as an out of class group (or individual) assignment.

%\subsection*{Activity~\ref{A:contradiction}}
%In this activity, students are asked to complete the proof by contradiction that was started in 
%Preview Activity~\ref{PA:contradicton}.  Specific directions are given for the algebraic steps to be performed, and of course, there are other algebraic steps that could be performed to reach a contradiction.
%
%\subsection*{Activity~\ref{A:exploreproof}}
%This activity will provide pracice at working with congruence notation.  If the definitions and notations are used correctly, students can arrive at a contradiction fairly easily.

\subsection*{Activity~\ref{A:lineareq} (A Proof by Contradiction)}
One thing that I encourage students to do with a proposition such as this is to use letters other than $x$ and $y$ to represent the solution that is assumed to exist.  This is not necessary but it does help reinforce the concept of a solution of an equation.  Also, it makes a distinction between the variables in the equation and specific values substituted for the variables.

\subsection*{Activity~\ref{A:quadratic} (Exploring a Quadratic Equation)}
The proof by contradiction in this activity is more involved than the ones in the other activities.  Again, encourage the students to use a letter other than $x$ to represent an integer  solution of the equation that is assume to exist.  As indicated earlier,  this activity may be too long to do in class, but it is an excellent activity to use as an out of class group (or individual) assignment.
\hbreak

\subsection*{The Exercises}
Many of these exercises can take students a long time to complete.  Be careful not to assign too many.  Seven or eight exercises (including any of the activities) should be sufficient.  If I have time, I usually discuss Exercise~(\ref{exer:sec33-1}) in class but do not assign it.  
Exercises~(\ref{exer:sec33-2}) and~(\ref{exer:sec33-10}) are good exercises to do after 
Theorem~\ref{T:squareroot2}.  

\vskip6pt
\noindent
Typical Assignment:  Exerices 3, 4, 7, 8, 9, two parts of 14, 15 or 16.


%\hbreak
\endinput

For Exercise~(\ref{exer:sec33-6}), some students will try to use the quadratic formula.  This can work but it gets quite messy.  A proof by contradiction provides a nice alternative method for this exercise.
