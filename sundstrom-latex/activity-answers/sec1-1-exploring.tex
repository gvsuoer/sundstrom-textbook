\documentclass[11pt]{article}
\usepackage{c://pctex/activity}

\lhead{}
\chead{\textbf{\large{Exercise 10 - Section 1.1 \\ Exploring Propositions}}}
\rhead{}
\lfoot{\emph{Mathematical Reasoning: Writing and Proof, Third Ed.} \\Ted Sundstrom}
\cfoot{}
\rfoot{\copyright \the\year\, by Pearson Education, Inc.\\}

%\renewcommand{\labelenumi}{(\textbf{\alph{enumi}})}

\begin{document}
\begin{enumerate}
\item Some possible conjectures are:
\begin{itemize}
\item If $n$ is a natural number, then the units digit of $4^n$ must be 4 or 6.
\item The units digit of the successive powers of 4 alternate between 4 and 6.
\item If $n$ is an odd natural number, then the units of digit of $4^n$ is 4, and if $n$ is an even natural number, then the units of digit of $4^n$ is 6.
\end{itemize}

\item If $n$ is a natural number, then the units digit of $\left( 7^n - 2^n \right)$ is 5.

\item If $f(x) = e^{2x}$, then:
\begin{multicols}{3}
$f'(x) = 2e^{2x}$

$f''(x) = 2^2 e^{2x}$

$f'''(x) = 2^3 e^{2x}$

$f^{(4)}(x) = 2^4 e^{2x}$

$f^{(5)}(x) = 2^5 e^{2x}$

$f^{(6)}(x) = 2^6 e^{2x}$
\end{multicols}

\textbf{Conjecture}:  If $n$ is a natural number, then $f^{(n)}(x) = 2^n e^{2x}$.



\end{enumerate}

\end{document}


\item $\left( {a + b} \right)^2  = a^2  + b^2$  for all real numbers  $a$  and  $b$.

This proposition is false.  A counterexample is  $a = 2$ and $b = 1$.  For these values,
$\left( {a + b} \right)^2 = 9$ and $a^2 + b^2 = 5$.

\item There are integers  $x$  and  $y$  such that  $2x + 5y = 9$.  

This proposition is true as we can see by using  $x = 3$ and  $y = 7$.  We could also use  
$x =  - 2$ and  $y = 9$.  There are many other possible choices for  $x$  and  $y$.

\item If  $x$  and  $y$  are odd integers, then $x \cdot y$ is an odd integer.

This proposition appears to be true.  Any time we use an example where  $x$  is  an odd integer and  $y$  is an odd integer, the product  $x \cdot y$ is an odd integer.  However, we cannot claim that this is true based on examples since we cannot list all of the examples where  $x$  is  an odd integer and  $y$  is an odd integer.
