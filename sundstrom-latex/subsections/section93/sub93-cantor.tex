\subsection*{Cantor's Theorem}
\index{Cantor's Theorem}

We have now seen two different infinite cardinal numbers, $\aleph_0$ and $\boldsymbol{c}$.  It can seem surprising that there is more than one infinite cardinal number.  A reasonable question at this point is, ``Are there any other infinite cardinal numbers?''  The astonishing answer is that there are, and in fact, there are infinitely many different infinite cardinal numbers.  The basis for this fact is the following theorem, which states that a set is not equivalent to its power set.  The proof is  due to Georg Cantor (1845--1918),
\index{Cantor, Georg}%
and the idea for this proof was explored in \typeu Activity~\ref*{PA:powerset}. The basic idea of the proof is to prove that any function from a set $A$ to its power set cannot be a surjection.

\begin{theorem}[\textbf{Cantor's Theorem}]\label{T:cantor}
For every set $A$, $A$ and $\mathcal{P} ( A )$ do not have the same cardinality.
\end{theorem}
%
\begin{myproof}
Let $A$ be a set.  If $A = \emptyset$, then 
$\mathcal{P} ( A ) = \left\{ \emptyset \right\}$, which has cardinality 1.  Therefore, 
$\emptyset$ and $\mathcal{P} ( \emptyset )$ do not have the same cardinality.

Now suppose that $A \ne \emptyset$, and let $f\x  A \to \mathcal{P} ( A )$.  We will show that $f$ cannot be a surjection, and hence there is no bijection from $A$ to 
$\mathcal{P} ( A )$.  This will prove that $A$ is not equivalent to 
$\mathcal{P} ( A )$.  Define
\[
S = \left\{ x \in A \mid x \notin f ( x ) \right\}.
\]
Assume that there exists a $t$ in $A$ such that $f ( t ) = S$.  Now, either $t \in S$ or $t \notin S$.

\begin{itemize}
\item If $t \in S$, then $t \in \left\{ x \in A \mid x \notin f ( x ) \right\}$.  By the definition of $S$, this means that $t \notin f ( t )$.  However, 
$f ( t ) = S$ and so we conclude that $t \notin S$.  But now we have $t \in S$ and $t \notin S$.  This is a contradiction.

\item If $t \notin S$, then $t \notin \left\{ x \in A \mid x \notin f ( x ) \right\}$.  By the definition of $S$, this means that $t \in f ( t )$.  However, 
$f ( t ) = S$ and so we conclude that $t \in S$.  But now we have $t \notin S$ and $t \in S$.  This is a contradiction.
\end{itemize}
So in both cases we have arrived at a contradiction.  This means that there does not exist a 
$t$ in $A$ such that $f ( t ) = S$.  Therefore, any function from $A$ to 
$\mathcal{P} ( A )$ is not a surjection and hence not a bijection.  Hence, $A$ and 
$\mathcal{P} ( A )$ do not have the same cardinality. 
\end{myproof}
%\hbreak
%
\begin{corollary}\label{C:cantor}
$\mathcal{P} ( \mathbb{N} )$ is an infinite set that is not countably infinite.
\end{corollary}
%
\begin{myproof}
Since $\mathcal{P} ( \mathbb{N} )$ contains the infinite subset 
$\left\{ \left\{ 1 \right\}, \left\{ 2 \right\}, \left\{ 3 \right\} \ldots \,\right\}$, we can use 
Theorem~\ref{T:subsetisinfinite}, to conclude that $\mathcal{P} ( \mathbb{N} )$ is an infinite set.  By Cantor's Theorem (Theorem~\ref{T:cantor}), $\mathbb{N}$ and 
$\mathcal{P} ( \mathbb{N} )$ do not have the same cardinality.  Therefore, 
$\mathcal{P} ( \mathbb{N} )$ is not countable and hence is an uncountable set.
\end{myproof}
%
\subsection*{Some Final Comments about Uncountable Sets}
\begin{enumerate}
\item We have now seen that any open interval of real numbers is uncountable and has cardinality \
$\boldsymbol{c}$.  In addition, $\mathbb{R}$ is uncountable and has cardinality $\boldsymbol{c}$.  Now, Corollary~\ref{C:cantor} tells us that $\mathcal{P} ( \mathbb{N} )$ is uncountable.  A question that can be asked is, 
\begin{center}
``Does $\mathcal{P} ( \mathbb{N} )$ have the same cardinality as $\mathbb{R}$?''
\end{center}
The answer is yes, although we are not in a position to prove it yet.  A proof of this fact uses the following theorem, which is known as the Cantor-Schr\"{o}der-Bernstein Theorem.
\label{sec94:comment1}%
\end{enumerate}

\begin{theorem}[\textbf{Cantor-Schr\"{o}der-Bernstein}]\label{T:bernstein}
Let $A$ and $B$ be sets.  If there exist injections $f\x A \to B$ and $g\x B \to A$, then 
$A \approx B$.
\end{theorem}
\index{Cantor-Schr\"{o}der-Bernstein Theorem}%


In the statement of this theorem, notice that it is not required that the function $g$ be the inverse of the function $f$.  We will not prove the Cantor-Schr\"{o}der-Bernstein Theorem here.  
The following items will show some uses of this important theorem. 
%but a proof is given in Section~\ref{S:CSBproof}.  
%The following item will indicate another use of this theorem.

\begin{enumerate} \setcounter{enumi}{1}
\item The Cantor-Schr\"{o}der-Bernstein Theorem can also be used to prove that the closed interval 
$[ 0, 1 ]$ is equivalent to the open interval $( 0, 1 )$.  
See Exercise~(\ref{exer:closedinterval}) on page~\pageref{exer:closedinterval}.

\item Another question that was posed earlier is,
\begin{center}
``Are there other infinite cardinal numbers other than $\aleph_0$ and $\boldsymbol{c}$?''
\end{center}
Again, the answer is yes, and the basis for this is Cantor's Theorem (Theorem~\ref{T:cantor}).
We can start with $\text{card} ( \mathbb{N} ) = \aleph_0$.  We then define the following infinite cardinal numbers:
\begin{multicols}{2}
\begin{list}{}
\item $\text{card} ( \mathcal{P} ( \mathbb{N} ) ) = \alpha_1$.

\item $\text{card} ( \mathcal{P} ( \mathcal{P} ( \mathbb{N} ) ) ) = \alpha_2$.

\item $\text{card} ( \mathcal{P} ( \mathcal{P} ( \mathcal{P} ( \mathbb{N} ) ) ) ) = \alpha_3$. 
\label{sec94:comment3}%
\item $\vdots$
\end{list}
\end{multicols}

%\begin{multicols}{2}
%\begin{list}{}
%\item $\text{card} ( \mathcal{P} ( \mathbb{N} ) )$.
%
%\item $\text{card} ( \mathcal{P} ( \mathcal{P} ( \mathbb{N} ) ) )$.
%
%\item $\text{card} ( \mathcal{P} ( \mathcal{P} ( \mathcal{P} ( \mathbb{N} ) ) ) )$. 
%\label{sec94:comment3}%
%
%\item $\vdots$
%\end{list}
%\end{multicols}
Cantor's Theorem tells us that these are all different cardinal numbers, and so we are just using the lowercase Greek letter $\alpha$ (alpha) to help give names to these cardinal numbers.  In fact, although we will not define it here, there is a way to ``order'' these cardinal numbers in such a way that 
%\[
%\aleph_0 < \text{card} ( \mathcal{P} ( \mathbb{N} ) ) < \text{card} ( \mathcal{P} ( \mathcal{P} ( \mathbb{N} ) ) ) < \aleph_3 < \cdots.
%\]
\[
\aleph_0 < \alpha_1 < \alpha_2 < \alpha_3 < \cdots.
\]
Keep in mind, however, that even though these are different cardinal numbers, Cantor's Theorem does not tell us that these are the only cardinal numbers.

\item In Comment~(\ref{sec94:comment1}), we indicated that 
$\mathcal{P} ( \mathbb{N} )$ and $\mathbb{R}$ have the same cardinality.  Combining this with the notation in Comment~(\ref{sec94:comment3}), this means that 
\[
\alpha_1 = \boldsymbol{c}.
\]
However, this does not necessarily mean that $\boldsymbol{c}$ is the ``next largest'' cardinal number after 
$\aleph_0$.  A reasonable question is, ``Is there an infinite set with cardinality between 
$\aleph_0$ and $\boldsymbol{c}$?''  Rewording this in terms of the real number line, the question is, ``On the real number line, is there an infinite set of points that is not equivalent to the entire line and also not equivalent to the set of natural numbers?''  This question was asked by Cantor, but he was unable to find any such set.  He conjectured that no such set exists.  That is, he conjectured that $\boldsymbol{c}$ is really the next cardinal number after 
$\aleph_0$.  This conjecture has come to be known as the \textbf{Continuum Hypothesis}.
\index{Continuum Hypothesis}%
Stated somewhat more formally, the Continuum Hypothesis is
\begin{center}
There is no set $X$ such that $\aleph_0 < \text{card} ( X ) < \boldsymbol{c}$.
\end{center}
The question of whether the Continuum Hypothesis is true or false is one of the most famous problems in modern mathematics.  

Through the combined work of Kurt G\"{o}del
\index{G\"{o}del, Kurt}%
 in the 1930s and Paul Cohen
\index{Cohen, Paul}%
 in 1963, it has been proved that the Continuum Hypothesis cannot be proved or disproved from the standard axioms of set theory.  This means that either the Continuum Hypothesis or its negation can be added to the standard axioms of set theory without creating a contradiction.

\end{enumerate}
\hbreak 


\endinput

