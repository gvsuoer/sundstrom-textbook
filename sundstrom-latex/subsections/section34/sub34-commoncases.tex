\subsection*{Some Common Situations to Use Cases}
When using cases in a proof, the main rule is that the cases must be chosen so that they exhaust all possibilities for an object $x$ in the hypothesis of the original proposition.  Following are some common uses of cases in proofs.

\newpar
\begin{tabular}{ l l l }
When the hypothesis is,   & Case 1:  $n$ is an even integer.   & \\
  ``$n$ is an integer.''    & Case 2:  $n$ is an odd integer.     & \\
 & & \\
\end{tabular}

\noindent
\begin{tabular}{ l l l }
When the hypothesis is,         &  Case 1: $m$ and $n$ are even. &  \\
``$m$ and $n$ are integers.''   &  Case 2: $m$ is even and $n$ is odd. & \\
                                &  Case 3: $m$ is odd and $n$ is even. & \\
                                &  Case 4: $m$ and $n$ are both odd. \\
 & & \\
\end{tabular}

\noindent
\begin{tabular}{ l l l } 
When the hypothesis is,     & Case 1:  $x$ is rational.  & \\
 ``$x$ is a real number.''  & Case 2:  $x$ is irrational.  & \\
  &  &    \\
\end{tabular}

\noindent
\begin{tabular}{l l l }
 When the hypothesis is,    & Case 1: $x = 0$. \quad OR & Case 1:  $x>0$.  \\
 ``$x$ is a real number.''  & Case 2: $x \ne 0$.    & Case 2:  $x = 0$. \\
                            &                      & Case 3:  $x<0$. \\
                            %& Often, Cases (1) and (2) can be combined. \\
  &  &   \\
\end{tabular}

\noindent
\begin{tabular}{ l l l }
When the hypothesis is,          & Case 1: $a = b$. \quad  OR  &   Case 1: $a > b$. \\
``$a$ and $b$ are real           & Case 2: $a \ne b$.        &  Case 2: $a = b$. \\
      numbers.''                 &                          &  Case 3: $a < b$. \\
  &  &  \\
\end{tabular}

\hbreak


\endinput


\endinput
