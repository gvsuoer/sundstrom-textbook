\section*{Section \ref{S:equivrelations}}

\begin{list}{\bf{\ref{exer:sec72-1}.}}
\item The relation $R$ is not reflexive on $A$ and is not symmetric.  However, it is transitive since the conditional statement ``For all $x, y, z \in A$, if $x \mathrel{R} y$ and $y \mathrel{R} z$, then $x \mathrel{R} z$'' is a true conditional statement since the hypothesis will always be false.
\end{list}


\begin{list}{\bf{\ref{exer71:notidentity}.}}
\item There are many possible equivalence relations on this set.  Perhaps one of the easier ways to determine one is to first decide what elements will be equivalent.  For example, suppose we say that we want 1 and 2 to be equivalent (and of course, all elements will be equivalent to themselves).  So if we use the symbol $\sim$ for the equivalence relation, then we need $1 \sim 2$ and $2 \sim 1$.  Using set notation, we can write this equivalence relation as
\[
\{ (1, 1), (2, 2), (3, 3), (4, 4), (5, 5), (1, 2), (2, 1) \}.
\]
\end{list}



\begin{list}{\bf{\ref{exer:sec72-2}.}}
\item The relation $R$ is not reflexive on $A$.  For example, $\left( 4, 4 \right) \notin R$.
The relation $R$ is symmetric.  If $\left( a, b \right) \in R$, then 
$\left| a \right| + \left| b \right| = 4$.  Therefore, $\left| b \right| + \left| a \right| = 4$, and hence, $\left( b, a \right) \in R$.
The relation $R$ is not transitive.  For example, $\left( 4, 0 \right) \in R$, 
$\left( 0, 4 \right) \in R$, and $\left( 4, 4 \right) \notin R$.
The relation $R$ is not an equivalence relation.
\end{list}


\begin{list}{\bf{\ref{exer:equivrelwithfunction}.}}
\item \begin{list}{\bf{(a)}}
\item The relation $\sim$ is an equivalence relation.  

For $a \in \mathbb{R}$, $a \sim a$ since $f \left( a \right) = f \left( a \right)$.  So, $\sim$ is reflexive.

For $a, b \in \mathbb{R}$, if $a \sim b$, then  $f \left( a \right) = f \left( b \right)$.  So, 
$f \left( b \right) = f \left( a \right)$. Hence, $b \sim a$ and $\sim$ is symmetric.

For $a, b, c  \in \mathbb{R}$, if $a \sim b$ and $b \sim c$, then  
$f \left( a \right) = f \left( b \right)$ and $f \left( b \right) = f \left( c \right)$.  So, 
$f \left( a \right) = f \left( c \right)$. Hence, $a \sim c$ and $\sim$ is transitive.

\end{list}
\end{list}

\begin{list}{}
\item \begin{list}{\bf{(b)}}
\item $C = \left \{ -5, 5 \right \}$
\end{list}
\end{list}


\begin{list}{\bf{\ref{exer:oneequivonenot}.}}
\item \begin{list}{\bf{(a)}} 
\item The relation $\sim$ is an equivalence relation on $\Z$.  It is reflexive since for each integer $a$, $a + a = 2a$ and hence, $2$ divides $a + a$.  Now let $a, b \in \Z$ and assume that 2 divides $a + b$.  Since $a + b = b + a$, 2 divides $b + a$ and hence, $\sim$ is symmetric.  Finally, let $a, b, c \in \Z$ and assume that 
$a \sim b$ and $b \sim c$.  Since 2 divides $a + b$, $a$ and $b$ must both be odd or both be even.  In the case that $a$ and $b$ are both odd, then $b \sim c$ implies that $c$ must be odd.  Hence, $a + c$ is even and $a \sim c$.  A similar proof shows that if $a$ and $b$ are both even, then $a \sim c$.  Therefore, $\sim$ is transitive.
\end{list}
\end{list}


\begin{list}{\bf{\ref{exer:sec72-circles}.}}
\item \begin{list}{\bf{(c)}}
\item The set $C$ is a circle of radius 5 with center at the origin.
\end{list}
\end{list}
\hbreak
\endinput


