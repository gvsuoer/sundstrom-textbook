\section*{Section~\ref{S:contradiction}}

\subsection*{Progress Check \ref{pr:start-con}}
\begin{enumerate}
\item There exists a real number $x$ such that $x$ is irrational and $\sqrt[3]{x}$ is rational.
\item There exists a real number $x$ such that $\left(x + \sqrt{2} \right)$ is rational and 
$\left(-x + \sqrt{2} \right)$ is rational.
\item There exist integers $a$ and $b$ such that 5 divides $ab$ and 5 does not divide $a$ and 5 does not divide $b$.
%\item There exist real numbers $x$ and $y$ such that $x$ is rational and $y$ is irrational and  
%$x + y$ is rational.
\item There exist real numbers $a$ and $b$ such that $a > 0$ and $b > 0$ and  
$\dfrac{2}{a} + \dfrac{2}{b} = \dfrac{4}{a + b}$.
\end{enumerate}


\subsection*{Progress Check \ref{pr:exploreproof}}
\begin{enumerate}
\item Some integers that are congruent to  2  modulo  4 are  $-6, -2, 2, 6, 10$, and some integers that are congruent to  3  modulo  6 are:  $-9, -3, 3, 9, 15$.  There are no integers that are in both of the lists.

\item For this proposition, it is reasonable to try a proof by contradiction since the conclusion is stated as a negation.

\item 
\begin{myproof}
\setcounter{equation}{0}
We will use a proof by contradiction.  Let  $n \in \mathbb{Z}$ and assume that  
$n \equiv 2 \pmod 4$ and that $n \equiv 3 \pmod 6$.  Since  $n \equiv 2 \pmod 4$, we know that  4  divides  $n - 2$.  Hence, there exists an integer  $k$  such that
\begin{equation} \label{eq:exploreproof1}
n - 2 = 4k.
\end{equation}
%
We can also use the assumption that  $n \equiv 3 \pmod 6$ to conclude that  6  divides  $n - 3$
 and that there exists an integer  $m$  such that
\begin{equation} \label{eq:exploreproof2}
n - 3 = 6m.
\end{equation}
%
If we now solve equations~(\ref{eq:exploreproof1})  and~(\ref{eq:exploreproof2})  for  $n$  and set the two expressions equal to each other, we obtain
\[
4k + 2 = 6m + 3.
\]
However, this equation can be rewritten as
\[
2\left( {2k + 1} \right) = 2\left( {3m + 1} \right) + 1.
\]
Since  $2k + 1$ is an integer and  $3m + 1$ is an integer, this last equation is a contradiction since the left side is an even integer and the right side is an odd integer.  Hence, we have proven that if  $n \equiv 2 \pmod 4$, then  $n\not  \equiv 3 \pmod 6$.
\end{myproof}
\end{enumerate}



\subsection*{Progress Check~\ref{prog:persquare}}
\begin{enumerate}
  \item $x^2 + y^2 = (2m + 1)^2 + (2n + 1)^2 = 2 \left( 2m^2 + 2m + 2n^2 + 2n + 1 \right)$.
  \item Using algebra to rewrite the last equation, we obtain
\[
4m^2 + 4m + 4n^2 + 4n + 2 = 4k^2.
\]
If we divide both sides of this equation by 2, we see that 
$2m^2 + 2m + 2n^2 + 2n + 1 = 2k^2$ or
\[
2 \left(m^2 + m + n^2 + n \right) + 1 = 2k^2.
\]
However, the left side of the last equation is an odd integer and the right side is an even integer.  This is a contradiction, and so we have proved that for all integers $x$ and $y$, if $x$ and $y$ are odd integers, then there does not exist an integer $z$ such that 
$x^2 + y^2 = z^2$.
\end{enumerate}
\hbreak



\endinput
