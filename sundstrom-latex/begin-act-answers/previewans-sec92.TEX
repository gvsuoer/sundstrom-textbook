\documentclass[11pt]{article}
\usepackage{../../styles/activity}

\usepackage{xr}
\externaldocument{0-MR}

\lhead{}
%\chead{\textbf{\Large{\hspace{0pt}Beginning Activities for Section~9.2}}\\\hspace{0pt}\emph{Mathematical Reasoning: Writing and Proof}}
\bahead{9.2}
\rhead{}
\lfoot{}
\rfoot{}
\cfoot{\hspace{0pt}\scalebox{0.4}{\includegraphics{cc-by-nc-sa.eps}}}
\graphicspath{{./epsfigs/}}

\begin{document}
\subsection*{Beginning Activity 1 (Introduction to Infinite Sets)}
\begin{enumerate}
\item The contrapositive of ``If $A$ is a finite set, then $A$ is not equivalent to any of its proper subsets" is
\begin{center}
If $A$ is equivalent to one of its proper subsets, then $A$ is an infinite set.
\end{center}
The contrapositive of ``For each set $A$, if $A$ is a finite set, then for each proper subset of $B$ of $A$, $A \not\approx B$'' is
\begin{center}
For each set $A$, if there exists a proper subset $B$ of $A$ such that $A \approx B$, then $A$ is an infinite set.
\end{center}
 This means that if we can show that there exists a set $B$ such that $B \subset A$ and 
$A \approx B$, then we can conclude that $A$ is an infinite set.

\item \begin{enumerate}
\item Since $D^+$ is a proper subset of $\mathbb{N}$ and $\mathbb{N} \approx D^+$, we conclude that $\mathbb{N}$ is an infinite set.  

\item We can also conclude that $D^+$ is an infinite set since if $D^+$ is finite, then there exists a $k \in \mathbb{N}$ such that $D^+ \approx \mathbb{N}_k$.  Since we proved in part~(a) that $\N \approx D^+$, we can then use part~(c) of Theorem~9.1, to conclude that $\approx$, $\mathbb{N} \approx \mathbb{N}_k$ and this is a contradiction since $\mathbb{N}$ is infinite and $\mathbb{N}_k$ is finite.
\end{enumerate}

\item \begin{enumerate}
\item If $0 < b < 1$, then $\left( 0, b \right)$ is a proper subset of 
$\left( 0, 1 \right)$ and 
$\left( 0, 1 \right) \approx \left( 0, b \right)$.  Hence, $\left( 0, 1 \right)$ is an infinite set.

\item If $b > 1$, then $\left( 0, 1 \right)$ is a proper subset of $\left( 0, b \right)$ and 
$\left( 0, 1 \right) \approx \left( 0, b \right)$.  Hence, $\left( 0, b \right)$ is an infinite set.
\end{enumerate}
\end{enumerate}
\hbreak



\newpage
\subsection*{Beginning Activity 2 (A Function from $\boldsymbol{\mathbb{N}}$ to 
$\boldsymbol{\mathbb{Z}}$)}

\begin{enumerate}
\item If the pattern continues, 
\begin{multicols}{2}
$f \left( 8 \right) = 4$

$f \left( 10 \right) = 5$

$f \left( 12 \right) = 6$

$f \left( 9 \right) = -4$

$f \left( 11 \right) = -5$

$f \left( 13 \right) = -6$
\end{multicols}

\item If the pattern continues, it would appear that $f$ is a bijection.

\item It appears that if $n$ is even, then $f \left( n \right) = \dfrac{n}{2}$.

\item It appears that if $n$ is odd, then $f \left( n \right) = \dfrac{1-n}{2}$.

\item Define $f:\mathbb{N} \to \mathbb{Z}$ where
%
\begin{equation} \notag
f \left( n \right) = 
\begin{cases}
\dfrac{n}{2}         &\text{if $n$ is even} \\
                      &                      \\
\dfrac{1-n}{2}       &\text{if $n$ is odd}
\end{cases}
\end{equation}

\item \begin{enumerate}
\item The results are consistent with the pattern.

\item $f \left( 1000 \right) = 500$ and $f \left( 1001 \right) = -500$.

\item $f \left( 2000 \right) = 1000$.
\end{enumerate}
\end{enumerate}
We will prove that this function is a bijection in Theorem~9.13, and hence $\N \approx \Z$.
\hbreak






\end{document}








\subsection*{Beginning Activity 3 (Rational Numbers between Rational Numbers)}
Let $a, b \in \mathbb{Q}$ with $a < b$.
\begin{enumerate}
\item Since the rational numbers are closed with respect to addition and with respect to division by nonzero rational numbers, $c = \dfrac{a + b}{2}$ is a rational number.  In addtion, since 
$a < b$,
\begin{align}
a + a &< a + b   & a + b &< b + b \notag \\
2a &< a + b      & a + b &< 2b \notag \\
a &< \frac{a + b}{2} &  \frac{a + b}{2} &< b \notag
\end{align}
This proves that $a < \dfrac{a + b}{2} < b$.

\item Since the rational numbers are closed with respect to addition and with respect to division by nonzero rational numbers, $c_2 = \dfrac{c_1 + b}{2}$ is a rational number.  In addtion, since 
$c_1 < b$,
\begin{align}
c_1 + c_1 &< c_1 + b   & c_1 + b &< b + b \notag \\
2c_1 &< c_1 + b      & c_1 + b &< 2b \notag \\
c_1 &< \frac{c_1 + b}{2} &  \frac{c_1 + b}{2} &< b \notag
\end{align}
This proves that $c_1 < \dfrac{c_1 + b}{2} < b$ or that $c_1 < c_2 < b$.


\item For each $k \in \mathbb{N}$, define
\[
c_{k+1} = \frac{c_k + b}{2}.
\]
From Part~(1), we know that $c_1 \in \mathbb{Q}$ and $a < c_1 < b$.  For each $n \in \mathbb{N}$, let $P \left( n \right)$ be, 
\begin{center}
$c_{n+1} \in \mathbb{Q}$ and $a < c_n < c_{n+1} < b$.
\end{center}
\noindent
\textbf{Basis Step}:  For $n = 1$, we use Part~(2) to conclude that $c_2 \in \mathbb{Q}$ and that 
$c_1 < c_2 < b$.  Hence, $a < c_1 < c_2 < b$ and $P \left( 1 \right)$ is true.

\noindent
\textbf{Inductive Step}:  Now let $k \in \mathbb{N}$ and assume that $P \left( k \right)$ is true.  That is, 
\begin{center}
$c_{k+1} \in \mathbb{Q}$ and $a < c_k < c_{k+1} < b$.
\end{center}

We need to prove that $P(k+2)$ is true or that 
\begin{center}
$c_{k+2} \in \mathbb{Q}$ and $a < c_{k+1} < c_{k+2} < b$.
\end{center}

Since $c_{k+2} = \dfrac{c_{k+1} + b}{2}$, we see that $c_{k+2} \in \mathbb{Q}$.  In addition,
since $c_{k+1} < b$,
\begin{align}
c_{k+1} + c_{k+1} &< c_{k+1} + b   & c_{k+1} + b &< b + b \notag \\
2c_{k+1} &< c_{k+1} + b      & c_{k+1} + b &< 2b \notag \\
c_{k+1} &< \frac{c_{k+1} + b}{2} &  \frac{c_{k+1} + b}{2} &< b \notag
\end{align}
This proves that $c_{k+1} < \dfrac{c_{k+1} + b}{2} < b$ or that $c_{k+1} < c_{k+2} < b$.

Hence, we may conclude that $a < c_{k+1} < c_{k+2} < b$ and that 
$P \left( k + 1 \right)$ is true.  Hence, by mathematical induction, $P \left( n \right)$ is true for all $n \in \mathbb{N}$ and we may conclude that for each $n \in \mathbb{N}$, 
$c_{n+1} \in \mathbb{Q}$ and $a < c_n < c_{n+1} < b$.

\vskip6pt
So, we can define a bijection $f: \mathbb{N} \to \left\{ c_k \mid k \in \mathbb{N} \right\}$ by 
$f \left( k \right) = c_k$ for each $k \in \mathbb{N}$.  This proves that 
$\left\{ c_k \mid k \in \mathbb{N} \right\}$ is a countably infinite set where each element is a rational number between $a$ and $b$.


\end{enumerate}
