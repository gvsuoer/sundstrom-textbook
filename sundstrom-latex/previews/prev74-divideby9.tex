\begin{previewactivity}[\textbf{The Remainder When Dividing by 9}] \label{PA:remainderdivide9} \hfill \\
If  $a$  and  $b$  are integers with  $b > 0$, then from the Division Algorithm, we know that there exist unique integers  $q$  and  $r$ such that  
\[
a = bq + r \text{ and } 0 \leq r < b.
\]
In this activity, we are interested in the remainder  $r$.  Notice that  $r = a - bq$.  So, given $a$ and $b$, if we can calculate  $q$, then we can calculate  $r$.  

We can use the ``int'' function on a calculator to calculate  $q$.  [The ``int'' function is the ``greatest integer function.''  If  $x$  is a real number, then  
$\operatorname{int}( x )$  is the greatest integer that is less than or equal to  $x$.]

So, in the context of the Division Algorithm,  
$q = \operatorname{int} \!\left( {\dfrac{a}{b}} \right)$.  Consequently,
\[
r = a - b \cdot \operatorname{int} \!\left( {\frac{a}{b}} \right).
\]
If  $n$  is a positive integer, we will let  $s\left( n \right)$  denote the sum of the digits of  $n$.  For example,  if  $n = 731$, then
\[
s( {731} ) = 7 + 3 + 1 = 11.
\]
\noindent
For each of the following values of  $n$, calculate
\begin{itemize}
  \item The remainder when $n$ is divided by 9, and
  \item The value of $s( n )$ and the remainder when $s( n )$ is divided by 9.
\end{itemize}

%\vskip6pt
%\begin{tabular}{ l l}
%  & The remainder when $n$ is divided by 9; \\
%  & The value of $s( n )$ and the remainder when $s( n )$ is divided by 9.  \\ 
%%  & The remainder when $s\left( n \right)$ is divided by 9.  \\
%\end{tabular}
%
\begin{multicols}{3}
\begin{enumerate}
\item $n=498$
\item $n=7319$
\item $n=4672$
\item $n=9845$
\item $n=51381$
\item $n=305877$
\end{enumerate}
\end{multicols}
\noindent
What do you observe?
\end{previewactivity}
\hbreak

\endinput
