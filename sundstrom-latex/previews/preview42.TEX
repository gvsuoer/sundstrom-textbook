\begin{previewactivity}[\textbf{Exploring a Proposition about Factorials}] \label{PA:factorials} \hfill

\begin{defbox}{nfactorial}{If  $n$  is a natural number, we define \textbf{$\boldsymbol{n}$  factorial},
\index{factorial}%
 denoted by $n!$ \label{sym:factorial},  to be the product of the first  $n$  natural numbers.  In addition, we define  $0!$ to be equal to  1.}
\end{defbox} 
Using this definition, we see that
\begin{align*}
0! &= 1  &  3! &= 1 \cdot 2 \cdot 3 = 6 \\
1! &=1   &  4! &= 1 \cdot 2 \cdot 3 \cdot 4 = 24\\
2! &= 1 \cdot 2 = 2  &  5! &= 1 \cdot 2 \cdot 3 \cdot 4 \cdot 5 = 120.
\end{align*}
%
\noindent
In general, we write  $n! = 1 \cdot 2 \cdot 3   \cdots \left( {n - 1} \right) \cdot n$ or  $n! = n \cdot \left( {n - 1} \right)  \cdots 2 \cdot 1$.  Notice that  for any natural number $n$, $n! = n \cdot (n-1)!$.
%
\begin{enumerate}
\item Compute the values of $2^n$ and $n!$ for each natural number  $n$  with  $1 \leq n \leq 7$. 
\label{pa:5211}
\end{enumerate}
Now let $P(n)$ be the open sentence, ``$n! > 2^n$.''
\setcounter{oldenumi}{\theenumi}
\begin{enumerate} \setcounter{enumi}{\theoldenumi}
  \item Which of the statements $P(1)$ through $P(7)$ are true?
\item Based on the evidence so far, does the the following proposition appear to be true or false? \label{pa:5212}
For each natural number  $n$ with $n \geq 4$,
$n! > 2^n $.
\end{enumerate}
\setcounter{equation}{0}
Let $k$ be a natural number with $k \geq 4$.  Suppose that we want to prove that if $P(k)$ is true, then $P(k+1)$ is true.  (This could be the inductive step in an induction proof.)  To do this, we would be assuming that $k! > 2^k$ and would need to prove that $(k+1)! > 2^{k+1}$.  Notice that if we multiply both sides of the inequality $k! > 2^k$ by $(k + 1)$, we obtain
\begin{equation} \label{pa:52inequality}
(k + 1)\cdot k! > (k + 1) 2^k. 
\end{equation}
\setcounter{oldenumi}{\theenumi}
\begin{enumerate} \setcounter{enumi}{\theoldenumi}
\item In the inequality in~(\ref{pa:52inequality}), explain why $(k + 1) \cdot k! = (k + 1)!$.
\item Now look at the right side of the inequality in~(\ref{pa:52inequality}).  Since we are assuming that $k \geq 4$, we can conclude that $(k+1) > 2$.  Use this to help explain why $(k + 1)2^k > 2^{k+1}$.
\item Now use the inequality in~(\ref{pa:52inequality}) and the work in steps~(4) and~(5) to explain why 
$(k+1)! > 2^{k+1}$.
\end{enumerate}

%\item If the proposition in Part~(\ref{pa:5212}) is true, construct a proof.  If it is false, rewrite it (by adding a condition on the natural number  $n$) so that the new proposition appears to be true (based on the data in Part~(\ref{pa:5211})).
\end{previewactivity}
\hrule
%
%
%
\begin{previewactivity}[\textbf{Prime Factors of a Natural Number}] \label{PA:primefactors} \hfill \\
Recall that a natural number  $p$  is  a \textbf{prime number}
\index{prime number}%
 provided that it is greater than 1 and the only natural numbers that divide  $p$  are  1  and  $p$.  A natural number other than 1 that is not a prime number is a \textbf{composite number}.
\index{composite number}%
  The number 1 is neither prime nor composite.
\begin{enumerate}
\item Give examples of four natural numbers that are prime and four natural numbers that are composite.

\item Write each of the natural numbers  20, 40, 50, and 150  as a product of prime numbers.  \label{PA:primefactors2}

%\item Repeat Part~(\ref{PA:primefactors2}) using  50  and  150. \label{PA:primefactors3}

\item Do you think that any composite number can be written as a product of prime numbers?

\item Write a useful description of what it means to say that a natural number is a composite number (other than saying that it is not prime).

\item Based on your work in Part~(\ref{PA:primefactors2}), do you think it would be possible to use induction to prove that any composite number can be written as a product of prime numbers?
\end{enumerate}
\end{previewactivity}
\hbreak

%\begin{previewactivity}[Subsets of a Set with Four Elements] \label{PA:subsetsofaset4} \hfill \\
%Review Preview Activity~\ref{PA:subsets} from Section~\ref{S:setoperations}.  In this activity, we saw that a set with one element has two subsets, a set with two elements has four subsets, and a set with three elements has eight subsets.  The following list shows the eight subsets of the set  
%$B = \left\{ {a,b,c} \right\}$.
%\begin{multicols}{4}
%$\emptyset $
%
%$\left\{ a, b \right\}$
%
%$\left\{ a \right\}$
%
%$\left\{ {a,c} \right\}$
%
%$\left\{ b \right\}$
%
%$\left\{ {b,c} \right\}$
%
%
%$\left\{ {c} \right\}$
%
%$\left\{ {a,b,c} \right\}$
%
%\end{multicols}
%\noindent
%Now let  $A = \left\{ {a, b, c, x} \right\}$.
%
%\begin{enumerate}
%\item Are the eight subsets of  $B$ also subsets of  $A$?  Explain.
%
%\item Create subsets of  $A$  by starting with the sets listed above and ``adding''  $x$  to each subset.  That is, if  $C$  is a subset of  $B$ (one of the eight listed above), then create a subset of  $A$ by using  $C \cup \left\{ x \right\}$. \label{pa:5222}
%
%\item How many subsets of  $A$ did you form in Part~(\ref{pa:5222})?  Are all of the subsets of  $A$  subsets of $B$ or listed in Part~(\ref{pa:5222})?  Explain.
%
%\item Use this work to explain why  
%$\mathcal{P}(A) = \mathcal{P}(B) \cup \left\{ {C \cup \left\{ x \right\} \mid {C \in \mathcal{P}(B)} } \right\}$.
%
%\end{enumerate}
%\end{previewactivity}
%\hbreak

\endinput
