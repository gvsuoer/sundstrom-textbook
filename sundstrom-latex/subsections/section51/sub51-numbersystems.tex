\subsection*{Standard Number Systems}
We can use set notation to specify and help describe our standard number systems.  The starting point is the set of \textbf{natural numbers},
\index{natural numbers}%
 for which we use the roster method.
\[
\N = \left\{ 1, 2, 3, 4, \ldots \;\right\}
\]
The \textbf{integers}
\index{integers}%
 consist of the natural numbers, the negatives of the natural numbers, and zero.  If we let $\N^- = \left\{ \ldots, -4, -3, -2, -1 \right\}$, then we can use set union and write
\[
\Z = \N^- \cup \left\{ 0 \right\} \cup \N.
\]
So we see that $\N \subseteq \Z$, and in fact, $\N \subset \Z$.

We need to use set builder notation for the set $\Q$ of all \textbf{rational numbers}, which consists of quotients of integers.
\[
\Q = \left\{ \left. \frac{m}{n}  \right| m, n \in \Z \text{ and } n \ne 0 \right\}
\]
Since any integer $n$ can be written as $n = \dfrac{n}{1}$, we see that $\Z \subseteq \Q$.  

We do not yet have the tools to give a complete description of the real numbers.
\index{real numbers}%
  We will simply say that the \textbf{real numbers} consist of the rational numbers and the \textbf{irrational numbers}.  In effect, the irrational numbers are the complement of the set of rational numbers $\Q$ in $\R$.  So 
\index{irrational numbers}%
  we can use the notation $\Q^c = \left\{ x \in \R \mid x \notin \Q \right\}$ and write
\[
\R = \Q \cup \Q^c \qquad \text{and} \qquad \Q \cap \Q^c = \emptyset.
\]
A number system that we have not yet discussed is the set of \textbf{complex numbers}.
\index{complex numbers}%
  The complex numbers, $\C$,  consist of all numbers of the form $a + bi$, where $a, b \in \R$ and $i = \sqrt{-1}$ (or $i^2 = -1$).  That is,
\[
\C = \left\{  a + bi \left| \hskip3pt a, b \in \R \text{ and } i = \sqrt{-1} \right. \right\}\!.
\]
We can add and multiply complex numbers as follows:  If $a, b, c, d \in \R$, then
\[
\begin{aligned}
\left( a + bi \right) + \left( c + di \right) &= \left(a + c \right) + \left(b + d \right)i, \text{ and} \\
\left( a + bi \right) \left( c + di \right) &= ac + adi + bci + bdi^2 \\
                                            &= \left(ac - bd \right) + \left(ad + bc \right)i.
\end{aligned}
\]
%\hbreak

\endinput
