\section*{Section \ref{S:moreaboutfunctions}}
\renewcommand{\labelenumi}{(\textbf{\alph{enumi}})}

\begin{list}{\bf{\ref{exer:sec62-1}.}}
\item \begin{list}{\bf{(a)}}
\item $f( 0 ) = 4$, $f( 1 ) = 0, f( 2 ) = 3$, 
$f( 3 ) = 3$, $f( 4 ) = 0$
\end{list}
\end{list}


\begin{list}{}
\item \begin{list}{\bf{(b)}}
\item $g( 0 ) = 4$, $g( 1 ) = 0, g( 2 ) = 3$, 
$g( 3 ) = 3$, $g( 4 ) = 0$
\end{list}
\end{list}


\begin{list}{}
\item \begin{list}{\bf{(c)}}
\item The two functions are equal.
\end{list}
\end{list}


%\begin{list}{\bf{\ref{exer:sec62-2}.}}
%\item \begin{list}{\bf{(c)}}
%\item The two functions are not equal.  For example, $f(1) = 5$ and $g(1) = 4$.
%\end{list}
%\end{list}


\begin{list}{\bf{\ref{exer62:realfunction}.}}
\item \begin{enumerate}
\item $f(2) = 9$, $f(-2) = 9$, $f(3) = 14$, $f(\sqrt{2} = 7$

\item $g(0) = 5$, $g(2) = 9$, $g(-2) = 9$, $g(3) = 14$, $g(\sqrt{2}) = 7$

\item The function $f$ is not equal to the function $g$ since they do not have the same domain.

\item The function $h$ is equal to the function $f$ since if $x \ne 0$, then 
$\dfrac{x^2 + 5x}{x} = x^2 + 5$.

\end{enumerate}

\end{list}


\begin{list}{\bf{\ref{exer:sec62-3}.}} 
\item \begin{list}{\bf{(a)}}
\item $\left\langle {a_n } \right\rangle $,  where  $a_n  = \dfrac{1}{{n^2 }}$  for each  
$n \in \mathbb{N}$.  The domain is  $\mathbb{N}$, and $\Q$ can be the codomain.
\end{list}
\end{list}
%
\begin{list}{}
\item \begin{list}{\bf{(d)}}
\item $\left\langle {a_n } \right\rangle $,  where  $a_n  = \cos ( {n\pi } )$  for each  $n \in \mathbb{N} \cup {0}$.  The domain is  $\mathbb{N} \cup {0}$, and $\left\{ -1, 1 \right\}$ can be the codomain.  This sequence is equal to the sequence in Part~(c).
\end{list}
\end{list}


\begin{list}{\bf{\ref{sym:projfunc}.}}
\item \begin{list}{\bf{(a)}}
\item $p_1(1, x) = 1$, $p_1(1, y) = 1$, $p_1(1, z) = 1$, $p_1(2, x) = 2$, $p_1(2, y) = 2$, 
$p_1(2, z) = 2$
\end{list}
\end{list}

\begin{list}{}
\item \begin{list}{\bf{(c)}}
\item $\text{range}( p_1 ) = A$, $\text{range}( p_2 ) = B$
\end{list}
\end{list}




\begin{list}{\bf{\ref{exer:sec62-6}.}}
\item Start of the inductive step:  Let $P ( n )$ be ``A convex polygon with $n$ sides has $\dfrac{n ( n-3 )}{2}$ diagonals.''  
Let  $k \in D$  and assume that $P ( k )$ is true, that is, a convex polygon with  $k$  sides has  $\dfrac{{k( {k - 3} )}}{2}$  diagonals.   Now let  $Q$  be convex polygon with  $( {k + 1} )$ sides.  Let  $v$  be one of the $( {k + 1} )$ vertices of $Q$  and let  $u$  and  $w$  be the two vertices adjacent to  $v$.  By drawing the line segment from  $u$  to  $w$ and omitting the vertex  $v$, we form a convex polygon with  $k$  sides.  Now complete the inductive step.   
\end{list}


\begin{list}{\bf{\ref{exer:sec61-7}.}}
\item \begin{list}{\bf{(a)}}
\item $f( { - 3, 4} ) = 9$, $f( { - 2, - 7} ) =  - 23$
\end{list}
\end{list}
%
\begin{list}{}
\item \begin{list}{\bf{(b)}}
\item $\text{range}(f) = \left\{ { {( {m, n} ) \in \mathbb{Z} \times \mathbb{Z} } \mid m = 4 - 3n} \right\}$
\end{list}
\end{list}


\begin{list}{\bf{\ref{exer:sec61-8}.}}
\item \begin{enumerate}
\item $g ( 3, 5 ) = ( 6, -2 )$, \qquad
$g ( -1, 4 ) = ( -2, -5 )$.
\addtocounter{enumi}{1}
%\item $( 0, 0 )$ is the only preimage of $( 0, 0 )$.

\item The set of  preimages of $( 8, -3 )$ is $\left\{ ( 4, 7 ) \right\}$. 

%\item The set of  preimages of $( 1, 1 )$ is $\emptyset$ since the first coordinate of $f(m, n)$ is always an even integer.

%\item Part~(d) shows that the statement is false since there does not exist an 
%$( m, n ) \in \mathbb{Z} \times \mathbb{Z}$ such that 
%$f ( m, n ) = ( 1, 1 )$.
\end{enumerate}

\end{list}



\begin{list}{\bf{\ref{exer:determinant}.}}
\item \begin{list}{\bf{(a)}}
\item $\det \left[ {\begin{array}{*{20}c}
   3 & 5  \\
   4 & 1  \\
\end{array} } \right] =  - 17, \det \left[ {\begin{array}{*{20}c}
   1 & 0  \\
   0 & 7  \\
 \end{array} } \right] = 7, \text{and det}\left[ {\begin{array}{*{20}c}
   3 & { - 2}  \\
   5 & 0  \\
 \end{array} } \right] = 10$
\end{list}
\end{list}

\hbreak
\endinput


