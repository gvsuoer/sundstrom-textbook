\subsection*{Notation for Relations}
The mathematical relations in Table~\ref{Ta:standardrelations} all used a relation symbol between the two elements that form the ordered pair in  $A \times B$.  For this reason, we often do the same thing for  a general relation from the set  $A$  to the set  $B$.  So if  $R$  is a relation from  $A$  to  $B$, and  $x \in A$ and  $y \in B$, we use the notation  

\label{sym:xrelatedy}
\begin{center}
\begin{tabular}{c l l}
$x \mathrel{R} y$  &  to mean  &  $\left( {x, y} \right) \in R$; and \\
$x \mathrel{\not \negthickspace R} y$  &  to mean  &  
$\left( {x, y} \right) \notin R$. \\
\end{tabular}
\end{center}
In some cases, we will even use a generic relation symbol for defining a new relation or speaking about relations in a general context.  Perhaps the most commonly used symbol is  
``$\sim $'',  read ``tilde'' or ``squiggle'' or ``is related to.''  When we do this, we will write

\label{sym:xtwiddley} 
\begin{center}
\begin{tabular}{c l l}
$x \sim y$  &  means the same thing as  &  $\left( {x, y} \right) \in R$; and \\
$x \nsim y$  &  means the same thing as  &  $\left( {x, y} \right) \notin R$.  \\
\end{tabular}
\end{center}
%
\hbreak
%
\begin{prog}[\textbf{The Divides Relation}] \label{prog:dividesrelation} \hfill \\
Whenever we have spoken about one integer dividing another integer, we have worked with the ``divides''
\index{divides}%
\index{relation!divides}%
 relation on  $\mathbb{Z}$.  In particular, we can write
\[
D = \left\{ { {\left( {m, n} \right) \in \mathbb{Z} \times \mathbb{Z} } \mid m\text{  divides  }n} \right\}\!.
\]
In this case, we have a specific notation for ``divides,'' and we write
\begin{center}
$m \mid n$ \quad if and only if  \quad $\left( {m, n} \right) \in D$.
\end{center}

\begin{enumerate}
\item What is the domain of the ``divides'' relation?  What is the range of the ``divides'' relation?

\item Are the following statements true or false?  Explain.

\begin{enumerate}
    \item For every  nonzero integer $a$, $a \mid a$.

    \item For all nonzero integers $a$ and $b$, if  $a \mid b$, then  $b \mid a$.

    \item For all  nonzero integers $a$, $b$, and $c$,  if  $a \mid b$ and   $b \mid c$, then  
     $a \mid c$.

\end{enumerate}
\end{enumerate}
\end{prog}
\hbreak

\endinput
