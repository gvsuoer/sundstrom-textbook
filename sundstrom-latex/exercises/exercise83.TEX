\section*{Exercises \ref{S:diophantine}}

\begin{enumerate}

\item Prove Part~(\ref{T:lindiop2-1}) of Theorem~\ref{T:lindioph2}: \label{exer:lindioph2}

Let $a$, $b$, and $c$ be integers with $a \ne 0$ and $b \ne 0$, and let 
$d = \gcd( a,b )$.  If $d$ does not divide $c$, then the linear Diophantine equation 
$ax + by = c$ has no solution.



\item Prove Corollary~\ref{C:lindioph2}. \label{exer:cor-lindioph2}

Let $a$, $b$, and $c$ be integers with $a \ne 0$ and $b \ne 0$.  If $a$ and $b$ are relatively prime, then the linear Diophantine equation $ax + by = c$ has infinitely many solutions.  In addition, if  $\left( x_0, y_0 \right)$ is a particular solution of this equation, then all the solutions of the equation are given by
\[
x = x_0 + b k  \qquad y = y_0 - a k\!,
\]
where $k \in \mathbb{Z}$.




\item Determine all solutions of the following linear Diophantine equations. \label{exer:sec83-2}

\begin{multicols}{2}
\begin{enumerate}
  \yitem $9x + 14y = 1$

  \yitem $18x + 22y = 4$

  \yitem $48x - 18y = 15$

  \yitem $12x + 9y = 6$

  \item $200x + 49y = 10$

  \item $200x + 54y = 21$

  \item $10x - 7y = 31$

  \item $12x + 18y = 6$

\end{enumerate}
\end{multicols}


\xitem   A certain rare artifact is supposed to weigh exactly 25 grams.  Suppose that you have an accurate balance scale and 500 each of 27 gram weights and  50 gram weights.  Explain how to use Theorem~\ref{T:lindioph2} to devise a plan to check the weight of this artifact.  
\label{exer:balancing}%

\noindent
\hint  Notice that $\gcd( {50, 27} ) = 1$.  Start by writing 1 as a linear combination of 50 and 27.

\xitem On the night of a certain banquet, a caterer offered the choice of two dinners, a steak dinner for \$25 and a vegetarian dinner for \$16.  At the end of the evening, the caterer presented the host with a bill (before tax and tips) for \$1461.  What is the minimum number of people who could have attended the banquet?  What is the maximum number of people who could have attended the banquet? \label{exer:sec83-5}

\item The goal of this exercise is to determine all (integer) solutions of the linear Diophantine equation in three variables $12x_1 + 9x_2 + 16x_3 = 20$.  \label{exer:lindioph3}
\begin{enumerate}
\yitem First, notice that $\gcd( {12, 9} ) = 3$.  Determine formulas that will generate all solutions for the linear Diophantine equation \linebreak
$3y + 16x_3 = 20$. \label{exer:lindioph3a}

\yitem Explain why the solutions (for $x_1$ and $x_2$) of the Diophantine equation 
$12x_1 + 9x_2 = 3y$ can be used to generate solutions for 
\linebreak
$12x_1 + 9x_2 + 16x_3 = 20$.

\yitem Use the general value for $y$ from Exercise~(\ref{exer:lindioph3a}) to determine the solutions of $12x_1 + 9x_2 = 3y$.
\label{exer:lindioph3c}

\item Use the results from Exercises~(\ref{exer:lindioph3a})  and~(\ref{exer:lindioph3c})  to determine formulas that will generate all solutions for the Diophantine equation 
\linebreak
$12x_1 + 9x_2 + 16x_3 = 20$.
\label{exer:lindioph3d}

\note  These formulas will involve two arbitrary integer parameters.  Substitute specific values for these integers and then check the resulting solution in the original equation.  Repeat this at least three times.

\item Check the general solution for $12x_1 + 9x_2 + 16x_3 = 20$ from Exercise~(\ref{exer:lindioph3d}).

\end{enumerate}

\item Use the method suggested in Exercise~(\ref{exer:lindioph3}) to determine formulas that will generate all solutions of the Diophantine equation  
$8x_1 +4x_2 - 6x_3 = 6$.  Check the general solution.

\item Explain why the Diophantine equation $24x_1 - 18x_2 + 60x_3 = 21$ has no solution.

\item The purpose of this exercise will be to prove that the nonlinear Diophantine equation
$3x^2 - y^2 = -2$ has no solution.
\label{exer:nonlineardioph}

\begin{enumerate}
\item Explain why if there is a solution of the Diophantine equation
$3x^2 - y^2 = -2$, then that solution must also be a solution of the congruence 
$3x^2 - y^2 \equiv -2 \pmod 3$.

\item If there is a solution to the congruence $3x^2 - y^2 \equiv -2 \pmod 3$, explain why there then must be an integer $y$ such that $y^2 \equiv 2 \pmod 3$.

\item Use a proof by contradiction to prove that the Diophantine equation $3x^2 - y^2 = -2$ has no solution.

\end{enumerate}

\item Use the method suggested in Exercise~(\ref{exer:nonlineardioph}) to prove that the Diophantine equation $7x^2 + 2 = y^3$ has no solution.

\end{enumerate}


\subsection*{Explorations and Activities}
\setcounter{oldenumi}{\theenumi}
\begin{enumerate} \setcounter{enumi}{\theoldenumi}
\item \textbf{Linear Congruences in One Variable}.  Let $n$ be a natural number and let $a, b \in \mathbb{Z}$ with $a \ne 0$.  A congruence of the form $ax \equiv b \pmod n$ is called a \textbf{linear congruence in one variable}.  \label{A:lincongruence}
\index{linear congruence}%
This is called a linear congruence since the variable $x$ occurs to the first power.  

A \textbf{solution of a linear congruence in one variable} is defined similarly to the solution of an equation.  A solution is an integer that makes the resulting congruence true when the integer is substituted for the variable $x$.  For example,

\begin{itemize}
\item The integer $x = 3$ is a solution for the congruence 
$2x \equiv 1 \pmod 5$ since $2 \cdot 3 \equiv 1 \pmod 5$ is a true congruence.

\item The integer $x = 7$ is not a solution for the congruence 
$3x \equiv 1 \pmod 6$ since $3 \cdot 7 \equiv 1 \pmod 6$ is a not a true congruence.

\end{itemize}

\begin{enumerate}
\item Verify that $x = 2$ and $x = 5$ are the only solutions the linear congruence $\mod{4x}{2}{6}$ with 
$0 \leq x < 6$.

\item Show that the linear congruence $\mod{4x}{3}{6}$ has no solutions with $0 \leq x < 6$. \label{lincongruence2}
\item Determine all solutions of the linear congruence $\mod{3x}{7}{8}$ with $0 \leq x < 8$.  

\end{enumerate}
The following parts of this activity show that we can use the results of Theorem~\ref{T:lindioph2} to help find all solutions of the linear congruence 
$6x \equiv 4 \pmod 8$.

\begin{enumerate}
\setcounter{enumii}{3}
\item Verify that $x = 2$ and $x = 6$ are the only solutions for the linear congruence 
$6x \equiv 4 \pmod 8$ with $0 \leq x < 8$. \label{lincongruence1}
\item Use the definition of ``congruence'' to rewrite the congruence \linebreak
$6x \equiv 4 \pmod 8$ in terms of ``divides.'' \label{lincongruence2}

\item Use the definition of ``divides'' to rewrite the result in part~(\ref{lincongruence2}) in the form of an equation.  (An existential quantifier must be used.) \label{lincongruence3}

\item Use the results of parts~(\ref{lincongruence1}) and~(\ref{lincongruence3}) to write an equation that will generate all the solutions of the linear congruence 
$6x \equiv 4 \pmod 8$.

\hint  Use Theorem~\ref{T:lindioph2}.  This can be used to generate solutions for $x$ and the variable introduced in part~(\ref{lincongruence3}).  In this case, we are interested only in the  solutions for $x$.

\end{enumerate}
Now let $n$ be a natural number and let $a, c \in \mathbb{Z}$ with $a \ne 0$.  A general linear congruence of the form $ax \equiv c \pmod n$ can be handled in the same way that we handled in 
$6x \equiv 4 \pmod 8$.

\begin{enumerate}
\setcounter{enumii}{7}
\item Use the definition of ``congruence'' to rewrite $ax \equiv c \pmod n$ in terms of ``divides.'' \label{lincongruence5}

\item Use the definition of ``divides'' to rewrite the result in part~(\ref{lincongruence5}) in the form of an equation.  (An existential quantifier must be used.) \label{lincongruence6}

\item Let $d = \gcd( {a, n} )$.  State and prove a theorem about the solutions of the linear congruence $ax \equiv c \pmod n$ in the case where $d$ does not divide $c$.

\hint  Use Theorem~\ref{T:lindioph2}.

\item Let $d = \gcd( {a, n} )$.  State and prove a theorem about the solutions of the linear congruence $ax \equiv c \pmod n$ in the case where $d$ divides $c$.

\end{enumerate}

\end{enumerate}



\hbreak






\endinput
