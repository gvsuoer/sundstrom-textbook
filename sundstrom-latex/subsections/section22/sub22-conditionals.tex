\subsection*{Logical Equivalencies Related to Conditional Statements}
\index{conditional statement!logical equivalencies}
%
The first two logical equivalencies in the following theorem were established in \typeu Activity~\ref*{PA:logequiv}, and the third logical equivalency was established in \typeu  Activity~\ref*{PA:converse}.  
\begin{theorem} \label{T:logical-condition}
For statements $P$ and $Q$,
\begin{enumerate}
  \item The conditional statement  $P \to Q$ is logically equivalent to  $\mynot  P \vee Q$.
  \item The statement  $\mynot  \left( {P \to Q} \right)$ is logically equivalent to  $P \wedge \mynot  Q$.
  \item The conditional statement  $P \to Q$ is logically equivalent to its contrapositive  $\mynot  Q \to \mynot  P$.
\end{enumerate}
\end{theorem}
%
%%The truth table that establishes the third logical equivalency is repeated here.
%
%$$
%\BeginTable
%\BeginFormat
%|c|c|c|c|c|c|
%\EndFormat
%\_6
%  |    $P$ | $Q$ \|6 $P \to Q$  |  $\mynot  \left( {P \to Q} \right)$  |  $\mynot  Q$  |  $P \wedge \mynot  Q$ | \\+22 \_6
%   |    T  |  T  \|6 T  |  F  |  F  |  F | \\ 
%   |    T  |  F  \|6 F  |  T  |  T  |  T | \\ 
%   |    F  |  T  \|6 T  |  F  |  F  |  F | \\ 
%   |    F  |  F  \|6 T  |  F  |  T  |  F |  \\ \_6
%\EndTable
%$$
%
%
%\begin{center}
%    \begin{tabular}{|c|c ||c|c|c|c|}
%     \hline
%      $P$ & $Q$ & $P \to Q$  &  $\mynot  \left( {P \to Q} \right)$  &  $\mynot  Q$  &  $P \wedge \mynot  Q$  \\ \hline
%       T  &  T  & T  &  F  &  F  &  F  \\ \hline
%       T  &  F  & F  &  T  &  T  &  T  \\ \hline
%       F  &  T  & T  &  F  &  F  &  F  \\ \hline
%       F  &  F  & T  &  F  &  T  &  F  \\ \hline
%     \end{tabular}
%\end{center}
%\hrule
%
%\enlargethispage{\baselineskip}
%\begin{example}[A Conditional Statement as a Disjunction]\label{E:conditionalasor} \hfill \\
%The logical equivalency $\left( {P \to Q} \right) \equiv \left( {\mynot  P \vee Q} \right)$ was illustrated in Beginning Activity~\ref{PA:conditional2}.  To see this, let  $P$  stand for 
%`` You do not clean your room''  and let  $Q$  stand for 
%``You cannot watch TV.''   Then the conditional statement $\left( {P \to Q} \right)$ is
%%
%\begin{center}
%If you do not clean your room, then you cannot watch TV.
%\end{center}
%%
%This is logically equivalent to the statement  $\left( {\mynot  P \vee Q} \right)$, which can be written as follows:
%%
%\begin{center}
%You clean your room or you cannot watch TV.
%\end{center}
%%
%\end{example}
%\hrule
%
\endinput
