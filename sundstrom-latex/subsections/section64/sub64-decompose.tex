\subsection*{Decomposing Functions} \index{decomposing functions}%

We use the \textbf{chain rule}
\index{chain rule}%
 in calculus to find the derivative of a composite function.  The first step in the process is to recognize a given function as a composite function.  This can be done in many ways, but the work in \typeu Activity~\ref*{PA:verbaldescriptions} can be used to decompose a function in a way that works well with the chain rule.  The use of the terms ``inner function'' and ``outer function'' can also be helpful.  The idea is that we use the last step in the process to represent the outer function, and the steps prior to that to represent the inner function.  So for the function, 
\[
f\x \mathbb{R} \to \mathbb{R}\text{  by  }f( x ) = ( {3x + 2} )^3 ,
\]
the last step in the verbal description table was to cube the result.  This means that we will use the function  $g$ (the cubing function) as the outer function and will use the prior steps as the inner function.  We will denote the inner function by  $h$.  So we let  $h\x \R \to \R$ by  
$h( x ) = 3x + 2$ and  $g\x \R \to \R$ by  $g( x ) = x^3 $.  Then
\begin{align*}
  ( {g \circ h} )( x ) &= g\left( {h( x )} \right) \\ 
                       &= g( {3x + 2} ) \\ 
                       &= ( {3x + 2} )^3  \\ 
                       &= f( x ). \\ 
\end{align*}
We see that  $g \circ h = f$\! and, hence, we have  ``decomposed''  the function  $f$.  It should be noted that there are other ways to write the function $f$ as a composition of two functions, but the way just described is the one that works well with the chain rule.  In this case, the chain rule gives
\begin{align*}
f'(x) &= ( g \circ h )'(x) \\
      &= g'(h(x))\; h'(x) \\
      &= 3(h(x))^2 \cdot 3 \\
      &= 9 (3x + 2)^2
\end{align*}
%\hbreak

\begin{prog}[\textbf{Decomposing Functions}] \label{prog:decompose} \hfill \\
Write each of the following functions as the composition of two functions.

\begin{enumerate}
%\item $f\x \mathbb{R} \to \mathbb{R}$ defined by $f( x ) = \sqrt {3x^2  + 2} $.
%
%\item $g\x \mathbb{R} \to \mathbb{R}$ defined by 
%$g( x ) = \sin ( {3x^2  + 2} )$.
%
%\item $h\x \mathbb{R} \to \mathbb{R}$ defined by $h( x ) = e^{3x^2  + 2} $.
%
%\item $k\x \mathbb{R} \to \mathbb{R}$ defined by 
%$k( x ) = \ln ( {3x^2  + 2} )$.

\item $F\x  \R \to \R$ by $F(x) = ( x^2 +3 )^3$

\item $G\x  \R \to \R$ by $G(x) = \ln ( x^2 + 3 )$

\item $f\x  \Z \to \Z$ by $f(x) = | x^2 - 3 |$

\item $g\x  \R \to \R$ by $g(x) = \cos \! \left( \dfrac{2x-3}{x^2+1} \right)$

\end{enumerate}

%\item Let  $h\x \mathbb{R} \to \mathbb{R}$ be defined by  $h( x ) = 3x + 2$ and  $g\x \mathbb{R} \to \mathbb{R}$ be defined by  $g( x ) = x^3 $.  Determine formulas for the composite functions  $g \circ h$  and  $h \circ g$.  What does this tell you about the operation of composition of functions?
%\end{enumerate}
\end{prog}
\hbreak


\endinput

