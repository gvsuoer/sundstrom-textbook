\section*{Exercises for Section~\ref{S:predicates}}
%
\begin{enumerate}
\xitem Use the roster method to specify the elements in each of the following sets and then write a sentence in English describing the set. 
\label{exer:sec21-1}
%
\begin{multicols}{2}
\begin{enumerate}
  \item $\left\{ x \in \mathbb{R} \mid 2x^2 + 3x - 2 = 0 \right\}$
  \item $\left\{ {x \in \mathbb{Z}} \mid 2x^2  + 3x - 2 = 0 \right\}$
  \item $\left\{ {x \in \mathbb{Z}} \mid x^2  < 25 \right\}$
  \item $\left\{ {x \in \mathbb{N}} \mid x^2  < 25 \right\}$
  \item $\left\{ {y \in \mathbb{Q}} \mid \left| {y - 2} \right| = 2.5 \right\}$
  \item $\left\{ {y \in \mathbb{Z}} \mid \left| {y - 2} \right| \leq 2.5 \right\}$
\end{enumerate}
\end{multicols}

\xitem Each of the following sets is defined using the roster method. \label{exer:setbuilder}

\begin{multicols}{2}
$A = \left\{1, 4, 9, 16, 25, \ldots \right\}$

$B = \left\{\ldots, -\pi^4, -\pi^3, -\pi^2, -\pi, -1 \right\}$

$C = \left\{3, 9, 15, 21, 27, \ldots \right\}$

$D = \left\{0, 4, 8, \ldots, 96, 100 \right\}$
\end{multicols}

\begin{enumerate}
\item Determine four elements of each set other than the ones listed using the roster method.

\item Use set builder notation to describe each set.
\end{enumerate}

%\pagebreak
\xitem Let 
\label{exer:sec21-2}%  
$A = \left\{ x \in \mathbb{R}  \mid  x\left( {x + 2} \right)^2 \left( x - \frac{3}{2} \right) = 0 \right\}$.  Which of the following sets are equal to the set $A$ and which are subsets of $A$?
%
\begin{multicols}{2}
\begin{enumerate}
  \item $\left\{ { - 2, 0,3} \right\}$
  \item $\left\{ {\dfrac{3}{2}, - 2,0} \right\}$
  \item $\left\{ { - 2, - 2,0,\dfrac{3}{2}} \right\}$
  \item $\left\{ { - 2, \dfrac{3}{2}} \right\}$
\end{enumerate}
\end{multicols}


\item \label{exer:sec23-4new} Use the roster method to specify the truth set for each of the following open sentences.  The universal set for each open sentence is the set of integers $\Z$.
\begin{enumerate}
\yitem $n + 7 = 4$.
\yitem $n^2 = 64$.
\item $\sqrt{n} \in \N$ and $n$ is less than 50.
\item $n$ is an odd integer that is greater than 2 and less than 14.
\item $n$ is an even integer that is greater than 10.
\end{enumerate}

%
\item Use set builder notation to specify the following sets: \label{exer:sec21-3}
  \begin{enumerate}
    \yitem The set of all integers greater than or equal to 5.
    \item The set of all even integers.
    \yitem The set of all positive rational numbers.
    \item The set of all real numbers greater than 1 and less than 7.
    \yitem The set of all real numbers whose square is greater than 10.
  \end{enumerate}
%
\item For each of the following sets, use English to describe the set and when appropriate,  use the roster method to specify all of the elements of the set.\label{exer:sec23-sets}
\begin{enumerate}
\begin{multicols}{2}
\item $\left\{ \left. x \in \R \right| -3 \leq x \leq 5 \right\}$

\item $\left\{ \left. x \in \Z \right| -3 \leq x \leq 5 \right\}$

\item $\left\{ \left. x \in \R \right| x^2 = 16 \right\}$

\item $\left\{ \left. x \in \R \right| x^2 + 16 = 0 \right\}$

\item $\left\{ \left. x \in \Z \right| x \text{ is odd } \right\}$

\item $\left\{ \left. x \in \R \right| 3x - 4 \geq 17 \right\}$
\end{multicols}
\end{enumerate}
\end{enumerate}


%\xitem Each of the following sentences is a predicate or a statement.  Assume that the universal set for each variable in these sentences is the set of all real numbers.  If a sentence is an open sentence (predicate), determine its truth set.  If a sentence is a statement, determine whether it is true or false. \label{exer:sec21-4}
%  \begin{enumerate}
%    \item $\forall a \in \mathbb{R},\;a + 0 = a$.
%    \item $3x - 5 = 9$.
%    \item $\sqrt x  \in \mathbb{R}$.
%    %\item $\left( \forall x \in \mathbb{R}\right) \left( \sin( {2x}) = 2( {\sin x})( {\cos x})$\right).
%    \item $\sin( {2x} ) = 2( {\sin x} )( {\cos x})$.
%    \item $\forall x \in \R, \sin( {2x} ) = 2( {\sin x})( {\cos x})$.
%    \item $\exists x \in \mathbb{R}\mathbf{ }\text{ such that }x^2  + 1 = 0$.
%    \item $\forall x \in \mathbb{R},\;x^3  \geq x^2$.
%    \item $x^2  + 1 = 0$. 
%    \item If  $x \geq 1$, then  $x^2  \geq 1$.
%    \item $\forall x \in \mathbb{R}, \exists y \in \mathbb{R}\text{ such that } x + y = 0$.
%    \item $\exists y \in \mathbb{R}\text{ such that }\forall x \in \mathbb{R}, x + y = 0$.
%    \item $\sqrt x  \in \mathbb{Z}$.
%  \end{enumerate}
%
%\markboth{Chapter \ref{C:logic}. Logical Reasoning}{\ref{S:logop}. Statements and Logical Operators}

\subsection*{Explorations and Activities}
\setcounter{oldenumi}{\theenumi}
\begin{enumerate} \setcounter{enumi}{\theoldenumi}
\item \textbf{Closure Explorations}.  In Section~\ref{S:prop}, we studied some of the closure properties of the standard number systems.  (See page~\pageref{ss:closure}.)  We can extend this idea to other sets of numbers.  So we say that:  
\begin{itemize}
  \item A set $A$ of numbers is \textbf{closed under addition} provided that whenever  $x$  and  $y$  are in the set $A$,   $x + y$ is in the set $A$.
\index{closed under addition}%

  \item A set $A$ of numbers is \textbf{closed under multiplication} provided that whenever  $x$  and  $y$  are are in the set $A$,   
$x \cdot y$ is in the set $A$.
\index{closed under multiplication}%

  \item A set $A$ of numbers is \textbf{closed under subtraction} provided that whenever  $x$  and  $y$  are are in the set $A$,   
$x - y$ is in the set $A$.
\index{closed under subtraction}%
\end{itemize}


For each of the following sets, make a conjecture about whether or not it is closed under addition and whether or not it is closed under multiplication.  In some cases, you may be able to find a counterexample that will prove the set is not closed under one of these operations.

\begin{multicols}{2}
\begin{enumerate}
\item The set of all odd natural numbers

\item The set of all even integers

\item $A = \left\{ 1, 4, 7, 10, 13, \ldots \right\}$

\item $B = \left\{ \ldots , -6, -3, 0, 3, 6, 9, \ldots \right\}$

\item $C = \left\{ \left. 3n + 1 \right| n \in \Z \right\}$

\item $D = \left\{ \left. \dfrac{1}{2^n} \right| n \in \N \right\}$

\end{enumerate}
\end{multicols}
\end{enumerate}

\hbreak



\endinput
