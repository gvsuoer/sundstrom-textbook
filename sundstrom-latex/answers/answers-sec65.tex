\section*{Section \ref{S:inversefunctions}}
\renewcommand{\labelenumi}{(\textbf{\alph{enumi}})}

\begin{list}{\bf{\ref{exer:sec65-2}.}}
\item \begin{list}{\bf{(b)}}
\item $f^{ - 1}  = \left\{ {( {c,a} ),( {b,b} ),( {d,c} ),( {a,d} )} \right\}$
\end{list}
\end{list}

\begin{list}{}
\item \begin{list}{\bf{(d)}}
\item For each $x \in S$, $\left( {f^{ - 1}  \circ f} \right)( x ) = x = ( {f \circ f^{ - 1} } )( x )$.  This illustrates \\
Corollary~\ref{C:inversecomposition}.
\end{list}
\end{list}

\begin{list}{\bf{\ref{exer:solveequation}.}}
\item \begin{enumerate}
\item This is a use of Corollary~\ref{C:inversecomposition} since the cube root function and the cubing function are inverse functions of each other and consequently, the composition of the cubing function with the cube root function is the identity function.

\item This is a use of Corollary~\ref{C:inversecomposition} since the natural logarithm function  and the exponential function with base $e$ are inverse functions of each other and consequently, the composition of the natural logarithm function with the exponential function with base $e$ is the identity function.

\item They are similar because they both use the concept of an inverse function to ``undo'' one side of the equation.
\end{enumerate}
\end{list}


\begin{list}{\bf{\ref{exer:inversecomposition}.}}
\item Using the notation from Corollary~\ref{C:inversecomposition}, if $y = f(x)$ and 
$x = f^{-1}(y)$, then
\begin{align*}
\left(f \circ f^{-1} \right) (y) &= f ( f^{-1} ( y ) ) \\
                                 &= f ( x ) \\
                                 &= y
\end{align*}
\end{list}



\begin{list}{\bf{\ref{exer:compequalidentity}.}}
\item \begin{list}{\bf{(a)}}
\item Let $x, y \in A$ and assume that 
$f ( x ) = f ( y )$.  Apply $g$ to both sides of this equation to prove that 
$( g \circ f ) ( x ) = ( g \circ f ) ( y )$.
Since $g \circ f = I_A $, this implies that $x = y$ and hence that $f$ is an injection.
\end{list}
\end{list}

\begin{list}{}
\item \begin{list}{\bf{(b)}}
\item Start by assuming that $f \circ g = I_B $, and then let $ y \in B$.  You need to prove there exists an $x \in A$ such that $f ( x ) = y$.
\end{list}
\end{list}



%\begin{list}{\bf{\ref{exer:sec65-5}.}}
%\item \begin{list}{\bf{(a)}}
%\item $x = \dfrac{1}{2} ( \ln{y} + 1 ) $
%%\qquad (b) $g:\mathbb{R}^ +   \to \mathbb{R}$ by 
%%$g ( y ) = \dfrac{1}{2} ( \ln{y} + 1 )$
%\end{list}
%\end{list}

%\begin{list}{}
%\item \begin{list}{\bf{(b)}}
%\item $g:\mathbb{R}^ +   \to \mathbb{R}$ by 
%$g ( y ) = \dfrac{1}{2} ( \ln{y} + 1 )$
%\end{list}
%\end{list}



\begin{list}{\bf{\ref{exer:sec65-6}.}}
%\item \begin{list}{(a)}
%\item The inverse of $f$ is not a function and the inverse of $g$ is a function.
\item \begin{enumerate}
\item $f:\mathbb{R} \to \mathbb{R}$ is defined by $f\left( x \right) = e^{ - x^2 } $.  Since this function is not an injection, the inverse of $f$ is not a function.

\item $g:\mathbb{R}^*  \to \left( {0, 1} \right]$ is defined by $g\left( x \right) = e^{ - x^2 }$.  In this case, $g$ is a bijection and hence, the inverse of $g$ is a function.

To see that $g$ is an injection, assume that $x, y \in \mathbb{R}^*$ and that 
$e^{-x^2} = e^{-y^2}$.  Then, $x^2 = y^2$ and since $x, y \geq 0$, we see that $x = y$.  To see that $g$ is a surjection, let $y \in \left( 0, 1 \right]$.  Then, $\ln y < 0$ and $- \ln y > 0$, and $g \left( \sqrt{-\ln y} \right) = y$.
\end{enumerate}
\end{list}
%\end{list}

%\begin{list}{}
%\item \begin{list}{(b)}
%\item The inverse of $g$ is a function.
%\end{list}
%\end{list}
\hbreak


\endinput


