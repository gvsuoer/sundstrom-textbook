\section*{Exercises for Section~\ref{S:cases}}
\begin{enumerate}
%
\xitem In \typeu Activity~\ref*{PA:cases}, we proved that if  $n$  is an integer, then  $ n^2 + n $   is an even integer.  We define two integers to be \textbf{consecutive integers}
\index{consecutive integers}%
\index{integers!consecutive}%
 if one of the integers is one more than the other integer.  This means that we can represent consecutive integers  as  $m$  and $m+1$, where  $m$  is some integer.

Explain why the result proven in \typeu Activity~\ref*{PA:cases}  can be used to prove that the product of any two consecutive integers is divisible by 2.
\label{exer:consecutive}%

%\item Extending the idea in Exercise~(\ref{exer:consecutive}), we can represent three consecutive integers as  $m$, $m + 1$, and  $m + 2$,  where  $m$  is an integer. 
%\label{exer:sec34-2}
%
%\begin{enumerate}
%  \item Explain why we can also represent three consecutive integers as  $k - 1$, $k$, and 
%$k + 1$  where  $k$  is an integer.
%
%  \yitem Explain why Proposition~\ref{P:3divides} proves that the product of any three consecutive integers is divisible by 3.
%
%\end{enumerate}

\xitem  Prove that if  $u$ is an odd integer, then the equation $x^2 + x - u = 0$ has no solution that is an integer.
\label{exer:sec33-6}%

\xitem Prove that if $n$ is an odd integer, then $n = 4k +1$ for some integer $k$ or
$n = 4k + 3$ for some integer $k$.
\label{exer:sec35-5}%


\xitem Prove the following proposition: \label{exer:sec34-quadratic}
\begin{list}{}
\item For each integer $a$, if $a^2 = a$, then $a = 0$ or $a = 1$.
\end{list}


\item \label{exer:dividesproduct}
\begin{enumerate} \item Prove the following proposition: 

  \begin{list}{}
    \item For all integers  $a$, $b$, and  $d$  with $d \ne 0$,  if  $d$  divides  $a$  or  $d$  divides  $b$, then  $d$  divides the product  $ab $.
  \end{list}

\hint  Notice that the hypothesis is a disjunction.  So use two cases. 
\label{exer:dividesproducta}%

  \item Write the contrapositive of the proposition in Exercise~(\ref{exer:dividesproducta}).

  \yitem Write the converse of the proposition in Exercise~(\ref{exer:dividesproducta}).  Is the converse true or false?  Justify your conclusion.
\label{exer:dividesproductc}%
  \end{enumerate}

%\item \begin{enumerate}
%\item Complete the details for the proof of Case 3 of Proposition~\ref{P:3divides}.
%\item Complete the details for the proof of Case 3 of Proposition~\ref{P:3dividesver2}.
%\end{enumerate}

%\item \label{exer:sec34-4} \begin{enumerate} \item Let  $n \in \mathbb{N}$ and let  $a \in \mathbb{Z}$.  Explain why  $n \mid a$  if and only if  
%\linebreak $a \equiv 0 \pmod n$. 
%
%  \item Let  $a \in \mathbb{Z}$.  Explain why  if  $a \not \equiv 0 \pmod 3$, then  $a \equiv 1\pmod 3$ or  $a \equiv 2 \pmod 3$.
%
%  \item Is the following proposition true or false?  Justify your conclusion.
%
%  \begin{list}{}
%  \item Let  $a \in \mathbb{Z}$.  If  $a \not \equiv 0 \pmod 3$, then  
%$a^2  \equiv 1 \pmod 3$.
%  \end{list}
%\end{enumerate}

%\item Prove the following proposition by proving its contrapositive.  (\hint:  Use case analysis.  There are several cases.)  \label{exer:congto3}
%
%  \begin{list}{}
%  \item Let  $a$  and  $b$  be integers.  If  $ab \equiv 0 \pmod 3$, then  $a \equiv 0 \pmod 3$  or  $b \equiv 0 \pmod 3$.
%  \end{list}


%\item Explain why the following proposition is equivalent to the proposition in Exercise~(\ref{exer:congto3}).
%
%  \begin{list}{}
%  \item Let  $a$  and  $b$  be integers.  If $3 \mid ab$  , then  $3 \mid a$   or $3 \mid b$  .
%  \end{list}
%\label{exer:3divprod}

%\item \begin{enumerate}
%\item Prove the following proposition.
%\begin{list}{}
%\item If $r$ is a real number such that $r^2 = 3$, then $r$ is irrational.
%\end{list}
%
%\item Prove that the real number $\sqrt{12}$ is an irrational number.
%\end{enumerate}


%\item The proposition in Exercise~(\ref{exer:3divprod}) was proven in Exercise~(\ref{exer:congto3}) using the concept of congruence.  This result can be proven without using the concept of congruence.  The idea is basically the same but we use the Division Algorithm  directly instead of using congruence.  Notice that if  3 does not divide an integer, then the remainder when that integer is divided by 3 must be 1 or 2.  This allows us to use two cases for that integer.  
%
%Prove the proposition in Exercise~(\ref{exer:3divprod}) by using this idea.  \hint  You will still have to prove the contrapositive and use case analysis.


%\item \begin{enumerate}
%\item Is the following proposition true or false?  Justify your conclusion with a proof if it is true or with a counterexample if it is false.
%
%\begin{list}{}
%\item For every integer $a$, if $a \not \equiv 0 \pmod 5$, then $a^2 \equiv 1 \pmod 5$ or 
%$a^2 \equiv 4 \pmod 5$.
%\end{list}
%
%\item Does there exist an integer $a$ such that $a^2 = 5,158,232,468,953,153$?  Use your work in Part~(a) to justify your conclusion.
%\end{enumerate}

%\item Is the following proposition true or false.  Justify your conclusion with a proof or a counterexample.
%\begin{list}{}
%\item Let $n$ be a natural number.  If 3 does not divide $\left( n^2 + 2 \right)$, then $n$ is not a prime number or $n=3$.
%\end{list}


\item 
%If  $n \in \mathbb{Z}$ and  $m = n + 1$, then  $m$  and  $n$  are said to be \textbf{consecutive integers.} \label{exer:sec35-4}
%\index{consecutive integers}%
%\index{integers!consecutive}%
Are the following propositions true or false?  Justify all your conclusions.  If a biconditional statement is found to be false, you should clearly determine if one of the conditional statements within it is true.  In that case, you should state an appropriate theorem for this conditional statement and prove it.
\label{exer:sec34-6}%
\begin{enumerate}
%\item For all integers $m$ and $n$, if  $m$  and  $n$  are two consecutive integers, then  4  divides $\left( {m^2  + n^2  - 1} \right)$.
\yitem For all integers $m$ and $n$, $m$  and  $n$  are consecutive integers if and only if  4  divides $\left( {m^2  + n^2  - 1} \right)$.
\item For all integers $m$ and $n$, 4 divides $\left( m^2 - n^2 \right) $ if and only if 
$m$ and $n$ are both even or $m$ and $n$ are both odd.
\end{enumerate}


\item Is the following proposition true or false?  Justify your conclusion with a counterexample or a proof.
\label{exer:sec34-nsquared}%

\begin{list}{}
\item For each integer $n$, if $n$ is odd, then $8 \mid \left( n^2 - 1 \right)$.
\end{list}


\xitem Prove that there are no natural numbers $a$ and $n$ with $n \geq 2$ and 
$a^2 + 1 = 2^n$.\label{exer:a2plus1not2n}%

%\item Prove the following propositions:
\item Are the following propositions true or false?  Justify each conclusion with a counterexample or a proof.
\begin{enumerate}
\item For all integers $a$ and $b$ with $a \ne 0$, the equation $ax + b = 0$ has a rational number solution.

\item For all integers $a$, $b$, and $c$, if $a$, $b$, and $c$ are odd, then the equation 
$ax^2 + bx + c = 0$ has no solution that is a rational number.

\hint  Do not use the quadratic formula.  Use a proof by contradiction and recall that any rational number can be written in the form $\dfrac{p}{q}$, where $p$ and $q$ are integers, $q > 0$, and $p$ and $q$ have no common factor greater than 1.

\item For all integers $a$, $b$, $c$, and $d$, if $a$, $b$, $c$, and $d$ are odd, then the equation $ax^3 + bx^2 + cx + d = 0$ has no solution that is a rational number.
\end{enumerate}


\item \label{exer:absvalue} \begin{enumerate}
\yitem Prove Part~(\ref{P:absvalue-2}) of Proposition~\ref{P:absvalue}.
\begin{list}{}
\item For each $x \in \R$, $\left| -x \right| = \left| x \right|$.
\end{list}

\item Prove Part~(\ref{T:absvalue-2}) of Theorem~\ref{T:absvalue}.
\begin{list}{}
\item For all real numbers $x$ and $y$, $\left| xy \right| = \left| x \right| \left| y \right|$.
\end{list}
\end{enumerate}


\item Let $a$ be a positive real number.  In Part~(\ref{T:absvalue-1}) of 
Theorem~\ref{T:absvalue}, we proved that for each real number $x$, $\left| x \right| < a$ if and only if $-a < x < a$.  It is important to realize that the sentence $-a < x < a$ is actually the conjunction of two inequalities.  That is, $-a < x < a$ means that $-a < x$ and $x < a$.
\label{exer:absvaluex}%

\begin{enumerate}
\yitem Complete the following statement:  For each real number $x$, $\left| x \right| \geq a$ if and only if \ldots.

\item Prove that for each real number $x$, $\left| x \right| \leq a$ if and only if $-a \leq x \leq a$.

\item Complete the following statement:  For each real number $x$, $\left| x \right| > a$ if and only if \ldots.
\end{enumerate}

\item Prove each of the following:
\label{exer:moreabsvalue}%
\begin{enumerate}
\item For each nonzero real number $x$ , $\abs{x^{-1}} = \dfrac{1}{\abs{x}}$.

\item For all real numbers $x$ and $y$, $\abs{x - y} \geq \abs{x} - \abs{y}$.

\hint  An idea that is often used by mathematicians is to  add 0 to an expression ``intelligently''.  In this case, we know that $\left(-y \right) + y = 0$.  Start by adding this ``version'' of 0 inside the absolute value sign of $\abs{x}$.

\item For all real numbers $x$ and $y$, $\abs{\abs{x} - \abs{y}} \leq \abs{x - y}$.
\end{enumerate}

\item \textbf{Evaluation of proofs}  \hfill \\
See the instructions for Exercise~(\ref{exer:proofeval}) on 
page~\pageref{exer:proofeval} from Section~\ref{S:directproof}.

\begin{enumerate}
\item \textbf{Proposition}. For all nonzero integers $a$ and $b$, if $a + 2b \ne 3$ and 
$9a + 2b \ne 1$, then the equation $ax^3 + 2bx = 3$ does not have a solution that is a natural number.

\begin{myproof}
We will prove the contrapositive, which is
\begin{list}{}
\item For all nonzero integers $a$ and $b$, if the equation $ax^3 + 2bx = 3$ has a solution that is a natural number, then $a + 2b = 3$ or $9a + 2b = 1$.
\end{list}
So we let $a$ and $b$ be nonzero integers and assume that the natural number $n$ is a solution of the equation $ax^3 + 2bx = 3$.  So we have
\begin{align*}
an^3 + 2bn &= 3 \qquad \text{or}\\
n \left( an^2 + 2b \right) &= 3.
\end{align*}
So we can conclude that $n = 3$ and $an^2 + 2b = 1$.  Since we now have the value of $n$, we can substitute it in the equation $an^3 + 2bn = 3$ and obtain $27a + 6b = 3$.  Dividing both sides of this equation by 3 shows that $9a + 2b = 1$.  So there is no need for us to go any further, and  this concludes the proof of the contrapositive of the proposition.  
\end{myproof}

\item \textbf{Proposition}. For all nonzero integers $a$ and $b$, if $a + 2b \ne 3$ and 
$9a + 2b \ne 1$, then the equation $ax^3 + 2bx = 3$ does not have a solution that is a natural number.

\setcounter{equation}{0}
\begin{myproof}
We will use a proof by contradiction.  Let us assume that there exist nonzero integers $a$ and 
$b$ such that $a + 2b = 3$ and $9a + 2b = 1$ and $an^3 + 2bn = 3$, where $n$ is a natural number.  First, we will solve one equation for $2b$; doing this, we obtain
\begin{align}
a + 2b &= 3 \notag \\
    2b &= 3 - a.
\end{align}
We can now substitute for $2b$ in $an^3 + 2bn = 3$.  This gives
\begin{align}
an^3 + (3 - a)n &= 3 \notag \\
an^3 + 3n - an  &= 3 \notag \\
n \left( an^2 + 3 - a \right) &= 3.
\end{align}

By the closure properties of the integers, $\left( an^2 + 3 - a \right)$ is an integer and, hence, equation~(2) implies that $n$ divides 3.  So $n = 1$ or $n = 3$.  When we substitute 
$n = 1$ into the equation $an^3 + 2bn = 3$, we obtain $a + 2b = 3$.  This is a contradiction since we are told in the proposition that $a + 2b \ne 3$.  This proves that the negation of the proposition is false and, hence, the proposition is true.
\end{myproof}
\end{enumerate}
\end{enumerate}


\subsection*{Explorations and Activities}
\setcounter{oldenumi}{\theenumi}
\begin{enumerate} \setcounter{enumi}{\theoldenumi}
\item \textbf{Proof of the Triangle Inequality}.  \label{exer:triangleineq}
\begin{enumerate}
\item  Verify that the triangle inequality is true for several different real numbers $x$ and 
$y$.  Be sure to have some examples where the real numbers are negative.

\item Explain why the following proposition is true:
For each real number $r$, $- \left| r \right| \leq r \leq \left| r \right|$.
\label{A:triangleinequality-2}%

\item Now let $x$ and $y$ be real numbers.  Apply the result in 
Part~(\ref{A:triangleinequality-2}) to both $x$ and $y$.  Then add the corresponding parts of the two inequalities to obtain another inequality.  Use this to prove that 
$\left| x + y \right| \leq \left| x \right| + \left| y \right|$.
\end{enumerate}
\index{absolute value|)} %

\end{enumerate}



%\item Translate the statement in Part~(a) into  a corresponding statement dealing with congruence modulo 8.


%\item Prove the following proposition:
%\begin{list}{}
%\item Let $a, b \in \mathbb{Z}$.  If 3 divides $\left( a^2 + b^2 \right)$, then 3 divides $a$ and 3 divides $b$.
%\end{list}

%\item Use the ideas presented in Activity~\ref{A:lasttwo} to complete the following: \label{exer:sec34-8}
%
%  \begin{enumerate}
%    \item Determine the last two digits in the decimal representation of  $3^{3356} $.
%    \item Determine the last two digits in the decimal representation of  $7^{403} $.
%  \end{enumerate}

%\item Explore the statements in Exercises~(\ref{exer:sec35-9a}) and~(\ref{exer:sec35-9b}) by considering several examples where the hypothesis is true. \label{exer:sec34-9}
%
%\begin{enumerate}
%  \item If an integer  $a$  is divisible by both  4  and  6, then it divisible by  24.  \label{exer:sec35-9a}
%
%  \item If an integer  $a$  is divisible by both  2  and  3, then it divisible by  6.  \label{exer:sec35-9b}
%
%  \item What can you conclude from the examples in Exercise~(\ref{exer:sec35-9a})?
%
%  \item What can you conclude from the examples in Exercise~(\ref{exer:sec35-9b})?
%
%\end{enumerate}
%%
%The proof of the following proposition [from Exercise~(\ref{exer:sec35-9b}] uses cases.  In this proof, however, we use cases and a proof by contradiction to prove that a certain integer cannot be odd.  Hence, it must be even.  Complete the proof of the proposition.
%
%\textbf{Proposition.}
%Let  $a \in \mathbb{Z}$.  If  2  divides  $a$  and  3  divides  $a$, then  6  divides  $a$.
%
%\noindent
%\textbf{\emph{Proof}}:  Let  $a \in \mathbb{Z}$ and assume that  2  divides  $a$  and  3  divides  $a$.  We will prove that  6 divides  $a$.  Since  3  divides  $a$, there exists an integer  $n$  such that
%\[
%a = 3n.
%\]
%The integer  $n$  is either even or it is odd.  We will show that it must be even by obtaining a contradiction if it assumed to be odd.  So, assume that  $n$  is odd. \ldots
%


\hbreak
%\markboth{Chapter~\ref{C:proofs}. Constructing Proofs}{\ref{S:constructive}. Constructive Proofs}

\endinput
