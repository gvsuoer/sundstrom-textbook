\subsection*{Set Builder Notation} \label{sym:setbuilder}
Sometimes it is not possible to list all the elements of a set.  For example, if the universal set is  $\mathbb{R}$, we cannot list all the elements of the truth set of  ``$x^2  < 4$.''  In this case, it is sometimes convenient to use the so-called \textbf{set builder notation}
\index{set builder notation}%
 in which the set is defined by stating a rule that all elements of the set must satisfy.  If  
$P(x)$ is a predicate in the variable  $x$, then the notation
\[
\left\{ x \in U \mid P(x) \right\}
\]
\label{sym:setbuilder}%
stands for the set of all elements  $x$  in the universal set  $U$  for which  $P(x)$ is true.  If it is clear what set is being used for the universal set, this notation is sometimes shortened to $\left\{ x \mid {P(x)} \right\}$.  This is usually read as ``the set of all $x$  such that  
$P(x)$.''  The vertical bar stands for the phrase ``such that.''  Some writers will use a colon (:)  instead of the vertical bar.

For a non-mathematical example,  $P$  could be the property that a college student is a mathematics major.  Then  $\left\{ x \mid {P(x)} \right\}$ denotes the set of all college students who are mathematics majors.  This could be written as
\[
\left\{ x \mid  x\text{ is a  college student who is a mathematics major} \right\}.
\]
%\hbreak
%


\begin{example}[\textbf{Truth Sets}] \hfill \\
Assume the universal set is  $\R$ and $P( x )$ is ``$x^2  < 4$.''  We can describe the truth set of $P(x)$ as the set of all real numbers whose square is less than 4.  We can also use set builder notation to write the truth set of $P( x )$ as
\[
\left\{ {x \in \R} \mid  x^2  < 4  \right\}.
\]
However, if we solve the inequality  $x^2  < 4$, we obtain  $ - 2 < x < 2$.  So we could also write the truth set as
\[
\left\{ {x \in \R} \mid - 2 < x < 2 \right\}.
\]
We could read this as the set of all real numbers that are greater than $-2$ and less than 2.  We can also write
\[
\left\{ {x \in \R} \mid x^2  < 4 \right\} = \left\{ {x \in \mathbb{R}} \mid - 2 < x < 2 \right\}.
\]
\end{example}
\hbreak

\begin{prog}[\textbf{Working with Truth Sets}]\label{pr:truthset} \hfill \\
Let  $P( x )$ be the predicate ``$x^2  \leq 9$.''
\begin{enumerate}
  \item If the universal set is  $\mathbb{R}$, describe the truth set of $P(x)$ using English and write the truth set of  $P( x )$ using set builder notation.  
  \item If the universal set is  $\mathbb{Z}$, then what is the truth set of  
$P( x )$?  Describe this set using English and then use the roster method to specify all the elements of this truth set.
  \item Are the truth sets in Parts (1) and (2) equal?  Explain.
\end{enumerate}

%The purpose of this activity is to show that the truth set of a predicate depends on the predicate and on the universal set.
\end{prog}
\hbreak

So far, our standard form for set builder notation has been 
$\left\{ x \in U \mid P( x ) \right\}$.  It is sometimes possible to modify this form and put the predicate first.  For example, the set
\[
A = \left\{ 3n + 1 \mid n \in \N \right\}
\]
describes the set of all natural numbers of the form $3n + 1$ for some natural number.  By substituting 1, 2, 3, 4, and so on, for $n$, we can use the roster method to write
\[
A = \left\{ 3n + 1 \mid n \in \N \right\} = \left\{4, 7, 10, 13, \ldots \right\}.
\]
We can sometimes ``reverse this process'' by starting with a set specified by the roster method and then writing the same set using set builder notation.

\begin{example}[\textbf{Set Builder Notation}] \hfill \\
Let $B = \left\{ \ldots, -11, -7, -3, 1, 5, 9, 13, \ldots \right\}$.  The key to writing this set using set builder notation is to recognize the pattern involved.  We see that once we have an integer in $B$, we can obtain another integer in $B$ by adding 4.  This suggests that the predicate we will use will involve multiplying by 4.

Since it is usually easier to work with positive numbers, we notice that $1 \in B$ and 
$5 \in B$.  Notice that

\begin{center}
$1 = 4 \cdot 0 + 1$ \qquad and \qquad $5 = 4 \cdot 1 + 1$.
\end{center}

This suggests that we might try $\left\{ 4n + 1 \mid n \in \Z \right\}$.  In fact, by trying other integers for $n$, we can see that

\[
B = \left\{ \ldots, -11, -7, -3, 1, 5, 9, 13, \ldots \right\} = \left\{ 4n + 1 \mid n \in \Z \right\}.
\]
\end{example}
\hbreak

\begin{prog}[\textbf{Set Builder Notation}]\label{pr:setbuilder} \hfill \\
Each of the following sets is defined using the roster method. 
\label{exer:setbuilder}%

\begin{multicols}{2}
$A = \left\{1, 5, 9, 13, \ldots \right\}$

$B = \left\{\ldots, -8, -6, -4, -2 , 0 \right\}$

$C = \left\{\sqrt{2}, \left( \sqrt{2}\right)^3, \left( \sqrt{2}\right)^5,  \ldots \,\right\}$

$D = \left\{1, 3, 9, 27, \ldots \right\}$
\end{multicols}

\begin{enumerate}
\item Determine four elements of each set other than the ones listed using the roster method.

\item Use set builder notation to describe each set.
\end{enumerate}
\end{prog}
\hbreak


\endinput

