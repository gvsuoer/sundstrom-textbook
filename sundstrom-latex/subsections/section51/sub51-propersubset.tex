\subsection*{Set Equality, Subsets, and Proper Subsets} \label{ss:propersubset} 
In Section~\ref{S:predicates}, we introduced some basic definitions used in set theory, what it means to say that two sets are equal and what it means to say that one set is a subset of another set.  See the definitions on page~\pageref{D:setequality}.  We need one more definition.

\begin{defbox}{propersubset}{
Let $A$ and $B$ be two sets contained in some universal set $U$.  The set  $A$  is a \textbf{proper subset} \label{sym:propersub}
\index{proper subset}%
\index{subset!proper}%
of $B$ provided that $A \subseteq B$ and  $A \ne B$.  When $A$ is a proper subset of $B$, we write $A \subset B$.}
\end{defbox}

  
One reason for the definition of proper subset is that each set is a subset of itself.  That is, 
\begin{center}
If  $A$  is a set, then  $A \subseteq A$.
\end{center}
%
However, sometimes we need to indicate that a set  $X$  is a subset of  $Y$  but  $X \ne Y$. For example, if
\[
X = \left\{ {1, 2} \right\}\text{  and  }Y = \left\{ {0, 1, 2, 3} \right\}\!,
\]
then  $X \subset Y$.  We know that  $X \subseteq Y$ since each element of  $X$  is an element of  $Y$, but  $X \ne Y$ since  $0 \in Y$  and  $0 \notin X$.  (Also, $3 \in Y$  and   $3 \notin X$.)   Notice that the notations  $A \subset B$  and  $A \subseteq B$ are used in a manner similar to inequality notation for numbers ($a < b$  and  $a \leq b$).

It is often very important to be able to describe precisely what it means to say that one set is not a subset of the other.  In the preceding example,  $Y$  is not a subset of  $X$  since there exists an element of $Y$ 
(namely, 0) that is not in $X$.  

In general, the subset relation is described with the use of a universal quantifier since $A \subseteq B$ means that for each element $x$ of $U$, if $x \in A$, then  $x \in B$.   So when we negate this, we use an existential quantifier as follows:
\begin{center}
\begin{tabular}{l l l}
 $A \subseteq B$  &  \qquad means \qquad &  $\left( {\forall x \in U} \right)\left[ {\left( {x \in A} \right) \to \left( {x \in B} \right)} \right]$. \\
  &  &  \\
$A \not\subseteq B$  &  \qquad means \qquad &    $\mynot  \left( {\forall x \in U} \right)\left[ {\left( {x \in A} \right) \to \left( {x \in B} \right)} \right]$  \\
  &  &  $\left( {\exists x \in U} \right) \mynot  \left[ {\left( {x \in A} \right) \to \left( {x \in B} \right)} \right]$ \\
  &  &  $\left( {\exists x \in U} \right)\left[ {\left( {x \in A} \right) \wedge \left( {x \notin B} \right)} \right]$. \\ 
\end{tabular}
\end{center}
So we see that  $A \not\subseteq B$ \label{sym:notsubset2}  means that there exists an  $x$ in $U$  such that  $x \in A$  and  $x \notin B$.

Notice that if $A = \emptyset$, then the conditional statement, ``For each $x \in U$, 
if $x \in \emptyset$, then $x \in B$'' must be true since the hypothesis will always be false.  Another way to look at this is to consider the following statement:

\begin{center}
$\emptyset \not \subseteq B$ means that there exists an $x \in \emptyset$ such that $x \notin B$.
\end{center}
However, this statement must be false since there does not exist an $x$ in $\emptyset$.  Since this is false, we must conclude that $\emptyset \subseteq B$.  Although the facts that $\emptyset \subseteq B$ and $B \subseteq B$ may not seem very important, we will use these facts later, and hence we  summarize them in Theorem~\ref{T:subsets}.

\begin{theorem} \label{T:subsets}
For any set  $B$, $\emptyset  \subseteq B$  and  $B \subseteq B$.  
\end{theorem}

In Section~\ref{S:predicates}, we also defined two sets to be equal when they have precisely the same elements.  For example,
\[
\left\{ {\left. {x \in \mathbb{R}\,} \right|\;x^2  = 4} \right\} = \left\{ { - 2,\;2} \right\}\!.
\]
If the two sets  $A$  and  $B$  are equal, then it must be true that every element of  $A$  is an element of  $B$, that is,  $A \subseteq B$, and it must be true that every element of  $B$  is an element of  $A$, that is, $B \subseteq A$.  Conversely, if  $A \subseteq B$  and   $B \subseteq A$, then  $A$  and  $B$  must have precisely the same elements.  This gives us the following test for set equality:
\begin{theorem} \label{T:setequality}
Let  $A$  and  $B$  be subsets of some universal set  $U$\!.  Then $A = B$  if and only if  $A \subseteq B$  and   $B \subseteq A$.
\end{theorem}

\noindent
%This theorem will provide a useful method for proving that two sets are equal.
\hbreak
\begin{prog}[\textbf{Using Set Notation}] \label{prog:setnotation} \hfill \\
Let the universal set be $U = \left\{ {1,2,3,4,5,6} \right\}$, and let
\[
  A = \left\{ {1,2,4} \right\}\!, \quad  B = \left\{ {1,2,3,5} \right\}\!, \quad   
  C = \left\{ {\left. {x \in U} \right|x^2  \leq 2} \right\}\!. 
\]
In each of the following, fill in the blank with one or more of the symbols  $ \subset, \\ \subseteq, =, \ne, \in ,\text{or } \notin $ so that the resulting statement is true.  For each blank, include all symbols that result in a true statement.  If none of these symbols makes a true statement, write nothing in the blank.
\begin{center}
\begin{tabular}{r p{0.8in} l p{0.5in} r p{0.8in} l }
  $A$   &   & $B$   &   & $\emptyset$   &   & $A$ \\ \cline{2-2} \cline{6-6}
   5    &   & $B$   &   &  $\left\{ 5 \right\}$  & &  $B$ \\ \cline{2-2} \cline{6-6}
  $A$   &   & $C$   &   &  $\left\{ {1,2} \right\}$  &  &  $C$ \\ \cline{2-2} \cline{6-6}
  $\left\{ {1, 2} \right\}$ &  &  $A$ &  &  $\left\{ {4,2,1} \right\}$ & & $A$ \\ \cline{2-2} \cline{6-6}
  6     &  &  $A$ &  &  $B$  &  & $\emptyset$ \\ \cline{2-2} \cline{6-6}
\end{tabular}
\end{center}
\end{prog}
\hbreak

\endinput

