\section*{Section \ref{S:predicates} Open Sentences and Sets}

\begin{enumerate}
\item \begin{enumerate}
\item This is the set of all real number solutions of the equation $2x^2 + 3x - 2 = 0$, which is $\left\{ {\dfrac{1}{2}, - 2} \right\}$. 
\item This is the set of all integer solutions of the equation $2x^2 + 3x - 2 = 0$, which is $\left\{ {- 2} \right\}$.
\item This is the set of all integers whose square is less than 25, which is 
$\left\{ -4, -3, -2, -1, 0, 1, 2, 3, 4 \right\}$.
\item This is the set of all natural numbers whose square is less than 25, which is 
$\left\{ {1,2,3,4} \right\}$
\item This is the set of all rational numbers that are 2 units from 2.5 on the number line, which is $\left\{ {-0.5,4.5} \right\}$.
\item This is the set of all integers that are less than or equal to 2.5 units from 2 on the number line, which is $\{0, 1, 2, 3, 4\}$.
\end{enumerate}

\item Some possible answers are
\begin{align*}
A &= \left\{ n^2 \mid n \in \N \right\}  \qquad B = \left\{ -\pi^n \mid n \text{~is a nonnegative integer} \right\} \\
C &= \left\{ 6n + 3 \mid n \text{~is a nonnegative integer} \right\}  = \left\{6n - 3 \mid n \in \N \right\}\\
D &= \left\{ 4n \mid n \in \Z \text{~and~} 0 \leq n \leq 25 \right\}
\end{align*}

\item The sets in (b) and (c) are equal to the given set.


\item \begin{multicols}{2}
\begin{enumerate}
\item $\{ -3 \}$
\item $\{ -8, 8 \}$
\item $\{ 1, 4, 9, 16, 25, 36, 49 \}$
\item $\{ 3, 5, 7, 9, 11, 13 \}$
\item $\{ 12, 14, 16, 18, \ldots \}$
\end{enumerate}
\end{multicols}



\item \begin{multicols}{2}
\begin{enumerate}
\item $\left\{ x\in \mathbb{Z} \mid x \geq 5 \right\}$
\item $\left\{ x\in \mathbb{Z} \mid x \text{ is even} \right\}$
\item $\left\{ x\in \mathbb{Q} \mid x > 0 \right\}$
\item $\left\{ x\in \mathbb{R} \mid  1 < x < 7 \right\}$
\item $\left\{ x\in \mathbb{R} \mid x^2 > 10 \right\}$
\end{enumerate}
\end{multicols}


\item \begin{enumerate}
\item The set $\left\{ \left. x \in \R \right| -3 \leq x \leq 5 \right\}$ is the set of all real numbers that are greater than or equal to $-3$ and less than or equal to 5.

\item The set $\left\{ \left. x \in \Z \right| -3 \leq x \leq 5 \right\}$ is the set of all integers that are greater than or equal to $-3$ and less than or equal to 5.  This set is equal to $\{ -3, -2, -1, 0, 1, 2, 3, 4, 5 \}$.

\item The set $\left\{ \left. x \in \R \right| x^2 = 16 \right\}$ is the set of all real numbers whose square is equal to 16.  This set is equal to $\{-4, 4\}$.

\item The set $\left\{ \left. x \in \R \right| x^2 + 16 = 0 \right\}$ is the set of all real numbers $x$ such that $x^2 + 16 = 0$.  This set is equal to the empty set.

\item The set $\left\{ \left. x \in \Z \right| x \text{ is odd } \right\}$ is the set of all odd integers, which is \\$\{ \ldots -3, -1,  1, 3, \ldots \}$.

\item The set $\left\{ \left. x \in \R \right| 3x - 4 \geq 17 \right\}$ is the set of all real numbers $x$ such that $3x - 4 \geq 17$.  This is equal to the set of all real numbers that are greater than or equal to 7.
\end{enumerate}
\end{enumerate}


\subsection*{Explorations and Activities}
\setcounter{oldenumi}{\theenumi}
\begin{enumerate} \setcounter{enumi}{\theoldenumi}
\item \begin{enumerate}
\item The set of all odd integers is not closed with respect to addition. For example, 3 and 5 are odd integers,
but $3 + 5 = 8$, and 8 is not an odd integer. In Theorem~\ref{T:xyodd} in Section~\ref{S:direct}, we proved that the set of all odd integers is closed with respect to multiplication.

\item The set of all even integers is closed with respect to addition. See Exercise~(\ref{exer:evenoddadd}), Part~(b) from Section~\ref{S:direct}. The set of all even integers is also closed with respect to multiplication. This is a consequence
of the result in Exercise~(\ref{exer:evenoddmult}), Part (a) from Section~\ref{S:direct}.

\item The set $A = \{1, 4, 7, 10,\ldots \}$ is not closed with respect to addition. For example, 1 and 4 are
elements of $A$, but $1 + 4 = 5$ and 5 is not in the set $A$. The set $A$, however, appears to be closed with
respect to multiplication. Any example of a product of two elements of $A$ will be in the set $A$. To
formally prove this, it is a good idea to first write the set $A$ using set builder notation.
\[
A = \{3n + 1 \mid n \text{ is a nonnegative integer} \}.
\]
If $x, y \in A$, then there exist non-negative integers $m$ and $n$ such that $x = 3m + 1$ and $y = 3n + 1$.
Then,
\begin{align*}
xy &= (3m + 1)(3n + 1) \\
   &= 9mn + 3m + 3n + 1 \\
   &= 3(3mn + m + n ) + 1
\end{align*}
Since $(3mn + m + n)$ is a non-negative integer, this shows that $xy \in A$, and hence, the set $A$ is closed under multiplication.

\item Any examples that are tried indicate that the set $B = \{ \ldots -6, -3, 0, 3, 6, 9, \ldots \}$ appears to be closed under addition and closed under multiplication. To formally prove this, it is a good idea to
first write the set $B$ using set builder notation.
\[
B = \{ 3n \mid n \in \Z \}.
\]
If $x, y \in B$, then there existintegers $m$ and $n$ such that $x = 3m$ and $y = 3n$. Then,
\begin{align*}
x + y &= 3m + 3n \\
      &= 3(m + n)
\end{align*}
Since $m + n$ is an integer, this shows that $x + y \in B$, and hence, the set $B$ is closed under addition.
In addition,
\begin{align*}
x y &= (3m)(3n) \\
      &= 3(3mn)
\end{align*}
Since $3mn$ is an integer, this shows that $x y \in B$, and hence, the set $B$ is closed under multiplication.

\item The set $C = \{3n + 1 \mid n \in \Z \}$ is closed under multiplication but is not closed under addition. The proof of these results is similar to the proof of the results in Part~(c) for the set $A$.

\item The set $D = \left\{ \left. \dfrac{1}{2^n} \right| n \in \N \right\}$ is not closed under addition. For example, $\dfrac{1}{2}$ and $\dfrac{1}{4}$ are in $D$, but
\[
\frac{1}{2} + \frac{1}{4} = \frac{3}{4}
\]
is not in $D$. The set $D$, however, is closed under multiplication. If $x, y \in D$, then there exist
natural numbers $m$ and $n$ such that $x = \dfrac{1}{2^m}$ and $y = \dfrac{1}{2^n}$.  Then
\begin{align*}
xy &= \frac{1}{2^m} \cdot \frac{1}{2^n} \\
   &= \frac{1}{2^m + n}
\end{align*}
Since $m + n$ is a natural number, the last equation shows that $xy \in D$ and hence, $D$ is closed under
multiplication.
\end{enumerate}


%\item \begin{enumerate}
%\item The set of all real numbers greater than or equal to $-3$ and less than or equal to 
%$5$.
%\item The set of all integers greater than or equal to $-3$ and less than or equal to 
%$5$.
%\item The set of all real numbers whose square is equal to 16.
%\item The set of all real numbers whose square plus 16 is equal to zero.
%\item The set of all odd integers.
%\item The set of all real numbers greater than or equal to 7.
%\end{enumerate}
%
%
%
%
%\item \begin{multicols}{2}
%\begin{enumerate}
%\item True statement.
%\item Predicate.  Truth set is $\left\{ \dfrac{14}{3} \right\}$.
%\item Predicate.  Truth set is $x \in \mathbb{R} \mid x \geq 0$.
%\item Predicate.  Truth set is $\mathbb{R}$.
%\item True statement.
%\item False statement.
%\item False statement.
%\item Predicate.  Truth set is $\emptyset$.
%\item True statement.
%\item True statement.
%\item False statement.
%\item Predicate.  Truth set is $\left\{ 1, 4, 9, 16, 25, \ldots \right\}$.
%\end{enumerate}
%\end{multicols}
%
%\pagebreak
%\item \begin{enumerate}
%\item It is not a statement since $x$ is an unquantified variable.
%\item It is a true statement.
%\item It is a false statement.
%\item It is a true statement.
%\item $\left\{ -20, -10, -5, -4, -2, -1, 1, 2, 4, 5, 10, 20 \right\}$
%\end{enumerate}

\end{enumerate}
\hbreak
\endinput
