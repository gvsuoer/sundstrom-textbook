\subsection*{Proofs of Biconditional Statements}
\index{proof!biconditional statement}%
\index{biconditional statement!proof of}%

In \typeu Activity \ref*{PA:biconditional}, we used the following logical equivalency:
\[
\left( {P \leftrightarrow Q} \right) \equiv \left( {P \to Q} \right) \wedge \left( {Q \to P} \right).
\]
This logical equivalency suggests one method for proving a biconditional statement written in the form  ``$P$  if and only if  $Q$.''  This method is to construct separate proofs of the two conditional statements  $P \to Q$  and  $Q \to P$.  For example, since we have now proven each of the following:

\begin{itemize}
  \item For each integer $n$, if  $n$  is an even integer, then  $n^2 $ is an even integer. (Exercise (\ref{exer:x2odd}) on page~\pageref{exer:x2odd} in Section \ref{S:direct})
  \item For each integer $n$, if  $n^2 $ is an even integer, then  $n$  is an even integer. (Theorem~\ref{T:n2odd})
\end{itemize}

\noindent
we can state the following theorem.
%\hbreak
\begin{theorem}\label{T:n2isodd} 
For each integer $n$,   $n$  is an even integer if and only if  $n^2 $ is an even integer.
\end{theorem}

%\noindent
%\textbf{Note:}  This gives one method for proving a biconditional statement; it is probably the most common way to prove a biconditional statement.
\hbreak
%
\subsection*{Writing Guidelines} \index{writing guidelines}%
When proving a biconditional statement using the logical equivalency \linebreak
$\left( {P \leftrightarrow Q} \right) \equiv \left( {P \to Q} \right) \wedge \left( {Q \to P} \right)$,  we actually need to prove two conditional statements.  The proof of each conditional statement can be considered as one of two parts of the proof of the biconditional statement.  Make sure that the start and end of each of these parts is indicated clearly.  This is illustrated in the proof of the following proposition.
%\hbreak
%
\begin{proposition}\label{P:iffexample}
Let  $x \in \mathbb{R}$.  The real number  $x$  equals  2  if and only if  
\linebreak
$x^3  - 2x^2  + x = 2$.
\end{proposition}
%
\begin{myproof}
We will prove this biconditional statement by proving the following two conditional statements:

\begin{itemize}
  \item For each real number $x$, if  $x$   equals  2 , then  $x^3  - 2x^2  + x = 2$.
  \item For each real number $x$, if   $x^3  - 2x^2  + x = 2$, then  $x$  equals 2.
\end{itemize}

For the first part, we assume  $x = 2$ and prove that $x^3  - 2x^2  + x = 2$.  We can do this by substituting  $x = 2$ into the expression  $x^3  - 2x^2  + x$.  This gives
\[
\begin{aligned}
  x^3  - 2x^2  + x &= 2^3  - 2\left( {2^2 } \right) + 2 \\ 
   &= 8 - 8 + 2 \\ 
   &= 2. \\ 
\end{aligned}
\]
This completes the first part of the proof.  

\newpar
For the second part, we assume that  $x^3  - 2x^2  + x = 2$ and from this assumption, we will prove that  $x = 2$.  We will do this by solving this equation for  $x$.  To do so, we first rewrite the equation  $x^3  - 2x^2  + x = 2$ by subtracting 2 from both sides:
\[
x^3  - 2x^2  + x - 2 = 0.
\]

We can now factor the left side of this equation by factoring an  $x^2$  from the first two terms and then factoring  $\left( {x - 2} \right)$ from the resulting two terms.  This is shown below.
\[
\begin{aligned}
  x^3  - 2x^2  + x - 2 &= 0 \\ 
  x^2 \left( {x - 2} \right) + \left( {x - 2} \right) &= 0 \\ 
  \left( {x - 2} \right)\left( {x^2  + 1} \right) &= 0 \\ 
\end{aligned}
\]
Now, in the real numbers, if a product of two factors is equal to zero, then one of the factors must be zero.  So this last equation implies that
\[
x - 2 = 0\text{  or  }x^2  + 1 = 0.
\]

The equation  $x^2  + 1 = 0$ has no real number solution.  So since  $x$  is a real number, the only possibility is that  $x - 2 = 0$.  From this we can conclude that  $x$  must be equal to  2.

Since we have now proven both conditional statements, we have proven that  $x = 2$  if and only if   $x^3  - 2x^2  + x = 2$.
%
\end{myproof}
%
\hbreak

\endinput
