\section*{Exercises for Section~\ref{S:moremethods}}
\begin{enumerate}

%\item Review each of the following parts of the summary for Chapter~\ref{C:proofs}:
%\begin{enumerate}
%\item Important Theorems and Results about Even and Odd Integers, page~\pageref{SS:evenodd}.
%\item Important Theorems and Results about Divisors, \pageref{SS:divisors}.
%\end{enumerate}

%Exercise 2
\xitem Let $n$ be an integer.  Prove each of the following:\label{exer:ncubed}%
  \begin{enumerate}
    \item If  $n$  is even, then  $n^3 $  is even.
    \item If  $n^3 $  is even, then  $n$  is even.
    \item The integer  $n$ is even if and only if  $n^3 $  is an even integer.
    \item The integer  $n$  is odd if and only if  $n^3 $  is an odd integer.
  \end{enumerate}
  %Exercise 3
\item In Section \ref{S:directproof}, we defined congruence modulo  $n$   where  $n$  is a natural number.  If  $a$  and  $b$  are integers, we will use the notation  
$a \not\equiv b \pmod n$ to mean that  $a$  is not congruent to  $b$  modulo  $n$.
\label{exer:sec32-2}%
  \begin{enumerate}
    \yitem Write the contrapositive of the following conditional statement:

    \begin{list}{}
      \item For all integers $a$ and $b$, if  $a \not\equiv 0 \pmod 6$ and  
$b \not\equiv 0 \pmod 6$, then  $ab \not\equiv 0 \pmod 6$.
    \end{list}
    
    \item Is this statement true or false?  Explain.
  \end{enumerate}
%Exercise 3
\xitem   \begin{enumerate}
    \item Write the contrapositive of the following statement:
      \begin{center}
       For all positive real numbers $a$ and $b$, if $\sqrt {ab}  \ne \dfrac{{a + b}}{2}$, then  $a \ne b$.
      \end{center}

    \item Is this statement true or false?  Prove the statement if it is true or provide a counterexample if it is false.
  \end{enumerate}

%Exercise 4
\xitem Are the following statements true or false?  Justify your conclusions.  %This means to prove the statement if it is true or provide a counterexample if it is false. 
\label{exer:sec32-4}%
  \begin{enumerate}
   \item For each $a \in \Z$, if  $a \equiv 2 \pmod 5$, then  $a^2  \equiv 4 \pmod 5$.
   \item For each $a \in \Z$, if  $a^2  \equiv 4 \pmod 5$, then  
$a \equiv 2 \pmod 5$.
   \item For each $a \in \Z$, $a \equiv 2 \pmod 5$ if and only if   $a^2  \equiv 4 \pmod 5$.
  \end{enumerate}

\item Is the following proposition true or false?
\label{exer:sec32-abeven}%
\begin{list}{}
\item For all integers $a$ and $b$, if $ab$ is even, then $a$ is even or $b$ is even.
\end{list}
Justify your conclusion by writing a proof if the proposition is true or by providing a counterexample if it is false.




\xitem Consider the following proposition:
\label{exer:sec32-congmod7}%
For each integer $a$, $a \equiv 3 \pmod 7$ if and only if 
$\left( a^2 + 5a \right) \equiv 3 \pmod 7$.

\begin{enumerate}
\item Write the proposition as the conjunction of two conditional statements.
\item Determine if the two conditional statements in Part~(a) are true or false.  If a conditional statement is true, write a proof, and if it is false, provide a counterexample.
\item Is the given proposition true or false?  Explain.
\end{enumerate}


\item Consider the following proposition:
\label{exer:sec32-congmod8}%
For each integer $a$, $a \equiv 2 \pmod 8$ if and only if $\left(a^2 + 4a \right) \equiv 4 \pmod 8$.

\begin{enumerate}
\item Write the proposition as the conjunction of two conditional statements.
\item Determine if the two conditional statements in Part~(a) are true or false.  If a conditional statement is true, write a proof, and if it is false, provide a counterexample.
\item Is the given proposition true or false?  Explain.
\end{enumerate}


%Exercise 7
\item For a right triangle, suppose that the hypotenuse has length $c$ feet and the lengths of the sides are $a$ feet and $b$ feet.
\label{exer:sec32-6}%
  \begin{enumerate}
    \item What is a formula for the area of this right triangle?  What is an isosceles triangle?
    \item State the Pythagorean Theorem for right triangles.
    %\item What is an isosceles triangle?
    \yitem Prove that the right triangle described above is an isosceles triangle if and only if the area of the right triangle is $\dfrac{1}{4} c^2$.
  \end{enumerate} 

%\item For a right triangle, suppose that the hypotenuse has length  $c$  feet and the lengths of %the other sides are  $a$  feet and  $b$  feet.
%
%In Exercise~(\ref{exer:righttri}) in Section \ref{S:directproof}, we proved that if the right triangle is an isosceles triangle, then the area of the right triangle is  $c^2/4$.
%
%Prove the following statement:
%
%\begin{list}{}
%\item The right triangle is an isosceles triangle if and only if the area of the right triangle %is  $c^2/4$.
%\end{list} \label{exer:sec32-6}

%Exercise 6
\xitem A real number  $x$ is defined to be a \textbf{rational number} \label{exer:rational}
\index{rational numbers}%
 provided 
\[
\text{there exist integers } m  \text{ and }  n  \text{ with }  n \ne 0  \text{ such that }  x = \frac{m}{n}.  
\]

A real number that is not a rational number is called an \textbf{irrational number.}
\index{irrational numbers}%

It is known that if  $x$  is a positive rational number, then there exist positive integers  $m$  and  $n$ with  $n \ne 0$ such that  $x = \dfrac{m}{n}$.

Is the following proposition true or false?  Explain.

\begin{list}{}
  \item For each positive real number $x$,  if  $x$  is irrational, then  $\sqrt x $
is irrational.
\end{list}
\label{exer:sec32-rational}%

%Exercise 7
\xitem Is the following proposition true or false?  Justify your conclusion.
\label{exer:sec32-8}%

\begin{list}{}
  \item For each integer $n$, $n$  is even if and only if  4  divides  $n^2 $.
\end{list}

%Exercise 8
%\item For a right triangle, suppose that the hypotenuse has length $c$ feet and the lengths of the sides are $a$ feet and $b$ feet.
%  \begin{enumerate}
%    \item What is a formula for the area of this right triangle?
%    \item What can be concluded from the Pythagorean Theorem for right triangles?
%    \item What is an isosceles triangle?
%    \item Prove that if the right triangle described above is an isosceles triangle, then the %area of the right triangle is $\frac{1}{4} c^2$.
%  \end{enumerate}

\item Prove that for each integer $a$, if $a^2 - 1$ is even, then 4 divides $a^2 - 1$.

\item Prove that for all integers $a$ and $m$, if $a$ and $m$ are the lengths of the sides of a right triangle and $m + 1$ is the length of the hypotenuse, then $a$ is an odd integer.

\item Prove the following proposition:
\label{exer:rationalbetween}%
\begin{list}{}
\item If  $p, q \in \mathbb{Q}$ with  $p < q$, then there exists an  $x \in \mathbb{Q}$ with  
$p < x < q$.
\end{list}


\item Are the following propositions true or false?  Justify your conclusion. 
\label{exer:existintegers}%

  \begin{enumerate}
    \item There exist integers  $x$  and  $y$ such that  $4x + 6y = 2$.
    \item There exist integers  $x$  and  $y$ such that  $6x + 15y = 2$.
    \item There exist integers  $x$  and  $y$ such that  $6x + 15y = 9$.
  \end{enumerate}

\xitem Prove that there exists a real number  $x$  such that  $x^3  - 4x^2  = 7$.
\label{exer:IVT}%

\item Let  $y_1 , y_2 , y_3 , y_4 $ be  real numbers.  The \textbf{mean}, $\overline y $,  of these four numbers is defined to be the sum of the four numbers divided by 4.  That is,
\[
\overline y  = \frac{{y_1  + y_2  + y_3  + y_4 }}{4}.
\]
Prove that there exists a  $y_i $ with  $1 \leq i \leq 4$ such that  $y_i  \geq \overline y $.

\hint  One way is to let  $y_{\text{max}}$ be the largest of  
$y_1 , y_2 , y_3 , y_4 $.


\item  Let  $a$  and  $b$  be natural numbers such that  $a^2  = b^3 $.  Prove each of the propositions in 
%Parts~(\ref{exer:a2equalb3-a}) through~(\ref{exer:a2equalsb3-d}).  
Parts (a) through (d).  (The results of Exercise~(\ref{exer:ncubed}) and Theorem~\ref{T:n2isodd} may be helpful.)
\label{exer:sec32-16}%

\begin{enumerate}
  \item If  $a$  is even, then  4  divides  $a$. 
\label{exer:a2equalb3-a}%
  \yitem If  4  divides  $a$, then  4  divides  $b$.
  \item If  4  divides  $b$, then  8  divides  $a$.
  \item If  $a$  is even, then  8 divides  $a$.  
\label{exer:a2equalsb3-d}%
  \item Give an example of natural numbers  $a$  and  $b$  such that  $a$  is even and  $a^2  = b^3 $, but  $b$  is not divisible by  8.
\end{enumerate}

\xitem Prove the following proposition:
\label{exer:sec32-equation17}%
\begin{list}{}
\item Let $a$ and $b$ be integers with $a \ne 0$.  If $a$ does not divide $b$, then the equation 
$ax^3 + bx + \left( b + a \right) = 0$ does not have a solution that is a natural number.
\end{list}
\hint It may be necessary to factor a sum  of cubes.  Recall that 
\[
u^3 + v^3 = \left( u + v \right) \left( u^2 - uv + v^2 \right).
\]

\item \textbf{Evaluation of Proofs}  \hfill \\
See the instructions for Exercise~(\ref{exer:proofeval}) on 
page~\pageref{exer:proofeval} from Section~\ref{S:directproof}.

\begin{enumerate}
\item \textbf{Proposition}. If $m$ is an odd integer, then $\left(m + 6\right)$ is an odd integer.

\begin{myproof}
For $m + 6$ to be an odd integer, there must exist an integer $n$ such that
\[
m + 6 = 2n + 1.
\]
By subtracting 6 from both sides of this equation, we obtain %$m = 2n - 5$.  Therefore,
\[
\begin{aligned}
m &= 2n - 6 + 1 \\
  &= 2 \left(n - 3 \right) + 1.
\end{aligned}
\]
By the closure properties of the integers, $\left(n - 3 \right)$ is an integer, and hence, the last equation implies that $m$ is an odd integer.  This proves that if $m$ is an odd integer, then $m+6$ is an odd integer.
\end{myproof}

\item \textbf{Proposition}. For all integers $m$ and $n$, if $mn$ is an even integer, then $m$ is even or $n$ is even.

\begin{myproof}
For either $m$ or $n$ to be even, there exists an integer $k$ such that $m = 2k$ or $n = 2k$.  So if we multiply $m$ and $n$, the product will contain a factor of 2 and, hence, $mn$ will be even.
\end{myproof}
\end{enumerate}
\end{enumerate}


\subsection*{Explorations and Activities}
\setcounter{oldenumi}{\theenumi}
\begin{enumerate} \setcounter{enumi}{\theoldenumi}
%\item \textbf{Some Propositions about Congruence}.  Determine if each of the following biconditional statements is true or false.  If a statement is true, then write a proof of the statement.  If a statement is false, then provide a counterexample to prove it is false.  In addition, if a biconditional statement is found to be false, determine if one of the conditional statements within it is true.  In that case,  state an appropriate theorem for this conditional statement and prove it.
%\begin{enumerate}
%  \item For each integer $a$, $a \equiv 3 \pmod 7$ if and only if $\left(a^2 + 5a \right) \equiv 3 \pmod 7$.
%  \item For each integer $a$, $a \equiv 2 \pmod 8$ if and only if $\left(a^2 + 4a \right) \equiv 4 \pmod 8$.
%\end{enumerate}


\item \textbf{Using a Logical Equivalency}.  Consider the following proposition:
\begin{list}{}
  \item \textbf{Proposition}.  For all integers $a$  and  $b$, if  3  does not divide  $a$  and  3  does not divide  $b$, then 3  does not divide the product  $a \cdot b$.
\end{list}
%
\begin{enumerate}
  \item Notice that the hypothesis of the proposition is stated as a conjunction of two negations  (``3  does not divide  $a$  and  3  does not divide  $b$'').   Also, the conclusion is stated as the negation of a sentence  (``3 does not divide the product $a \cdot b$.''). This often indicates that we should consider using a proof of the contrapositive. If we use the symbolic form  $\left( {\mynot  Q \wedge \mynot  R} \right) \to \mynot  P$  as a model for this proposition, what is  $P$, what is  $Q$, and what is  $R$?

  \item Write a symbolic form for the contrapositive of   $\left( {\mynot  Q \wedge \mynot  R} \right) \to \mynot  P$.

  \item Write the contrapositive of the proposition as a conditional statement in English.
\end{enumerate}

We do not yet have all the tools needed to prove the proposition or its contrapositive.  However, later in the text, we will learn that the following proposition is true.

\begin{flushleft}
\textbf{Proposition X.}
Let  $a$  be an integer.  If  3  does not divide  $a$, then there exist integers  $x$  and  $y$  such that  $3x + ay = 1$.
\end{flushleft}

\begin{enumerate}
\setcounter{enumii}{3}
  \item \begin{enumerate} \item Find  integers  $x$  and  $y$  guaranteed by Proposition X when  $a = 5$.
      \item Find  integers  $x$  and  $y$  guaranteed by Proposition X when  $a = 2$.
      \item Find  integers  $x$  and  $y$  guaranteed by Proposition X when  $a =  - 2$.
      \end{enumerate}

  \item Assume that Proposition  X  is true and use it to help construct a proof of the contrapositive of the given proposition.  In doing so, you will most likely have to use the logical equivalency $P \to \left( {Q \vee R} \right) \equiv \left( {P \wedge \mynot  Q} \right) \to R$.

\end{enumerate}

\end{enumerate}





\hbreak


\endinput

%Exercise 5
%\item It is known from geometry that the sum of the interior angles of a quadrilateral is $360^ \circ  $ .  Prove the following statement:
%
%  \begin{list}{}
%    \item If no interior angle of a quadrilateral is an obtuse angle, then the quadrilateral is a rectangle.
%  \end{list}
%
%\item One of the most famous unsolved problems in mathematics is a conjecture made by Christian Goldbach in a letter to Leonhard Euler in 1742.  The conjecture made in this letter is now known as \textbf{Goldbach's Conjecture}.
%\index{Goldbach's Conjecture}%
%  The conjecture is:
%
%\begin{list}{}
%\item Every even integer greater than 2 can be expressed as the sum of two (not necessarily distinct) prime numbers.
%\end{list}
%
%As of January 1, 2002, it is not known if this conjecture is true or false, although most mathematicians believe it to be true. \label{exer:goldbach}
%
%\begin{enumerate}
%\item Describe one way to prove that Goldbach's Conjecture is false.
%
%\item Prove the following:
%\begin{list}{}
%\item If there exists an odd integer greater than 5 that is not the sum of three prime numbers, then Goldbach's Conjecture is false.
%\end{list}
%\end{enumerate}
%
