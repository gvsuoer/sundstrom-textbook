\begin{previewactivity}[\textbf{Review of Congruence Modulo} $\boldsymbol{n}$]\label{PA:reviewofcongruence} \hfill
\begin{enumerate}
\item Let  $a, b \in \mathbb{Z}$ and let  $n \in \mathbb{N}$.  On page~\pageref{congruence} of Section~\ref{S:directproof}, we defined what it means to say that  $a$  is congruent to  $b$  modulo  $n$.  Write this definition and state two different conditions that are equivalent to the definition.

\item %Let  $a, b \in \mathbb{Z}$ and let  $n \in \mathbb{N}$.  We use  
%$a \equiv b \pmod n$ as a notation for  ``$a$  is congruent to  $b$  modulo  $n$.''  
Explain why congruence modulo  $n$  is a relation on  $\mathbb{Z}$.

\item Carefully review Theorem~\ref{T:modprops} and the proofs given on page~\pageref{T:modprops} of Section~\ref{S:divalgo}.  In terms of the properties of relations introduced in \typeu Activity~\ref*{PA:propsofrelaitons}, what does this theorem say about the relation of congruence modulo  $n$  on the integers?

\item Write a complete statement of Theorem~\ref{T:congtorem} on page~\pageref{T:congtorem} and Corollary~\ref{C:congtorem}.

\item Write a proof of the symmetric property for congruence modulo  $n$.  That is, prove the following:

\begin{list}{}
\item Let  $a, b \in \mathbb{Z}$ and let  $n \in \mathbb{N}$.  If  
$a \equiv b \pmod n$, then  $b \equiv a \pmod n$.
\end{list}

\end{enumerate}
\end{previewactivity}
\hbreak





\endinput
