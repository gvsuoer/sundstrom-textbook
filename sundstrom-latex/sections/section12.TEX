\section{Constructing Direct Proofs} \label{S:direct}
%\markboth{Chapter \ref{C:intro}. Introduction to Writing}{\ref{S:direct}. Constructing Direct Proofs}
\setcounter{previewactivity}{0}
%\input{focus/focus12}
\begin{previewactivity}[\textbf{Definition of Even and Odd Integers}]\label{PA:even} \hfill \\
Definitions play a very important role in mathematics.  A direct proof of a proposition in mathematics is often a demonstration that the proposition follows logically from certain definitions and previously proven propositions.  A \textbf{definition}
\index{definition}%
 is an agreement that a particular word or phrase will stand for some object, property, or other concept that we expect to refer to often.  In many elementary proofs, the answer to the question, ``How do we prove a certain proposition?'', is often answered by means of a definition.  For example, in Progress Check~\ref{pr:explores} on page~\pageref{pr:explores}, all of the examples should have indicated that the following conditional statement is true:
\begin{center}
If  $x$  and  $y$  are odd integers, then $x \cdot y$ is an odd integer.
\end{center}
In order to construct a mathematical proof of this conditional statement, we need a precise definition of what it means to say that an integer is an even integer and what it means to say that an integer is an odd integer.
%
\begin{defbox}{D:even}{An integer  $a$  is an \textbf{even integer} 
\index{even integer}%
 provided that there exists an integer  $n$  such that  $a = 2n$. An integer  $a$  is an 
\textbf{odd integer}
\index{odd integer}%
 provided there exists an integer  $n$  such that  $a = 2n + 1$.}
\end{defbox}
\newpar
\label{def:even}
Using this definition, we can conclude that the integer 16 is an even integer since $16 = 2 \cdot 8$ and 8 is an integer.  By answering the following questions, you should obtain a better understanding of these definitions.  These questions are not here just to have questions in the textbook.  Constructing and answering such questions is a way in which many mathematicians will try to gain a better understanding of a definition.
%
%\begin{flushleft}
%\fbox{\parbox{5in}{\begin{definition}
%An integer  $a$  is an \textbf{even integer} if there exists an integer  $n$  such that  $a = 2n$. An integer  $a$  is an \textbf{odd integer} if there exists an integer  $n$  such that  $a = 2n + 1$.
%\end{definition}}}
%\end{flushleft}
\begin{enumerate}
\item Use the definitions given above to
\begin{enumerate}
\item Explain why  28, $-42$, 24, and 0 are even integers.

\item Explain why 51, $-11$, 1, and $-1$  are odd integers.
\end{enumerate}
\end{enumerate}

\noindent
It is important to realize that mathematical definitions are not made randomly.  In most cases, they are motivated by a mathematical concept that occurs frequently.
%
\begin{enumerate}
\setcounter{enumi}{1}
\item Are the definitions of even integers and odd integers consistent with your previous ideas about even and odd integers?	
\end{enumerate}
\hbreak
\end{previewactivity}

\endinput

\begin{previewactivity}[\textbf{Thinking about a Proof}]\label{PA:thinking} \hfill \\
Consider the following proposition:  
\begin{flushleft}
\textbf{Proposition.}  If  $x$  and  $y$  are odd integers, then $ x \cdot y$  is an odd integer.
\end{flushleft}
Think about how you might go about proving this proposition. A \textbf{direct proof}
\index{direct proof}%
\index{proof!direct}%
 of a conditional statement is a demonstration that the conclusion of the conditional statement follows logically from the hypothesis of the conditional statement.  Definitions and previously proven propositions are used to justify each step in the proof.    To help get started in proving this proposition, answer the following questions:
\begin{enumerate}
  \item The proposition is a conditional statement.  What is the hypothesis of this conditional statement?  What is the conclusion of this conditional statement?
  \item If  $x = 2$ and  $y = 3$, then  $x \cdot y = 6$, and 6 is an even integer.  Does this example prove that the proposition is false?  Explain.
  \item If  $x = 5$ and  $y = 3$, then  $x \cdot y = 15$.  Does this example prove that the proposition is true?  Explain.
\end{enumerate}
In order to prove this proposition, we need to prove that whenever both $x$ and $y$ are odd integers, 
$x \cdot y$ is an odd integer.  Since we cannot explore all possible pairs of integer values for $x$ and $y$, we will use the definition of an odd integer to help us construct a proof.  

\begin{enumerate} \setcounter{enumi}{3}
  \item To start a proof of this proposition, we will assume that the hypothesis of the conditional statement is true.  So in this case, we assume that both $x$ and $y$ are odd integers.  We can then use the definition of an odd integer to conclude that there exists an integer $m$ such that $x = 2m + 1$.  Now use the definition of an odd integer to make a conclusion about the integer $y$.
\label{PA:prev12-Q4}

\note The definition of an odd integer says that a certain other integer exists.  This definition may be applied to both $x$ and $y$.  However, do not use the same letter in both cases.  To do so would imply that 
$x = y$ and we have not made that assumption.  To be more specific, if $x = 2m + 1$ and $y = 2m +1$, then 
$x = y$.
  \item We need to prove that if the hypothesis is true, then the conclusion is true.  So, in this case, we need to prove that $x \cdot y$ is an odd integer.  At this point, we usually ask ourselves a so-called 
\textbf{backward question}.  In this case, we ask, ``Under what conditions can we conclude that $x \cdot y$ is an odd integer?''  Use the definition of an odd integer to answer this question.% and be careful to use a different letter for the new integer than was used in Part~(\ref{PA:prev12-Q4}).
\end{enumerate}
\hbreak
\end{previewactivity}


\endinput



\subsection*{Properties of Number Systems}\label{SS:properties}
At the end of Section~\ref{S:prop}, we introduced notations for the standard number systems we use in mathematics and discussed their closure properties.  For this text, it is assumed that the reader is familiar with these closure properties and the basic rules of algebra that apply to all real numbers that are given in %That is, it is assumed the reader is familiar with the properties of the real numbers shown in 
Table~\ref{Ta:propertiesofreals}. 
\begin{table}[h]
$$
\BeginTable
\BeginFormat 
| p(1.5in)|p(3.0in)| 
\EndFormat
"   " For all real numbers $x$, $y$, and $z$ "  \\+02  \_
| Identity Properties |  $x+0=x$ and $x \cdot 1=x$ |   \\ \_1 
| Inverse Properties  |  $x + \left( { - x} \right) = 0$ and if $x \ne 0$, then 
$x \cdot \dfrac{1}{x} = 1$.  |  \\+55 \_1
|Commutative Properties  |  \Lower{$x + y = y + x$ and  $x  y = y  x$} | \\ \_1
|Associative Properties  |   \Lower{$\left( {x + y} \right) + z = x + \left( {y + z} \right)$ and 
                        $\left( {x  y} \right)  z = x  \left( {y  z} \right)$} | \\ \_1
|Distributive Properties |  \Lower{$x\left( {y + z} \right) = x  y + x  z$ and  
                       $\left( {y + z} \right)x = y  x + z  x$}  | \\ \_ 
\EndTable
$$
    \caption{Properties of the Real Numbers}
    \label{Ta:propertiesofreals}
\end{table}
\endinput

%


\input{subsections/section12/sub12-construct-proof}



\subsection*{Writing Guidelines for Mathematics Proofs}
\index{writing guidelines|(}%
%In this section, the emphasis is on constructing an outline of a proof using a know-show table.  However, some proof writing will be done, and 
At the risk of oversimplification, doing mathematics can be considered to have two distinct stages.  The first stage is to convince yourself that you have solved the problem or proved a conjecture.  This stage is a creative one and is quite often how mathematics is actually done.  The second equally important stage is to convince other people that you have solved the problem or proved the conjecture.  This second stage often has little in common with the first stage in the sense that it does not really communicate the process by which you solved the problem or proved the conjecture.   However, it is an important part of the process of communicating mathematical results to a wider audience.

A \textbf{mathematical proof} is a convincing argument (within the accepted standards of the mathematical community) that a certain \index{proof}%
mathematical statement is necessarily true.  A proof generally uses deductive reasoning and logic but also contains some amount of ordinary language (such as English).  A mathematical proof that you write should convince an appropriate audience that the result you are proving is in fact true. So we do not consider a proof complete until there is a well-written proof.  So it is important to introduce some writing guidelines.  The preceding proof was written according to the following basic guidelines for writing proofs.  More writing guidelines will be given in Chapter~\ref{C:proofs}.
\begin{enumerate}
%\item \label{writing:know}%
%\textbf{Know Your Audience}. 
%
%Every writer should have a clear idea of the intended audience for a piece of writing.  In that way, the writer can give the right amount of information at the proper level of sophistication to communicate effectively.  This is especially true for mathematical writing.  For example, if a mathematician is writing a solution to a textbook problem for a solutions manual for instructors, the writing would be brief with many details omitted.  However, if the writing was for a students' solution manual, more details would be included.  %This is why the instructions for Beginning Activity~\ref{PA:equation} stated that your descriptions should be written for someone who already knows basic algebra and how to solve quadratic equations.


\item \textbf{Begin with a carefully worded statement of the theorem or result to be proven.}
This should be a simple declarative statement of the theorem or result.  Do not simply rewrite the problem as stated in the textbook or given on a handout.  Problems often begin with phrases such as ``Show that'' or ``Prove that.''  This should be reworded as a simple declarative statement of the theorem.  Then skip a line and write ``Proof''  in italics or boldface font (when using a word processor).  Begin the proof on the same line.  Make sure that all paragraphs can be easily identified.  Skipping a line between paragraphs or indenting each paragraph can accomplish this.

As an example, an exercise in a text might read, ``Prove that if $x$  is an odd integer, then $x^2$ is an odd integer.''  This could be started as follows:

\textbf{Theorem.} 
If  $x$  is an odd integer, then $x^2$ is an odd integer.

\textbf{\emph{Proof}}:  We assume that  $x$  is an odd integer  $\ldots$

\item \textbf{Begin the proof with a statement of your assumptions.}
Follow the statement of your assumptions with a statement of what you will prove.

\noindent
\textbf{Theorem.} 
If  $x$  is an odd integer, then $x^2$ is an odd integer.

%\begin{flushleft}
\noindent
\emph{\textbf{Proof}}.  We assume that  $x$  is an odd integer and will prove that $x^2$   is an odd integer.
%\end{flushleft}

\item \textbf{Use the pronoun ``we.''}
If a pronoun is used in a proof, the usual convention is to use ``we'' instead of ``I.''  The idea is to stress that you and the reader are doing the mathematics together.  It will help encourage the reader to continue working through the mathematics.  Notice that we started the proof of Theorem~\ref{T:xyodd} with ``We assume that $\ldots$ .''

%If a pronoun is used in a proof, the usual convention is to use ``we'' instead of ``I.''  The idea is that the author and the reader are proving the theorem together.



\item \textbf{Use italics for variables when using a word processor.}
When using a word processor to write mathematics, the word processor needs to be capable of producing the appropriate mathematical symbols and equations.  The mathematics that is written with a word processor should look like typeset mathematics.  This means that italics is used for variables, boldface font is used for vectors, and regular font is used for mathematical terms such as the names of the trigonometric and logarithmic functions.  

For example, we do not write sin (x) or \emph{sin (x)}.  The proper way to typeset this is $\sin (x)$.



%\item \textbf{Do not use $*$ for multiplication or \^{} for exponents.} \\
%Leave this type of notation for writing computer code.  The use of this notation makes it difficult for humans to read.  In addition, avoid using $/$ for division when using a complex fraction.  
%
%For example, it is very difficult to read 
%$\left(x^3 -3x^2 + 1/2 \right)/\left(2x/3 - 7\right)$; the fraction
%\[
%\frac{x^3 - 3x^2 +\dfrac{1}{2}}{\dfrac{2x}{3} - 7}
%\]
%is much easier to read.


%\item \textbf{Use complete sentences and proper paragraph structure.}
%
%Good grammar is an important part of any writing.  Therefore, conform to the accepted rules of grammar.  Pay careful attention to the structure of sentences.  Write proofs using \textbf{complete sentences} but avoid run-on sentences.  Also, do not forget punctuation, and always use a spell checker when using a word processor.


\item \textbf{Display important equations and mathematical expressions.}
Equations and manipulations are often an integral part of mathematical exposition.  Do not write equations, algebraic manipulations, or formulas in one column with reasons given in another column. 
%(as is often done in geometry texts).
   Important equations and manipulations should be displayed.  This means that they should be centered with blank lines before and after the equation or manipulations, and if the left side of the equations does not change, it should not be repeated.  For example,

Using algebra, we obtain	
\begin{align}
  x \cdot y &= \left( {2m + 1} \right)\left( {2n + 1} \right)  \notag \\ 
            &= 4mn + 2m + 2n + 1  \notag \\ 
            &= 2\left( {2mn + m + n} \right) + 1.  \notag  
\end{align} 
Since  $m$  and  $n$  are integers, we conclude that $ \ldots $ .

%\item \textbf{Do not use a mathematical symbol at the beginning of a sentence.}
%For example, we should not write, ``Let $n$ be an integer.  $n$ is an odd integer provided that \ldots''  Many people find this hard to read and often have to re-read it to understand it.  It would be better to write, ``An integer $n$ is an odd integer provided that \ldots''


\item \textbf{Tell the reader when the proof has been completed.}
Perhaps the best way to do this is to simply write, ``This completes the proof.''  Although it may seem repetitive, a good alternative is to finish a proof with a sentence that states precisely what has been proven.  In any case, it is usually good practice to use  some ``end of proof symbol'' such as  $\blacksquare$.
\index{writing guidelines|)}%


%\item \textbf{Keep it simple}.
%
%It is often difficult to understand a mathematical argument no matter how well it is written.  Do not let your writing help make it more difficult for the reader.  Use simple, declarative sentences and short paragraphs, each with a simple point.
\end{enumerate}
\hbreak

\begin{prog}[\textbf{Proving Propositions}] \label{prog:proving} \hfill \\
Construct a know-show table for each of the following propositions and then write a formal proof for one of the propositions.
\begin{enumerate}
  \item If $x$ is an even integer and $y$ is an even integer, then $x + y$ is an even integer.
  \item If $x$ is an even integer and $y$ is an odd integer, then $x + y$ is an odd integer.
  \item If $x$ is an odd integer and $y$ is an odd integer, then $x + y$ is an even integer.
\end{enumerate}
\end{prog}
\hbreak



\endinput




\subsection*{Some Comments about Constructing Direct Proofs}
\index{direct proof|(}%
\index{proof!direct|(}%
\begin{enumerate}
  \item When we constructed the know-show table prior to writing a proof for Theorem~\ref{T:xyodd}, we had only one answer for the backward question and one answer for the forward question.  Often, there can be more than one answer for these questions.  For example, consider the following statement:
\begin{center}
If  $x$  is an odd integer, then  $x^2$ is an odd integer.
\end{center}
The backward question for this could be, ``How do I prove that an integer is an odd integer?''  One way to answer this is to use the definition of an odd integer, but another way is to use the result of 
Theorem~\ref{T:xyodd}.  That is, we can prove an integer is odd by proving that it is a product of two odd integers.

The difficulty then is deciding which answer to use.  Sometimes we can tell by carefully watching the interplay between the forward process and the backward process.  Other times, we may have to work with more than one possible answer.  
\label{proofcomment1}%
%\hrulefill

%\begin{prog}[Constructing a Know-Show Table]\label{pr:kstable1} \hfill \\
%Construct a know-show table for the following statement that uses the result of 
%Theorem~\ref{T:xyodd}:
%
%\begin{list}{}
%\item If  $x$  is an odd integer, then  $x^2$ is an odd integer.
%\end{list}
%%\hrulefill
%\end{prog}
%
\item Sometimes we can use previously proven results to answer a forward question or a backward question.  This was the case in the example given in 
Comment~(\ref{proofcomment1}), where Theorem~\ref{T:xyodd} was used to answer a backward question.

\item Although we start with two separate processes (forward and backward), the key to constructing a proof is to find a way to link these two processes.  This can be difficult.  One way to proceed is to use the know portion of the table to motivate answers to backward questions and to use the show portion of the table to motivate answers to forward questions.

\item Answering a backward question can sometimes be tricky.  If the goal is the statement  $Q$, we must construct the know-show table so that if we know that  $Q$1 is true, then we can conclude that $Q$ is true.  It is sometimes easy to answer this in a way that if it is known that  $Q$ is true, then we can conclude that $Q$1 is true.  For example, suppose the goal is to prove 
\[
y^2  = 4,
\]
where  $y$  is a real number.  A backward question could be, ``How do we prove the square of a real number equals four?''  One possible answer is to prove that the real number equals 2.  Another way is to prove that the real number equals $-2$.  This is an appropriate backward question, and these are appropriate answers.

However, if the goal is to prove
\[
y = 2,
\]
where  $y$  is a real number, we could ask, ``How do we prove a real number equals 2?''  It is not appropriate to answer this question with ``prove that the square of the real number equals 4.''  
%That is, we should not have the show portion of the table as follows:
%$$
%\BeginTable
%\BeginFormat
%|p(0.4in)|p(1.6in)|p(1.6in)|
%\EndFormat
%\_
%  | $Q1$  |   $y^2=4$             |           |  \\ \_1
%  | $Q$   |  $y=2$                |  Square root of both sides | \\ \_
%  |\textbf{Step}  |  \textbf{Show}  |  \textbf{Reason} | \\+20 \_
%\EndTable
%$$
%\begin{center}
%\begin{tabular}[h]{|p{0.4in}|p{1.6in}|p{1.6in}|}
%  \hline
%  $Q1$  &   $y^2=4$             &             \\ \hline
%  $Q$  &  $y=2$  &  Square root of both sides \\ \hline
%  \textbf{Step}  &  \textbf{Show}  &  \textbf{Reason} \\ \hline
%\end{tabular}
%\end{center}
This is because if $y^2=4$, then it is not necessarily true that $y=2$.

\item Finally, it is very important to realize that not every proof can be constructed by the use of a simple know-show table.  Proofs will get more complicated than the ones that are in this section.  The main point of this section is not the know-show table itself, but the way of thinking about a proof that is indicated by a know-show table.  In most proofs, it is very important to specify carefully what it is that is being assumed and what it is that we are trying to prove.  The process of asking the ``backward questions'' and the ``forward questions'' is the important part of the know-show table.  It is very important to get into the ``habit of mind'' of working backward from what it is we are trying to prove and working forward from what it is we are assuming.  Instead of immediately trying to write a complete proof, we need to stop, think, and ask questions such as

\begin{itemize}
\item Just exactly what is it that I am trying to prove?
\item How can I prove this?
\item What methods do I have that may allow me to prove this?
\item What are the assumptions?
\item How can I use these assumptions to prove the result?
\end{itemize}
\index{direct proof|)}%
\index{proof!direct|)}%

\end{enumerate}
%\hrule




\begin{prog}[\textbf{Exploring a Proposition}]\label{A:kstable2} \hfill \\
Construct a table of values for  $\left( {3m^2  + 4m + 6} \right)$
 using at least six different integers for  $m$.  Make one-half of the values for  $m$  even integers and the other half odd integers.  Is the following proposition true or false?  

\begin{center}
If $m$ is an odd integer, then $\left(3m^2 + 4m + 6 \right)$ is an odd integer.
\end{center}
Justify your conclusion.  This means that if the proposition is true, then you should write a proof of the proposition.  If the proposition is false, you need to provide an example of an odd integer for which $\left(3m^2 + 4m + 6 \right)$ is an even integer.
\end{prog}
\hbreak


\begin{prog}[\textbf{Constructing and Writing a Proof}] \label{pr:pythag} \hfill \\
The \textbf{Pythagorean Theorem}
\index{Pythagorean Theorem}%
for right triangles states that if $a$ and $b$ are the lengths of the legs of a right triangle and $c$ is the length of the hypotenuse, then $a^2 + b^2 = c^2$.  For example, if $a = 5$ and $b = 12$ are the lengths of the two sides of a right triangle and if $c$ is the length of the hypotenuse, then the $c^2 = 5^2 + 12^2$ and so 
$c^2 = 169$.  Since $c$ is a length and must be positive, we conclude that $c = 13$.

\newpar
Construct and provide a well-written proof for the following proposition.

\newpar
\textbf{Proposition}.  If $m$ is a real number and $m$, $m + 1$, and $m + 2$ are the lengths of the three sides of a right triangle, then $m = 3$.

\newpar
Although this proposition uses different mathematical concepts than the one used in this section, the process of constructing a proof for this proposition is the same forward-backward method that was used to construct a proof for Theorem~\ref{T:xyodd}.  However, the backward question, ``How do we prove that $m = 3$?'' is simple but may be difficult to answer.  The basic idea is to develop an equation from the forward process and show that $m = 3$ is a solution of that equation.
\end{prog}
\hbreak


\endinput




