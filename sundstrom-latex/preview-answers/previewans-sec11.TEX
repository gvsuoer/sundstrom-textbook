\documentclass[11pt]{article}
%\usepackage{../../styles/activity}
\usepackage{c://pctex/beginact}


\usepackage{xr}
%\externaldocument{../../0-MR}
\externaldocument{../../0-MR-links}

\lhead{}
%\chead{\textbf{\Large{\hspace{0pt}Beginning Activities for Section~1.2}}\\ \beghead} 
\bahead{1.1}
\rhead{}
\lfoot{}
\rfoot{}
\cfoot{\hspace{0pt}\scalebox{0.4}{\includegraphics{cc-by-nc-sa.eps}}}

\begin{document}

\subsection*{Beginning Activity 1 (Statements)}
The sentences in (1), (2), (4), and (5) are statements.

\noindent 
The sentence in (3) is not a statement since it involves a variable $x$ and we do not know what $x$ represents.

\noindent
The sentence is (6) is not a statement since it is a question and not a declarative sentence.
 \hbreak

\subsection*{Beginning Activity 2 (Conditional Statements)}
\begin{enumerate}
  \item If it is raining, then Laura is at the theatre. 
\begin{enumerate}
\item This statement is true when it is raining and Laura is at the theatre.
\item This statement is false when it is raining and Laura is not at the theatre.
\item This statement is true when it is not raining and Laura is at the theatre.
\item This statement is true when it is not raining and Laura is not at the theatre.
\end{enumerate}

\item \begin{tabular}[t]{| c | p{2.0in} | p{2.0in} |} \hline
  &  Hypothesis  &  Conclusion \\ \hline
%(a)  &  $n$ is a prime number.  &  $n^2$ has three positive divisors. \\ \hline
%(b)  &  $a$ is an irrational number and $b$ is an irrational number.  &  $a \cdot b$ is an irrational number. \\ \hline
%(c)  &  $p$ is a prime number.  &  $p = 2$ or $p$ is an odd number.  \\ \hline
%(d)  &  $p$ is a prime number and $p \ne 2$. & $p$ is an odd number. \\ \hline
%(e)  &  $p \ne 2$ and $p$ is an even number &  $p$ is not prime.  \\ \hline
(a) & $x$ is a positive real number. & $\sqrt{x}$ is a positive real number.  \\  \hline
(b) & $\sqrt{x}$ is not a real number. & $x$ is a negative real number. \\ \hline
(c) & the lengths of the diagonals of a parallelogram are equal. & the parallelogram is a rectangle. \\ \hline
\end{tabular}

\end{enumerate}
\hbreak

\end{document}

