\section*{Exercises~\ref{S:constructive}}
\begin{enumerate}
%
\item Prove the following proposition: \label{exer:sec35-1}
\begin{list}{}
\item If  $p, q \in \mathbb{Q}$ with  $p < q$, then there exists an  $x \in \mathbb{Q}$ with  $p < x < q$.
\end{list}

\item Is the following proposition true or false?  Justify your conclusion.

\begin{list}{}
\item Let  $m, b \in \mathbb{R}$ with  $m \ne 0$.  If  $u \in \mathbb{R}$, then there exists an  $x \in \mathbb{R}$  such that  $mx + b = u$.
\end{list}
\label{exer:35start}


\item Prove the following proposition: \label{exer:sec35-3}
\begin{list}{}
\item If  $p, q \in \mathbb{Q}$ with  $q \ne 0$, then  $\frac{\textstyle p}{\textstyle q}$  is a rational number.
\end{list}

\item If  $n \in \mathbb{Z}$ and  $m = n + 1$, then  $m$  and  $n$  are said to be \textbf{consecutive integers.} \label{exer:sec35-4}
\index{consecutive integers}%
\index{integers!consecutive}%
Is the following proposition true or false?  Justify your conclusion.
\begin{list}{}
\item If  $m$  and  $n$  are two consecutive integers, then  4  divides \\$\left( {m^2  + n^2  - 1} \right)$.
\end{list}

\item Let  $x \in \mathbb{R}$.  Prove that there exists a real number  $\delta $
 with  $\delta  > 0$ such that $\left| {\left( {2x + 1} \right) - 7} \right| < 0.01$  whenever  $\left| {x - 3} \right| < \delta $. \label{exer:sec35-6}

\underline{Note}:  The symbol  $\delta $ is the lowercase letter delta from the Greek alphabet.  It is often used in mathematics to represent a small positive real number.

\item \begin{enumerate} \label{exer:sec35-solution}
\item Prove that there exists a real number  $x$  such that  $x^3  - 4x^2  = 7$. \label{exer:35end}
\item Prove that there is no integer $x$ such that  $x^3  - 4x^2  = 7$.
\end{enumerate}

\item Classify each of your proofs in Exercises~(\ref{exer:35start}) through~(\ref{exer:35end}) as a constructive proof or a nonconstructive proof.

\item Are the following propositions true or false?  Justify your conclusion. 
\label{exer:sec35-9}

  \begin{enumerate}
    \item There exist integers  $x$  and  $y$ such that  $4x + 6y = 2$.
    \item There exist integers  $x$  and  $y$ such that  $6x + 15y = 2$.
    \item There exist integers  $x$  and  $y$ such that  $6x + 15y = 9$.
  \end{enumerate}

\item Let  $y_1 , y_2 , y_3 , y_4 $ be  real numbers.  The mean, $\overline y $,  of these four numbers is defined to be the sum of the four numbers divided by 4.  That is,
\[
\overline y  = \frac{{y_1  + y_2  + y_3  + y_4 }}{4}.
\]
Prove that there exists a  $y_i $ with  $1 \leqslant i \leqslant 4$ such that  $y_i  \geqslant \overline y $.

\underline{Hint}:  One way is to let  $y_{\text{max}}$ be the largest of  
$y_1 , y_2 , y_3 , y_4 $.

\item Prove that there do not exist three consecutive natural numbers such that the cube of the largest is equal to the sum of the cubes of the other two. \label{exer:sec35-10}

\underline{Hint}: Three consecutive natural numbers can be represented by $n$, $n + 1$, and 
$n + 2$, where $n \in \mathbb{N}$, or three consecutive natural numbers can be respresented by 
$m - 1$, $m$, and $m + 1$, where $m \in \mathbb{N}$.

\end{enumerate}
\hbreak
\endinput
