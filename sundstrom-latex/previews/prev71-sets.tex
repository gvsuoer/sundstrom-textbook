\begin{previewactivity}[\textbf{The Solution Set of an Equation with Two Variables}] \label{PA:eqn2variables} \hfill \\
In Section~\ref{S:predicates}, we introduced the concept of the \textbf{truth set of an open sentence with one variable}.
\index{truth set}%
  This was defined to be the set of all elements in the universal set that can be substituted for the variable to make the open sentence a true proposition.  Assume that  $x$  and  $y$  represent real numbers.  Then the equation
\[
4x^2  + y^2  = 16
\]
is an open sentence with two variables.  An element of the truth set of this open sentence (also called a solution of the equation) is an ordered pair  $\left( {a, b} \right)$ of real numbers so that when  $a$  is substituted for  $x$  and  $b$  is substituted for  $y$, the predicate becomes a true statement (a true equation in this case).  We can use set builder notation to describe the truth set $S$ of this equation with two variables as follows:
\[
S = \left\{ (x, y) \in \R \times \R \mid 4x^2 + y^2 = 16 \right\}\!.
\]
When a set is a truth set of an open sentence that is an equation, we also call the set the 
\textbf{solution set}
\index{solution set}%
 of the equation.
\begin{enumerate}
\item List four different elements of the set $S$\!.

\item The graph of the equation  $4x^2  + y^2  = 16$  in the $xy$-coordinate plane is an ellipse.  Draw the graph and explain why this graph is a representation of the truth set (solution set) of the equation $4x^2 + y^2 = 16$.

%\item Write a description of the solution set  $S$  of the equation  $x^2  + y^2  = 25$ using set builder notation.

%\pagebreak
\item Describe each of the following sets as an interval of real numbers:
\label{PA:eqn2variables3}
\begin{enumerate}
\item $A = \left\{ x \in \R \mid \text{ there exists a } y \in \R \text{ such that } 4x^2 + y^2 = 16 \right\}$.

\item $B = \left\{ y \in \R \mid \text{ there exists an } x \in \R \text{ such that } 4x^2 + y^2 = 16 \right\}$.
\end{enumerate}

\end{enumerate}
\end{previewactivity}
\hbreak
%

%\begin{previewactivity}[A Set of Order Pairs] \label{PA:orderedpairs} \hfill \\
%\noindent
%For another exampLet $\R^* = \left\{ y \in \R \mid y \geq 0 \right\}$ and let
%$F = \left\{ (x, y) \in \R \times \R^* \mid y = x^2 \right\}$.
%
%\begin{enumerate}
%\item List five different ordered pairs that are in the set $F$.
%
%\item Use the roster method to specify the elements of each of the following the sets:
%\begin{multicols}{2}
%\begin{enumerate}
%\item $A = \left\{ x \in \R \mid (x, 4) \in F \right\}$
%\item $B = \left\{ x \in \R \mid (x, 10) \in F \right\}$
%\item $C = \left\{ y \in \R^* \mid (5, y) \in F \right\}$
%\item $D = \left\{ y \in \R^* \mid (-3, y) \in F \right\}$
%\end{enumerate}
%\end{multicols}
%
%\item Can the set  $F$  be used to define a function from the set  $\R$  to the set  $\R^*$?  Explain.
%
%\end{enumerate}
%\end{previewactivity}
%\hbreak

\endinput
