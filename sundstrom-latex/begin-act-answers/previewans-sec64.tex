\documentclass[11pt]{article}
\usepackage{../../styles/activity}

\usepackage{xr}
\externaldocument{0-MR}

\lhead{}
%\chead{\textbf{\Large{\hspace{0pt}Beginning Activities for Section~6.4}}\\\hspace{0pt}\emph{Mathematical Reasoning: Writing and Proof}}
\bahead{6.4}
\rhead{}
\lfoot{}
\rfoot{}
\cfoot{\hspace{0pt}\scalebox{0.4}{\includegraphics{cc-by-nc-sa.eps}}}

\begin{document}
\input table

\subsection*{Beginning Activity 1 (Constructing a New Function)}
\begin{tabular}[t]{| c | c | c | p{0.5in}  l} \cline{1-3}
$x$  &  $f\left( x \right)$  &  $g\left( {f\left( x \right)} \right)$ \\ \cline{1-3}
$a$  &  $p$  &  $t$  & & This represents a function from \\ \cline{1-3}
$b$  &  $q$  &  $t$  & & $A$  to  $C$ since each element of  $A$  is\\ \cline{1-3}
$c$  &  $r$  &  $s$  & & associated with exactly one element of  $C$. \\ \cline{1-3}
$d$  &  $r$  &  $s$  & \\ \cline{1-3}
\end{tabular}

\eighth
\noindent
This method of constructing a new function is called the \textbf{composition of} $\boldsymbol{f}$ \textbf{and} $\boldsymbol{g}$ and will be studied in this section.
\hbreak


%\BeginTable
%\BeginFormat
%| l | c |
%\EndFormat
%\_
%| \textbf{Verbal Description}  |  \textbf{Symbolic Result} | \\+22 \_
%| Choose an input.	  |   $x$    |       \\  \hline
%|Square it.	  |  $x^2$   | \\+22 \_
%|Multiply by 3.	  |  $3x^2$   | \\+22 \_
%|Add 2.    |  $3x^2 + 2$  | \\+22 \_
%|Take the square root.  |  $\sqrt {3x^2  + 2} $ | \\+22 \_
%\EndTable

\newpage
\subsection*{Beginning Activity 2 (Verbal Descriptions of Functions)}
\begin{enumerate} \begin{multicols}{2}

\item 
\begin{tabular}[t]{| l | c|} \hline
\textbf{Verbal Description}  &  \textbf{Symbolic Result}  \\ \hline
Choose an input.	  &  $x$           \\  \hline
Square it.	  &  $x^2$          \\  \hline
Multiply by 3.	        &  $3x^2$      \\  \hline
Add 2.    &  $3x^2 + 2$  \\  \hline
Take the square root.  &  $\sqrt {3x^2  + 2} $ \\ \hline
\end{tabular}

\item
\begin{tabular}[t]{| l | c|} \hline
\textbf{Verbal Description}  &  \textbf{Symbolic Result}  \\ \hline
Choose an input.	  &  $x$           \\  \hline
Square it.	  &  $x^2$          \\  \hline
Multiply by 3.	        &  $3x^2$      \\  \hline
Add 2.    &  $3x^2 + 2$  \\  \hline
Take the sine of the result.  &  $\sin \left( {3x^2  + 2} \right) $ \\ \hline
\end{tabular}
\end{multicols}


\item
\begin{tabular}[t]{| l | c|} \hline
\textbf{Verbal Description}  &  \textbf{Symbolic Result}  \\ \hline
Choose an input.	  &  $x$           \\  \hline
Square it.	  &  $x^2$          \\  \hline
Multiply by 3.	        &  $3x^2$      \\  \hline
Add 2.    &  $3x^2 + 2$  \\  \hline
Apply the exponential function.  &  $e^{3x^2  + 2} $\\ \hline
\end{tabular}

\eighth
Notice that the first four steps of each of these functions were identical.  The only difference was in the last step.

\item
\begin{tabular}[t]{| l | c|} \hline
\textbf{Verbal Description}  &  \textbf{Symbolic Result}  \\ \hline
Choose an input.	  &  $x$           \\  \hline
Raise it to the fourth power.	  &  $x^4$          \\  \hline
Add 3.	        &  $x^4 + 3$      \\  \hline
Apply the natural logarithm function.  &  $\ln \left( x^4 + 3 \right)$\\ \hline
\end{tabular}


\item
\begin{tabular}[t]{| l | c|} \hline
\textbf{Verbal Description}  &  \textbf{Symbolic Result}  \\ \hline
Choose an input.	  &  $x$           \\  \hline
Multiply by 4 and add 3.	  &  $4x + 3$          \\  \hline
Apply the sine function.          & $ \sin (4x + 3)$ \\ \hline
Square the input $(x)$ and add 1.	        &  $x^2 + 1$      \\  \hline
Divide $\sin(4x + 3)$ by $\left( x^2 + 1 \right)$.  &  $\dfrac{\sin(4x + 3)}{x^2 + 1} $\\ \hline
Apply the cube root function.   &  $\sqrt[3]{\dfrac{\sin(4x + 3)}{x^2 + 1}} $ \\ \hline
\end{tabular}

\end{enumerate}


\end{document}
