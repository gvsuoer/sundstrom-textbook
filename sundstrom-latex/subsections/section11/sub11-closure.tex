\subsection*{Closure Properties of Number Systems}\label{ss:closure} 
\index{closure properties|(}%
The primary number system used in algebra and calculus is the \textbf{real number system}.
\index{real number system}%
  We usually use the symbol $\mathbb{R}$ 
\label{sym:reals}%
%\index{$\mathbb{R}$}%
 to stand for the set of all real numbers.  The real numbers consist of the rational numbers
\index{rational numbers}%
%\index{$\mathbb{Q}$}%
 and the irrational numbers.
\index{irrational numbers}%
  The \textbf{rational numbers} are those real numbers that can be written as a quotient of two integers (with a nonzero denominator), and the \textbf{irrational numbers} are those real numbers that cannot be written as a quotient of two integers.   That is, a rational number can be written in the form of a fraction, and an irrational number cannot be written in the form of a fraction.  Some common irrational numbers are $\sqrt{2}$, $\pi$, and $e$.  We usually use the symbol $\mathbb{Q}$  to represent the set of all rational numbers.  (The letter $\mathbb{Q}$ 
\label{sym:rationals}%
is used because rational numbers are quotients of integers.)  There is no standard symbol for the set of all irrational numbers.

Perhaps the most basic number system used in mathematics is the set of 
\textbf{natural numbers}.
\index{natural numbers}%
The natural numbers consist of the positive whole numbers such as 1, 2, 3, 107, and 203.  We will use the symbol $\N$ 
\label{sym:natural}%
to stand for the set of natural numbers.  Another basic number system that we will be working with is the set of 
\textbf{integers}.
\index{integers}%
  The integers consist of zero, the natural numbers, and the negatives of the natural numbers.  If  $n$  is an integer, we can write  $n = \dfrac{n}{1}$.  So each integer is a rational number and hence also a real number.

We will use the letter  $\mathbb{Z}$ 
\label{sym:integers}%
%\index{$\mathbb{Z}$}%
 to stand for the set of integers.  (The letter  $\mathbb{Z}$ is from the German word, \emph{Zahlen}, for numbers.)  Three of the basic properties of the integers are that  the set  $\mathbb{Z}$ is \textbf{closed under addition}, 
\index{closed under addition}%
the set  $\mathbb{Z}$ is \textbf{closed under multiplication}, 
\index{closed under multiplication}%
and the set of integers is \textbf{closed under subtraction}.  
\index{closed under subtraction}%
This means that
\begin{itemize}
  \item If  $x$  and  $y$  are integers, then  $x + y$ is an integer;

  \item If  $x$  and  $y$  are integers, then  $x \cdot y$ is an integer; and

  \item If  $x$  and  $y$  are integers, then  $x - y$ is an integer.

\end{itemize}

Notice that these so-called closure properties are defined in terms of conditional statements.  This means that if we can find one instance where the hypothesis is true and the conclusion is false, then the conditional statement is false.
%\hbreak  

\begin{example}[\textbf{Closure}]\label{ex:closure} \hfill 
%Following are two examples dealing with closure: %\hfill
\begin{enumerate}
\item In order for the set of natural numbers to be closed under subtraction, the following conditional statement would have to be true:  If $x$ and $y$ are natural numbers, then $x - y$ is a natural number.  However, since 5 and 8 are natural numbers,  $5 - 8 = -3$, which is not a natural number, this conditional statement is false.  Therefore, the set of natural numbers is not closed under subtraction.


\item We can use the rules for multiplying fractions and the closure rules for the integers to show that the rational numbers are closed under multiplication.  If $\dfrac{a}{b}$ and $\dfrac{c}{d}$ are rational numbers (so $a$, $b$, $c$, and $d$ are integers and $b$ and $d$ are not zero), then
\[
\frac{a}{b} \cdot \frac{c}{d} = \frac{ac}{bd}.
\]
Since the integers are closed under multiplication, we know that $ac$ and $bd$ are integers and since $b \ne 0$ and $d \ne 0$, $bd \ne 0$.  So $\dfrac{ac}{bd}$ is a rational number and this shows that the rational numbers are closed under multiplication.
\end{enumerate}
\end{example}
\hbreak

%\subsection*{Test Your Understanding}
%\begin{test} \hfill \label{test:closure}
\begin{prog}[\textbf{Closure Properties}]\label{pr:closure} \hfill \\
Answer each of the following questions.
\begin{enumerate}
\item Is the set of rational numbers closed under addition?  Explain.

\item Is the set of integers closed under division?  Explain.

\item Is the set of rational numbers closed under subtraction?  Explain.

%\item Is the set of rational numbers closed under division?  Explain.

%\item Is the set of nonzero rational numbers closed under division?  Explain.
\end{enumerate}
\end{prog}
\index{closure properties|)}%
\hbreak
\newpage

\endinput
