\begin{previewactivity}[Functions from Previous Courses] \label{PA:previousfunctions} \hfill \\
A \textbf{function}
\index{function}%
 can be thought of as a procedure for associating with each element of some set, called the \textbf{domain of the function},
\index{domain!of a function}%
\index{function!domain}%
 exactly one element of another set, called the \textbf{codomain of the function}.
\index{codomain}%
\index{function!codomain}%
  This is procedure can be considered an input-output-rule.  The function takes the input, which is an element of the domain, and produces an output, which is an element of the codomain.  So the notation $f( x ) = x^2 \sin x$ means the following:

\begin{itemize}
\item $f$  is the name of the function.

\item $f( x )$  is a real number.  It is the output of the function when the input is the real number  $x$.  For example,
\[
\begin{aligned}
  f\left( {\frac{\pi }{2}} \right) &= \left( {\frac{\pi }{2}} \right)^2 \sin \left( {\frac{\pi }
{2}} \right) \\ 
                                   &= \frac{{\pi ^2 }}{4} \cdot 1 \\ 
                                   &= \frac{{\pi ^2 }}{4}. \\ 
\end{aligned}
\]
\end{itemize}
For this function, it is understood that the domain of the function is the set  $\R$ of all real numbers.  In this situation, we think of the domain as the set of all possible inputs.  That is, the domain is the set of all possible real numbers  $x$  for which a real number output can be determined.

In this situation, we often consider the function  $f$  to be a rule that assigns to each real number input  $x$  a unique output  $f( x )$.  This is closely related to the equation  
$y = x^2 \sin x $.  With this equation, we frequently think of  $x$  as the input and  $y$  as the output.  In fact, we sometimes write  $y = f( x )$.

Which of the following equations can be used to define a function with  $x \in \mathbb{R}$
as the input and  $y$  as the output?

\begin{multicols}{2}
\begin{enumerate}

\item $y = x^2  - 2$

\item $y^2  = x + 3$

\item $y = \dfrac{1}{2}x^3  - 1$

\item $y = \dfrac{1}{2} x\sin x$

\item $x^2  + y^2  = 4$

\item $y = 2x - 1$

\item $y = \dfrac{x}{{x - 1}}$

\end{enumerate}
\end{multicols}
\end{previewactivity}
\hbreak
\begin{previewactivity}[The Birthday Function] \label{PA:birthdayfunction} \hfill \\
\index{birthday function}%
The domain and codomain of the functions in Preview Activity~\ref{PA:previousfunctions} is the set  $\R$ of all real numbers, or some subset of   $\R$.  In most of these cases, the way in which the function associates elements of the domain with elements of the codomain is by a rule determined by some mathematical expression.  For example, when we say that $f$  is the function such that
\[
f( x ) = \dfrac{x}{{x - 1}},
\]
then the algebraic rule that determines the output of the function  $f$  when the input is  $x$  is  $\dfrac{x}{{x - 1}}$.  In this case, we would say that the domain of  $f$  is the set of all real numbers not equal to  1 since division by zero is not defined.

However, the concept of a function is much more general than this.  The way in which a function associates elements of the domain with elements of the codomain can have many different forms.  This input-output rule can be a formula, a graph, a table, a random process, a computer algorithm, or a verbal description.  Following is such an example.

Let  $b$  be the function that assigns to each person his or her birthday (month and day).  The domain of the function  $b$  is the set of all people and the codomain of  $b$  is the set of all days in a leap year (i.e., January 1 through December 31, including February 29).

\begin{enumerate}
\item Explain why  $b$  really is a function.

\item In 1995, Andrew Wiles
\index{Wiles, Andrew}%
 became famous for publishing a proof of Fermat's Last Theorem.  (See A. D. Aczel, \textit{Fermat's Last Theorem: Unlocking the Secret of an Ancient Mathematical Problem}, Dell Publishing, New York, 1996.)  Andrew Wiles's birthday is April 11, 1953.  Translate this fact into functional notation using the ``birthday function'' $b$.  That is, fill in the spaces for the following question marks:
\[
b( {\,?\,} ) = \,?.
\]
\item Is the following statement true or false?  Explain.

\begin{list}{}
\item For each day  $D$  of the year, there exists a person  $x$  such that  
\linebreak $b( x ) = D$.
\end{list}

\item Is the following statement true or false?  Explain.

\begin{list}{}
\item For any people  $x$  and  $y$,  if  $x$  and  $y$  are different people, then  
\linebreak $b( x ) \ne b( y )$.
\end{list}

\end{enumerate}
\end{previewactivity}
\hbreak
%
\begin{previewactivity}[The Sum of the Divisors Function] \label{PA:sumofdivisors} \hfill \\
\index{sum of divisors function}%
Let  $s$  \label{sym:sumdivisors} be the function that associates with each natural number the sum of its distinct natural number factors.  For example,
\[
\begin{aligned}
  s( 6 ) &= 1 + 2 + 3 + 6 \\ 
                    &= 12. \\ 
\end{aligned} 
\]
\begin{enumerate}
\item Calculate  $s( k )$ for each natural number  $k$  from  1  through 15.

\item Are the numbers $\sqrt{5}$, $\pi$, and $-6$ in the domain of the function $s$?  What is the domain of the function  $s$?

%\item Is  $s\left( {\sqrt 5 } \right)$  defined?  Explain.  Is  $s\left( \pi  \right)$  defined?  Is  $s\left( { - 6} \right)$  defined?

\item Does there exist a natural number  $n$  such that  $s( n ) = 5$?  Justify your conclusion.

\item Is it possible to find two different natural numbers  $m$  and  $n$  such that  \linebreak
$s( m ) = s( n )$?  Explain.

\item Are the following statements true or false?

\begin{enumerate}
  \item For each  $m \in \mathbb{N}$, there exists a natural number  $n$  such that  
$s( n ) = m$.

  \item For all $m, n \in \mathbb{N}$, if  $m \ne n$, then  $s( m ) \ne s( n )$.
\end{enumerate}

\end{enumerate}
\end{previewactivity}
\hbreak


\endinput
