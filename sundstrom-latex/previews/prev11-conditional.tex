%
%
\begin{previewactivity}[\textbf{Conditional Statements}]\label{PA:conditional} \hfill \\
Given statements $P$ and $Q$, a statement of the form ``If $P$ then $Q$'' is called a 
\textbf{conditional statement}.
\index{conditional statement}%
\index{statement!conditional}%
 It seems reasonable that the truth value (true or false) of the conditional statement 
``If $P$ then $Q$'' depends on the truth values of $P$  and  $Q$.  The statement ``If $P$ then $Q$'' means that $Q$  must be true whenever $P$ is true.  The statement $P$ is called the \textbf{hypothesis}
\index{conditional statement!hypothesis}%
 of the conditional statement, and the statement $Q$ is called the \textbf{conclusion}
\index{conditional statement!conclusion}%
 of the conditional statement.  We will now explore some examples.

\begin{enumerate}
\item ``If it is raining, then Laura is at the theater.''
Under what conditions is this conditional statement false?  For example,
\begin{enumerate}
\item Is it false if it is raining and Laura is at the theater?
\item Is it false if it is raining and Laura is not at the theater?
\item Is it false if it is not raining and Laura is at the theater?
\item Is it false if it is not raining and Laura is not at the theater?
\end{enumerate}

\item Identify the hypothesis and the conclusion for each of the following conditional statements. 
  \begin{enumerate}
%  \item If  $n$  is a prime number, then  $n^2$ has three positive factors.
%  \item If  $a$  is an irrational number and  $b$  is an irrational number, then  $a \cdot b$ is an irrational number.
% \item If  $p$  is a prime number, then  $p = 2$ or   $p$ is an odd number.
%  \item If $p$ is a prime number and $p \ne 2$, then $p$ is an odd number.
%  \item If $p \ne 2$ and $p$ is an even number, then $p$ is not prime.
\item If $x$ is a positive real number, then $\sqrt{x}$ is a positive real number.
\item If $\sqrt{x}$ is not a real number, then $x$ is a negative real number.
\item If the lengths of the diagonals of a parallelogram are equal, then the parallelogram is a rectangle.
  \end{enumerate}

\end{enumerate}
\end{previewactivity}
\hbreak
\endinput
