\subsection*{Countably Infinite Sets}

\begin{theorem}\label{T:addonetocountable}
If $A$ is a countably infinite set, then $A \cup \left\{ x \right\}$ is a countably infinite set.
\end{theorem}
%
\begin{myproof}
Let $A$ be a  countably infinite set.  Then there exists a bijection $f\x \mathbb{N} \to A$.  Since $x$ is either in $A$ or not in $A$, we can consider two cases.

\newpar
If $x \in A$, then $A \cup \left\{ x \right\} = A$ and $A \cup \left\{ x \right\}$ is countably infinite.

\newpar
If $x \notin A$, define $g\x \mathbb{N} \to A \cup \left\{ x \right\}$ by
\begin{equation} \notag
g( n ) = 
\begin{cases}
x                        &\text{if $n = 1$} \\
f ( n - 1 )   &\text{if $n > 1$.}
\end{cases}
\end{equation}
The proof that the function $g$ is a bijection is Exercise~(\ref{exer:addonetocountable}).  Since $g$ is a bijection, we have proved that $A \cup \left\{ x \right\} \approx \N$ and hence, $A \cup \left\{ x \right\}$ is countably infinite.
\end{myproof}
%
\begin{theorem}\label{T:addfinitetocountable}
If $A$ is a countably infinite set and $B$ is a finite set, then $A \cup B$ is a countably infinite set.
\end{theorem}
%
\begin{myproof}
Exercise~(\ref{exer:addfinitetocountable}) on page~\pageref{exer:addfinitetocountable}.
\end{myproof}
%

Theorem~\ref{T:addfinitetocountable} says that if we add a finite number of elements to a countably infinite set, the resulting set is still countably infinite.  In other words, the cardinality of the new set is the same as the cardinality of the original set.  Finite sets behave very differently in the sense that if we add elements to a finite set, we will change the cardinality.  What may even be more surprising  is the result in Theorem~\ref{T:unionofcountable} that states that the union of two countably infinite (disjoint) sets is countably infinite.  The proof of this result is similar to the proof that the integers are countably infinite  (Theorem~\ref{T:ZequivtoN}).  In fact, if $A = \{a_1, a_2, a_3, \ldots \}$ and $B = \{b_1, b_2, b_3, \ldots \}$, then we can use the following diagram to help define a bijection from $\N$ to $A \cup B$.


\begin{figure}[h]
$$
\BeginTable
\BeginFormat
| c | c | c | c | c | c | c | c | c | c | c |
\EndFormat
" 1 " 2 " 3 " 4 " 5 " 6 " 7 " 8 " 9 " 10 " $\cdots$ " \\
" $\downarrow$ " $\downarrow$ " $\downarrow$ " $\downarrow$ " $\downarrow$ " $\downarrow$ " $\downarrow$ " $\downarrow$ " $\downarrow$ " $\downarrow$ " $\cdots$ "\\
" $a_1$ " $b_1$ " $a_2$ " $b_2$ " $a_3$ " $b_3$ " $a_4$ " $b_4$ " $a_5$ " $b_5$ "  $\cdots$ " \\
\EndTable
$$
\caption{A Function from $\N$ to $A \cup B$} \label{fig:functionNtoUnion}
\end{figure}


\begin{theorem}\label{T:unionofcountable}
If $A$ and $B$ are disjoint countably infinite sets,
\index{countably infinite sets!union of}%
 then $A \cup B$ is a countably infinite set.
\end{theorem}
%
\begin{myproof}
Let $A$ and $B$ be countably infinite sets and let $f\x \mathbb{N} \to A$ and 
$g\x \mathbb{N} \to B$ be bijections.  Define $h\x \mathbb{N} \to A \cup B$ by
\begin{equation} \notag
h( n ) = 
\begin{cases}
f \!\left( \dfrac{n+1}{2} \right)                        &\text{if $n$ is odd} \\
                                                       & \\
g \!\left( \dfrac{n}{2} \right)                          &\text{if $n$ is even}.
\end{cases}
\end{equation}
It is left as Exercise~(\ref{exer:unionofcountable}) on 
page~\pageref{exer:unionofcountable} to prove that the function $h$ is a bijection.
\end{myproof}

Since we can write the set of rational numbers $\Q$ as the union of the set of nonnegative rational numbers and the set of negative rational numbers, we can use the results in Theorem~\ref{T:positiverationals}, Theorem~\ref{T:addonetocountable}, and Theorem~\ref{T:unionofcountable} to prove the following theorem.

\begin{theorem}\label{T:Qiscountable}
The set $\mathbb{Q}$ of all rational numbers is countably infinite.
\end{theorem}
%
\begin{myproof}
Exercise~(\ref{exer:Qiscountable}) on page~\pageref{exer:Qiscountable}.
\end{myproof}
%\hbreak
%
\index{countably infinite sets!subsets of|(}%
In Section~\ref{S:finitesets}, we proved that any subset of a finite set is finite 
(Theorem~\ref{T:finitesubsets}).  A similar result should be expected for countable sets. We first prove that every subset of $\mathbb{N}$ is countable.  For an infinite subset $B$ of 
$\mathbb{N}$, the idea of the proof is to define a function $g\x  \mathbb{N} \to B$ by removing the elements from $B$ from smallest to the next smallest to the next smallest, and so on.  
We do this by defining the function $g$ recursively as follows:
\begin{itemize}
\item Let $g ( 1 )$ be the smallest natural number in $B$.
\item Remove $g ( 1 )$ from $B$ and let $g ( 2 )$ be the smallest natural number in \linebreak
$B - \left\{ g ( 1 ) \right\}$.
\item Remove $g ( 2 )$ and let $g ( 3 )$ be the smallest natural number in 
\linebreak
$B - \left\{ g ( 1 ), g ( 2 ) \right\}$.
\item We continue this process.  The formal recursive definition of $g\x  \mathbb{N} \to B$ is included in the proof of Theorem~\ref{T:subsetsofN}.
\end{itemize}
%
\begin{theorem}\label{T:subsetsofN}
Every subset of the natural numbers is countable.
\end{theorem}
%
\begin{myproof}
Let $B$ be a subset of $\mathbb{N}$.  If $B$ is finite, then $B$ is countable.  So we next assume that $B$ is infinite.  We will next give a recursive definition of a function 
$g\x  \mathbb{N} \to B$ and then prove that $g$ is a bijection.
\begin{itemize}
\item Let $g ( 1 )$ be the smallest natural number in $B$.
\item For each $n \in \mathbb{N}$, the set 
$B - \left\{ g ( 1 ), g ( 2 ), \ldots, g ( n ) \right\}$ is not empty since $B$ is infinite.  Define $g ( n + 1 )$ to be the smallest natural number in 
\linebreak
$B - \left\{ g ( 1 ), g ( 2 ), \ldots, g ( n ) \right\}$.
\end{itemize}
The proof that the function $g$ is a bijection is Exercise~(\ref{exer:subsetofN}) on page~\pageref{exer:subsetofN}.  
\end{myproof}
%
\begin{corollary}\label{C:subsetofcountable}
Every subset of a countable set is countable.
\end{corollary}
%
\begin{myproof}
Exercise~(\ref{exer:subsetofcountable}) on page~\pageref{exer:subsetofcountable}.
\end{myproof}
\index{countably infinite sets!subsets of|)}%

\hbreak

\endinput
