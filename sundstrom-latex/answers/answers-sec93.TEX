\subsection*{Section~\ref{S:uncountablesets}}

\begin{list}{\bf{\ref{exer:sec94-1}.}}
\item \begin{list}{\bf{(a)}}
\item One such bijection is $f\x ( 0, \infty ) \to \mathbb{R}$ by $f ( x ) = \ln x$ for all 
$x \in ( 0, \infty )$
\end{list}
\end{list}

\begin{list}{}
\item \begin{list}{\bf{(b)}}
\item One such bijection is $g\x ( 0, \infty ) \to ( a, \infty )$ by $g( x ) = x + a$ for all $x \in ( 0, \infty )$.  The function $g$ is a bijection and so 
$\left( 0, \infty \right) \approx \left( a, \infty \right)$.  Then use Part~(a).
%\item Find a bijection from $\left( 0, \infty \right)$ to  $\left( a, \infty \right)$.
\end{list}
\end{list}


\begin{list}{\bf{\ref{exer:irrationaluncount}.}}
\item Use a proof by contradiction.  Let $\mathbb{H}$ be the set of irrational numbers and assume that $\mathbb{H}$ is countable.  Then 
$\mathbb{R} = \mathbb{Q} \cup \mathbb{H}$ and $\mathbb{Q}$ and $\mathbb{H}$ are disjoint.  Use Theorem~\ref{T:unionofcountable}, to obtain a contradiction.
\end{list}


\begin{list}{\bf{\ref{exer:supersetofuncount}.}}
\item By Corollary~\ref{C:subsetofcountable}, every subset of a countable set is countable.  So if 
$B$ is countable, then $A$ is countable. %Use Corollary~\ref{C:subsetofcountable} on page~\pageref{C:subsetofcountable}.
\end{list}


\begin{list}{\bf{\ref{exer93:uncountablesets}.}}
\item By Cantor's Theorem (Theorem~\ref{T:cantor}), $\mathbb{R}$ and 
$\mathcal{P} \left( \mathbb{R} \right)$ do not have the same cardinality.
\end{list}

\hbreak
\endinput
