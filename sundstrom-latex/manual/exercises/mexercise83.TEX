\section*{Section \ref{S:diophantine} Linear Diophantine Equations}

\begin{enumerate}
\item Prove the contrapositive.  If the linear Diophantine equation $ax + by = c$ has a solution, then there exist integers $m$ and $n$ such that $am + bn = c$.  This means that $c$ is a linear combination of $a$ and $b$ and hence by Theorem~\ref{T:gcddivideslincombs}, $d \mid c$.


\item If $a$ and $b$ are relatively prime, then $d = \gcd \left( a, b \right) = 1$.  Hence, $d \mid c$.  Now use Theorem~\ref{T:lindioph2}.



\item \begin{enumerate}
\item $x = -3 + 14k$, \quad  $y = 2 - 9k$

\item $x = -1 + 11k$, \quad  $y = 1 - 9k$

\item No solution

\item $x = 2+3k$, \quad  $y = -2-4k$

\item $x = -120 + 49k, y = 490 - 200k$

\item No solution

\item $x = 1 - 7k, y = -3 - 10k$

\item $x = 2 + 3k, y = -1 - 2k$
\end{enumerate}


\item There are several possible solutions to this problem.  Each solution can be generated from the solutions of the Diophantine equation $27x + 50y = 25$.  For example, one form of the general solution for this equation is
\[
x = 25 + 50k, y = -13 - 27k.
\]
If we use the solution $x = 25$ and $y = -13$, we see that if 25 of the 27 gram weights are put on one side of the balance and the artifact and 13 of the  50 gram weights are put on the other side of the balance, then the scale should balance.

Another way is to use the solution $x = -25$ and $y = 14$.  If we put the artifact and 25 of the 27 gram weights on one side and 14 of the 50 gram weights on the other side.

\underline{Note}:  If the Euclidean Algorithm is used, we get $27 \cdot 13 + 50 \left( -7 \right) = 1$.  Then, 
$27 \left( 325 \right) + 50 \left( -175 \right) = 25$ and so the general solution of the linear Diophantine equation is $x = 325 + 50k$ and $y = -175 - 27k$.  Using $k = -6$ gives $x = 25$ and 
$y = -13$.

\item This problem can be solved using solutions of the linear Diophantine equation 
$25x + 16y = 1461$.  From the Euclidean Algorithm, we obtain
\[
\begin{aligned}
25 \left( -7 \right) + 16 \cdot 11 &= 1 \\
25 \left( -10227 \right) + 16 \left( 16071 \right) &= 1461. 
\end{aligned}
\]
The general solution of the linear Diophantine equation is
\[
\begin{aligned}
x &= -10227 + 16k \\
y &= 16071 - 25k. \\
\end{aligned}
\]
For this problem, we need $x \geq 0$ and $y \geq 0$.  The first inequality implies that 
$k > 639$, and the second inequality implies that $k < 643$.

\begin{multicols}{2}
When $k = 640$, $x = 13$ and $y = 71$.

When $k = 641$, $x = 29$ and $y = 46$.

When $k = 642$, $x = 45$ and $y = 21$.
\end{multicols}

These are the only possible solutions with $x \geq 0$ and $y \geq 0$.  So either 66, 75, or 84 people attended.

\item \begin{enumerate}
\item $y = 12 + 16k, x_3 = -1 - 3k$

\item If $3y = 12x_1 + 9x_2$ and $3y + 16x_3 = 20$, we can substitute for $3y$ and obtain 
$12x_1 + 9x_2 + 16x_3 = 20$.

\item Rewrite the equation $12x_1 + 9x_2 = 3y$ as $4x_1 + 3x_2 = y$.  A general solution for this linear Diophantine equation is
\[
\begin{aligned}
x_1 &= y + 3n \\
x_2 &= -y - 4n. \\
\end{aligned}
\]
\item $x_1 = 12 + 16k + 3n$, $x_2 = -12 - 16k - 4n$, $x_3 = -1 - 3k$.

\item \[
\begin{aligned}
12x_1 + 9x_2 + 16x_3 &= 12 \left( 12 + 16k + 3n \right) + 9 \left( -12 -16k - 4n \right) \\
                     &\qquad + 16 \left( -1 - 3k \right) \\
                     &= \left( 144 + 192k + 36n \right) + \left( -108 - 144k - 36 n \right) \\
                     &\qquad + \left( -16 - 48k \right) \\
                     &= 20.
\end{aligned}
\]
\end{enumerate}

\item First solve the Diophantine equation $4y - 6x_3 = 6$.
\[
\begin{aligned}
y &= 0 - 3k \\
x_3 &= -1 - 2k. \\
\end{aligned}
\]
Next, solve $8x_1 + 4x_2 = 4y$.
\[
\begin{aligned}
x_1 &= y + n \\
x_2 &= =y -2n. \\
\end{aligned}
\]
So the solutions of $8x_1 + 4x_2 - 6x_3 = 6$ can be written as
\[
\begin{aligned}
x_1 &= 0 - 3k + n \\
x_2 &= 0 + 3k - 2n \\
x_3 &= -1 - 2k, \\
\end{aligned}
\]
where $k$ and $n$ are integers.

\item The Diophantine equation $24x_1 - 18x_2 + 60x_3 = 21$ has no solution since the left side of the equation is a multiple of 6 for all integers $x_1$, $x_2$, and $x_3$.

\item \begin{enumerate}
\item If two integers are equal, then they are congruent modulo 3.

\item For each integer $x$, $3x^2 \equiv \pmod 3$.  Therefore, if 
\[
3x^2 - y^2 \equiv -2 \pmod 3, 
\]
then $-y^2 \equiv -2 \pmod 3$ and hence $y^2 \equiv 2 \pmod 3$.

\item If there is a solution to the Diophantine equation $3x^2 - y^2 = -2$, then by Parts~(a) and~(b), we see that $y^2 \equiv 2 \pmod 3$.  This is a contradiction to the fact that for all integers $y$, $y^2 \not \equiv 2 \pmod 3$.
\end{enumerate}

\item Use congruence modulo 7.  If the Diophantine equation $7x^2 + 2 = y^3$ has a solution, then there exists an integer $y$ such that $y^3 \equiv 2 \pmod 7$.  Verify that
\begin{multicols}{3}
$0^3 \equiv 0 \pmod 7$

$1^3 \equiv 1 \pmod 7$

$2^3 \equiv 1 \pmod 7$

$3^3 \equiv 6 \pmod 7$

$4^3 \equiv 1 \pmod 7$

$5^3 \equiv 6 \pmod 7$

$6^3 \equiv 6 \pmod 7$
\end{multicols}
\end{enumerate}


\subsection*{Explorations and Activities}
\setcounter{oldenumi}{\theenumi}
\begin{enumerate} \setcounter{enumi}{\theoldenumi}
\item \begin{enumerate}
  \item The following table shows that $x = 2$ and  $x = 5$ are the only solutions of the congruence \\ $4x \equiv 2 \pmod 6$ with $0 \leq x < 6$.

\begin{center}
\begin{tabular}[t]{ c | c  c  c | c } 
$x$  &  $4x \pmod 6$ & & $x$  &  $4x \pmod 6$ \\ \cline{1-2} \cline{4-5}
0  &  0  &  &  3  &  0 \\ \cline{1-2} \cline{4-5}
1  &  4  &  &  4  &  4 \\ \cline{1-2} \cline{4-5}
2  &  2  &  &  5  &  2 \\ \cline{1-2} \cline{4-5}
\end{tabular}
\end{center}


  \item The table in Part (a) shows that the congruence $4x \equiv 3 \pmod 6$ has no solution $x$ with $0 \leq x < 6$.



\item The following table shows that $x = 5$  is the only solution of the congruence \\ $3x \equiv 7 \pmod 8$ with $0 \leq x < 8$.

\begin{center}
\begin{tabular}[t]{ c | c  c  c | c } 
$x$  &  $3x \pmod 8$ & & $x$  &  $3x \pmod 8$ \\ \cline{1-2} \cline{4-5}
0  &  0  &  &  4  &  4 \\ \cline{1-2} \cline{4-5}
1  &  3  &  &  5  &  7 \\ \cline{1-2} \cline{4-5}
2  &  6  &  &  6  &  2 \\ \cline{1-2} \cline{4-5}
3  &  1  &  &  7  &  5 \\ \cline{1-2} \cline{4-5}
\end{tabular}
\end{center}


\item The following table shows that $x = 2$ and  $x = 6$ are the only solutions of the congruence \\ $6x \equiv 4 \pmod 8$ with $0 \leq x < 8$.

\begin{center}
\begin{tabular}[t]{ c | c  c  c | c } 
$x$  &  $6x \pmod 8$ & & $x$  &  $6x \pmod 8$ \\ \cline{1-2} \cline{4-5}
0  &  0  &  &  4  &  0 \\ \cline{1-2} \cline{4-5}
1  &  6  &  &  5  &  6 \\ \cline{1-2} \cline{4-5}
2  &  4  &  &  6  &  4 \\ \cline{1-2} \cline{4-5}
3  &  2  &  &  7  &  2 \\ \cline{1-2} \cline{4-5}
\end{tabular}
\end{center}


\item $6x \equiv 4 \pmod 8$ if and only if  $8 \mid \left( {6x-4} \right)$.

\item $8 \mid \left( {6x-4} \right)$ if and only if there exists an integer $m$ such that 
$6x - 4 = 8m$.

\item Using Parts~(e) and~(f), we see that $6x \equiv 4 \pmod 8$ if and only if there exists an integer $k$ such that $6x - 4 = 8m$.  We rewrite this equation in the form of a linear Diophantine equation in two variables as follows:
\[
6x - 8m = 4.
\]
We notice that $x = 2, m = 1$ is a solution of this equation.  If we let 
$d = \gcd \left( {6,-8} \right) = 2$, we can use Theorem~8.22 to solve this equation.  The solution is
\begin{align}
x &= 2 + \frac{-8}{2} k & y &= 1 - \frac{6}{2} k \notag \\
x &= 2 -4k              & y &= 1 - 3k     \notag  
\end{align}
where $k$ is an integer.

This means that all solutions of the congruence $6x \equiv 4 \pmod 8$ can be written in the form 
$x = 2 - 4k$, where $k$ is an integer.

\item $ax \equiv c \pmod n$ if and only if  $n \mid \left( {ax-c} \right)$.

\item $n \mid \left( {ax-c} \right)$ if and only if there exists an integer $m$ such that 
$ax - c = nm$.
\end{enumerate}

We now see that $ax \equiv c \pmod n$ if and only if there exists an integer $m$ such that 
$ax - c = nm$.  We write this last equation in the form of a linear Diophantine equation in two variables as follows:
\[
ax - nm = c.
\]
By letting $d = \gcd \left( {a,n} \right) = \gcd \left( {a,-n} \right)$, we can use Theorem~\ref{T:lindioph2} to obtain the following theorems.

\begin{enumerate} \setcounter{enumii}{9}
\item Let $n$ be a natural, let $a$ and $c$ be integers with $a \ne 0$, and let 
$d = \gcd \left( {a, n} \right)$.  If  $d$ does  not divide $c$, then the linear congruence 
$ax \equiv c \pmod n$ has no solution.

\eighth
For the proof, we can say that $ax \equiv c \pmod n$ if and only if there exists an integer $m$ such that
\[
ax - nm = c.
\]
Theorem~8.25 states that if $d$ does not divide $c$, then this equation has no solution, and hence, the congruence $ax \equiv c \pmod n$ has no solution.

\item Let $n$ be a natural number, let $a$ and $c$ be integers with $a \ne 0$, and let 
$d = \gcd \left( {a, n} \right)$.  If  $d$ divides $c$, then the linear congruence 
$ax \equiv c \pmod n$ has infinitely many solutions.  In addition, if $x_0$ is a particular solution of this congruence, then all the solutions of the congruence are given by
\[
x = x_0 - \frac{n}{d} k
\]
where $k \in \mathbb{Z}$.

\eighth
Again, we can say that $ax \equiv c \pmod n$ if and only if there exists an integer $m$ such that
\[
ax - nm = c.
\]
We can then use Theorem~\ref{T:lindioph2} to conclude that this equation (and corresponding congruence) has infinitely many solutions, and if $x_0$ is a particular solution of this congruence, then any solution can be written in the form 
\begin{align*}
x &= x_0 + \frac{-n}{d} k \quad \text{or} \\
x &= x_0 - \frac{n}{d} k
\end{align*}
where $k \in \mathbb{Z}$.



\end{enumerate}

\end{enumerate}

\hbreak

\endinput
