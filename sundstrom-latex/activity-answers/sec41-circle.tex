\documentclass[11pt]{article}
\usepackage{c://pctex/activity}

\lhead{}
\chead{\textbf{\large{Exercise 20 -- Section 4.1\\Regions of a Circle}}}
\rhead{}
\lfoot{\emph{Mathematical Reasoning: Writing and Proof, Third Ed.} \\Ted Sundstrom}
\cfoot{}
\rfoot{\copyright \the\year\, by Pearson Education, Inc.\\}


\begin{document}
\setcounter{equation}{0}
\begin{enumerate}
\item There are 8 regions when there are 4 equally spaced points on the circle.

\item The pattern seems to indicate that when a point is added, the number of regions doubles.  If this pattern continues, there would be 16 regions when there are 5 equally spaced points on the circle.

\item If the pattern continues, there would be 32 regions when there are 6 equally spaced points on the circle.

\item There are 16 regions when there are 5 equally spaced points on the circle, and there are 30 
regions when there are 6 equally spaced points on the circle.

\item This activity is intended to show that we cannot assume that a pattern continues, we must prove that the pattern continues.  One way to do this is to use mathematical induction, and this activity shows that the inductive step is a necessary part of a proof by induction.

\end{enumerate}

\end{document}
