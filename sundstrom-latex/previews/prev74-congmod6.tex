\begin{previewactivity}[\textbf{Congruence Modulo 6}] \label{PA:congruencemod6} \hfill \\
For this activity, we will only use the relation of congruence modulo 6 on the set of integers.
\begin{enumerate}
\item Find five different integers  $a$  such that  $a \equiv 3 \pmod 6$ and find five different integers  $b$  such that  $b \equiv 4 \pmod 6$.  That is, find five different integers in $[3]$, the congruence class of  3 modulo 6  and five different integers in $[4]$, the congruence class of 4 modulo 6.  
\label{PA:congruencemod6-1}

%\item Find  five different integers  $b$  such that  $b \equiv 3 \pmod 6$. 
%\label{PA:congruencemod6-2}

\item Calculate  $s = a + b$ using several values of  $a$ in $[3]$ and several values of  $b$ in $[4]$ from Part~(\ref{PA:congruencemod6-1}).  For each sum $s$ that is calculated, find  $r$  so that  
$0 \leq r < 6$ and  $s \equiv r \pmod 6$.  What do you observe?

\item Calculate  $p = a \cdot b$ using several values of  $a$ in $[3]$ 
and several values of  $b$  in $[4]$ from Part~(\ref{PA:congruencemod6-1}).  For each product $p$ that is calculated, find  $r$  so that  $0 \leq r < 6$ and  $p \equiv r \pmod 6$.  What do you observe?

\item Calculate  $q = a^2$ using several values of  $a$ in $[3]$ from Part~(\ref{PA:congruencemod6-1}).  For each product $q$ that is calculated, find  $r$  so that  $0 \leq r < 6$ and  $q \equiv r \pmod 6$.  What do you observe?

\end{enumerate}
\end{previewactivity}
\hbreak

\endinput
