\section*{Section \ref{S:indexfamily}}
\begin{multicols}{2}
\begin{list}{\bf{\ref{exer:sec45-example1}.}}
\item \begin{list}{\bf{(a)}}
\item $\left\{ 3, 4 \right\}$ 
\end{list}
\end{list}

\begin{list}{}
\item \begin{list}{\bf{(d)}}
\item $\left\{ 3, 4, 5, 6, 7, 8, 9, 10 \right\}$
\end{list}
\end{list}
\end{multicols}



\begin{multicols}{2}
\begin{list}{\bf{\ref{exer:sec45-example2}.}}
\item \begin{list}{\bf{(a)}}
\item $\left\{ 5, 6, 7, \ldots \,\right\}$
\end{list}
\end{list}

\begin{list}{}
\item \begin{list}{\bf{(c)}}
\item $\emptyset$
\end{list}
\end{list}

\begin{list}{}
\item \begin{list}{\bf{(d)}}
\item $\left\{1, 2, 3, 4 \right\}$
\end{list}
\end{list}

\begin{list}{}
\item \begin{list}{\bf{(f)}}
\item $\emptyset$
\end{list}
\end{list}
\end{multicols}


\begin{multicols}{2}
\begin{list}{\bf{\ref{exer:indexpositivereals}.}}
\item \begin{list}{\bf{(a)}}
\item $\left\{ x \in \R \left| -100 \leq x \leq 100 \right. \right\}$
\end{list}
\end{list}

\begin{list}{}
\item \begin{list}{\bf{(b)}}
\item $\left\{ x \in \R \left| -1 \leq x \leq 1 \right. \right\}$
\end{list}
\end{list}
\end{multicols}



\begin{list}{\bf{\ref{exer:indexproperties}.}}
\item \begin{list}{\bf{(a)}}
\item We let $\beta \in \Lambda$ and let $x \in A_\beta$.  Then $x \in A_\alpha$, for at least one 
$\alpha \in \Lambda$ and, hence, 
$x \in \bigcup\limits_{\alpha \in \Lambda}^{}A_\alpha$.  This proves that 
$A_\beta \subseteq \bigcup\limits_{\alpha \in \Lambda}^{}A_\alpha$.
\end{list}
\end{list}


\begin{list}{\bf{\ref{exer:distributeindex}.}}
\item \begin{list}{\bf{(a)}}
\item We first let 
$x \in B \cap \left(\:\bigcup\limits_{\alpha \in \Lambda}^{}A_{\alpha} \right)$.  Then $x \in B$ and $x \in \bigcup\limits_{\alpha \in \Lambda}^{}A_{\alpha} $.  This means that there exists an $\alpha \in \Lambda$ such that $x \in A_\alpha$.  Hence, 
$x \in B \cap A_\alpha$, which implies that 
$x \in \bigcup\limits_{\alpha \in \Lambda}^{} \left( B \cap A_{\alpha} \right)$.  This proves that 
$B \cap \left(\bigcup\limits_{\alpha \in \Lambda}^{}A_{\alpha} \right) 
\subseteq \bigcup\limits_{\alpha \in \Lambda}^{} \left( B \cap A_{\alpha} \right)$.
 
We now let 
$y \in \bigcup\limits_{\alpha \in \Lambda}^{} \left( B \cap A_{\alpha} \right)$.  So there exists an $\alpha \in \Lambda$ such that $y \in B \cap A_{\alpha}$.  Then $y \in B$ and 
$y \in A_{\alpha}$, which implies that $y \in B$ and  
$y \in \bigcup\limits_{\alpha \in \Lambda}^{}A_{\alpha}$.  Therefore, 
$y \in B \cap \left(\:\bigcup\limits_{\alpha \in \Lambda}^{}A_{\alpha} \right)$, and this proves that 
$\bigcup\limits_{\alpha \in \Lambda}^{} \left( B \cap A_{\alpha} \right) \subseteq B \cap \left(\bigcup\limits_{\alpha \in \Lambda}^{}A_{\alpha} \right)$.
\end{list}
\end{list}



\begin{list}{\bf{\ref{exer:indexsubsets}.}}
\item \begin{list}{\bf{(a)}}
\item Let $x \in B$.  For each $\alpha \in \Lambda$, $B \subseteq A_\alpha$ and, hence, 
$x \in A_\alpha$.  This means that for each 
$\alpha \in \Lambda$, $x \in A_\alpha$  and, hence, 
$x \in \bigcap\limits_{\alpha \in \Lambda}^{}A_\alpha$.  Therefore, 
$B \subseteq \bigcap\limits_{\alpha \in \Lambda}^{}A_\alpha$.
\end{list}
\end{list}


\begin{list}{\bf{\ref{exer:gendemorgan}.}}
\item \begin{list}{\bf{(a)}}
\item We first rewrite the set difference and then use a distributive law.
\begin{align*}
\left(\bigcup\limits_{\alpha \in \Lambda}^{}A_{\alpha} \right) - B 
            &= \left(\bigcup\limits_{\alpha \in \Lambda}^{}A_{\alpha} \right) \cap B^c \\
            &= \bigcup\limits_{\alpha \in \Lambda}^{}\left( A_\alpha \cap B^c \right) \\
            &=\bigcup\limits_{\alpha \in \Lambda}^{} \left( A_{\alpha} - B \right) \\
\end{align*}
\end{list}
\end{list}
\hbreak
\endinput

