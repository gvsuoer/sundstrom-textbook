\subsection*{Proof of \mathversion{bold} $\left( P \to Q \right)$ Using a Proof by Contradiction}
%\textbf{{Proof of} {\mathversion{bold} $P \to Q$} Using a Proof by Contradiction}
\index{contradiction}%
\index{proof!by contradiction}%

\begin{itemize}
\item \textbf{When is it indicated}?  This type of proof is often used when the  conclusion is stated in the form of a negation, but the hypothesis is not.  This often works well if the conclusion contains the operator ``or'';  that is, if the conclusion is in the form of a disjunction.  In this case, the negation will be a conjunction.

\item \textbf{Description of the process}.  Assume  $P$  and $\mynot Q$   and work forward from these two assumptions until a contradiction is obtained.

\item \textbf{Why the process makes sense}.  The statement  $P \to Q$
  is either true or false.  In a proof by contradiction, we show that it is true by eliminating the only other possibility (that it is false).  We show that  $P \to Q$
  cannot be false by assuming it is false and reaching a contradiction.  Since we assume that  $P \to Q$
  is false, and the only way for a conditional statement to be false is for its hypothesis to be true and its conclusion to be false,  we assume that  $P$ is true and that  $Q$  is false (or, equivalently, that  $\mynot  Q$
  is true).  When we reach a contradiction, we know that our original assumption that  $P \to Q$
  is false is incorrect.  Hence,  $P \to Q$
  cannot be false, and so it must be true.
\end{itemize}
