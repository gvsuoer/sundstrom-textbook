\documentclass[11pt]{article}
\usepackage{c://pctex/activity}

\lhead{}
\chead{\textbf{\large{Exercise 14 - Section 1.2 \\More Work with Pythagorean Triples}}}
\rhead{}
\lfoot{\emph{Mathematical Reasoning: Writing and Proof, Third Ed.} \\Ted Sundstrom}
\cfoot{}
\rfoot{\copyright \the\year\, by Pearson Education, Inc.\\}


\begin{document}
\begin{enumerate}
  \item \textbf{Proposition}.  If $m$ is a natural number and $m$, $m + 7$, and $m + 8$ form a Pythagorean triple, then $m = 5$.  Hence, the only Pythagorean triple of the form $m$, $m + 7$, and $m + 8$, where $m$ is a natural number is 5, 12, and 13.
\begin{myproof}
Let $m$ be a natural number.  In order to determine Pythagorean triples of the form $m$, $m + 7$, and $m + 8$, we will find all natural number solutions of the equation $m^2 + (m + 7)^2 = (m + 8)^2$.
\begin{align*}
m^2 + (m + 7)^2 &= (m + 8)^2 \\
2m^2 + 14m + 49 &= m^2 + 16m + 64 \\
m^2 - 2m - 15 &= 0 
\end{align*}
We can factor the left side of the last equation and obtain $(m - 5)(m + 3) = 0$. Hence, the only solutions of the equation are when $m - 5 = 0$ or when $m + 3 = 0$.  Hence, the only solutions of 
$m^2 + (m + 7)^2 = (m + 8)^2$ are $m = 5$ and $m = -3$.  However, $-3$ is not a natural number and so the only natural number solution is $m = 5$.  This means that the only possibility for a Pythagorean triple of the form $m$, $m + 7$, and $m + 8$ is when $m = 5$.  It can be verified that 5, 12, and 13 form a Pythagorean triple and so this is the only Pythagorean triple of the form $m$, $m + 7$, and $m + 8$.
\end{myproof}


  \item \textbf{Proposition}.  There is no Pythagorean triple of the form $m$, $m + 11$, and $m + 12$, where $m$ is a natural number.

\begin{myproof}
We can obtain a Pythagorean triple of the form $m$, $m + 11$, and $m + 12$ only if $m$ is a natural number and $m^2 + (m + 11)^2 = (m + 12)^2$.  However, if we solve this equation for $m$, we obtain
\begin{align*}
m^2 + \left( m^2 + 22m + 121 \right) &= m^2 + 24m + 144 \\
m^2 - 2m - 23 &= 0
\end{align*}
Using the quadratic formula, we obtain $m = \dfrac{2 + \sqrt{92}}{2}$ or $m = \dfrac{2 - \sqrt{92}}{2}$.  Since neither of these numbers are natural numbers, we conclude that the equation 
$m^2 + (m + 11)^2 = (m + 12)^2$ has no solution that is a natural number.  This means that there is no Pythagorean triple of the form $m$, $m + 11$, and $m + 12$, where $m$ is a natural number.
\end{myproof}

\end{enumerate}

\end{document}
