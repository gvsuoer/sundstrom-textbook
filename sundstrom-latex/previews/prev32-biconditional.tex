\begin{previewactivity}[\textbf{A Biconditional Statement}]\label{PA:biconditional} \hfill
\index{biconditional statement}%
\index{statement!biconditional}%
%
\begin{enumerate}
  \item In Exercise (\ref{exer:bicond}) from Section \ref{S:logequiv}, we constructed a truth table to prove that the biconditional statement, $P \leftrightarrow Q$, is logically equivalent to  $\left( {P \to Q} \right) \wedge \left( {Q \to P} \right)$.  Complete this exercise if you have not already done so.  \label{PA:biconditional1}

  \item Suppose that we want to prove a biconditional statement of the form  
\makebox{$P \leftrightarrow Q$}.  Explain a method for completing this proof based on the logical equivalency in part~(\ref{PA:biconditional1}).

  \item Let  $n$  be an integer.  Assume that we have completed the proofs of the following two statements:

  \begin{itemize}
    \item If  $n$  is an odd integer, then  $n^2 $ is an odd integer.
    \item If  $ n^2 $ is an odd integer, then  $n$  is an odd integer.
  \end{itemize}

(See Exercise~(\ref{exer:x2odd}) from Section~\ref{S:direct} and \typeu Activity \ref*{PA:attempt}.)  Have we completed the proof of the following proposition?

  \begin{list}{}
    \item For each integer $n$, $n$  is an odd integer  if and only if  $ n^2 $ is an odd integer.
  \end{list}

Explain.
 
\end{enumerate}
\hbreak
\end{previewactivity}

\endinput
