\subsection*{The Definition of a Function}
The concept of a function is much more general than the idea of a function used in calculus or precalculus.  In particular, the domain and codomain do not have to be subsets of  $\mathbb{R}$.  In addition, the way in which a function associates elements of the domain with elements of the codomain can have many different forms.  This input-output rule can be a formula, a graph, a table, a random process, a computer algorithm, or a verbal description.  Two such examples were introduced in \typeu Activity~\ref*{PA:otherfunctions}.

For the \textbf{birthday function}, 
\index{birthday function}%
 the domain would be the set of all people and the codomain would be the set of all days in a leap year.  For the \textbf{sum of the divisors function},
\index{sum of divisors function}%
 the domain is the set  $\mathbb{N}$ of natural numbers, and the codomain could also be  $\mathbb{N}$.  In both of these cases, the input-output rule was a verbal description of how to assign an element of the codomain to an element of the domain.

We formally define the concept of a function as follows:
%
\begin{defbox}{function}{A \textbf{function}
\index{function}%
 from a set  $A$  to a set  $B$  is a rule that associates with each element  $x$  of the set  $A$  exactly one element of the set  $B$.  A function from  $A$  to  $B$ is also called a 
\textbf{mapping}
\index{mapping}%
 from  $A$  to  $B$.}
\end{defbox}
%
\noindent
\textbf{Function Notation}.  When we work with a function, we usually give it a name.  The name is often a single letter, such as  $f$  or  $g$.  If  $f$  is a function from the set  $A$  to the set  $B$, we will write  $f \colon A \to B$.
\label{sym:function}%
  This is simply shorthand notation for the fact that  $f$  is a function from the set  $A$  to the set  $B$.  In this case, we also say that  $f$  maps  $A$  to  $B$.

\begin{defbox}{domainfunction}{Let  $f \colon A \to B$.  (This is read, ``Let  $f$  be a function from  $A$  to  $B$.'')  The set  $A$  is called the \textbf{domain}
\index{domain!of a function}%
\index{function!domain}%
of the function  $f$, and we write  $A = \text{dom}( f )$.\label{sym:domfunc}  The set  $B$ is called the \textbf{codomain}
\index{codomain}%
\index{function!codomain}%
of the function  $f$, and we write  $B = \text{codom}( f )$. \label{sym:codomain}
\vskip6pt
If  $a \in A$, then the element of  $B$  that is associated with  $a$  is denoted by  
$f( a )$ 
\label{sym:fofx}and is called the \textbf{image of  $\boldsymbol{a}$
\index{image!of an element}%
  under  $\boldsymbol{f}$}\!.  If  $f( a ) = b$, with  $b \in B$, then  $a$  is called a 
\textbf{preimage of  $\boldsymbol{b}$
\index{preimage!of an element}%
  for  $\boldsymbol{f}$}\!.}  
\end{defbox}
%
\noindent
\textbf{Some Function Terminology with an Example}.  When we have a function \linebreak
$f \colon A \to B$, we often write $y = f(x)$.  In this case, we consider $x$ to be an unspecified object that can be chosen from the set $A$, and we would say that  $x$  is the \textbf{independent variable}
\index{independent variable}%
\index{variable!independent}%
 of the function  $f$  and  $y$  is the \textbf{dependent variable}
\index{dependent variable}%
\index{variable!dependent}%
 of the function  $f$.

For a specific example, consider the function $g \x \mathbb{R} \to \mathbb{R}$, where  $g( x )$ is defined by the formula 
\[
g( x ) = x^2  - 2. \label{example:function}
\]
Note that this is indeed a function since given any input  $x$  in the domain, $\R$, there is exactly one output  $g( x )$ in the codomain, $\R$.  For example,
\[
\begin{aligned}
  g( { - 2} )        &= \left( { - 2} \right)^2  - 2 = 2, \\ 
  g( 5 )             &= 5^2  - 2 = 23, \\
  g( {\sqrt 2 } )    &= \left( {\sqrt 2 } \right)^2  - 2 = 0,  \\
  g( { - \sqrt 2 } ) &= \left( { - \sqrt 2 } \right)^2  - 2 = 0. \\ 
\end{aligned}
\]
So we say that the image of  $-2$ under $g$  is  2, the image of  5 under $g$ is  23, and so on.  

Notice in this case that the number  0  in the codomain has two preimages, $ - \sqrt 2 \text{ and }\sqrt 2 $.  This does not violate the mathematical definition of a function since the definition only states that each input must produce one and only one output.  That is, each element of the domain has exactly one image in the codomain.  Nowhere does the definition stipulate that two different inputs must produce different outputs.

Finding the preimages of an element in the codomain can sometimes be difficult.  In general, if  $y$ is in the codomain, to find its preimages, we need to ask, ``For which values of  $x$  in the domain will we have  
$y = g( x )$?''  For example, for the function  $g$, to find the preimages of  5, we need to find all  $x$  for which  $g( x ) = 5$.  In this case, since  $g( x ) = x^2  - 2$, we can do this by solving the equation
\[
x^2  - 2 = 5.
\]
The solutions of this equation are  $ - \sqrt 7 $  and  $\sqrt 7 $.  So for the function  $g$, the preimages of  5  are  $ - \sqrt 7 $  and  $\sqrt 7 $.  We often use set notation for this  and say that the set of preimages of 5 for the function $g$ is 
$\left\{ -\sqrt{7}, \sqrt{7} \right\}$.
%\vskip10pt

Also notice that for this function, not every element in the codomain has a preimage.  For example, there is no input  $x$  such that  $g\left( x \right) =  - 3$.  This is true since for all real numbers  $x$,  $x^2  \geq 0$  and hence  $x^2  - 2 \geq  - 2$.  This means that for all  $x$  in  $\mathbb{R}$, $g\left( x \right) \geq  - 2$.
%\noindent

Finally, note that we introduced the function $g$ with the sentence, ``Consider the function  
$g\x \mathbb{R} \to \mathbb{R}$, where  $g( x )$  is defined by the formula  
$g( x ) = x^2  - 2$.''  This is one correct way to do this, but we will frequently shorten this to, ``Let  $g\x \mathbb{R} \to \mathbb{R}$ be defined by  $g( x ) = x^2  - 2$'', or ``Let  
$g\x \mathbb{R} \to \mathbb{R}$, where  $g( x ) = x^2  - 2$.''

\begin{prog}[\textbf{Images and Preimages}] \label{pr:images} \hfill \\
Let $f\x \R \to \R$ be defined by $f(x) = x^2 - 5x$ for all $x \in \R$, and let 
$g\x \Z \to \Z$ be defined by $g(m) = m^2 - 5m$ for all $m \in \Z$.

\begin{enumerate}
\item Determine $f ( -3 )$ and $f \left( \sqrt 8 \right)$.

\item Determine $g ( 2 )$ and $g ( -2 )$.

\item Determine the set of all preimages of 6 for the function $f$\!.

\item Determine the set of all preimages of 6 for the function $g$\!.

\item Determine the set of all preimages of 2 for the function $f$\!.

\item Determine the set of all preimages of 2 for the function $g$\!.
\end{enumerate}
\end{prog}
\hbreak


\endinput
