\section*{Section \ref{S:finitesets}}

\subsection*{Progress Check~\ref{prog:equivsets}}
\begin{enumerate}
%\item One example of a bijection $f\x A \to B$ is to define $f$ so that  $f ( 1 ) = a$, 
%$f ( 2 ) = b$, and $f ( 3 ) = c$.

\item We first prove that $f\x A \to B$ is an injection.  So let $x, y \in A$ and assume that $f(x) = f(y)$.  Then $x + 350 = y + 350$ and we can conclude that $x = y$.  Hence, $f$ is an injection.  To prove that $f$ is a surjection, let $b \in B$.  Then $351 \leq b \leq 450$ and hence, $1 \leq b - 350 \leq 100$ and so $b - 350 \in A$.  In addition, $f(b - 350) = (b - 350) + 350 = b$.  This proves that $f$ is a surjection.  Hence, the function $f$ is a bijection, and so, $A \approx B$.

\item If $x$ and $t$ are even integers and $F ( x ) = F ( t )$, then 
$x + 1 = t + 1$ and, hence, $x = t$.  Therefore, $F$ is an injection.

To prove that $F$ is a surjection, let $y \in D$.  This means that $y$ is an odd integer and, hence, $y - 1$ is an even integer.  In addition,
\[
F ( y - 1 ) = ( y - 1 ) + 1 = y.
\]
Therefore, $F$ is a surjection and hence, $F$ is a bijection.  We conclude that $E \approx D$.

\item Let $x, t \in ( 0, 1 )$ and assume that $f ( x ) = f ( t )$.  Then 
$bx = bt$ and, hence, $x = t$.  Therefore, $f$ is an injection.

To prove that $f$ is a surjection, let $y \in ( 0, b )$.  Since $0 < y < b$, we conclude that $0 < \dfrac{y}{b} < 1$ and that
\[
f \!\left( \dfrac{y}{b} \right) = b \!\left( \dfrac{y}{b} \right) = y.
\]
Therefore, $f$ is a surjection and hence $f$ is a bijection.  Thus, 
$( 0, 1 ) \approx ( 0, b )$.
\end{enumerate}

\hbreak





\endinput
