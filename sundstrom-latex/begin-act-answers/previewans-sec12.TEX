\documentclass[11pt]{article}
%\usepackage{../../styles/activity}
\usepackage{c://pctex/beginact}


\usepackage{xr}
%\externaldocument{../../0-MR}
\externaldocument{../../0-MR-links}

\lhead{}
%\chead{\textbf{\Large{\hspace{0pt}Beginning Activities for Section~1.2}}\\ \beghead} 
\bahead{1.2}
\rhead{}
\lfoot{}
\rfoot{}
\cfoot{\hspace{0pt}\scalebox{0.4}{\includegraphics{cc-by-nc-sa.eps}}}

\begin{document}

\subsection*{Beginning Activity 1 (Definition of Even and Odd Integers)}
\begin{enumerate}
\item \begin{enumerate}
\item The integer 28 is an even integer since $28 = 2 \cdot 14$ and 14 is an integer. \\
The integer $-42$ is an even integer since $-42 = 2 \cdot (-21)$ and $-21$ is an integer. \\
The integer 24 is an even integer since $24 = 2 \cdot 12$ and 12 is an integer. \\
The integer 0 is an even integer since $0 = 2 \cdot 0$ and 0 is an integer. \\

\item The integer 51 is an odd integer since $51 = 2 \cdot 25 + 1$ and 25 is an integer. \\
The integer $-11$ is an odd integer since $-11 = 2 \cdot (-6) + 1$ and $-6$ is an integer. \\
The integer 51 is an odd integer since $51 = 2 \cdot 25 + 1$ and 25 is an integer. \\
The integer 1 is an odd integer since $1 = 2 \cdot 0 + 1$ and 0 is an integer. \\
The integer $-1$ is an odd integer since $-1 = 2 \cdot (-1) + 1$ and $-1$ is an integer. \\
\end{enumerate}
\end{enumerate}
We will use these definitions of even and odd integers throughout the book.  However, just as important as the definitions is the fact that this beginning activity illustrates how mathematicians think about and use definitions.  When we encounter a new definition, we should do the following:
\begin{itemize}
  \item Write out some specific examples of the definition.
  \item Write out some specific non-examples of the definition.
  \item Write a carefully worded negation of the definition.
\end{itemize}
In this beginning activity, we focused mainly on the first item in this list.  For a non-example of an even integer, we might consider using an odd integer such as 23.  (We probably know 23 is not even, but we want to see if our formal definition is consistent with our prior knowledge.)  So in order for 23 to be an even integer, there must exist an integer $n$ such that
\[
23 = 2n.
\]
For this equation to be true, we can use our knowledge of elementary algebra to solve for $n$ and obtain
\[
n = \dfrac{23}{2}.
\]
Since $\dfrac{23}{2}$ is not an integer, we can conclude that there does not exist an integer $n$ such that $23 = 2n$.  Hence, 23 is not an even integer.  This illustrates the  last item in the list above as we have almost written a carefully worded negation of an even integer.  We see that
\begin{itemize}
  \item An integer $a$ is not an even integer provided that there does not exist an integer $n$ such that $a = 2n$.
  \item An integer $a$ is not an odd integer provided that there does not exist an integer $n$ such that $a = 2n + 1$.
\end{itemize}
Later in this text (Section 2.4), we will see other ways to write the negations of these definitions.
%\begin{enumerate}
%\item $8 = 2 \cdot 4$   and 4 is an integer. \qquad $-12 = 2 \cdot \left( -6 \right)$  and -6 is an integer.\\
%$24 = 2 \cdot 12$ and 12 is an integer. \qquad $0 = 2 \cdot 0$ and 0 is an integer.
%
%
%\item $7 = 2 \cdot 3 + 1$  and 3 is an integer. \quad  $-11 = 2 \cdot \left( -6 \right) + 1$ and  -6 is an integer. \\
%$51 = 2 \cdot 25 + 1$ and 51 is an integer. \quad $1 = 2 \cdot 0 + 1$ and 0 is an integer. \\
%$-1 = 2 \cdot \left( -1 \right) + 1$ and -1 is an integer.
%\end{enumerate}
\hbreak

\subsection*{Beginning Activity 2 (Thinking about a Proof)}
\begin{enumerate}
\item The hypothesis of the conditional statement is ``$x$  and  $y$  are odd integers'', and the conclusion of the conditional statement is ``$x \cdot y$ is an odd integer.''

\item It does not prove the conditional statement is false since it provides an example where the hypothesis of the conditional statement is false.  In this situation, the conditional statement is true.

\item This does not prove the conditional statement is true.  It is only one example.  To prove the conditional statement is true, we must prove that the conclusion of the statement, $x \cdot y$ is an odd integer, is true whenever the hypothesis, $x$  and  $y$  are odd integers,  is true.  We do not have to worry about the situation when the hypothesis is false.  So we must be able to prove that no matter what odd integers we choose for  $x$  and  $y$, the product  
$x \cdot y$  will always be odd.

\item Since $y$ is odd, there exists an integer $n$ such that $y = 2n + 1$.

\item We can prove that $x \cdot y$ is an odd integer by proving that there exists an integer $q$ such that 
$x \cdot y = 2q + 1$.

Please note that once again, we used a different letter ($q$) for the definition of an odd integer since we have already used $m$ and $n$ using the definition of an odd integer for $x$ and $y$.

%\item To prove the conditional statement is true, we must prove that the conclusion of the statement, $x \cdot y$ is an odd integer, is true whenever the hypothesis, $x$  and  $y$  are odd integers,  is true.  We do not have to worry about the situation when the hypothesis is false.
%
%\item We start the proof by assuming that  $x$  and  $y$  are odd integers.
%
%\item We need to prove that the product  $x \cdot y$  is an odd integer.
%
%\item In order to prove that  $x \cdot y$ is an odd integer, we need to prove that there exists an integer  $q$  such that  $x \cdot y = 2q + 1$.
\end{enumerate}
This proposition will be proven in Section~1.2 as Theorem 1.3.  The idea will be to use $x = 2m + 1$ and $y = 2n + 1$ to prove that there exists an integer $q$ such that 
$x\cdot y = 2q + 1$.  
\hbreak

\end{document}

