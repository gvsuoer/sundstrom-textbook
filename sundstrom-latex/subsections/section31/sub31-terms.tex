\subsection*{Some Mathematical Terminology}
In Section~\ref{S:direct}, we introduced the idea of a direct proof.  Since then, we have used some common terminology in mathematics without much explanation.  Before we proceed further, we will discuss some frequently used mathematical terms.

A \textbf{proof}
\label{proof}%
\index{proof}%
 in mathematics is a convincing argument that some mathematical statement is true.  A proof should contain enough mathematical detail to be convincing to the person(s) to whom the proof is addressed.  In essence, a proof is an argument that communicates a mathematical truth to another person (who has the appropriate mathematical background).  A proof must use correct, logical reasoning and be based on previously established results.  These previous results can be axioms, definitions, or previously proven theorems.  These terms are discussed below.

Surprising to some is the fact that in mathematics, there are always \textbf{undefined terms}.
\label{undefined}%
\index{undefined term}%
  This is because if we tried to define everything, we would end up going in circles.  Simply put, we must start somewhere.  For example, in Euclidean geometry, the terms ``point,'' ``line,'' and ``contains'' are undefined terms.  In this text, we are using our number systems such as the natural numbers and integers as undefined terms.  We often assume that these undefined objects satisfy certain properties.  These assumed relationships are accepted as true without proof and are called axioms (or postulates).  An \textbf{axiom} 
\label{axiom}%
\index{axiom}%
 is a mathematical statement that is accepted without proof.  Euclidean geometry starts with undefined terms and a set of postulates and axioms.  For example, the following statement is an axiom of Euclidean geometry:

\newpar
\setlength{\hangindent}{20pt}
\indent
\emph{Given any two distinct points, there is exactly one line that contains these two points.}


\begin{center}
\fbox{\parbox{4.68in}{The closure properties of the number systems discussed in Section~\ref{S:prop} and the properties of the number systems in Table~\ref{Ta:propertiesofreals} on page~\pageref{Ta:propertiesofreals} are being used as axioms in this text.}}
\end{center}

%The closure properties of the number systems discussed in Section~\ref{S:prop} are being used as axioms in this text.

A \textbf{definition}
\label{definition}%
\index{definition}%
 is simply an agreement as to the meaning of a particular term.  For example, in this text, we have defined the terms ``even integer'' and ``odd integer.''  Definitions are not made at random, but rather, a definition is usually made because a certain property is observed to occur frequently.  As a result, it becomes convenient to give this property its own special name.  Definitions that have been made can be used in developing mathematical proofs.  In fact, most proofs require the use of some definitions.

In dealing with mathematical statements, we frequently use the terms ``conjecture,'' ``theorem,'' ``proposition,'' ``lemma,'' and ``corollary.''  A \textbf{conjecture}
\label{conjecture}%
\index{conjecture}%
 is a statement that we believe is plausible.  That is, we think it is true, but we have not yet developed a proof that it is true.  A \textbf{theorem} 
\label{theorem}%
\index{theorem}%
 is a mathematical statement for which we have a proof.  A term that is often considered to be synonymous with ``theorem'' is \textbf{proposition}. 
\label{proposition}%.  %One difference, however, is that a proposition can be false.  In this case, if the proposition involves a universal quantifier, we often show it is false by giving a counterexample.  

Often the proof of a theorem can be quite long.  In this case, it is often easier to communicate the proof in smaller ``pieces.''    These supporting pieces are often called lemmas.  A 
\textbf{lemma} 
\label{lemma}%
\index{lemma}%
 is a true mathematical statement that was proven mainly to help in the proof of some theorem.  Once a given theorem has been proven, it is often the case that other propositions follow immediately from the fact that the theorem is true.  These are called corollaries of the theorem.  The term \textbf{corollary} 
\label{corollary}%
\index{corollary}%
 is used to refer to a theorem that is easily proven once some other theorem has been proven.
\hbreak

\endinput
