\section*{Section~\ref{S:divalgo} The Division Algorithm and Congruence}

\subsection*{Preview Activity 1 (Quotients and Remainders)}

\begin{enumerate}
\item 
$$
\BeginTable
\BeginFormat
| c | c | c | c | c | c | c | c | c | c | c | 
\EndFormat
\_
| $q$ | 1 | 2 | 3 | 4 | 5 | 6 | 7 | 8 | 9 | 10 | \\+33 \_
| $r$ | 23  | 19 | 15 | 11  |  7 |  3 | $-1$  | $-5$  | $-9$     | $-13$     |  \\+33 \_
| $4q + r$ | 27  | 27  | 27 | 27 | 27 | 27 | 27 | 27 | 27 | 27 | \\+33 \_
\EndTable
$$


\item The smallest value for  $r$  from part (1) is  3. $( {27 = 4 \cdot 6 + 3} )$

\item When 27 is divided by 4, the quotient is 6 and the remainder is 3.  These are the values for $q$ and $r$ from part~(2).

\item 
$$
\BeginTable
\BeginFormat
| c | c | c | c | c | c | c | c | 
\EndFormat
\_
| $q$ | $-7$ | $-6$ | $-5$ | $-4$ | $-3$ | $-2$ | $-1$ |  \\+33 \_
| $r$ | 18 | 13 | 8  | 3  | $-2$  | $-7$  | $-12$  |  \\+33 \_
| $5q + r$ | $-17$  | $-17$ | $-17$ | $-17$ | $-17$ | $-17$ | $-17$ |  \\+33 \_
\EndTable
$$

\item When $-17$ is divided by 5, the quotient is $-4$ and the remainder is 3. $(-17 = 5 \cdot (-4) + 3)$

\end{enumerate}
\hbreak



\subsection*{Preview Activity 2 (Some Work with Congruence Modulo \emph{n})}
\begin{enumerate}
\item \begin{enumerate}
\item Let  $n \in \mathbb{N}$.  If  $a$  and  $b$  are integers, then we say that \textbf{$\boldsymbol{a}$  is congruent to  $\boldsymbol{b}$  modulo  $\boldsymbol{n}$}
\index{congruent modulo $n$}%
  provided that  $n$  divides  $a - b$.  A standard notation for this is   
$a \equiv b \pmod n$.  This is read as ``$a$  is congruent to  $b$  modulo  $n$''   or  ``$a$  is congruent to  $b$  mod  $n$ .''

\item When $a \equiv b \pmod n$, there exists an integer $k$ such that $a - b = nk$.
\end{enumerate}

\item $B = \left\{ b \in \Z \mid b \equiv 5 \pmod 6 \right\} = \{ \ldots, -13, -7, -1, 5, 11, 17, 23, \ldots \}$.

\item 
$$
\BeginTable
\BeginFormat
| c | c | c | c |
\EndFormat
\_
| $a \in A$ | $b \in B$ | $a + b$ | $r$, where $(a+b) \equiv r \pmod 6 \text{ and } 0 \leq r < 6$ | \\ \_
| 9 | 5 | 14 | 2 | \\ \_
| $-9$ | 17 | 8 | 2 | \\ \_
| 3 | $-7$ | $-4$ | 2 | \\ \_
| 21 | 23 | 44| 2 | \\ \_
\EndTable
$$
 
\item \textbf{Proposition}.  For all integers $a$ and $b$, if $a \equiv 3 \pmod 6$ and $b \equiv 5 \pmod 6$, then \\
$(a + b) \equiv 2 \pmod 6$.

\begin{myproof}
We assume that $a$ and $b$ are integers and that $a \equiv 3 \pmod 6$ and $b \equiv 5 \pmod 6$.  We will prove that 
$(a + b) \equiv 2 \pmod 6$.  Since $a \equiv 3 \pmod 6$ and $b \equiv 5 \pmod 6$, there exist integers $k$ and $m$ such that
\[
a - 3 = 6k \quad \text{and} \quad b - 5 = 6m.
\]
We can then write $a = 6k + 3$ and $b = 6m + 5$ and obtain
\begin{align*}
(a + b) - 2 &= (6k + 3) + (6m + 5) - 2 \\
            &= 6k + 6m + 6 \\
            &= 6(k + m + 1)
\end{align*}
Since $(k + m + 1) \in \Z$, the last equation shows that 6 divides $(a + b) - 2$, which implies that
$(a + b) \equiv 2 \pmod 6$.  This proves that if $a \equiv 3 \pmod 6$ and $b \equiv 5 \pmod 6$, then \\
$(a + b) \equiv 2 \pmod 6$.
\end{myproof}

\end{enumerate}
\hbreak



\newpage

\endinput
