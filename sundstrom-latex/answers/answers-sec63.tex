\section*{Section \ref{S:typesoffunctions}}
\renewcommand{\labelenumi}{(\textbf{\alph{enumi}})}

\begin{list}{\bf{\ref{exer:sec63-mod5function}.}}
\item \begin{list}{\bf{(a)}}
\item Notice that $f(0) = 4$, $f(1) = 0$, $f(2) = 3$, $f(3) = 3$, and $f(4) = 0$.  So the function $f$ is not an injection and is not a surjection.
\end{list}
\end{list}


\begin{list}{}
\item \begin{list}{\bf{(c)}}
\item Notice that $F(0) = 4$, $F(1) = 0$, $F(2) = 2$, $F(3) = 1$, and $F(4) = 3$.  So the function $F$ is an injection and is a surjection.
\end{list}
\end{list}


\begin{list}{\bf{\ref{exer:sec63-5}.}}
\item \begin{list}{\bf{(a)}}
\item The function $f$ is an injection.  To prove this, let $x_1, x_2 \in \mathbb{Z}$ and assume that $f ( x_1 ) = f ( x_2 )$.  Then,
\begin{align*}
3x_1 + 1 &= 3x_2 + 1 \\
    3x_1 &= 3x_2 \\
     x_1 &= x_2. \\
\end{align*}
Hence, $f$ is an injection.  Now, for each $x \in \mathbb{Z}$, $3x + 1 \equiv 1 \pmod 3$, and hence 
$f ( x ) \equiv 1 \pmod 3$.  This means that there is no integer $x$ such that 
$f ( x ) = 0$.  Therefore, $f$ is not a surjection.
\end{list}
\end{list}

\begin{list}{}
\item \begin{list}{\bf{(b)}}
\item The proof that $F$ is an injection is similar to the proof in Part~(a) that $f$ is an injection.  To prove that $F$ is a surjection, let $y \in \mathbb{Q}$.  Then, 
$\dfrac{y-1}{3} \in \mathbb{Q}$ and $F \left( \dfrac{y-1}{3} \right) = y$ and hence, $F$ is a surjection.
\end{list}
\end{list}

\begin{list}{}
\item \begin{list}{\bf{(h)}}
\item Since $h(1) = h(4)$, the function $h$ is not an injection.  Using calculus, we can see that the function $h$ has a maximum when $x = 2$ and a minimum when $x = -2$, and so for each $x \in \R$, $h(-2) \leq h(x) \leq h(2)$ or
\[
-\frac{1}{2} \leq h(x) \leq \frac{1}{2}.
\]
This can be used to prove that $h$ is not a surjection.  

We can also prove that there is no $x \in \R$ such that $h(x) = 1$ using a proof by contradiction.  If such an $x$ were to exist, then $\dfrac{2x}{x^2 + 4} = 1$ or $2x = x^2 + 4$.  Hence, $x^2 - 2x + 4 = 0$.  We can then use the quadratic formula to prove that $x$ is not a real number.  Hence, there is no real number $x$ such that $h(x) = 1$ and so $h$ is not a surjection.
\end{list}
\end{list}



\begin{list}{\bf{\ref{exer:forexample}.}}
\item \begin{list}{\bf{(a)}}
\item Let $F\x \mathbb{R} \to \mathbb{R}$ be defined by $F( x ) = 5x + 3$ for all 
$x \in \mathbb{R}$.  Let  $x_1, x_2 \in \mathbb{R}$ and assume that 
$F( x_1 ) = F( x_2 )$.  Then  $5x_1 + 3 = 5x_2 + 3$.  Show that this implies that $x_1 = x_2$ and, hence, $F$ is an injection.  

Now let $y \in \mathbb{R}$.  Then $\dfrac{y - 3}{5} \in \mathbb{R}$.  Prove that 
$F \!\left( \dfrac{y - 3}{5} \right) = y$.   Thus, $F$ is a surjection and hence $F$ is a bijection.
\end{list}
\end{list}

\begin{list}{}
\item \begin{list}{\bf{(b)}}
\item The proof that $G$ is an injection is similar to the proof in Part~(a) that $F$ is an injection.  Notice that for each $x \in \Z$, $G(x) \equiv 3 \pmod 5$.  Now explain why $G$ is not a surjection.
\end{list}
\end{list}



\begin{list}{\bf{\ref{exer:sec63-9}.}}
\item The birthday function is not an injection since there are two different people with the same birthday.  The birthday function is a surjection since for each day of the year, there is a person that was born on that day.
\end{list}

\begin{list}{\bf{\ref{exer:sec63-11}.}}
\item \begin{list}{\bf{(a)}}
\item The function  $f$  is an injection and a surjection.  To prove that $f$  is an injection, we assume that  
$( {a, b} ) \in \mathbb{R} \times \mathbb{R}$, 
$( {c, d} ) \in \mathbb{R} \times \mathbb{R}$, and that  
$f( {a, b} ) = f( {c, d} )$.  This means that
\[
( {2a, a + b} ) = ( {2c, c + d} ).
\]
Since this equation is an equality of ordered pairs, we see that
\[
\begin{aligned}
       2a &= 2c \text{, and} \\ 
    a + b &= c + d. \\ 
\end{aligned}
\]
The first equation implies that $a = c$.  Substituting this into the second equation shows that 
$b = d$.  Hence, 
\[
( {a, b} ) = ( {c, d} ),
\]
and we have shown that if $f( {a, b} ) = f( {c, d} )$, then  
$( {a, b} ) = ( {c, d} )$.  Therefore,  $f$  is an injection.

Now, to determine if  $f$  is a surjection, we let  
$( {r, s} ) \in \mathbb{R} \times \mathbb{R}$. To find an ordered pair 
$( {a, b} ) \in \mathbb{R} \times \mathbb{R}$ such that  
$f( {a, b} ) = ( {r, s} )$, we need
\[
( {2a, a + b} ) = ( {r, s} ).
\]
That is, we need
\[
\begin{aligned}
      2a &= r\text{, and} \\ 
   a + b &= s. \\ 
\end{aligned}
\]
Solving this system for  $a$  and  $b$  yields  
\[
a = \frac{{r}}{2} \text{ and } b = \frac{{2s - r}}{2}.
\]
Since  $r, s \in \mathbb{R}$, we can conclude that  $a \in \mathbb{R}$ and 
$b \in \mathbb{R}$ and hence that  
\mbox{$( {a, b} ) \in \mathbb{R} \times \mathbb{R}$}.  So,
\[
\begin{aligned}
  f( {a, b} ) &= f\left( {\frac{{r}}{2}, \frac{{2s - r}}{2}} \right) \\ 
                         &= \left( {2\left( {\frac{{r}}{2}} \right),\frac{{r}}{2} + \frac{{2s - r}}{2}} \right) \\ 
                         &= ( {r, s} ). \\ 
\end{aligned} 
\]
This proves that for all  $( {r, s} ) \in \mathbb{R} \times \mathbb{R}$, there exists  $( {a, b} ) \in \mathbb{R} \times \mathbb{R}$ such that  
$f( {a, b} ) = ( {r, s} )$.  Hence, the function  $f$  is a surjection.  Since  $f$  is both an injection and a surjection,  it is a bijection.
\end{list}
\end{list}

\begin{list}{}
\item \begin{list}{\bf{(b)}}
\item The proof that the function $g$ is an injection is similar to the proof that $f$ is an injection in Part~(a).  Now use the fact that the first coordinate of $g(x, y)$ is an even integer to explain why the function $g$ is not a surjection.
\end{list}
\end{list}

\hbreak
\endinput


