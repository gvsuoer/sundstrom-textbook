In \typeu Activity~\ref*{PA:logequiv}, we introduced the concept of logically equivalent expressions and the notation $X \equiv Y$  to indicate that statements $X$  and  $Y$  are logically equivalent.  
The following theorem gives two important logical equivalencies.  They are sometimes referred to as \textbf{De Morgan's Laws}.
\index{De Morgan's Laws!for statements}%
%\pagebreak

\begin{nametheorem}[\textbf{De Morgan's Laws}]\label{T:demorgan} \hfill \\
For statements $P$ and $Q$, 
\begin{itemize}
    \item The statement  $\mynot  \left( {P \wedge Q} \right)$ is logically equivalent to  $\mynot  P \vee \mynot  Q$.  This can be written as  $\mynot  \left( {P \wedge Q} \right) \equiv \;\mynot  P \vee \mynot  Q$.
  \item The statement  $\mynot  \left( {P \vee Q} \right)$ is logically equivalent to  $\mynot  P \wedge \mynot  Q$.  This can be written as  $\mynot  \left( {P \vee Q} \right) \equiv \;\mynot  P \wedge \mynot  Q$.
  \end{itemize}
\end{nametheorem}
%\hrule \vskip6pt
%
The first equivalency in Theorem~\ref{T:demorgan} was established in \typeu Activity~\ref*{PA:logequiv}. Table~\ref{Ta:Demorgan} establishes the second equivalency.

\begin{table}[h]
$$
\BeginTable
\BeginFormat
|c|c|c|c|c|c|c|
\EndFormat
\_6
   |   $P$ | $Q$ \|6 $P \vee Q$  |  $\mynot  \left( {P \vee Q} \right)$  |  $\mynot  P$  |  $\mynot  Q   $  |  $\mynot  P \wedge \mynot  Q$ | \\+22 \_6
   |    T  |  T  \|6 T  |  F  |  F  |  F  |  F | \\
   |    T  |  F  \|6 T  |  F  |  F  |  T  |  F | \\ 
   |    F  |  T  \|6 T  |  F  |  T  |  F  |  F | \\ 
   |    F  |  F  \|6 F  |  T  |  T  |  T  |  T | \\ \_6
\EndTable
$$
\caption{Truth Table for One of De Morgan's Laws}
\label{Ta:Demorgan}
\end{table}
%\begin{center}
%    \begin{tabular}{|c|c ||c|c|c|c|c|}
%     \hline
%      $P$ & $Q$ & $P \vee Q$  &  $\mynot  \left( {P \vee Q} \right)$  &  $\mynot  P$  &  $\mynot  Q   $  &  $\mynot  P \wedge \mynot  Q$  \\ \hline
%       T  &  T  & T  &  F  &  F  &  F  &  F  \\ \hline
%       T  &  F  & T  &  F  &  F  &  T  &  F  \\ \hline
%       F  &  T  & T  &  F  &  T  &  F  &  F  \\ \hline
%       F  &  F  & F  &  T  &  T  &  T  &  T \\ \hline
%     \end{tabular}
%\end{center}
It is possible to develop and state several different logical equivalencies at this time.  However, we will restrict ourselves to what are considered to be some of the most important ones.   Since many mathematical statements are written in the form of conditional statements, logical equivalencies related to conditional statements are quite important.
\hbreak


\endinput
