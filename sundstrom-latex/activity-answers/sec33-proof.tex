\documentclass[11pt]{article}
\usepackage{c://pctex/activity}

\lhead{}
\chead{\textbf{\large{Exercise 21 -- Section 3.3\\A Proof by Contradiction}}}
\rhead{}
\lfoot{\emph{Mathematical Reasoning: Writing and Proof, Third Ed.} \\Ted Sundstrom}
\cfoot{}
\rfoot{\copyright \the\year\, by Pearson Education, Inc.\\}


\begin{document}
\noindent
\textbf{Proposition}: Let  $a$, $b$, and $c$  be integers.  If  3  divides  $a$,  3  divides  $b$,  and  $c \equiv 1 \pmod 3$, then the equation $ax + by = c$ has no solution in which both  $x$  and  $y$  are integers.

\begin{myproof}
A proof by contradiction will be used.  So, we assume that the statement is false.  That is, we assume that $a$, $b$, and $c$ are integers, that  3  divides both  $a$  and  $b$, that  $c \equiv 1 \pmod 3$,  and that  the equation
\[
ax + by = c
\]
has a solution in which both  $x$  and  $y$  are integers.

So, let  $m$  and  $n$  be integers  such that 
\setcounter{equation}{0}
\begin{equation} \label{eq:act317a}
am  + bn  = c.
\end{equation}
Since  3  divides both  $a$  and  $b$, we know there exist integers  $q$  and  $r$  such that
\[
a = 3q\text{  and  }b = 3r.
\]
Substituting these equations into equation~(\ref{eq:act317a}) gives
\begin{align} 
  3qm + 3rn &= c \notag \\ 
  3\left( {qm + rn} \right) &= c.  \label{eq:act317b}
\end{align}
Now, equation~(\ref{eq:act317b}) tells us that  3  divides  $c$ and hence, that  
$c \equiv 0 \pmod 3$, but this contradicts the assumption that  $c \equiv 1 \pmod 3$.  Thus, the assumption that the proposition is false is incorrect and we have proven that if  3  divides  $a$,  3  divides  $b$,  and  $c \equiv 1 \pmod 3$, then the equation $ax + by = c$  has no solution in which both  $x$  and  $y$  are integers.
\end{myproof}



\end{document}
