\section*{Section~\ref{S:typesoffunctions}}

\subsection*{Progress Check~\ref{pr:injections}}
The functions $k$, $F$, and $s$ are injections.  The functions $f$ and $h$ are not injections.


\subsection*{Progress Check~\ref{pr:functionswithfinitedom}}
The functions $f$ and $s$ are surjections.  The functions $k$ and $F$ are not surjections.


\subsection*{Progress Check~\ref{pr:domainandcodomain}}
The function $f$ is an injection but not a surjection.  To see that it is an injection, let  $a, b \in \R$ and assume that $f(a) = f(b)$.  This implies that $e^{-a} = e^{-b}$. Now use the natural logarithm function to prove that $a = b$.  Since 
$e^{-x} > 0$ for each real number $x$, there is no $x \in \R$ such that $f(x) = -1$.  So $f$ is not a surjection.

\newpar
The function $g$ is an injection and is a surjection.  The proof that $g$ is an injection is basically the same as the proof that $f$ is an injection.  To prove that $g$ is a surjection, let $b \in \R^+$.  To construct the real number $a$ such that 
$g(a) = b$, solve the equation $e^{-a} = b$ for $a$.  The solution is $a = -\ln b$.  It can then be verified that 
$g(a) = b$.



\subsection*{Progress Check~\ref{pr:function2variables}}
\begin{enumerate}
  \item There are several ordered pairs $(a, b) \in \R \times \R$ such that $g(a, b) = 2$.  For example, $g(0, 2) = 2$, $g(-1, 4) = 2$, and $g(2, -2) = 2$.
  \item For each $z \in \R$, $g(0, z) = z$.
  \item Part~(1) implies that the function $g$ is not an injection.  Part~(2) implies that the function $g$ is a surjection since for each $z \in \R$, $(0, z)$ is in the domain of $g$ and $g(0, z) = z$.
\end{enumerate}

%There are several examples that can be used to prove that $g$ is not an injection.  For instance, $g(0, 1) = 2$ and 
%$g(2, 0) = 2$.  There are several ways to prove that $g$ is a surjection.  To start, let $z \in \R$.  The goal now is to find $(x, y) \in \R \times \R$ such that $g(x, y) = z$ or $x + 2y = z$.  One way to do this is to use $y = 0$ and notice that 
%$g(z, 0) = z$.  Therefore, $g$ is a surjection.
\hbreak


\endinput

