\subsection*{Conditional Statements}\label{SS:conditional}
We had our first encounter with conditional statements in \typeu Activity~\ref*{PA:conditional}.  %One of the most frequently used types of statements in mathematics is the so-called conditional statement.  Given statements $P$ and $Q$, a statement of the form ``If $P$ then $Q$'' is called a 
%\textbf{conditional statement}.
%\index{conditional statement}%
%\index{statement!conditional}%
% It seems reasonable that the truth value (true or false) of the conditional statement 
%``If $P$ then $Q$'' depends on the truth values of $P$  and  $Q$.  The statement ``If $P$ then $Q$'' means that $Q$  must be true whenever $P$ is true.  The statement $P$ is called the \textbf{hypothesis}
%\index{conditional statement!hypothesis}%
% of the conditional statement, and the statement $Q$ is called the \textbf{conclusion}
%\index{conditional statement!conclusion}%
% of the conditional statement.  
Since conditional statements are the most important type of statement in mathematics, we give a more formal definition.

\begin{defbox}{D:conditional}{A \textbf{conditional statement}
\index{conditional statement|(}%
\index{statement!conditional|(}%
 is a statement that can be written in the form  ``If $P$ then $Q$,'' where  $P$  and  $Q$  are sentences. For this conditional statement,  $P$  is called the \textbf{hypothesis}
\index{conditional statement!hypothesis}%
 and  
$Q$  is called the \textbf{conclusion}.}
\index{conditional statement!conclusion}%
\end{defbox}
Intuitively, ``If $P$ then $Q$'' means that  $Q$  must be true whenever  $P$  is true.  Because conditional statements are used so often, a  symbolic shorthand notation is used to represent the conditional statement ``If $P$ then $Q$.''  We will use the notation  $P \to Q$ 
\label{sym:cond}%
  to represent ``If $P$ then $Q$.''
When $P$ and $Q$ are statements, it seems reasonable that the truth value (true or false) of the conditional statement $P \to Q$
 depends on the truth values of  $P$  and  $Q$.  There are four cases to consider:
\begin{multicols}{2}
\begin{itemize}
  \item $P$ is true and $Q$ is true.
  \item $P$ is true and $Q$ is false.
  \item $P$ is false and $Q$ is true.
  \item $P$ is false and $Q$ is false.
\end{itemize}
\end{multicols}
The conditional statement  $P \to Q$ means that  $Q$  is true whenever  $P$ is true.  It says nothing about the truth value of  $Q$  when  $P$  is false.  Using this as a guide, we define the conditional statement  $P \to Q$  to be false  only when  $P$  is true and  $Q$  is false, that is, only when the hypothesis is true and the conclusion is false.  In all other cases, $P \to Q$ is true.  This is summarized in Table~\ref{Ta:first}, which is called a \textbf{truth table}
\index{truth table}%
\index{conditional statement!truth table}%
 for the conditional statement  $P \to Q$.  (In Table~\ref{Ta:first},  T  stands for ``true'' and  F  stands for ``false.'')
\begin{table}[h]
$$
\BeginTable
    \BeginFormat
    | c | c | c |
    \EndFormat
     \_6
      | $P$ | $Q$ \|6 $P \to Q$ | \\+22 \_6
      | T   |  T  \|6 T | \\ %\hline
      | T   |  F  \|6 F | \\ %\hline
      | F   |  T  \|6 T | \\ %\hline
      | F   |  F  \|6 T | \\ \_6
 \EndTable
 $$
     \caption{Truth Table for $P \to Q$}
     \label{Ta:first}
\end{table}

The important thing to remember is that the conditional statement $P~\to~Q$ has its own truth value.  It is either true or false (and not both).  Its truth value depends on the truth values for $P$ and $Q$, but some find it a bit puzzling that the conditional statement is considered to be true when the hypothesis $P$ is false.  We will provide a justification for this through the use of an example.

\begin{example}
Suppose that I say
\begin{center}
``If it is not raining, then Daisy is riding her bike.''
\end{center}
We can represent this conditional statement as $P \to Q$ where $P$ is the statement, ``It is not raining'' and $Q$ is the statement, ``Daisy is riding her bike.''

Although it is not a perfect analogy, think of the statement $P \to Q$ as being \emph{false} to mean that I lied and think of the statement $P \to Q$ as being \emph{true} to mean that I did not lie.  We will now check the truth value of $P \to Q$ based on the truth values of $P$ and $Q$.

\begin{enumerate}
  \item Suppose that both $P$ and $Q$ are true.  That is, it is not raining and Daisy is riding her bike.  In this case, it seems reasonable to say that I told the truth and that $P \to Q$ is true.
  \item Suppose that $P$ is true and $Q$ is false or that it is not raining and Daisy is not riding her bike.  It would appear that by making the statement, ``If it is not raining, then Daisy is riding her bike,''  I have not told the truth.  So in this case, the statement $P \to Q$ is false.
  \item Now suppose that $P$ is false and $Q$ is true or that it is raining and Daisy is riding her bike.  Did I make a false statement by stating that if it is not raining, then Daisy is riding her bike?  The key is that I did not make any statement about what would happen if it was raining, and so I did not tell a lie.  So we consider the conditional statement, ``If it is not raining, then Daisy is riding her bike,'' to be true in the case where it is raining and Daisy is riding her bike.
  \item Finally, suppose that both $P$ and $Q$ are false.  That is, it is raining and Daisy is not riding her bike.  As in the previous situation, since my statement was $P \to Q$, I made no claim about what would happen if it was raining, and so I did not tell a lie.  So the statement $P \to Q$ cannot be false in this case and so we consider it to be true.
\end{enumerate}

\end{example}

%For example, the conditional statement
%\begin{center}
%``If it is not raining, then Karen is riding her bike.''
%\end{center}
%is false only in the case when it is not raining and Karen is not riding her bike.  In all other cases, it is true.
\hbreak




\begin{prog}[\textbf{Explorations with Conditional Statements}] \label{prog:condition} \hfill
\begin{enumerate}
%\item ``If it is raining, then Laura is at the theater.''
%Under what conditions is this conditional statement false?  For example,
%\begin{enumerate}
%\item Is it false if it is raining and Laura is at the theater?
%\item Is it false if it is raining and Laura is not at the theater?
%\item Is it false if it is not raining and Laura is at the theater?
%\item Is it false if it is not raining and Laura is not at the theater?
%\end{enumerate}

%\item Which of the following conditional statements do you believe are true and which do you believe are false?
%\begin{multicols}{2}
%\begin{enumerate}
%\item If $3 + 2 = 5$, then $5 < 8$.
%\item If $3 + 2 = 5$, then $8 < 5$.
%\item If $8 < 5$, then $3 + 2 = 5$.
%\item If $8 < 5$, then $3 + 2 = 9$.
%\end{enumerate}
%\end{multicols}

\item Consider the following sentence:
\begin{center}
If $x$ is a positive real number, then $x^2 + 8x$ is a positive real number.
\end{center}
Although the hypothesis and conclusion of this conditional sentence are not statements, the conditional sentence itself can be considered to be a statement as long as we know what possible numbers may be used for the variable $x$.  From the context of this sentence, it seems that we can substitute any positive real number for $x$.  We can also substitute 0 for $x$ or a negative real number for $x$ provided that we are willing to work with a false hypothesis in the conditional statement.  (In Chapter~\ref{C:logic}, we will learn how to be more careful and precise with these types of conditional statements.)

\begin{enumerate}
\item Notice that if $x = -3$, then $x^2 + 8x = -15$, which is negative.  Does this mean that the given conditional statement is false?

\item Notice that if $x = 4$, then $x^2 + 8x = 48$, which is positive.  Does this mean that the given conditional statement is true?

\item Do you think this conditional statement is true or false?  Record the results for at least five different examples where the hypothesis of this conditional statement is true.
\end{enumerate}

%\item ``If  $x$  and  $y$  are odd integers, then  $x \cdot y$ is an odd integer.''
%   \begin{enumerate}
%   \item Notice that  if  $x = 7$ and  $y = 2$, then  $x \cdot y = 14$.  So the statement that  $x \cdot y$ is odd is false in this case.  Does this mean that the given conditional statement is false?
%   \item Do you think this statement is true or false?  Try and record at least five different examples where the hypothesis of this conditional statement is true.
%   \end{enumerate}

\item ``If  $n$  is a positive integer, then $(n^2-n+41)$    is a prime number.''  (Remember that a prime number is a positive integer greater than 1 whose only positive factors are 1  and itself.) 
\label{PA:conditional3}%

To explore whether or not this statement is true, try using (and recording your results) for $n=1$, $n=2$, 
$n=3$, $n=4$, $n=5$, and $n=10$.  Then record the results for at least four other values of  $n$.  Does this conditional statement appear to be true? 
\end{enumerate}
\end{prog}
\hbreak

\endinput
