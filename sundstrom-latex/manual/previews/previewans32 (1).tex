\section*{Section~\ref{S:moremethods} More Methods of Proof}

\subsection*{Preview Activity 1 (Using the Contrapositive)}
\begin{enumerate}
  \item The proposition appears to be true since for each example,   $n$  is odd whenever  $n^2$  is odd.
  \item It is really not possible to construct a now-show table for a direct proof.  Hopefully, you discovered this.  If we assume that  $n^2 $ is even, we can conclude that there exists an integer  $k$  such that  $n^2  = 2k$.  

It would seem that we could use a square root to obtain something about  $n$, but there are two problems with this.  One is that the square root of  $n^2 $ is equal to the absolute value of  
$n$, and the other is that we cannot really do anything algebraically with  $\sqrt {2k} $.  That is, we cannot really simplify the equation  $\sqrt {n^2 }  = \sqrt {2k} $.

Another possibility could be to divide both sides of the equation  $n^2  = 2k$ by $n$.  One problem with this is that we first have to establish that  $n \ne 0$.  In the situation where  $n \ne 0$, we would get  $n = \dfrac{{2k}}{n}$.  The problems here are that the integers are not closed with respect to division and this does not really help us prove that  $n$  is odd.

  \item The contrapositive is:  For each integer $n$, if $n$ is an even integer, then $n^2$ is an even integer.

\item 
\begin{tabular}[t]{| c | l | l |} \hline
\textbf{Step} & \textbf{Know} & \textbf{Reason} \\ \hline
$P$  &  $n$  is an even integer.  & Hypothesis \\ \hline
$P1$  &  $( {\exists q \in \mathbb{Z}} )( {n = 2q} )$ & Defintion of even integer \\ \hline
$P2$  &  $n^2  = ( {2q} )^2 $ &  Square both sides of the equation \\ \hline
$P3$  &  $n^2 = 4 ( q^2 ) = 2 ( 2q^2 ) $  &  Algebra \\ \hline
$P4$  &  $2q^2$ is an integer.  &  Closure Property of the integers \\ \hline
$Q1$  &  There exists an integer $k$ such that $n^2 = 2k$.  &  The integer is $2q^2$. \\ \hline
$Q$   &  $n^2$ is an even integer.  &  Defintion of even integer \\ \hline
\textbf{Step} & \textbf{Show} & \textbf{Reason} \\ \hline
\end{tabular}

\item Since the contrapositive is logically equivalent to the conditional statement, by proving the contrapositive, we have also proven the original proposition.

\end{enumerate}
\hbreak




\subsection*{\textbf{Preview Activity 2 (A Biconditional Statement)}}

\begin{enumerate} 
  \item 
$$
\BeginTable
\BeginFormat
| c | c | c | c | c | c |
\EndFormat
\_6
| $P$  |  $Q$    \|6  $P \to Q$  |  $Q \to P$  |  $(P \to Q) \wedge (Q \to P)$ | $P \leftrightarrow Q$ | \\+22 \_6
          | T | T \|6 T | T | T | T | \\ 
         | T | F  \|6 F | T | F | F | \\ 
          | F | T \|6 T | F | F | F | \\ 
          | F | F \|6 T | T | T | T |  \\ \_6
\EndTable
$$

\item We can prove the biconditional statement  $P \leftrightarrow Q$  by completing a proof for  $P \to Q$  and completing a proof for  $Q \to P$.

\item In this case, if use  $P \to Q$  to represent, ``If  $n$  is an odd integer, then  
$n^2 $ is an odd integer'', then  $Q \to P$  would represent,  ``If  $n^2$ is an odd integer, then  $n$  is an odd integer.''  The logical equivalency in Part~(1) tells us that if we have proven the conditional statement  $P \to Q$  and have proven its converse  $Q \to P$, then we have proven the biconditional statement  $P \leftrightarrow Q$.  In this case,  the biconditional statement  
$P \leftrightarrow Q$ is, ``The integer  $n$  is an odd  if and only if  $n^2 $ is an odd integer.''

\end{enumerate}
\hbreak


\newpage

\endinput
