\section*{Section \ref{S:contradiction}}
\renewcommand{\labelenumi}{(\textbf{\alph{enumi}})}

\begin{list}{\bf{\ref{exer:sec33-1}.}}
\item \begin{list}{\bf{(a)}}
\item $P \vee C$
\end{list}
\end{list}


\begin{list}{\bf{\ref{exer:sec32-truefalse}.}}
\item \begin{enumerate}
\item This statement is true.  Use a proof by contradiction.  So assume that there exist integers  $a$ and $b$ such that $a$ is even, $b$ is odd, and 4 divides $a^2 + b^2$.  So there exist integers $m$ and 
$n$ such that
\[
a = 2m \qquad \text{and} \qquad a^2 + b^2 = 4n.
\]
Substitute $a = 2m$ into the second equation and use algebra to rewrite in the form 
$b^2 = 4(n - m^2)$.  This means that $b^2$ is even and hence, that $b$ is even.  This is a contradiction to the assumption that $b$ is odd.


\item This statement is true.  Use a proof by contradiction.  So assume that there exist integers $a$ and $b$ such that  $a$ is even, $b$ is odd, and 6 divides $a^2 + b^2$.  So there exist integers $m$ and 
$n$ such that
\[
a = 2m \qquad \text{and} \qquad a^2 + b^2 = 6n.
\]
Substitute $a = 2m$ into the second equation and use algebra to rewrite in the form 
$b^2 = 2(3n - 2m^2)$.  This means that $b^2$ is even and hence, that $b$ is even.  This is a contradiction.

%\item This statement is true.  Use a proof by contradiction.  So assume that $a$ and $b$ are integers, $a$ is even, $b$ is odd, and 4 divides $a^2 + 2b^2$.  So there exist integers $m$ and 
%$n$ such that
%\[
%a = 2m \qquad \text{and} \qquad a^2 + 2b^2 = 4n.
%\]
%Substitute $a = 2m$ into the second equation and use algebra to rewrite in the form 
%$b^2 = 2(n - m^2)$.  This means that $b^2$ is even and hence, that $b$ is even.  This is a contradiction to the assumption that $b$ is odd.
\addtocounter{enumi}{1}
\item This statement is true.  Use a direct proof.  Let $a$ and $b$ be integers and assume they are odd.  So there exist integers $m$ and $n$ such that
\[
a = 2m + 1 \qquad \text{and} \qquad b = 2n + 1.
\]
We then see that
\begin{align*}
a^2 + 3b^2 &= 4m^2 + 4m + 1 + 12n^2 + 12n + 3 \\
           &= 4\left( m^2 + m + 3n^2 + 3n + 1 \right).
\end{align*}
This shows that 4 divides $a^2 + 3b^2$.
\end{enumerate}
\end{list}




%Exercise 3
\begin{list}{\bf{\ref{exer:sec33-2}.}}
\item \begin{list}{\bf{(a)}}
\item We would assume that there exists a positive real number $r$  such that $r^2 = 18$ and $r$ is a rational number.
\end{list}
\end{list}

\begin{list}{}
\item \begin{list}{\bf{(b)}}
\item Do not attempt to mimic the proof that the square root of 2 is irrational 
(Theorem~\ref{T:squareroot2}).  You should still use the definition of a rational number but then use the fact that  $\sqrt {18}  = \sqrt {9 \cdot 2}  = \sqrt 9 \sqrt 2  = 3\sqrt 2 $.  So, if we assume that $r = \sqrt{18} = 3 \sqrt{2}$ is rational, then $\dfrac{r}{3} = \dfrac{\sqrt{18}}{3}$ is rational since the rational numbers are closed under division.  Hence, $\sqrt{2}$ is rational and this is a contradiction to Theorem~\ref{T:squareroot2}.
\end{list}
\end{list}


%Exercise 4
\begin{list}{\bf{\ref{exer:sec33-3}.}}
%\item In each part, what is the contrapositive of the proposition?  Why does it seem like the contrapositive will not be a good approach?  For each statement, try a proof by contradiction.  
\item \begin{enumerate}
\item Use a proof by contradiction.  So, we assume that there exist real numbers $x$ and $y$ such that $x$ is rational, $y$ is irrational, and $x + y$ is rational.  Since the rational numbers are closed under addition, this implies that $\left( x + y \right) - x$ is a rational number.  Since $\left( x + y \right) - x = y$, we conclude that $y$ is a rational number and this contradicts the assumption that $y$ is irrational.

\item Use a proof by contradiction.  So, we assume that there exist nonzero real numbers $x$ and $y$ such that $x$ is rational, $y$ is irrational, and $xy$ is rational.  Since the rational numbers are closed under division by nonzero rational numbers, this implies that 
$\dfrac{xy}{x}$ is a rational number.  Since $\dfrac{xy}{x} = y$, we conclude that $y$ is a rational number and this contradicts the assumption that $y$ is irrational.
\end{enumerate}
\end{list}

%Exercise 5
\begin{list}{\bf{\ref{exer:sec33-4}.}}
%Two of the propositions are true and the other two are false. 
\item \begin{enumerate}
\item This statement is false.  A counterexample is $x = \sqrt{2}$.
\item This statement is true since the contrapositive is true.  The contrapositive is:
\begin{list}{}
\item For any real number $x$, if $\sqrt{x}$ is rational, then $x$ is rational.
\end{list}
If there exist integers $a$ and $b$ with $b \ne 0$ such that $\sqrt{x} = \dfrac{a}{b}$, then 
$x^2 = \dfrac{a^2}{b^2}$ and hence, $x^2$ is rational.
\end{enumerate} 
\end{list}

%Exercise 9
\begin{list}{\bf{\ref{exer:sec33-9}.}}
\item Recall that $\log _2 (32)$  is the real number  $a$  such that  $2^a  = 32$.  That is, 
$a = \log _2 (32)$  means that  $2^a  = 32$.  If we assume that  $a$ is rational, then there exist integers $m$ and $n$, with $n \ne 0$, such that $a = \dfrac{m}{n}$.
\end{list}


%Exercise 11
\begin{list}{\bf{\ref{exer:sec33IVT}.}}
\item \hint  The only factors of 7 are $-1, 1, -7,$ and $7$.
\end{list}


%Exercise 12
\begin{list}{\bf{\ref{exer:sec33-11}.}}
\item \begin{list}{\bf{(a)}}
\item What happens if you expand  $[ {\sin (\theta)  + \cos (\theta) } ]^2$?  Don't forget your trigonometric identities.
\end{list}
\end{list}


%Exercise 13
\begin{list}{\bf{\ref{exer:sec35-10}.}}
\item \hint Three consecutive natural numbers can be represented by $n$, $n + 1$, and 
$n + 2$, where $n \in \mathbb{N}$, or three consecutive natural numbers can be represented by 
$m - 1$, $m$, and $m + 1$, where $m \in \mathbb{N}$.
\end{list}
\hbreak
\renewcommand{\labelenumi}{\textbf{\arabic{enumi}.}}
\endinput
