\begin{previewactivity}[\textbf{Exploring Statements of the Form \mathversion{bold} $\left( \forall n \in \N \right)\left( P(n) \right)$}] \label{PA:exploringstatements} \hfill \\
%In Section~\ref{S:predicates}, we defined the \textbf{truth set}
%\index{truth set}%
% of a predicate 
%$P( x )$ to be the collection of objects in the universal set that make the predicate a true statement when substituted for  $x$.  
One of the most fundamental sets in mathematics is the set of natural numbers $\N$.  In this section, we will learn a new proof technique, called mathematical induction, that is often used to prove statements of the form 
$\left( \forall n \in \N \right)\left( P(n) \right)$.  In Section~\ref{S:otherinduction}, we will learn how to extend this method to statements of the form $\left( \forall n \in T \right)\left( P(n) \right)$, where $T$ is a certain type of subset of the integers $\Z$.

\newpar
For each natural number $n$, let $P(n)$ be the following open sentence:
\[
4 \text{ divides }  \left( {5^n  - 1} \right)\!.
\]
\begin{enumerate}
\item Does this open sentence become a true statement when  $n = 1$?  That is, is  1  in the truth set of $P(n)$?

\item Does this open sentence become a true statement when  $n = 2$?  That is, is  2  in the truth set of $P(n)$?

\item Choose at least four more natural numbers and determine whether the open sentence is true or false for each of your choices.
\end{enumerate}
All of the examples that were used should provide evidence that the following proposition is true:
\begin{center}
For each natural number  $n$, 4  divides  $\left( {5^n  - 1} \right)$.
\end{center}
We should keep in mind that no matter how many examples we try, we cannot prove this proposition with a list of examples because we can never check if 4 divides $\left( {5^n  - 1} \right)$ for every natural number $n$.  Mathematical induction will provide a method for proving this proposition.

\setcounter{equation}{0}
For another example, for each natural number $n$, we now let $Q( n )$ be the following open sentence:
\begin{equation} \label{eq:PAexplore}
1^2  + 2^2  + \, \cdots \, + n^2  = \frac{{n(n + 1)(2n + 1)}}{6}.
\end{equation}
The expression on the left side of the previous equation is the sum of the squares of the first  $n$  natural numbers.  So when  $n = 1$, the left side of equation~(\ref{eq:PAexplore}) is  
$1^2 $.  When  $n = 2$, the left side of equation~(\ref{eq:PAexplore}) is  $1^2  + 2^2 $.

\setcounter{oldenumi}{\theenumi}
\begin{enumerate} \setcounter{enumi}{\theoldenumi}
\item Does $Q ( n )$ become a true statement when
\begin{itemize}
\item  $n = 1$?  (Is  1  in the truth set of $Q ( n )$?)

\item $n = 2$?  (Is  2  in the truth set of $Q ( n )$?)

\item $n = 3$?  (Is  3  in the truth set of $Q ( n )$?)
\end{itemize}

\item Choose at least four more natural numbers and determine whether the open sentence is true or false for each of your choices.  A table with the columns $n$, $1^2  + 2^2  + \, \cdots \, + n^2$, and $\dfrac{{n(n + 1)(2n + 1)}}{6}$ may help you organize your work.
\end{enumerate}
All of the examples we have explored, should indicate the following proposition is true:
\begin{center}
For each natural number  $n$,  $1^2  + 2^2  + \, \cdots \, + n^2  = \dfrac{{n(n + 1)(2n + 1)}}{6}$.
\end{center}
In this section, we will learn how to use mathematical induction to prove this statement.
\end{previewactivity}
\hbreak


\endinput
