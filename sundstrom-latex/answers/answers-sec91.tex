\subsection*{Section~\ref{S:finitesets}}

\begin{list}{\bf{\ref{exer:sec92-1}.}}
\item One way to do this is to prove that the following function is a bijection:  $f\x A \times \left\{ x \right\} \to A$ by $f ( a, x ) = a$, for all 
$( a, x ) \in A \times \left\{ x \right\}$.
\end{list}


\begin{list}{\bf{\ref{exer91:evennaturals}.}}
\item One way to prove that $\N \approx E^+$ is to find a bijection from $\N$ to $E^+$.  One possibility is $f: \mathbb{N} \to E^+$ by $f \left( n \right) = 2n$ for all $n \in \mathbb{N}$.  (We must prove that this is a bijection.)
\end{list}


\begin{list}{\bf{\ref{exer:sec92corollary}.}}
\item Notice that $A = ( A - \left\{ x \right\} ) \cup \left\{x \right\}$.  Use 
Theorem~\ref{T:finitesubsets} to conclude that $A - \left\{ x \right\}$ is finite.  Then use 
Lemma~\ref{L:addone}.
\end{list}


\begin{list}{\bf{\ref{exer:sec92-finitesets}.}}
\item \begin{enumerate}
\item Since $A \cap B \subseteq A$, if $A$ is finite, then Theorem~\ref{T:finitesubsets} implies that $A \cap B$ is finite.
\item The sets $A$ and $B$ are subsets of $A \cup B$.  So if $A \cup B$ is finite, then $A$ and $B$ are finite.
\end{enumerate}
\end{list}


\begin{list}{\bf{\ref{exer:sec92-7}.}}
\item \begin{list}{\bf{(a)}}
\item Remember that two ordered pairs are equal if and only if their corresponding coordinates are equal.  So if $\left( a_1, c_1 \right) , \left( a_2, c_2 \right) \in A \times C$ and $h \left( a_1, c_1 \right) = h \left( a_2, c_2 \right)$, then 
$\left( f \left( a_1 \right), g \left( c_1 \right) \right) = 
\left( f \left( a_2 \right), g \left( c_2 \right) \right)$.  We can then conclude that 
$f \left( a_1 \right) = f \left( a_2 \right)$ and $g \left( c_1 \right) = g \left( c_2 \right)$.  Since $f$ and $g$ are both injections, this means that $a_1 = a_2$ and $c_1 = c_2$ and therefore, 
$\left( a_1, c_1 \right) = \left( a_2, c_2 \right)$.  This proves that $f$ is an injection.

Now let $\left( b, d \right) \in B \times D$.  Since $f$ and $g$ are surjections, there exists 
$a \in A$ and $c \in C$ such that $f \left( a \right) = b$ and $g \left( c \right) = d$.  Therefore, $h \left( a, c \right) = \left( b, d \right)$.  This proves that $f$ is a surjection.
\end{list}
\end{list}


\begin{list}{\bf{\ref{exer:sec927}.}}
\item \begin{list}{\bf{(a)}}
\item If we define the function $f$ by $f ( 1 ) = a$, $f ( 2 ) = b$, 
$f ( 3 ) = c$, $f ( 4 ) = a$, and $f ( 5 ) = b$, then we can use 
 $g ( a ) = 1$, $g ( b ) = 2$, and $g ( 3 ) = c$.  The function $g$ is an injection.
\end{list}
\end{list}




\hbreak

\endinput


