\begin{previewactivity}[Functions with Finite Domains] \label{PA:functionswithfinitedom} \hfill \\
Let  $A = \left\{ {1,2,3} \right\}$, $B = \left\{ {a,b,c,d} \right\}$, and 
$C = \left\{ {s,t} \right\}$.  Define

\begin{center}
\begin{tabular}{c | c | c}
$f\x A \to B$ by &  $g\x A \to B$ by &  $h\x A \to C$ by \\
$f( 1 ) = a $  &  $g( 1 ) = a $  &  $h( 1 ) = s $ \\
$f( 2 ) = b $  &  $g( 2 ) = b $  &  $h( 2 ) = t $ \\
$f( 3 ) = c $  &  $g( 3 ) = a $  &  $h( 3 ) = s $
\end{tabular}
\end{center}
%
\begin{enumerate}
\item Determine the range of each of these functions.

\item Which of these functions satisfy the following property for a function  $F$?

\begin{list}{}
\item For all  $x, y \in \text{dom}( F )$, if  $x \ne y$, then  
$F(x) \ne F(y)$.
\end{list}

\item Which of these functions satisfy the following property for a function  $F$?

\begin{list}{}
\item For all  $x, y \in \text{dom}( F )$, if  
$F( x ) = F( y )$, then  $x = y$.
\end{list}

\item Which of these functions have their range equal to their codomain?

\item Which of the these functions satisfy the following property for a function  $F$?

\begin{list}{}
\item For all  $y$ in the codomain of $F$, there exists an  
$x \in \text{dom}( F )$ such that  $F( x ) = y$.
\end{list}
\end{enumerate}
\end{previewactivity}
\hbreak
%
%\begin{previewactivity}[Creating Functions with Finite Domains]
%\label{PA:functionswithfinitedom2} \hfill
%
%If you have not already done so, complete the following parts of Activity~\ref{A:creatingfunctions}:
%Let  $A = \left\{ {a,b,c,d} \right\}$, $B = \left\{ {a,b,c} \right\}$, and  
%$C = \left\{ {s,t,u,v} \right\}$.  In each of the following exercises, draw an arrow diagram to represent your function when it is appropriate.
%\begin{enumerate}
%\item Create a function  $f:A \to C$ whose range is the set   $C$  or explain why it is not possible to construct such a function.
%
%\item Create a function  $f:B \to C$ whose range is the set   $C$  or explain why it is not possible to construct such a function.
%
%\item If possible, create a function  $f:A \to C$  that satisfies the following condition:
%
%\begin{center}
%For all  $x, y \in A$,  if  $x \ne y$, then  $f( x ) \ne f( y )$.
%\end{center}
%
%If it is not possible to create such a function, explain why.
%
%\item If possible, create a function  $f:A \to \left\{ {s, t, u} \right\}$ that satisfies the following condition:
%
%\begin{center}
%For all  $x, y \in A$,  if  $x \ne y$, then  $f( x ) \ne f( y )$.
%\end{center}
%
%If it is not possible to create such a function, explain why.
%\end{enumerate}
%\end{previewactivity}
%\hbreak
%

%\pagebreak
\begin{previewactivity}[Statements Involving Functions]
\label{PA:functionstatements} \hfill \\
Let $A$ and $B$ be nonempty sets and let $f\x A \to B$.  
%In Preview Activity~\ref{PA:functionswithfinitedom}, we determined whether or not certain functions satisfied some specified properties.  These properties were written in the form of statements.
\begin{enumerate}
\item Consider the following statement:
\begin{list}{}
\item For all $x, y \in A$, if $x \ne y$, then $f ( x ) \ne f ( y )$.
\end{list}

\begin{enumerate} \label{PA:functionstatements1}
\item Write the contrapositive of this conditional statement.

\item Write the negation of this conditional statement.
\end{enumerate}

\item Now consider the statement:
\label{PA:functionstatements2}%
\begin{list}{}
\item For all $y \in B$, there exists an $x \in A$ such that $f ( x ) = y$.
\end{list}
Write the negation of this statement.

\item Let $g:\R \to \R$ be defined by $g ( x ) = 5x + 3$, for all $x \in \R$.  Complete the following proofs of about the function $g$.
\label{PA:functionstatements3}

\begin{enumerate}
\item For all $a, b \in \R$, if $g ( a ) = g ( b )$, then $a = b$.

\noindent
\textbf{\emph{Proof}}.  We let $a, b \in \R$, and we assume that 
$g ( a ) = g ( b )$.  This means that
\[
5a + 3 = 5b + 3.
\]
(Now prove that in this situation, $a = b$.)

\item For all $b \in \R$, there exists an $a \in \R$ such that $g ( a ) = b$.

\noindent
\textbf{\emph{Proof}}.  We let $b \in \R$.  We need to find an $a \in \R$ such that 
$g ( a ) = b$.  In order for this to happen, we need $5a + 3 = b$.

\noindent
(Now solve the equation for $a$ and then show that for this real number $a$, 
$g ( a ) = b$.)
\end{enumerate}
\end{enumerate}

\end{previewactivity}
\hbreak


\begin{previewactivity}[A Function of Two Variables]  \label{PA:functionoftwovars} \hfill \\
If necessary, see page~\pageref{ss:functiontwovar} for a disucssion of functions of two variables.
Define  $f\x \mathbb{Z} \times \mathbb{Z} \to \mathbb{Z}$ by  $f( {m, n} ) = m - 3n$.

%\noindent
%\underline{Note}:  Since the domain of this function is  $\mathbb{Z} \times \mathbb{Z}$ and each element of  $\mathbb{Z} \times \mathbb{Z}$ is an ordered pair of integers, we frequently call this type of function a \textbf{function of two variables}.
%\index{function!of two variables}%

\begin{enumerate}
\item Calculate  $f( {3, 2} )$, $f( {3,  - 2} )$, 
$f( {0, 2} )$, $f( {2, 0} )$, and  $f( { - 6, 0} )$.

\item If  $m \in \mathbb{Z}$, what is  $f( {m, 0} )$?  If  $n \in \mathbb{Z}$, what is  $f( {0, n} )$?

\item What is the range of the function  $f$?  Explain. 

\item Is the following statement true or false?  Explain.

\begin{list}{}
\item For every  $y \in \mathbb{Z}$, there exists  
$( {m, n} ) \in \mathbb{Z} \times \mathbb{Z}$ such that  \\
$f( {m, n} ) = y$?  Explain.
\end{list}

\item Is the following statement true or false?  Explain.

\begin{list}{}
\item For all  $( {m, n} ), ( {p, q} ) \in \mathbb{Z} \times \mathbb{Z}$, if  $( {m, n} ) \ne ( {p, q} )$, then  \\
$f( {m, n} ) \ne f( {p, q} )$.
\end{list}

\item Is the following statement true or false?  Explain.

\begin{list}{}
\item For all  $( {m, n} ), ( {p, q} ) \in \mathbb{Z} \times \mathbb{Z}$, if  $f( {m, n} ) = f( {p, q} )$, then  \\
$( {m, n} ) = ( {p, q} )$.
\end{list}
\end{enumerate}
\end{previewactivity}
\hbreak

\endinput
