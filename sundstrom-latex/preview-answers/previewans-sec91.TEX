\documentclass[11pt]{article}
\usepackage{../../styles/activity}

\usepackage{xr}
\externaldocument{0-MR}

\lhead{}
%\chead{\textbf{\Large{\hspace{0pt}Beginning Activities for Section~9.1}}\\\hspace{0pt}\emph{Mathematical Reasoning: Writing and Proof}}
\bahead{9.1}
\rhead{}
\lfoot{}
\rfoot{}
\cfoot{\hspace{0pt}\scalebox{0.4}{\includegraphics{cc-by-nc-sa.eps}}}
\graphicspath{{./epsfigs/}}

\begin{document}
\subsection*{Beginning Activity 1 (Equivalent Sets, Part 1)}
As is indicated in the definition of equivalent sets, one of the fundamental tools for ``comparing the sizes of sets'' is the concept of a function.  In particular, we need to thoroughly understand the concepts of injections, surjections, and bijections.
\begin{enumerate}
\item \begin{enumerate}
\item The function $f$ is an injection provided that for all $x, y \in A$, if $x \ne y$, then 
$f \left( x \right) \ne f \left( y \right)$.
\item The function $f$ is not an injection provided that there exist $x, y \in A$, such that  
$x \ne y$ and $f \left( x \right) \ne f \left( y \right)$.
\item The function $f$ is a surjection provided that for all $y \in B$, there exists an $x \in A$ such that $f \left( x \right) = y$.
\item The function $f$ is not a surjection provided that there exists a $y \in B$ such that for all $x \in A$ $f \left( x \right) \ne y$.
\item The function $f$ is a bijection provided that $f$ is an injection and $f$ is a surjection.
\end{enumerate}

\item \begin{enumerate}
\item There are several bijections that can be used to prove that $A \approx B$.  One is:  
$f:A \to B$ where $f \left( 1 \right) = a$, $f \left( 2 \right) = b$, and 
$f \left( 3 \right) = c$.

\item The set $C$ is not equivalent to $B$.  For any function $f: C \to B$, the range of the $f$ contains at most two elements.  This means that $f$ cannot be a surjection and hence cannot be a bijection.

\item The following function shows that $X = \left\{ 1, 2, 3, \ldots, 10 \right\}$ is equivalent 
$Y = \left\{ 57, 58, 59, \ldots, 66 \right\}$.

$f:X \to Y$ by $f \left( x \right) = x + 56$, for all $x \in X$.
\end{enumerate}

\item Let $f:\mathbb{N} \to D^+$ by $f \left( x \right) = 2x + 1$, for all $x \in \mathbb{N}$.

Let $x, t \in \mathbb{N}$ and assume that $f \left( x \right) = f \left( t \right)$.  Then,
$2x - 1 = 2t - 1$ and hence, $x = t$.  Therefore, $f$ is an injection.  

To prove that $f$ is a surjection, let $y \in D^+$.  Since $y$ is odd, $y + 1$ is an even natural number, and hence $\dfrac{y+1}{2} \in \mathbb{N}$.  Also,
\[
f \left( \frac{y+1}{2} \right) = 2 \left( \frac{y+1}{2} \right) - 1 = y.
\]
Therefore, $f$ is a surjection and hence $f$ is a bijection.  Thus, $\mathbb{N} \approx D^+$.

\item Let $g:\R \to \R^+$ by $g(x) = e^x$ for all $x \in \R$.  If $a, b \in R$ and 
$g(a) = g(b)$, then $e^a = e^b$.  Taking the natural logarithm of both sides of this equation yields
\begin{align*}
\ln \left( e^a \right) &= \ln \left( e^b \right) \\
                     a &= b
\end{align*}

Therefore, $g$ is an injection.  To prove that $g$ is a surjection, let $y \in \R^+$.  Then, $\ln y \in R$ and 
\begin{align*}
g \left( \ln y \right) &= e^{\ln y} \\
                       &= y
\end{align*}
Hence, $g$ is a surjection.  This proves that $g$ is a bijection and hence, 
$\R \approx \R^+$.
\end{enumerate}
\hbreak




\subsection*{Beginning Activity 2 (Equivalent Sets, Part 2)}
\renewcommand{\labelenumi}{(\textbf{\alph{enumi}})}

\noindent
\textbf{Theorem 9.1}  Let $A$, $B$, and $C$ be sets.

\begin{enumerate}
\item For each set $A$, $A \approx A$.  \label{T:equivsets1}

\item For all sets $A$ and $B$, if $A \approx B$, then 
$B \approx A$.  \label{T:equivsets2}

\item For all sets $A$, $B$, and $C$, if $A \approx B$ and 
$B \approx C$, then $A \approx C$.  \label{T:equivsets3}
\end{enumerate}

\begin{myproof}
Let $A$ be a set.  The identity mapping $I_A$ on the set $A$ is a bijection.  Therefore, $A \approx A$.   (Recall that $I_A \left( x \right) = x$ for all $x \in A$.)

\newpar
Now, assume that $A$ and $B$ are sets and that $A \approx B$.  Then, there exists a bijection $f:A \to B$.  Hence, by Exercise~(9) in Section~6.5, 
$f^{-1}:B \to A$ is a bijection.  Therefore, $B \approx A$.

\newpar
Now, assume that $A, B, \text{ and }C$ are sets and that $A \approx B$ and 
$B \approx C$.  Then, there exist bijections $f:A \to B$ and $g:B \to C$.  By Theorem~6.20 in Section~6.4, 
$g \circ f:A \to C$ is a bijection.  Therefore, $A \approx C$.

\hbreak
\end{myproof}
\renewcommand{\theenumi}{\textbf{\arabic{enumi}}}

\subsection*{Proof of Exercise 9 from Section 6.5}
%\textbf{\emph{Proof}.}
\begin{myproof}
Let $A$ and $B$ be sets and assume $A \approx B$.  So there exists a bijection $f:A \ to B$ and hence, the inverse function $f^{-1}:B \to A$ is defined.

Now assume that $b_1, b_2 \in B$ and $f^{-1} \left( b_1 \right) = f^{-1} \left( b_2 \right)$.  We can then apply the function $f$ to both sides of this equation and obtain
\begin{align*}
f \left( f^{-1} \left( b_1 \right) \right) &= f \left( f^{-1} \left( b_1 \right) \right) \\
                                       b_1 &= b_2, \\
\end{align*}
This proves that for all $b_1, b_2 \in B$, if $f^{-1} \left( b_1 \right) = f^{-1} \left( b_2 \right)$, then $b_1 = b_2$.  Hence, $f^{-1}$ is an injection.  

Now, let $a \in A$ and let $f \left( a \right) = b$, where $b \in B$.   This means that $f^{-1}(b) = a$, and hence, $f^{-1}$ is a surjection.  Since $f^{-1}$ is both an injection and a surjection, we conclude that $f^{-1}: B \to A$ is a bijection, and hence, $B \approx A$.
\end{myproof}

\end{document}
