\section*{Section~\ref{S:logop} Statements and Logical Operators}
Plan about one class period for this section.


\subsection*{Main Topics}
Logical operators (conjunction, disjunction, negation), conditional statements, constructing truth tables, biconditional statements.

\subsection*{The Preview Activities}
\subsubsection*{Preview Activity~\ref{PA:compound} (Compound Statements)} 
The purpose of this preview activity is to introduce the concept of a compound statement using logical operators.  Students are asked to determine if some compound statements are true or false.  This is intended to begin a discussion of the standard logical operators and the truth tables of conjunctions, disjunctions, negations, and conditional statements.

\subsubsection*{Preview Activity~\ref{PA:truth} (Truth Values of Statements)} 
The purpose of this preview activity is to get the students thinking about the use of the operators (connectives) ``and'', ``or'', ``not'', and ``if-then.''  This material will, of course, be covered in the section.

%\subsubsection*{Preview Activity~\ref{PA:truthtables} (Truth Tables)}  
%The purpose of this preview activity is to introduce the idea of a logical operator and its corresponding truth table.  Students are asked to complete the standard truth tables as they think they should be completed.  This can sometimes provide a basis for classroom discussion about the conventions used in mathematics when the mathemtical definitions are presented in the section.
\hbreak



\subsection*{The Exercises}

It would be good to assign all the exercises, but this may be a bit too much.  If necessary, assign only one of Exercises~(\ref{exer:sec22-2}) and~(\ref{exer:sec22-3}), and assign only one of Exercise~(\ref{exer:sec22-8}) and Exercise~(\ref{exer:sec22-9}).  
Exercises~(\ref{exer:sec22-5}) and~(\ref{exer:sec22-6}) provide practice at constructing truth tables.

\vskip6pt
\noindent
Typical Assignment:  Exercises 1, 2, 4, 5, 6, 7, 8
\hbreak

\subsection*{Activities and Explorations}

%\subsubsection*{Activity~\ref{A:working} (Working with Conditional Statements)}
The activity in Exercise~(\ref{exer:working-conditional}) deals with other forms of the conditional statement.  In the text, most conditional statements are used in the standard ``if-then'' form.  If this activity is not used, there should be some presentation or discussion of the other forms of the condtional statements.

\newpar
Exercise~(\ref{exer:working-truth}) is intended to give students practice at determining the truth values of statements without relying on truth tables.
\hbreak

\endinput
