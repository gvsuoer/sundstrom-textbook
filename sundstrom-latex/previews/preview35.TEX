\begin{previewactivity}[Quotients and Remainders]\label{PA:quotients} \hfill
\begin{enumerate}
\item Determine several integers  $q$  and  $r$  so that  $27 = 4 \cdot q + r$. For example, we could write
\[
27 = 4 \cdot 2 + 19 \text{ and }  27 = 4\left( 9 \right) + \left( { - 9} \right).
\]
Include at least five other examples where  $r$  is positive and at least two other examples where  $r$  is negative.
\label{part1-34}%

\item What is the smallest positive value for  $r$  that you obtained in your examples from Part~(\ref{part1-34})? 
\label{part2-34}%

\item Division is not considered an operation on the set of integers since the quotient of two integers need not be an integer.  However, we have all divided one integer by another and obtained a quotient and a remainder.  For example, if we divide 113 by 5, we obtain a quotient of 22 and a remainder of 3.  We can write this as follows:
\[
\frac{{113}}{5} = 22 + \frac{3}{5}.
\]
What is the resulting equation if we multiply both sides of the equation above by 5 and then use the distributive law to ``clear the parentheses'' on the right side of the equation?

\item What are the quotient and the remainder when we divide  27  by  4?  How is this related to your answer for Part~(\ref{part2-34})?

\end{enumerate}
\end{previewactivity}
\hbreak

\begin{previewactivity}[Review of Congruence]\label{PA:congruencereview} \hfill 
\begin{enumerate}
\item Let $n$ be a natural number and let $a$ and $b$ be integers.
\begin{enumerate}
\item Write the definition of ``$a$ is congruent to $b$ modulo $n$,'' which is written 
$a \equiv b \pmod n$.

\item Use the definition of ``divides'' to complete the following:

\begin{list}{}
\item When we write $a \equiv b \pmod n$, we may conclude that there exists an integer $k$ such that \ldots.
\end{list}
\end{enumerate}

\item 
\begin{enumerate}
\item Write the reflexive, symmetric, and transitive properties of congruence?  See 
Theorem~\ref{T:modprops} on page~\pageref{T:modprops}.

\item Explain why $17 \equiv 5 \pmod 6$.  If $x$ is an integer and  $x \equiv 17 \pmod 6$, then what conclusion can be made about the integer $x$ using the transitive property of congruence?

\item Find an integer $r$ such that $16^2 \equiv r \pmod 6$ and $0 \leq r < 5$.  Is there more than one such integer?  Find an integer $s$ such that \linebreak
$5^3 \equiv s \pmod 6$ and 
$0 \leq x < 5$.  Is there more than one such integer?
\end{enumerate}
\end{enumerate}
\end{previewactivity}
\hbreak

\endinput
