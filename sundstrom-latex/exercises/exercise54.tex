\section*{Exercises for Section~\ref{S:cartesian}}
%
\begin{enumerate}
\xitem Let   $A = \left\{ {1,2} \right\}$,  $B = \left\{ {a,b,c,d} \right\}$, and  $C = \left\{ {1,a,b} \right\}$.  Use the roster method to list all of the elements of each of the following sets: \label{exer:sec44-1}

\begin{multicols}{2} \begin{enumerate}
\item $A \times B$
\item $B \times A$
\item $A \times C$
\item $A^2$
\item $A \times \left( {B \cap C} \right)$
\item $\left( {A \times B} \right) \cap \left( {A \times C} \right)$
\item $A \times \emptyset $
\item $B \times \left\{ 2 \right\}$
\end{enumerate}
\end{multicols}


\item Sketch a graph of each of the following Cartesian products in the Cartesian plane.
\label{exer:sec44-2}
%
\begin{multicols}{2}
\begin{enumerate}
  \item $\left[ {0,2} \right] \times \left[ {1,3} \right]$
  \item $\left( {0,2} \right) \times \left( {1,3} \right]$
  \item $\left[ {2,3} \right] \times \left\{ 1 \right\}$
  \item $\left\{ 1 \right\} \times \left[ {2,3} \right]$
  \item $\mathbb{R} \times \left( {2,4} \right)$
  \item $\left( {2,4} \right) \times \mathbb{R}$
  \item $\mathbb{R} \times \left\{ { - 1} \right\}$
  \item $\left\{ { - 1} \right\} \times \left[ {1, + \infty } \right)$
\end{enumerate}
\end{multicols}

\xitem Prove Theorem~\ref{T:propsofcartprod}, Part~(\ref{T:propsofcartprod1}): $A \times \left( {B \cap C} \right) = \left( {A \times B} \right) \cap \left( {A \times C} \right)$. \label{exer:sec44-3}
	
\xitem Prove Theorem~\ref{T:propsofcartprod}, Part~(\ref{T:propsofcartprod4}): $\left( {A \cup B} \right) \times C = \left( {A \times C} \right) \cup \left( {B \times C} \right)$. \label{exer:sec44-4}

\item Prove Theorem~\ref{T:propsofcartprod}, Part~(\ref{T:propsofcartprod5}): $A \times \left( {B - C} \right) = \left( {A \times B} \right) - \left( {A \times C} \right)$.

\item Prove Theorem~\ref{T:propsofcartprod}, Part~(\ref{T:propsofcartprod7}): If  $T \subseteq A$, then  $T \times B \subseteq A \times B$. \label{exer:sec44-6}

\item Let  $A = \left\{ 1 \right\}\!, B = \left\{ 2 \right\}\!, \text{and }C = \left\{ 3 \right\}$.
\label{exer:sec44-7}

\begin{enumerate}
  \item Explain why  $A \times B \ne B \times A$.
  \item Explain why  $\left( {A \times B} \right) \times C \ne A \times \left( {B \times C} \right)$.
\end{enumerate}

\item Let  $A$  and  $B$  be nonempty sets.  Prove that  $A \times B = B \times A$
if and only if  $A = B$.

\item Is the following proposition true or false? Justify your conclusion. \label{exer:sec44-9}

\begin{list}{}
  \item Let  $A$, $B$, and  $C$  be sets with  $A \ne \emptyset $.  If  
$A \times B = A \times C$\!, then  $B = C$\!.
\end{list}

Explain where the assumption that $A \ne \emptyset $ is needed.
\end{enumerate}

\subsection*{Explorations and Activities}
\setcounter{oldenumi}{\theenumi}
\begin{enumerate} \setcounter{enumi}{\theoldenumi}
  \item (\textbf{A Set Theoretic Definition of an Ordered Pair}) 
\label{exer:defoforderedpair}%
In elementary mathematics, the notion of an ordered pair introduced at the beginning of this section will suffice.  However, if we are interested in a formal development of the Cartesian product of two sets, we need a more precise definition of ordered pair.  Following is one way to do this in terms of sets.  This definition is credited to Kazimierz Kuratowski (1896 -- 1980).
\index{Kuratowski, Kazimierz}%
Kuratowski was a famous Polish mathematician whose main work was in the areas of topology and set theory.  He was appointed the Director of the Polish Academy of Sciences and served in that position for 19 years.

Let  $x$  be an element of the set  $A$, and let  $y$  be an element of the set  $B$.  The \textbf{ordered pair}
\index{ordered pair}%
 $\left( {x,y} \right)$ is defined to be the set  
$\left\{ {\left\{ x \right\},\left\{ {x,y} \right\}} \right\}$.  That is,
\[
\left( {x,y} \right) = \left\{ {\left\{ x \right\},\left\{ {x,y} \right\}} \right\}\!.
\]

\begin{enumerate}
\item Explain how this definition allows us to distinguish between the ordered pairs  $\left( {3,5} \right)$  and  $\left( {5,3} \right)$.

\item Let  $A$  and  $B$  be sets and let  $a,c \in A$  and  $b,d \in B$.  Use this definition of an ordered pair and the concept of set equality to prove that  $\left( {a,b} \right) = \left( {c,d} \right)$  if and only if  $a = c$ and $b = d$.
\end{enumerate}

An \textbf{ordered triple}
\index{ordered triple}%
 can be thought of as a single triple of objects, denoted by  $\left( {a,b,c} \right)$, with an implied order.  This means that in order for two ordered triples to be equal, they must contain exactly the same objects in the same order.  That is,  $\left( {a,b,c} \right) = \left( {p,q,r} \right)$ if and only if  $a = p$, $b = q$ and $c = r$.

\begin{enumerate} \setcounter{enumii}{2}

\item Let  $A$, $B$, and  $C$  be sets, and let  $x \in A,y \in B,\text{and }z \in C$.  Write a set theoretic definition of the ordered triple  $\left( {x,y,z} \right)$ similar to the set theoretic definition of ``ordered pair.''
\end{enumerate}

\end{enumerate}
\hbreak

\endinput
