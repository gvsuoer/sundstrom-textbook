\section*{Exercises for Section~\ref{S:divalgo}}
\begin{enumerate}
%
%\item In   Beginning Activities~\ref{PA:alogicalequiv}~and~\ref{PA:propintegers}, we proved that if  $n$  is an integer, then  $ n^2 + n $   is an even integer.  We define two integers to be \textbf{consecutive integers}
%\index{consecutive integers}%
%\index{integers!consecutive}%
% if one of the integers is one more than the other integer.  This means that we can represent consecutive integers  as  $m$  and $m+1$, where  $m$  is some integer.
%
%Explain why the result proven in Beginning Activities~\ref{PA:alogicalequiv}~and~\ref{PA:propintegers} can be used to prove that the product of any two consecutive integers is divisible by 2.  \label{exer:consecutive}
%
%\item Extending the idea in Exercise~(\ref{exer:consecutive}), we can represent three consecutive integers as  $m$, $m + 1$, and  $m + 2$,  where  $m$  is an integer. \label{exer:sec34-2}

%\begin{enumerate}
%  \item Explain why we can also represent three consecutive integers as  $k - 1$, $k$, and 
%$k + 1$  where  $k$  is an integer.
%
%  \item Explain why Proposition~\ref{P:3divides} proves that the product of any three consecutive integers is divisible by 3.
%
%\end{enumerate}

%\item  Prove that if  $u$ is an odd integer, then the equation $x^2 + x - u = 0$ has no solution that is an integer. \label{exer:sec33-6}
%
%\item Prove that if $n$ is an odd integer, then $n = 4k +1$ for some integer $k$ or
%$n = 4m-1$ for some integer $m$. \label{exer:sec35-5}
%
%\item \begin{enumerate} \item Prove the following proposition:
%
%  \begin{list}{}
%    \item Let  $a$, $b$, and  $d$  be integers.  If  $d$  divides  $a$  or  $d$  divides  $b$, then  $d$  divides the product  $ab $.
%  \end{list}
%
%\hint  Notice that the hypothesis is a disjunction.  So use two cases. \label{exer:dividesproducta}
%
%  \item Write the contrapositive of the proposition in Exercise~(\ref{exer:dividesproducta}).
%
%  \item Write the converse of the proposition in Exercise~(\ref{exer:dividesproducta}).  Is the converse true or false?  Justify your conclusion. \label{exer:dividesproductc}
%  \end{enumerate}

\item \label{exer:prop3divides} 
Complete the details for the proof of Case 3 of Proposition~\ref{P:3divides}.


\xitem \begin{enumerate} \label{exer:sec35-2}
 \item Use cases based on congruence modulo 3 and properties of congruence to prove that for each integer $n$, 
$\mod{n^3}{n}{3}$.
 \item Explain why the result in Part~(a) proves that for each integer $n$, 3 divides $\left(n^3 - n \right)$.  Compare this to the  proof of the same result in Proposition~\ref{P:3divides}.
\end{enumerate}




\xitem %Let $n \in \mathbb{N}$.  
Prove the symmetric property of congruence stated in Theorem~\ref{T:modprops}.
\label{exer:cong-symm}%

\xitem Consider the following proposition:  \label{exer:sec35-a2mod3}
For each integer $a$, if 3 divides $a^2$, then 3 divides $a$.
\begin{enumerate}
  \item Write the contrapositive of this proposition.
  \item Prove the proposition by proving its contrapositive.  \hint Consider using cases based on the Division Algorithm using the remainder for  ``division by 3.''  There will be two cases.
\end{enumerate}

\xitem \label{exer:sec34-4} \begin{enumerate} \item Let  $n \in \mathbb{N}$ and let  $a \in \mathbb{Z}$.  Explain why  $n$ divides $a$  if and only if  
\linebreak $a \equiv 0 \pmod n$. 

  \item Let  $a \in \mathbb{Z}$.  Explain why  if  $a \not \equiv 0 \pmod 3$, then  $a \equiv 1\pmod 3$ or  $a \equiv 2 \pmod 3$.

  \item Is the following proposition true or false?  Justify your conclusion.

  \begin{list}{}
  \item For each  $a \in \mathbb{Z}$, if $a \not \equiv 0 \pmod 3$, then $a^2  \equiv 1 \pmod 3$.
  \end{list}
\end{enumerate}


%\item \begin{enumerate}
% \item Use cases based on congruence modulo 3 and properties of congruence to prove that for each integer $n$, 
%$\mod{n^3}{n}{3}$.
% \item Explain why the result in Part~(a) proves that for each integer $n$, 3 divides $\left(n^3 - n \right)$.  Compare this to the  proof of the same result in Proposition~\ref{P:3divides}.
%\end{enumerate}



\xitem Prove the following proposition by proving its contrapositive.  (\hint  Use case analysis.  There are several cases.)
\label{exer:congto3}%

  \begin{list}{}
  \item For all integers  $a$  and  $b$, if  $ab \equiv 0 \pmod 3$, then  $a \equiv 0 \pmod 3$  or  $b \equiv 0 \pmod 3$.
  \end{list}

%\pagebreak
\xitem \label{exer:3divprod}%
\begin{enumerate}
\item Explain why the following proposition is equivalent to the proposition in Exercise~(\ref{exer:congto3}).

  \begin{list}{}
  \item For all integers  $a$  and  $b$,  if $3 \mid ab$  , then  $3 \mid a$   or 
$3 \mid b$.
  \end{list}
\item Prove that for each integer $a$, if 3 divides $a^2$, then 3 divides $a$.
\end{enumerate}



\item \label{exer:sqrt3} \begin{enumerate}
\yitem Prove that the real number $\sqrt{3}$ is an irrational number.  That is, prove that
\begin{list}{}
\item If $r$ is a positive real number such that $r^2 = 3$, then $r$ is irrational.
\end{list}

\item Prove that the real number $\sqrt{12}$ is an irrational number.
\end{enumerate}


\xitem Prove that for each natural number $n$, $\sqrt{3n + 2}$ is not a natural number.
\label{exer:notperfectsquare}%

\item Extending the idea in Exercise~(\ref{exer:consecutive}) of Section~\ref{S:cases}, we can represent three consecutive integers as  $m$, $m + 1$, and  $m + 2$,  where  $m$  is an integer. 
\label{exer:sec34-2}%
\begin{enumerate}
  \item Explain why we can also represent three consecutive integers as  $k - 1$, $k$, and 
$k + 1$,  where  $k$  is an integer.

  \yitem Explain why Proposition~\ref{P:3divides} proves that the product of any three consecutive integers is divisible by 3.

  \yitem Prove that the product of three consecutive integers is divisible by 6.
\end{enumerate}

%\item The proposition in Exercise~(\ref{exer:3divprod}) was proven in Exercise~(\ref{exer:congto3}) using the concept of congruence.  This result can be proven without using the concept of congruence.  The idea is basically the same but we use the Division Algorithm  directly instead of using congruence.  Notice that if  3 does not divide an integer, then the remainder when that integer is divided by 3 must be 1 or 2.  This allows us to use two cases for that integer.  
%
%Prove the proposition in Exercise~(\ref{exer:3divprod}) by using this idea.  \hint  You will still have to prove the contrapositive and use case analysis.


\item \begin{enumerate} \label{exer:squaremod5}
%\item Is the following proposition true or false?  Justify your conclusion with a proof if it is true or with a counterexample if it is false.
%
%\begin{list}{}
%\item For every integer $a$, if $a \not \equiv 0 \pmod 5$, then $a^2 \equiv 1 \pmod 5$ or 
%$a^2 \equiv 4 \pmod 5$.
%\end{list}
\item Use the result in Proposition~\ref{prop:congmod5}  to help prove that the integer $m = 5,344, 580,232,468,953,153$ is not a perfect square.  Recall that an integer $n$ is a perfect square provided that there exists an integer $k$ such that $n = k^2$.  \hint Use a proof by contradiction.
\item Is the integer $n = 782,456,231,189,002,288, 438$ a perfect square?  Justify your conclusion.
\end{enumerate}


\item 
\label{exer:sqrt5-irrational}%
\begin{enumerate}
\item Use the result in Proposition~\ref{prop:congmod5} to help prove that for each integer $a$, if 5 divides $a^2$, then 5 divides $a$.

\item Prove that the real number $\sqrt{5}$ is an irrational number.
\end{enumerate}


\item \begin{enumerate}
\item Prove that for each integer $a$, if $a \not \equiv 0 \pmod 7$, then 
$a^2 \not \equiv 0 \pmod 7$.

\item Prove that for each integer $a$, if 7 divides $a^2$, then 7 divides $a$.

\item Prove that the real number $\sqrt{7}$ is an irrational number.
\end{enumerate}

\item \label{exer:remainderbycong} \begin{enumerate}
\item If an integer has a remainder of 6 when it is divided by 7, is it possible to determine the remainder of the square of that integer when it is divided by 7?  If so, determine the remainder and prove that your answer is correct.

\item If an integer has a remainder of 11 when it is divided by 12, is it possible to determine the remainder of the square of that integer when it is divided by 12?  If so, determine the remainder and prove that your answer is correct.

\item Let $n$ be a natural number greater than 2.  If an integer has a remainder of $n - 1$ when it is divided by $n$, is it possible to determine the remainder of the square of that integer when it is divided by $n$?  If so, determine the remainder and prove that your answer is correct.
\end{enumerate}


\item Let $n$ be a natural number greater than 4 and let $a$ be an integer that has a remainder of $n - 2$ when it is divided by $n$.  Make whatever conclusions you can about the remainder of $a^2$ when it is divided by $n$.  Justify all conclusions.


\item Is the following proposition true or false?  Justify your conclusion with a proof or a counterexample.
\begin{list}{}
\item For each natural number $n$, if 3 does not divide $\left( n^2 + 2 \right)$, then $n$ is not a prime number or $n=3$.
\end{list}

\item \label{exer:sec34-new9}\begin{enumerate}
\item Is the following proposition true or false?  Justify your conclusion with a counterexample or a proof. 

\begin{list}{}
\item For each integer $n$, if $n$ is odd, then $n^2 \equiv 1 \pmod 8$.
\end{list}

\item Compare this proposition to the proposition in Exercise~(\ref{exer:sec34-nsquared}) from Section~\ref{S:cases}.  Are these two propositions equivalent? Explain.
\item Is the following proposition true or false?  Justify your conclusion with a counterexample or a proof.
\begin{list}{}
\item For each integer $n$, if $n$ is odd and $n$ is not a multiple of 3, then $n^2 \equiv 1 \pmod {24}$.
\end{list}

\end{enumerate}

\item Prove the following proposition:
\begin{list}{}
\item For all integers $a$ and $b$, if 3 divides $\left( a^2 + b^2 \right)$, then 3 divides $a$ and 3 divides $b$.
\end{list}

\item Is the following proposition true or false?  Justify your conclusion with a counterexample or a proof.
\label{exer:case-ind}%
\begin{list}{}
\item For each integer $a$, 3 divides $a^3 + 23a$.
\end{list}

\item Are the following statements true or false?  Either prove the statement is true or provide a counterexample to show it is false.
\label{exer:falsecongruence}%

\begin{enumerate}
\item For all integers $a$ and $b$, if  $a \cdot b \equiv 0 \pmod{6}$, then $a \equiv 0 \pmod{6}$ or $b \equiv 0 \pmod{6}$.

\item For all integers $a$ and $b$, if  $a \cdot b \equiv 0 \pmod{8}$, then $a \equiv 0 \pmod{8}$ or $b \equiv 0 \pmod{8}$.

\item For all integers $a$ and $b$, if  $a \cdot b \equiv 1 \pmod{6}$, then $a \equiv 1 \pmod{6}$ or $b \equiv 1 \pmod{6}$.

\item For all integers $a$ and $b$, if $ab \equiv 7 \pmod{12}$, then either 
$a \equiv 1 \pmod{12}$ or $a \equiv 7 \pmod{12}$.
\end{enumerate}


%\xitem Use the ideas presented in Activity~\ref{A:lasttwo} to complete the following: 
%\label{exer:sec34-8}%
%  \begin{enumerate}
%    \item Determine the last two digits in the decimal representation of  $3^{3356} $.
%    \item Determine the last two digits in the decimal representation of  $7^{403} $.
%  \end{enumerate}

\item \begin{enumerate}
\item Determine several pairs of integers $a$ and $b$ such that $a \equiv b \pmod 5$.  For each such pair, calculate $4a + b$, $3a + 2b$, and $7a + 3b$.  Are each of the resulting integers congruent to 0 modulo 5?

\item Prove or disprove the following proposition:
\begin{list}{}
\item Let $m$ and $n$ be integers such that $(m + n) \equiv 0 \pmod 5$ and let $a, b \in \Z$.  If $a \equiv b \pmod 5$, then  $(ma +nb) \equiv 0 \pmod 5$.
\end{list}
\end{enumerate}




\item \textbf{Evaluation of proofs}  \hfill \\
See the instructions for Exercise~(\ref{exer:proofeval}) on 
page~\pageref{exer:proofeval} from Section~\ref{S:directproof}.

\begin{enumerate}
\item \textbf{Proposition}. For all integers $a$ and $b$, if $(a + 2b) \equiv 0 \pmod 3$, then 
$(2a + b) \equiv 0 \pmod 3$.

\begin{myproof}
We assume $a, b \in \Z$ and $(a + 2b) \equiv 0 \pmod 3$.  This means that 3 divides $a + 2b$ and, hence, there exists an integer $m$ such that $a + 2b = 3m$.  Hence, $a = 3m - 2b$.  For 
$(2a + b) \equiv 0 \pmod 3$, there exists an integer $x$ such that $2a + b = 3x$.  Hence,
\begin{align*}
        %2a + b &= 3x \\
2(3m - 2b) + b &= 3x \\
       6m - 3b &= 3x \\
     3(2m - b) &= 3x \\
        2m - b &= x.
\end{align*}
Since $(2m - b)$ is an integer, this proves that 3 divides $(2a + b)$ and hence, 
$(2a + b) \equiv 0 \pmod 3$.
\end{myproof}

\item \textbf{Proposition}. For each integer $m$, 5 divides $\left(m^5 - m \right)$.

\begin{myproof}
Let $m \in \Z$.  We will prove that 5 divides $\left(m^5 - m \right)$ by proving that 
$\left(m^5 - m \right) \equiv 0 \pmod 5$.  We will use cases.

For the first case, if $m \equiv 0 \pmod 5$, then $m^5 \equiv 0 \pmod 5$ and, hence, 
$\left(m^5 - m \right) \equiv 0 \pmod 5$.

For the second case, if $m \equiv 1 \pmod 5$, then $m^5 \equiv 1 \pmod 5$ and, hence, 
$\left(m^5 - m \right) \equiv (1 - 1) \pmod 5$, which means that 
$\left(m^5 - m \right) \equiv 0 \pmod 5$.

For the third case, if $m \equiv 2 \pmod 5$, then $m^5 \equiv 32 \pmod 5$ and, hence, 
$\left(m^5 - m \right) \equiv (32 - 2) \pmod 5$, which means that 
$\left(m^5 - m \right) \equiv 0 \pmod 5$.
\end{myproof}
\end{enumerate}
\end{enumerate}


\subsection*{Explorations and Activities}
\setcounter{oldenumi}{\theenumi}
\begin{enumerate} \setcounter{enumi}{\theoldenumi}
\item \textbf{Using a Contradiction to Prove a Case Is Not Possible}.  Explore the statements in Parts~(a) and~(b) 
%Exercises~(\ref{exer:sec35-9a}) and~(\ref{exer:sec35-9b}) 
by considering several examples where the hypothesis is true.
\label{exer:sec34-9}%

\begin{enumerate}
  \item If an integer  $a$  is divisible by both  4  and  6, then it divisible by  24.
\label{exer:sec35-9a}%

  \item If an integer  $a$  is divisible by both  2  and  3, then it divisible by  6.
\label{exer:sec35-9b}%

  \item What can you conclude from the examples in Part~(a)?

  \item What can you conclude from the examples in Part~(b)?

\end{enumerate}
%
The proof of the following proposition based on Part~(b) uses cases.  In this proof, however, we use cases and a proof by contradiction to prove that a certain integer cannot be odd.  Hence, it must be even.  Complete the proof of the proposition.

\textbf{Proposition.}
Let  $a \in \mathbb{Z}$.  If  2  divides  $a$  and  3  divides  $a$, then  6  divides  $a$.

\noindent
\textbf{\emph{Proof}}:  Let  $a \in \mathbb{Z}$ and assume that  2  divides  $a$  and  3  divides  $a$.  We will prove that  6 divides  $a$.  Since  3  divides  $a$, there exists an integer  $n$  such that
\[
a = 3n.
\]
The integer  $n$  is either even or it is odd.  We will show that it must be even by obtaining a contradiction if it assumed to be odd.  So, assume that  $n$  is odd.  (Now complete the proof.)

\item \textbf{The Last Two Digits of a Large Integer}.  \\
Notice that $\mod{7,381,272}{72}{100}$ since 
$7,381,272 - 72 = 7,381,200$, which is divisible by 100.  In general, if we start with an integer whose decimal representation has more than two digits and subtract the integer formed by the last two digits, the result will be an integer whose last two digits are 00.  This result will be divisible by 100.  Hence, any integer with more than 2 digits is congruent modulo 100 to the integer formed by its last two digits.
\begin{enumerate}
\item Start by squaring both sides of the congruence  $3^4  \equiv 81 \pmod {100}$ to prove that $\mod{3^8}{61}{100}$ and then prove that  
$3^{16}  \equiv 21 \pmod {100}$.  What does this tell you about the last two digits in the decimal representation of  $3^{16} $?
\label{A:lasttwo1}%

\item Use the two congruences in Part~(\ref{A:lasttwo1}) and laws of exponents to determine  $r$  where  $3^{20}  \equiv r \pmod {100}$ and  $r \in \mathbb{Z}$ with  $0 \leq r < 100$
. What does this tell you about the last two digits in the decimal representation of  $3^{20} $?


\item Determine the last two digits in the decimal representation of  $3^{400} $.

%\item Determine the last two digits in the decimal representation of  $3^{3356} $.


\item Determine the last two digits in the decimal representation of  $4^{804} $.

\hint  One way is to determine the  ``mod 100 values''  for  $4^2$, $4^4$, $4^8$, $4^{16}$, $4^{32}$, $4^{64}$,~and so on.  Then use these values and laws of exponents to determine 
$r$,  where  $4^{804}  \equiv r \pmod {100}$ and  $r \in \mathbb{Z}$ with  $0 \leq r < 100$.
\item Determine the last two digits in the decimal representation of  $3^{3356} $.
\item Determine the last two digits in the decimal representation of  $7^{403} $.
\end{enumerate}

\end{enumerate}


\hbreak
%\markboth{Chapter~\ref{C:proofs}. Constructing Proofs}{\ref{S:constructive}. Constructive Proofs}

\endinput
