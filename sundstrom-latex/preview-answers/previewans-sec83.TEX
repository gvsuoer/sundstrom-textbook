\documentclass[11pt]{article}
\usepackage{../../styles/activity}

\usepackage{xr}
\externaldocument{0-MR}

\lhead{}
%\chead{\textbf{\Large{\hspace{0pt}Beginning Activities for Section~8.3}}\\\hspace{0pt}\emph{Mathematical Reasoning: Writing and Proof}}
\bahead{8.3}
\rhead{}
\lfoot{}
\rfoot{}
\cfoot{\hspace{0pt}\scalebox{0.4}{\includegraphics{cc-by-nc-sa.eps}}}
\graphicspath{{./epsfigs/}}

\begin{document}
\subsection*{Beginning Activity 1 (Integer Solutions for Linear Equations)}
\begin{enumerate}
  \item An integer solution for the equation is $x = 7$ since $6 \cdot 7 = 42$.

\item An integer solution for the equation is $x =  - 3$ since 
$7\left( { - 3} \right) =  - 21$.

\item The equation has no solution that is an integer.  There is no integer such that $4x = 9$.

\item The equation has no solution that is an integer.  There is no integer such that $-3x = 20$.

\item \textbf{Theorem 8.18}.  \emph{Let $a, b \in \mathbb{Z}$ with $a \ne 0$. \label{prop:lindiophone}
\begin{itemize}
\item If $a$ does not divide $b$, then the equation $ax = b$ has no solution that is an integer.  
\item If $a$ divides $b$, then the equation $ax = b$ has exactly one solution that is an integer.
\end{itemize}}

\begin{myproof}
For the first statement, we will prove the contrapositive, which is ``If the equation 
$ax = b$ has a solution that is an integer, then  $a$ divides $b$.''

So we assume the equation $ax = b$ has a solution that is an integer.  This means there exists an integer  $q$  such that  $aq = b$, and this means that  $a$ divides $b$.  This completes the proof of the contrapositive.

For the second statement, if $a$ divides $b$, then there exists an integer $q$ such that 
$aq = b$, and this proves that there exists at least one solution of the equation $ax = b$ that is an integer.  So we assume that $x_1$ and $x_2 $ are two solutions of the equation  $ax = b$.  Then, $ax_1  = b$  and   $ax_2  = b$ and hence  $ax_1  = ax_2 $.  Since  
$a \ne 0$,this implies that $x_1  = x_2 $.  This proves there is only one solution of the equation.
\end{myproof}

\end{enumerate}
\hbreak



\newpage
\subsection*{Beginning Activity 2 (Exploring Linear Equations in Two Variables)}
\begin{enumerate}
\item There are no integers $x$ and $y$ such that $2x + 6y = 25$.  If there were such integers, the left side of the equation would be an even integer and the right side would be an odd integer.   This is impossible.

\item There are no integers $x$ and $y$ such that $6x - 9y = 100$.  If there were such integers, the left side of the equation would be an integer that is a  multiple of 3 and the right side would be an integer that is not a multiple of 3.   This is impossible.

\item \begin{enumerate}
\item Examples are:  $x = 12,y =  - 5$  and  $x = 17,y =  - 8$.

\item Examples are:  $x =  - 3,y = 4$  and  $x =  - 8,y = 7$.

\item One possibility is:
\[
\begin{aligned}
  x &= 2 + 5k \\ 
  y &= 1 - 3k \\ 
\end{aligned} 
\]
where $k$  is an integer.
\end{enumerate}

\item \begin{enumerate}
\item Examples are:  $x = 10,y =  - 4$  and  $x = 13,y =  - 6$.

\item Examples are:  $x = 1,y = 2$  and  $x =  - 2,y = 4$.

\item One possibility is:  
\[
\begin{aligned}
  x &= 4 + 3k \\ 
  y &= 0 - 3k \\ 
\end{aligned} 
\]
where $k$  is an integer.
\end{enumerate}
\end{enumerate}
\hbreak

\end{document}
