\subsection*{Section~\ref{S:infinitesets}}

\begin{list}{\bf{\ref{exer:sec93-1}.}}
\item All except Part~(d) are true.
\end{list}



\begin{list}{\bf{\ref{exer:sec93-2}.}}
\item \begin{enumerate}
\item Prove that the function $f: \mathbb{N} \to F^+$ defined by $f \left( n \right) = 5n$ for all 
$n \in \mathbb{N}$ is a bijection. 
\setcounter{enumi}{4}
\item  One way is to define $f\x \N \to \N -\{4, 5, 6 \}$ by
\begin{equation} \notag
f( n ) = 
\begin{cases}
n                        &\text{if $n = 1$, $n = 2$, or $n = 3$} \\
f ( n + 3 )   &\text{if $n \geq 4$}.
\end{cases}
\end{equation}
and then prove that the function $f$ is a bijection.

It is also possible to use Corollary~\ref{C:subsetofcountable} to conclude that $\mathbb{N} - \left\{ 4, 5, 6 \right\}$ is countable, but it must also be proved that $\mathbb{N} - \left\{ 4, 5, 6 \right\}$ cannot be finite.  To do this, assume that $\mathbb{N} - \left\{ 4, 5, 6 \right\}$ is finite and then prove that $\N$ is finite, which is a contradiction.
\item  Let $A = \left\{ m \in \mathbb{Z} \mid m \equiv 2 \pmod 3 \right\} = 
\left\{ 3k + 2 \mid k \in \mathbb{Z} \right\}$.  Prove that the function $f: \mathbb{Z} \to A$ is a bijection, where $f \left( x \right) = 3x + 2$ for all $x \in \mathbb{Z}$.  This proves that 
$\mathbb{Z} \approx A$ and hence, $\mathbb{N} \approx A$.
\end{enumerate}
\end{list}



\begin{list}{\bf{\ref{exer:addfinitetocountable}.}}
\item For each $n \in \mathbb{N}$, let $P ( n )$ be ``If 
$\text{card} ( B ) = n$, then $A \cup B$ is a countably infinite set.''

Note that if $\text{card} ( B ) = k+1$ and $x \in B$, then 
$\text{card} ( B - \left\{ x \right\} ) = k$.  Apply the inductive assumption to 
$B - \left\{ x \right\}$.
\end{list}



\begin{list}{\bf{\ref{exer:unionofcountable}.}}
\item Let $m, n \in \mathbb{N}$ and assume that $h \left( n \right) = h \left( m \right)$. Then since $A$ and $B$ are disjoint, either $h \left( n \right)$ and $h \left( m \right)$ are both in $A$ or are both in $B$.  If they are both in $A$, then both $m$ and $n$ are odd and
\[
f \left( \frac{n + 1}{2} \right) = h \left( n \right) = h \left( m \right) = f \left( \frac{m + 1}{2} \right).
\]
Since $f$ is an injection, this implies that $\dfrac{n + 1}{2} = \dfrac{m + 1}{2}$ and hence that $n = m$.  Similary, if both $h \left( n \right)$ and $h \left( m \right)$ are in $B$, then $m$ and $n$ are even and $g \left( \dfrac{n}{2} \right) = g \left( \dfrac{m}{2} \right)$, and since $g$ is an injection, $\dfrac{n}{2} = \dfrac{m}{2}$ and $n = m$.  Therefore, $h$ is an injection.

Now let $y \in A \cup B$.  There are only two cases to consider:  $y \in A$ or $y \in B$.  If 
$y \in A$, then since $f$ is a surjection, there exists an $m \in \mathbb{N}$ such that 
$f \left( m \right) = y$.  Let $n = 2m - 1$.  Then $n$ is an odd natural number, 
$m = \dfrac{n + 1}{2}$,  and
\[
h \left( n \right) = f \left( \frac{n + 1}{2} \right) = f \left( m \right) = y.
\]
Now assume $y \in B$ and use the fact that $g$ is a surjection to help prove that there exists a natural number $n$ such that $h(n) = y$.

%If $y \in B$, then since $g$ is a surjection, there exists an $m \in \mathbb{N}$ such that 
%$g \left( m \right) = y$.  Let $n = 2m$.  Then $n$ is an even natural number, 
%$m = \dfrac{n}{2}$,  and
%\[
%h \left( n \right) = g \left( \frac{n}{2} \right) = g \left( m \right) = y.
%\]
We can then conclude that $h$ is a surjection.
%\item Notice that if $h ( n ) = h ( m )$, then since $A$ and $B$ are disjoint, either $h ( n )$ and $h ( m )$ are both in $A$ or are both in $B$.
%
%Also, if $y \in A \cup B$, then there are only two cases to consider:  $y \in A$ or $y \in B$.
\end{list}


\begin{list}{\bf{\ref{exer:Qiscountable}.}}
\item By Theorem~\ref{T:positiverationals}, the set $\mathbb{Q}^+$ of positive rational numbers is countably infinite. So by Theorem~\ref{T:addonetocountable}, 
$\mathbb{Q}^+ \cup \left\{ 0 \right\}$ is countably infinite.  Now prove that the set $\mathbb{Q}^-$ of all negative rational numbers is countably infinite and then use  Theorem~\ref{T:unionofcountable} to prove that $\mathbb{Q}$ is countably infinite.

\end{list}

\begin{list}{\bf{\ref{exer:countinf-finite}.}}
\item Since $A - B \subseteq A$, the set $A - B$ is countable.  Now assume $A - B$ is finite and show that this leads to a contradiction.
\end{list}
\hbreak

\endinput


