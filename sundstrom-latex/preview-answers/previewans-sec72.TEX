\documentclass[11pt]{article}
\usepackage{../../styles/activity}

\usepackage{xr}
\externaldocument{0-MR}

\lhead{}
%\chead{\textbf{\Large{\hspace{0pt}Beginning Activities for Section~7.2}}\\\hspace{0pt}\emph{Mathematical Reasoning: Writing and Proof}}
\bahead{7.2}
\rhead{}
\lfoot{}
\rfoot{}
\cfoot{\hspace{0pt}\scalebox{0.4}{\includegraphics{cc-by-nc-sa.eps}}}
\graphicspath{{./epsfigs/}}

\begin{document}


\subsection*{Beginning Activity 1 (Properties of Relations)}
\begin{enumerate}
\item The relation  $R$  is not reflexive on  $A$  provided that there exists an  $x \in A$  such that  $\left( {x, x} \right) \notin R$ (or equivalently,  $x \not \mathrel{R} x$).

\item The relation  $R$ is not symmetric provided that there exist  $x, y \in A$ such that  
$\left( {x, y} \right) \in R$ and  $\left( {y, x} \right) \notin R$ (or  $x \mathrel{R} y$
 and  $y \not \mathrel{R} x$).

\item The relation  $R$ is not transitive provided that there exist  $x, y, z \in A$ such that  $\left( {x, y} \right) \in R$, $\left( {y, z} \right) \in R$, and  
$\left( {x, z} \right) \notin R$ (or  $x \mathrel{R} y$, $y \mathrel{R} z$, and  
$x \not \mathrel{R} z$).

\item The directed graph should have a loop at each of the four vertices, an arrow from 1 to 3, and an arrow from 3 to 2.  

The loops at each vertex show that for each $x \in A$, $x \mathrel{R} x$ and so the relation $\mathrel{R}$ is reflexive.

The relation $\mathrel{R}$ is not symmetric since $1 \mathrel{R} 3$ but $3 \not \mathrel{R} 1$, and the relation $\mathrel{R}$ is not transitive since $1 \mathrel{R} 3$ and $3 \mathrel{R} 2$ but $1 \not \mathrel{R} 2$.

\item The directed graph should have a loop at vertex 2, an arrow from 1 to 4, an arrow from 4 to 1, an arrow from 2 to 4 and  an arrow from 4 to 2.
The relation $\mathrel{T}$ is not reflexive since $a \in A$ and  $a \not \mathrel{T} a$. 

The relation $\mathrel{T}$ is symmetric since whenever there is an arrow from a vertex $a$ to a vertex $b$, there is an arrow from $b$ to $a$.  This means that for all $a, b \in A$, if $a \mathrel{T} b$, then $b \mathrel{T} a$ is a true conditional statement.

The relation $\mathrel{T}$ is not transitive since $1 \mathrel{T} 4$ and $4 \mathrel{T} 2$ but $1 \not \mathrel{T} 2$.
\end{enumerate}

\noindent
\textbf{Note}:  There are other properties of relations that we will not study in this text.  Two such properties are:

\begin{itemize}
  \item A relation $R$ on a set $A$ is \textbf{irreflexive} provided that for all $x, y \in A$, if $x \mathrel{R} y$, then $x \ne y$.  This means that no element in $A$ is related to itself.  One such example is the ``less than'' relation on the real numbers.
  \item A relation $R$ on a set $A$ is \textbf{antisymmetric} provided that for all $x, y \in A$, if $x \mathrel{R} y$ and $y \mathrel{R} x$, then $x = y$.    One such example is the ``less than or equal to'' relation on the real numbers.
\end{itemize}

%\begin{enumerate}
%\item The relation $\sim$ is:
%\begin{itemize}
%\item Reflexive on $\Q$.  For each $a \in \Q$, $a - a = 0$ and $0 \in \Z$, and therefore, 
%$a \sim a$.
%
%\item Symmetric.  Let $a, b \in \Q$ and assume that $a \sim b$.  This means that 
%$a - b \in \Z$.  Therefore, $-(a - b) \in \Z$ and this means that $b - a \in \Z$, and hence, $b \sim a$.
%
%\item Transitive.  Let $a, b, c \in \Q$ and assume that $a \sim b$ and $b \sim c$.  This means that 
%$a - b \in \Z$ and that $b - c \in \Z$.  Therefore, 
%$\left((a - b) + (b - c) \right) \in \Z$ and this means that $a - c \in \Z$, and hence, 
%$a \sim c$.
%\end{itemize}
%
%\item The relation $\approx$ is:
%\begin{itemize}
%\item Not reflexive on $\Q$.  For example, $\dfrac{1}{3} \in \Q$ and 
%$\dfrac{1}{3} + \dfrac{1}{3} \notin \Z$, and hence, $\dfrac{1}{3} \not \approx \dfrac{1}{3}$.
%
%\item Symmetric.  Let $a, b \in \Q$ and assume that $a \approx b$.  This means that 
%$a + b \in \Z$, and therefore, $b + a \in \Z$.  Hence, $b \approx a$.
%
%\item Not transitive.  For example, $\dfrac{1}{3} \approx \dfrac{2}{3}$ and 
%$\dfrac{2}{3} \approx \dfrac{4}{3}$, but $\dfrac{1}{3} \not \approx \dfrac{4}{3}$.
%\end{itemize}
%
%\end{enumerate}
\hbreak

\newpage
\noindent
\subsection*{Beginning Activity 2 (Review of Congruence Modulo $\boldsymbol{n}$)}
\begin{enumerate}
\item Let  $n \in \mathbb{N}$.  If  $a$  and  $b$  are integers, then we say that  $a$  is congruent to  $b$  modulo  $n$  provided that  $n$  divides  $a - b$.  So,
\begin{align} \notag
  a &\equiv b \pmod n \text{  means  } \left( {\exists k \in \mathbb{Z}} \right)\left( {a - b = nk} \right) \\ \notag
  a &\equiv b \pmod n \text{  means  } \left( {\exists k \in \mathbb{Z}} \right)\left( {a = b + nk} \right) \\ \notag
\end{align}

\item Let  $a, b \in \mathbb{Z}$ and let  $n \in \mathbb{N}$.  If  $a \equiv b \pmod n$, then we have determined an ordered pair  $\left( {a, b} \right) \in \mathbb{Z} \times \mathbb{Z}$.  So, congruence modulo  $n$  determines the following subset of  $\mathbb{Z} \times \mathbb{Z}$:
\[
\left\{ {\left. {\left( {a, b} \right) \in \mathbb{Z} \times \mathbb{Z}\,} \right| a \equiv b\left( {\bmod n} \right)} \right\}.
\]
So, this is a relation on  $\mathbb{Z}$.

\item Theorem 3.4  states that the relation of congruence modulo  $n$  is reflexive, symmetric, and transitive. 

\item \textbf{Theorem 3.31}.  Let  $n \in \mathbb{N}$ and let  $a \in \mathbb{Z}$.  If  
$a = nq + r\text{  and  }0 \leqslant r < n$ for some integers  $q$  and  $r$, then  
$a \equiv r \pmod n$.

\textbf{Corollary 3.32}.  If  $n \in \mathbb{N}$, then each integer is congruent, modulo $n$, to precisely one of the integers $0, 1, 2,  \ldots , n - 1$.

\item \textbf{Symmetric Property}: Let  $a, b \in \mathbb{Z}$ and let  $n \in \mathbb{N}$.  If  
$a \equiv b \pmod n$, then  $b \equiv a \pmod n$.

\begin{myproof}  Let  $a, b \in \mathbb{Z}$ and let  $n \in \mathbb{N}$ and assume that  
$a \equiv b \pmod n$.  We will prove that  $b \equiv a \pmod n$.  Since  
$a \equiv b \pmod n$, we know that  $n \mid \left( {a - b} \right)$ and hence that there exists an integer  $k$  such that 
\[
a - b = nk.
\]
If we multiply both sides of this equation by -1, we get
\[
\begin{aligned}
  \left( { - 1} \right)\left( {a - b} \right) &= \left( { - 1} \right)nk \\ 
                                        b - a &= n\left( { - k} \right). \\ 
\end{aligned}
\]
Since  $ - k \in \mathbb{Z}$, this last equation implies that  $n \mid \left( {b - a} \right)$ 
 and hence that  $b \equiv a \pmod n$.  We have thus proven that if  
$a \equiv b \pmod n$, then  $b \equiv a \pmod n$.  This means that the relation of congruence modulo  $n$  is a symmetric relation.
\end{myproof}
\hbreak
\end{enumerate}


\end{document}
