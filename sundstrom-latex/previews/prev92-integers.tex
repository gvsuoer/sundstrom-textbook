\begin{previewactivity}[\textbf{A Function from $\boldsymbol{\N}$ to $\boldsymbol{\Z}$}]\label{PA:functionNtoZ} \hfill \\
In this activity, we will define and explore a function $f:\mathbb{N} \to \mathbb{Z}$.  We will start by defining $f ( n )$ for the first few natural numbers $n$.

\begin{center}
\begin{tabular}[h]{l p{2cm} l}
$f ( 1 ) = 0$  &  &  \\
$f ( 2 ) = 1$  &  &  $f ( 3 ) = -1$ \\
$f ( 4 ) = 2$  &  &  $f ( 5 ) = -2$ \\
$f ( 6 ) = 3$  &  &  $f ( 7 ) = -3$\\
\end{tabular}
\end{center}
Notice that if we list the outputs of $f$ in the order 
$f ( 1 ), f ( 2 ), f ( 3 ), \ldots$,  we create the following list of integers:
$0, 1, -1, 2, -2, 3, -3, \ldots$.  We can also illustrate the outputs of this function with the following diagram:
\begin{figure}[h]
$$
\BeginTable
\BeginFormat
| c | c | c | c | c | c | c | c | c | c | c |
\EndFormat
" 1 " 2 " 3 " 4 " 5 " 6 " 7 " 8 " 9 " 10 " $\cdots$ " \\
" $\downarrow$ " $\downarrow$ " $\downarrow$ " $\downarrow$ " $\downarrow$ " $\downarrow$ " $\downarrow$ " $\downarrow$ " $\downarrow$ " $\downarrow$ " $\cdots$ "\\
" 0 " 1 " $-1$ " 2 " $-2$ " 3 " $-3$ " 4 " $-4$ " 5 "  $\cdots$ " \\
\EndTable
$$
\caption{A Function from $\N$ to $\Z$} \label{fig:functionNtoZ}
\end{figure}


\begin{enumerate}
\item If the pattern suggested by the function values we have defined continues, what are 
$f ( 11 )$ and $f ( 12 )$?  What is $f ( n )$ for $n$ from 13 to 16? 
\label{PA:functionNtoZ1}%

\item If the pattern of outputs continues, does the function $f$ appear to be an injection?  Does $f$ appear to be a surjection?  (Formal proofs are not required.)
\end{enumerate}

We will now attempt to determine a formula for $f ( n )$, where $n \in \mathbb{N}$.  We will actually determine two formulas:  one for when $n$ is even and one for when $n$ is odd.

\begin{enumerate} \setcounter{enumi}{2}
\item Look at the pattern of the values of $f ( n )$ when $n$ is even.  What appears to be a formula for $f ( n )$ when $n$ is even? 
\label{PA:functionNtoZ3}%

\item Look at the pattern of the values of $f ( n )$ when $n$ is odd.  What appears to be a formula for $f ( n )$ when $n$ is odd? 
\label{PA:functionNtoZ4}%

\item Use the work in Part~(\ref{PA:functionNtoZ3}) and Part~(\ref{PA:functionNtoZ4}) to complete the following:  Define $f\x \mathbb{N} \to \mathbb{Z}$, where
\[
f ( n ) = \left\{ \begin{gathered}
  ?? \text{  if  }n \text{ is even} \hfill \\
\\
  ?? \text{  if  }n \text{ is odd}. \hfill \\ 
\end{gathered}  \right.
\]
\label{PA:functionNtoZ5}%

\item Use the formula in Part~(\ref{PA:functionNtoZ5}) to
\begin{enumerate}
\item Calculate $f ( 1 )$ through $f ( 10 )$.  Are these results consistent with the pattern exhibited at the start of this activity?

\item Calculate $f ( 1000 )$ and $f ( 1001 )$.

\item Determine the value of $n$ so that $f ( n ) = 1000$.
\end{enumerate}
\end{enumerate} 
\end{previewactivity}
\hbreak

\endinput
