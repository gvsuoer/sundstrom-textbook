\documentclass[11pt]{article}
\usepackage{../../styles/activity}

\usepackage{xr}
\externaldocument{0-MR}

\lhead{}
%\chead{\textbf{\Large{\hspace{0pt}Beginning Activities for Section~7.1}}\\\hspace{0pt}\emph{Mathematical Reasoning: Writing and Proof}}
\bahead{7.1}
\rhead{}
\lfoot{}
\rfoot{}
\cfoot{\hspace{0pt}\scalebox{0.4}{\includegraphics{cc-by-nc-sa.eps}}}
\graphicspath{{./epsfigs/}}

\begin{document}
\subsection*{Beginning Activity 1 (The United States of America)}
\begin{enumerate}
\item \begin{enumerate}
\item $B = \left\{ {\text{Wisconsin, Indiana, Ohio}} \right\}$ or 
$B = \left\{ {\text{Wisconsin, Indiana, Ohio, Michigan}} \right\}$ depending on whether or not Michigan is considered to have a land border with itself.	

\item $C = \left\{ {\text{Wisconsin, Indiana, Ohio}} \right\}$ or 
$C = \left\{ {\text{Wisconsin, Indiana, Ohio, Michigan}} \right\}$ depending on whether or not Michigan is considered to have a land border with itself.

\item $D = \left\{ {\text{Michigan, Minnesota, Iowa, Illinois}} \right\}$ or 
$D = \left\{ {\text{Michigan, Minnesota, Iowa, Illinois, Wisconsin}} \right\}$ depending on whether or not Wisconsin is considered to have a land border with itself.	
\end{enumerate}

\item Two examples are:

$\left( {\text{Michigan, Wisconsin}} \right) \in R$, 
$\left( {\text{Wisconsin, Iowa}} \right) \in R$, but  
$\left( {\text{Michigan, Iowa}} \right) \notin R$.

$\left( {\text{Florida, Georgia}} \right) \in R$, 
$\left( {\text{Georgia, S}\text{. Carolina}} \right) \in R$, but  
$\left( {\text{Florida, S}\text{. Carolina}} \right) \notin R$.

\item The statement, ``For all $x, y \in A$, if $(x, y) \in R$, then $(y, x) \in R$.'' is true since if $x$ and $y$ have a land border in common, then $y$ and $x$ have a land border in common.
%\item The set  $R$  does not define a function from  $A$  to  $A$ since there are examples where  $\left( {x, y} \right) \in R$, $\left( {x, z} \right) \in R$, and  $y \ne z$.  For example,  $\left( {\text{Michigan, Wisconsin}} \right) \in R$ and  
%$\left( {\text{Michigan, Ohio}} \right) \in R$.
\end{enumerate}
\hbreak



\subsection*{Beginning Activity 2 (The Solution Set of an Equation with Two Variables)}
\begin{enumerate}
\item The ordered pairs  $\left( {2, 0} \right)$, $\left( { -2, 0} \right)$, 
$\left( {0, 4} \right)$, $\left( {0,  -4} \right)$, $\left( {1, \sqrt{12}} \right)$, 
$\left( {1, -\sqrt{12}} \right)$, $\left( {-1, \sqrt{12}} \right)$, and 
$\left( {-1, -\sqrt{12}} \right)$ are some of the ordered pairs in the solution set.

\item A point $(a, b)$ in the coordinate plane is on the graph of the equation 
$4x^2 + y^2 = 16$ if and only if it is a solution of the equation.  That is, if and only if  $4a^2 + b^2 = 16$.  In this sense, the graph of the equation is a representation of the solution set of this equation.

\item \begin{multicols}{2}
\begin{enumerate}
\item $A = \left[ -2, 2 \right] = \left\{ x \in \R \left| -2 \leq x \leq 2 \right. \right\}$.
\item $B = \left[ -4, 4 \right] = \left\{ y \in \R \left| -4 \leq y \leq 4 \right. \right\}$.
\end{enumerate}
Notice in each case, we used both interval notation and set builder notation.  Either one of these notations is correct.
\end{multicols}
\end{enumerate}
\hbreak



%\subsection*{Beginning Activity 3 (A Set of Ordered Pairs)}
%\begin{enumerate}
%\item The ordered pairs $(0, 0)$, $(1, 1)$, $(-1, 1)$, $(2, 4)$, $(-2, 4)$, 
%$(\sqrt{2}, 2)$, and $(-\sqrt{2}, 2)$ are some of the ordered pairs in the set $F$.
%
%\item \begin{multicols}{2}
%\begin{enumerate}
%\item $A = \left\{-2, 2 \right\}$.
%\item $B = \left\{ -\sqrt{10}, \sqrt{10} \right\}$.
%\item $C = \left\{ 25 \right\}$.
%\item $D = \left\{ 9 \right\}$.
%\end{enumerate}
%\end{multicols}
%
%\item The set $F$ can be used to define a function from $\R$ to $\R^*$ since the first coordinate of each ordered pair in $F$ is a real number and the second coordinate is a non-negative real number, and each real number is the first coordinate of only one ordered pair in $F$.
%\end{enumerate}
%\hbreak
%



\end{document}
