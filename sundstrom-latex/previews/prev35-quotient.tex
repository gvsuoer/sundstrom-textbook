\begin{previewactivity}[\textbf{Quotients and Remainders}]\label{PA:quotients} \hfill
\begin{enumerate}
  \item Let $a = 27$ and $b = 4$.  We will now determine several pairs of integers $q$ and $r$ so that 
$27 = 4q + r$.  For example, if $q = 2$ and $r = 19$, we obtain 
$4\cdot 2 + 19 = 27$.  The following table is set up for various values of $q$.  For each $q$, determine the value of $r$ so that $4q + r = 27$. \label{prev35-part1} 
$$
\BeginTable
\BeginFormat
| c | c | c | c | c | c | c | c | c | c | c | 
\EndFormat
\_
| $q$ | 1 | 2 | 3 | 4 | 5 | 6 | 7 | 8 | 9 | 10 | \\+33 \_
| $r$ |   | 19 |  |   |   |   |   |  $-5$ |      |      |  \\+33 \_
| $4q + r$ | 27  | 27  | 27 | 27 | 27 | 27 | 27 | 27 | 27 | 27 | \\+33 \_
\EndTable
$$

\item What is the smallest positive value for  $r$  that you obtained in your examples from Part~(\ref{prev35-part1})? 
\label{prev35-part2}%
\end{enumerate}

Division is not considered an operation on the set of integers since the quotient of two integers need not be an integer.  However, we have all divided one integer by another and obtained a quotient and a remainder.  For example, if we divide 113 by 5, we obtain a quotient of 22 and a remainder of 3.  We can write this as $\dfrac{113}{5} = 22 + \dfrac{3}{5}$.  If we multiply both sides of this equation by 5 and then use the distributive property to ``clear the parentheses,'' we obtain
\begin{align*}
5 \cdot \frac{113}{5} &= 5 \left( 22 + \frac{3}{5} \right) \\
                113   &= 5 \cdot 22 + 3
\end{align*}
This is the equation that we use when working in the integers since it involves only multiplication and addition of integers.
\setcounter{oldenumi}{\theenumi}
\begin{enumerate} \setcounter{enumi}{\theoldenumi}
\item What are the quotient and the remainder when we divide  27  by  4?  How is this related to your answer for Part~(\ref{prev35-part2})?

\item Repeat part~(\ref{prev35-part1}) using $a = -17$ and $b = 5$.  So the object is to find integers $q$ and $r$ so that $-17 = 5q + r$.  Do this by completing the following table. \label{PA:quotients4}
$$
\BeginTable
\BeginFormat
| c | c | c | c | c | c | c | c | 
\EndFormat
\_
| $q$ | $-7$ | $-6$ | $-5$ | $-4$ | $-3$ | $-2$ | $-1$ |  \\+33 \_
| $r$ | 18 |  |   |   |   | $-7$  |   |  \\+33 \_
| $5q + r$ | $-17$  | $-17$ | $-17$ | $-17$ | $-17$ | $-17$ | $-17$ |  \\+33 \_
\EndTable
$$
\item The convention we will follow is that the remainder will be the smallest positive integer 
$r$ for which $-17 = 5q + r$ and the quotient will be the corresponding value of $q$.  Using this convention, what is the quotient and what is the remainder when $-17$ is divided by 5? \label{PA:quotients5}
\end{enumerate}
\end{previewactivity}
\hbreak


\endinput
