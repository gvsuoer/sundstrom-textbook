%\textbf{Preview Activity 2.2 A - Truth Values of Statements} \\
\begin{previewactivity}[\textbf{Compound Statements}]\label{PA:compound} \hfill \\
Mathematicians often develop ways to construct new mathematical objects from existing mathematical objects.  It is possible to form new statements from existing statements by connecting the statements with words such as ``and'' and ``or'' or by negating the statement.  A \textbf{logical operator}
\index{logical operator}%
 (or \textbf{connective})
\index{connective}%
 on mathematical statements is a word or combination of words that combines one or more mathematical statements to make a new mathematical statement.  A \textbf{compound statement}
\index{compound statement}%
\index{statement!compound}%
 is a statement that contains one or more operators.  Because some operators are used so frequently in logic and mathematics, we give them names and use special symbols to represent them.
\begin{itemize}
  \item The \textbf{conjunction}
\index{conjunction}%
 of the statements $P$ and $Q$ is the statement ``$P$ \textbf{and} $Q$'' and its denoted by $P \wedge Q$ \label{sym:and} .  The statement $P \wedge Q$ is true only when both $P$ and $Q$ are true.
  \item The \textbf{disjunction}
\index{disjunction}%
 of the statements $P$ and $Q$ is the statement ``$P$ \textbf{or} $Q$'' and its denoted by $P \vee Q$ \label{sym:or}.  The statement $P \vee Q$ is true only when at least one of $P$ or $Q$ is true.
  \item The \textbf{negation} 
\index{negation}%
of the statement $P$ is the statement ``\textbf{not} $P$'' and is denoted by $\mynot P$ \label{sym:not}.  The negation of $P$ is true only when $P$ is false, and $\mynot P$ is false only when $P$ is true.  
  \item The \textbf{implication} 
\index{implication}%
\index{conditional}%
or \textbf{conditional} is the statement ``\textbf{If } $P$ \textbf{then} $Q$'' and is denoted by $P \to Q$ \label{sym:cond2}.  The statement $P \to Q$ is often read as ``$P$ \textbf{implies} $Q$,'' and we have seen in Section~\ref{S:prop} that $P \to Q$ is false only when $P$ is true and $Q$ is false.
\end{itemize}
\newpar
\textbf{Some comments about the disjunction}.  \\It is important to understand the use of the operator ``or.''  In mathematics, we use the \textbf{``inclusive or''}
\index{inclusive or}%
 unless stated otherwise.  This means that  $P \vee Q$ is true when both  $P$  and  $Q$  are true and also when only one of them is true.  That is, $P \vee Q$  is true when at least one of  $P$  or  $Q$  is true, or $P \vee Q$  is false only when both $P$  and  $Q$  are false.

A different use of the word ``or'' is the \textbf{``exclusive or.''}
\index{exclusive or}%
  For the exclusive or, the resulting statement is false when both statements are true. That is, ``$P$ exclusive or $Q$'' is true only when exactly one of  $P$  or  $Q$  is true.  In everyday life, we often use the exclusive or.  When someone says, ``At the intersection, turn left or go straight,'' this person is using the exclusive or.

\newpar
\textbf{Some comments about the negation}.  Although the statement, $\mynot P$, can be read as ``It is not the case that $P$,'' there are often better ways to say or write this in English.  For example, we would usually say (or write): 
\begin{itemize}
  \item The negation of the statement, ``391 is prime'' is  ``391 is not prime.''
  \item The negation of the statement, ``$12 < 9$'' is  ``$12 \geq 9$.''
\end{itemize}


\begin{enumerate}
  \item For the statements
\begin{center}
$P$: 15 is odd \qquad \qquad $Q$: 15 is prime
\end{center}
write each of the following statements as English sentences and determine whether they are true or false.  Notice that $P$ is true and $Q$ is false.
\begin{multicols}{4}
\begin{enumerate}
  \item $P \wedge Q$.
  \item $P \vee Q$.
  \item $P \wedge \mynot Q$.
  \item $\mynot P \vee \mynot Q$.
\end{enumerate}
\end{multicols}

  \item For the statements
\begin{center}
$P$:  15 is odd \qquad \qquad $R$: $15 < 17$
\end{center}
write each of the following statements in symbolic form using the operators $\wedge$, $\vee$, and 
$\mynot$.
\begin{multicols}{2}
\begin{enumerate}
  \item $15 \geq 17$.
  \item 15 is odd or $15 \geq 17$.
  \item 15 is even or $15 < 17$.
  \item 15 is odd and $15 \geq 17$.
\end{enumerate}
\end{multicols}
\end{enumerate}
\end{previewactivity}
\hbreak




\endinput
We will now learn how mathematicians and logicians create new statements from existing statements.  One common way to do this is to insert the word ``and'' or ``or'' between two existing statements $P$ and $Q$.  The \textbf{conjunction} of the statements $P$ and $Q$ is the statement
\[
P \textbf{ and } Q.
\]
Another common mathematical practice is to have a notation for new objects.  The conjunction of $P$ and $Q$ is denoted by $P \wedge Q$.
 tr
