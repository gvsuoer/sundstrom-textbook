\begin{previewactivity}[\textbf{Recursively Defined Sequences}] \label{PA:recursivesequences} \hfill \\
In a proof by mathematical induction, we ``start with a first step'' and then prove that we can always go from one step to the next step.  We can use this same idea to define a sequence as well.  We can think of a \textbf{sequence}
\index{sequence}%
 as an infinite list of numbers that are indexed by the natural numbers (or some infinite subset of  $\mathbb{N} \cup \left\{ 0 \right\}$).  We often write a sequence in the following form:
\[
a_1, a_2, \ldots,a_n, \ldots. 
\]
The number $a_n$ is called the $n^\text{th}$ term of the sequence.  One way to define a sequence is to give a specific formula for the $n^\text{th}$ term of the sequence such as 
$a_n = \dfrac{1}{n}$.

Another way to define a sequence is to  give a specific definition of the first term (or the first few terms) and then state, in general terms, how to determine  $a_{n + 1} $  in terms of  $n$  and the first  $n$  terms  $a_1 ,a_2 , \ldots ,a_n $.  This process is known as \textbf{definition by recursion} and is 
\index{definition!by recursion}%
  also called a \textbf{recursive definition}.  \label{recursivedef}
\index{recursive definition}%
The specific definition of the first term is called the \textbf{initial condition},
\index{initial condition}%
 and the general definition of  $a_{n + 1} $  in terms of  $n$  and the first  $n$  terms  $a_1 ,a_2 , \ldots ,a_n $ is called the \textbf{recurrence relation}.
\index{recurrence relation}%
  (When more than one term is defined explicitly, we say that these are the initial conditions.)  For example, we can define a sequence recursively as follows:
\begin{list}{}
\item $b_1  = 16$, and	for each  $n \in \mathbb{N}$,  $b_{n + 1}  = \dfrac{1}{2}b_n $.
\end{list}

\newpar
Using $n = 1$ and then $n = 2$, we then see that
\begin{align*}
b_2 &= \frac{1}{2} b_1 &    b_3 &= \frac{1}{2} b_2 \\
    &= \frac{1}{2} \cdot 16 &    &= \frac{1}{2} \cdot 8 \\
    &= 8                    &    &= 4
\end{align*}

\begin{enumerate}

\item Calculate  $b_4 $ through  $b_{10} $. What seems to be happening to the values of  $b_n $
as  $n$  gets larger? \label{PA:recursivesequences1}

\item Define a sequence recursively as follows: \label{PA:recursivesequences2}

\begin{list}{}
\item $T_1  = 16$, and	for each  $n \in \mathbb{N}$,  $T_{n + 1}  = 16 + \dfrac{1}{2}T_n $.
\end{list}

Then $T_2 = 16 + \dfrac{1}{2} T_1 = 16 + 8 = 24$.  Calculate  $T_3 $ through  $T_{10} $.  What seems to be happening to the values of  $T_n $ as  $n$  gets larger?

\end{enumerate}

\noindent
The sequences in Parts~(\ref{PA:recursivesequences1}) and~(\ref{PA:recursivesequences2}) can be generalized as follows:  Let  $a$  and  $r$  be real numbers.  Define two sequences recursively as follows:

\begin{list}{}
\item $a_1  = a$, and for each  $n \in \mathbb{N}$,  $a_{n + 1}  = r \cdot a_n $.

\item $S_1  = a$, and for each  $n \in \mathbb{N}$,  $S_{n + 1}  = a + r \cdot S_n $.
\end{list}

\begin{enumerate}
\setcounter{enumi}{2}
\item Determine  formulas (in terms of  $a$  and  $r$) for  $a_2 $ through  $a_6 $.  What do you think  $a_n $ is equal to (in terms of  $a$, $r$, and  $n$)?

\item Determine  formulas (in terms of  $a$  and  $r$) for  $S_2 $ through  $S_6 $.  What do you think  $S_n $ is equal to (in terms of  $a$, $r$, and  $n$)?

\end{enumerate}


In \typeu Activity~\ref*{PA:factorials} in Section~\ref{S:otherinduction}, for each natural number $n$, we defined  $n!$, read  \textbf{$n$  factorial}, \label{factorial2}
\index{factorial}%
 as the product of the first  $n$  natural numbers.  
We also defined  $0!$  to be equal to 1.  Now recursively define a sequence of numbers   $a_0 ,a_1 ,a_2 , \ldots $  as follows:
%
%\begin{center}
%\fbox{\parbox{4in}{
\begin{list}{}
\item $a_0  = 1$, and 
\item 
\item for each nonnegative integer  $n$,  $a_{n + 1}  = \left( {n + 1} \right) \cdot a_n $.
\end{list}
%}}
%\end{center}
%
\vskip10pt
\noindent
Using $n=0$, we see that this implies that $a_1 =1 \cdot a_0 = 1 \cdot 1 = 1$\!.  Then using $n = 1$, we see that
\[
a_2 = 2 a_1 = 2 \cdot 1 = 2.
\]
\setcounter{oldenumi}{\theenumi}
\begin{enumerate} \setcounter{enumi}{\theoldenumi}
\item Calculate  $a_3, a_4, a_5$, and $a_6 $.

\item Do you think that it is possible to calculate  $a_{20} $ and  $a_{100} $\!?  Explain.

\item Do you think it is possible to calculate  $a_n $ for any natural number  $n$?  Explain.

\item Compare the values of  $a_0 ,a_1 ,a_2 ,a_3 ,a_4 ,a_5$, and $a_6 $ with those of  \\$0!,1!,2!,3!,4!,5!$, and $6!$.  What do you observe?  We will use mathematical induction to prove a result about this sequence in 
Exercise~(\ref{exer:sec51-factorial}).
\end{enumerate}

\end{previewactivity}
\hbreak

\endinput
