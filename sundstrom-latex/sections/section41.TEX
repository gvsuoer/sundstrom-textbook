\section{The Principle of Mathematical Induction} \label{S:mathinduction}
\setcounter{previewactivity}{0}
%
\begin{previewactivity}[\textbf{Exploring Statements of the Form \mathversion{bold} $\left( \forall n \in \N \right)\left( P(n) \right)$}] \label{PA:exploringstatements} \hfill \\
%In Section~\ref{S:predicates}, we defined the \textbf{truth set}
%\index{truth set}%
% of a predicate 
%$P( x )$ to be the collection of objects in the universal set that make the predicate a true statement when substituted for  $x$.  
One of the most fundamental sets in mathematics is the set of natural numbers $\N$.  In this section, we will learn a new proof technique, called mathematical induction, that is often used to prove statements of the form 
$\left( \forall n \in \N \right)\left( P(n) \right)$.  In Section~\ref{S:otherinduction}, we will learn how to extend this method to statements of the form $\left( \forall n \in T \right)\left( P(n) \right)$, where $T$ is a certain type of subset of the integers $\Z$.

\newpar
For each natural number $n$, let $P(n)$ be the following open sentence:
\[
4 \text{ divides }  \left( {5^n  - 1} \right)\!.
\]
\begin{enumerate}
\item Does this open sentence become a true statement when  $n = 1$?  That is, is  1  in the truth set of $P(n)$?

\item Does this open sentence become a true statement when  $n = 2$?  That is, is  2  in the truth set of $P(n)$?

\item Choose at least four more natural numbers and determine whether the open sentence is true or false for each of your choices.
\end{enumerate}
All of the examples that were used should provide evidence that the following proposition is true:
\begin{center}
For each natural number  $n$, 4  divides  $\left( {5^n  - 1} \right)$.
\end{center}
We should keep in mind that no matter how many examples we try, we cannot prove this proposition with a list of examples because we can never check if 4 divides $\left( {5^n  - 1} \right)$ for every natural number $n$.  Mathematical induction will provide a method for proving this proposition.

\setcounter{equation}{0}
For another example, for each natural number $n$, we now let $Q( n )$ be the following open sentence:
\begin{equation} \label{eq:PAexplore}
1^2  + 2^2  + \, \cdots \, + n^2  = \frac{{n(n + 1)(2n + 1)}}{6}.
\end{equation}
The expression on the left side of the previous equation is the sum of the squares of the first  $n$  natural numbers.  So when  $n = 1$, the left side of equation~(\ref{eq:PAexplore}) is  
$1^2 $.  When  $n = 2$, the left side of equation~(\ref{eq:PAexplore}) is  $1^2  + 2^2 $.

\setcounter{oldenumi}{\theenumi}
\begin{enumerate} \setcounter{enumi}{\theoldenumi}
\item Does $Q ( n )$ become a true statement when
\begin{itemize}
\item  $n = 1$?  (Is  1  in the truth set of $Q ( n )$?)

\item $n = 2$?  (Is  2  in the truth set of $Q ( n )$?)

\item $n = 3$?  (Is  3  in the truth set of $Q ( n )$?)
\end{itemize}

\item Choose at least four more natural numbers and determine whether the open sentence is true or false for each of your choices.  A table with the columns $n$, $1^2  + 2^2  + \, \cdots \, + n^2$, and $\dfrac{{n(n + 1)(2n + 1)}}{6}$ may help you organize your work.
\end{enumerate}
All of the examples we have explored, should indicate the following proposition is true:
\begin{center}
For each natural number  $n$,  $1^2  + 2^2  + \, \cdots \, + n^2  = \dfrac{{n(n + 1)(2n + 1)}}{6}$.
\end{center}
In this section, we will learn how to use mathematical induction to prove this statement.
\end{previewactivity}
\hbreak


\endinput

\begin{previewactivity}[\textbf{A Property of the Natural Numbers}] \label{PA:propertyofN} \hfill \\
Intuitively, the natural numbers begin with the number  1, and then there is 2, then 3, then 4, and so on.  Does this process of ``starting with 1'' and ``adding 1 repeatedly'' result in all the natural numbers?  We will use the concept of an inductive set to explore this idea in this activity.

\begin{defbox}{D:inductiveset}{A set  $T$  that is a subset of  $\mathbb{Z}$ is an 
\textbf{inductive set}
\index{inductive set}%
 provided that for each integer $k$, if $k \in T$, then  $k + 1 \in T$.}
\end{defbox}

%Consider the following property for a set   $T$  that is a subset of  $\mathbb{Z}$, the set of all integers.
%\begin{center}
%\textbf{Property I:} For every  $k \in \mathbb{Z}$, if  $k \in T$, then  $k + 1 \in T$.
%\end{center}

\begin{enumerate}
  \item Carefully explain what it means to say that a subset $T$ of the integers $\Z$ is not an inductive set.  This description should use an existential quantifier.

\item Use the definition of an inductive set to determine which of the following sets are inductive sets and which are not.  Do not worry about formal proofs, but if a set is  not inductive, be sure to provide a specific counterexample that proves it is not inductive.

\begin{multicols}{2}
\begin{enumerate}
\item $A = \left\{ {1,2,3, \ldots ,20} \right\}$

\item The set of natural numbers, $\mathbb{N}$

\item $B = \left\{ { {n \in \mathbb{N}} \mid n \geq 5} \right\}$
 
\item $S = \left\{ { {n \in \mathbb{Z}} \mid n \geq  - 3} \right\}$
 
\item $R = \left\{ { {n \in \mathbb{Z}} \mid n \leq  100} \right\}$

\item The set of integers, $\mathbb{Z}$
\item The set of odd natural numbers.
\end{enumerate}
\end{multicols}

\item This part will explore one of the underlying mathematical ideas for a proof by induction.  Assume that  $T \subseteq \mathbb{N}$ and assume that  $1 \in T$ and that  $T$ is an inductive set.  Use the definition of an inductive set to answer each of the following:  \label{PA:propertyofN6}

\begin{multicols}{2}
\begin{enumerate}
  \item Is  $2 \in T$?  Explain.
  \item Is  $3 \in T$?  Explain.
  \item Is  $4 \in T$?  Explain.
  \item Is  $100 \in T$?  Explain.
  \item Do you think that  $T = \mathbb{N}$?  Explain.
\end{enumerate}
\end{multicols}

\end{enumerate}
\end{previewactivity}
\hbreak
\endinput

%
\subsection*{Inductive Sets}
The two open sentences in \typeu Activity~\ref*{PA:exploringstatements} appeared to be true for all values of  $n$  in the set of natural numbers, $\mathbb{N}$.  That is, the examples in this \typel activity provided evidence that the following two statements are true.  

\begin{itemize}
\item For each natural number $n$,  4  divides  $\left( {5^n  - 1} \right)$. \label{conjecture1}
\item For each natural number  $n$, $1^2  + 2^2  + \, \cdots \, + n^2  = \dfrac{{n(n + 1)(2n + 1)}}{6}$. \label{conjecture2}
\end{itemize}
One way of proving statements of this form uses the concept of an inductive set introduced in \typeu Activity~\ref*{PA:propertyofN}.  The idea is to prove that if one natural number makes the open sentence true, then the next one also makes the open sentence true.  This is how we handle the phrase ``and so on'' when dealing with the natural numbers.
%
In \typeu Activity~\ref*{PA:propertyofN}, we saw that the number systems $\N$  and  $\Z$ and other sets are inductive.  What we are trying to do is somehow distinguish  $\N$ from the other inductive sets.  The way to do this was suggested in Part~(\ref{PA:propertyofN6}) of \typeu Activity~\ref*{PA:propertyofN}.  Although we will not prove it, the following statement should seem true.

\begin{description}
\item [Statement 1:] For each subset  $T$ of $\N$,  if  $1 \in T$  and  $T$  is inductive, then  $T = \mathbb{N}$. \label{inductivestatement1}
\end{description}
Notice that the integers,  $\mathbb{Z}$, and the set  
$S = \left\{ {n \in \mathbb{Z} \mid n \geq  - 3} \right\}$ both contain  1  and both are inductive, but they both contain numbers other than natural numbers.  For example, the following statement is false:

\begin{description}
\item [Statement 2:] For each subset $T$ of $\Z$, if  $1 \in T$  and  $T$  is inductive, then  $T = \mathbb{Z}$. \label{inductivestatement2}
\end{description}
The set  $S = \left\{ {n \in \mathbb{Z} \mid n \geq  - 3} \right\} = \left\{ { -3, -2, -1,0,1,2,3, \ldots \: } \right\}$ is a counterexample that shows that this statement is false.
%

\begin{prog}[\textbf{Inductive Sets}] \label{prog:inductivesets} \hfill \\
Suppose that $T$ is an inductive subset of the integers.  Which of the following statements are true, which are false, and for which ones is it not possible to tell?

\begin{multicols}{2}
\begin{enumerate}
\item $1 \in T$ and $5 \in T$\!.

\item If $1 \in T$, then $5 \in T$\!.

\item If $5 \notin T$\!, then $2 \notin T$\!.

\item For each integer $k$, if $k \in T$\!, then $k + 7 \in T$.

\item For each integer $k$, $k \notin T$ or $k + 1 \in T$.

\item There exists an integer $k$ such that $k \in T$ and $k + 1 \notin T$\!.

\item For each integer $k$, if $k + 1 \in T$\!, then $k \in T$\!.

\item For each integer $k$, if $k + 1 \notin T$\!, then $k \notin T$\!.
\end{enumerate}
\end{multicols}
\end{prog}
\hbreak

\endinput

\subsection*{The Principle of Mathematical Induction}
Although we proved that Statement~(2) is false, in this text, we will not prove that Statement~(1) is true.  One reason for this is that we really do not have a formal definition of the natural numbers.  However, we should be convinced that Statement~(1) is true.  We resolve this by making Statement~(1) an axiom for the natural numbers so that this becomes one of the defining characteristics of the natural numbers.
\begin{flushleft}
\fbox{\parbox{4.68in}{
\textbf{The Principle of Mathematical Induction} \\
\index{Principle of Mathematical Induction}%
\index{mathematical induction!Principle}%
If  $T$  is a subset of  $\mathbb{N}$ such that

\begin{enumerate}
  \item $1 \in T$\!, and
  \item For every  $k \in \mathbb{N}$, if  $k \in T$\!, then  
$\left( {k + 1} \right) \in T$\!,
\end{enumerate}
then  $T = \mathbb{N}$.
}}
\end{flushleft}
%\hbreak
%
\subsection*{Using the Principle of Mathematical Induction}
The primary use of the Principle of Mathematical Induction is to prove statements of the form
\[
\left( {\forall n \in \mathbb{N}} \right)\left( {P\left( n \right)} \right),
\]
where  $P( n )$ is some open sentence.  Recall that a universally quantified statement like the preceding one is true if and only if the truth set  $T$  of the open sentence $P( n )$ is the set  $\N$.  So our goal is to prove that  $T = \N$, which is the conclusion of the Principle of Mathematical Induction.  To verify the hypothesis of the Principle of Mathematical Induction, we must

\begin{enumerate}
\item Prove that  $1 \in T$\!.  That is, prove that  $P( 1 )$ is true.

\item Prove that if  $k \in T$\!, then  $\left( {k + 1} \right) \in T$\!.  That is, prove that if  $P( k )$ is true, then  $P( {k + 1} )$ is true.

\end{enumerate}
The first step is called the \textbf{basis step}
\index{mathematical induction!basis step}%
\index{basis step}%
 or the \textbf{initial step}, and the second step is called the \textbf{inductive step}.
\index{mathematical induction!inductive step}%
\index{inductive step}%
  This means that a proof by mathematical induction will have the following form:
%\hbreak
\begin{flushleft}
\fbox{\parbox{4.68in}{
\textbf{Procedure for a Proof by Mathematical Induction} \\
\noindent
To prove:	$\left( {\forall n \in \mathbb{N}} \right)\left( {P\left( n \right)} \right)$

\begin{tabular}{r l}
                   &                                 \\
Basis step:        &    Prove  $P( 1 )$.  \\
                   &                                 \\
Inductive step:    &  Prove that for each  $k \in \mathbb{N}$, \\
                   &  if  $P( k )$ is true, then  $P( {k + 1} )$ is true. \\
\end{tabular}
\vskip10pt
\noindent
We can then conclude that  $P( n )$ is true for all  $n \in \mathbb{N}$.
}}
\end{flushleft}

%\hbreak
\newpar
Note that in the inductive step, we want to prove that the conditional statement ``for each $k \in \N$, if 
$P(k)$ then $P(k + 1)$''  is true.  So we will start the inductive step by assuming that   $P( k )$  is true.  This assumption is called the \textbf{inductive assumption}
\index{inductive assumption}%
 or the \textbf{inductive hypothesis.}
\index{inductive hypothesis}%

%\vskip6pt
The key to constructing a proof by induction is to discover how  $P( {k + 1} )$
 is related to  $P( k )$ for an arbitrary natural number  $k$.  For example, in 
\typeu Activity~\ref*{PA:exploringstatements}, one of the open sentences $P(n)$ was
\[
1^2  + 2^2  + \, \cdots \, + n^2  = \frac{{n(n + 1)(2n + 1)}}{6}.
\]
Sometimes it helps to look at some specific examples such as  $P( 2 )$ and  $P( 3 )$.  The idea is not just to do the computations, but to see how the statements are related.  This can sometimes be done by writing the details instead of immediately doing computations.
\begin{align*}
P(2) &\qquad \text{is} & 1^2  + 2^2  &= \dfrac{{2 \cdot 3 \cdot 5}}{6} \\
P(3) &\qquad \text{is} & 1^2  + 2^2  + 3^2  &= \dfrac{{3 \cdot 4 \cdot 7}}{6}
\end{align*}
%\begin{center}
%\begin{tabular}{l l l} \\
%$P( 2 )$  &  is  &	$1^2  + 2^2  = \dfrac{{2 \cdot 3 \cdot 5}}{6}$.  \\
%$P( 3 )$  & 	is  &	$1^2  + 2^2  + 3^2  = \dfrac{{3 \cdot 4 \cdot 7}}{6}$. \\
%\end{tabular}
%\end{center}
In this case, the key is the left side of each equation.  The left side of  $P( 3 )$
  is obtained from the left side of  $P( 2 )$  by adding one term, which is $3^2$.  This suggests that we might be able to obtain the equation for $P( 3 )$  by adding  
$3^2$  to both sides of  the equation in  $P( 2 )$.  Now for the general case, if  
$k \in \mathbb{N}$, we look at  $P( {k + 1})$ and compare it to  $P( k )$.
\begin{align*}
P(k)  &\quad \text{is} & 1^2  + 2^2  +  \cdots  + k^2  &= \dfrac{{k(k + 1)(2k + 1)}}{6} \\
P(k+1)  &\quad \text{is} & 1^2  + 2^2  +  \cdots  + \left( {k + 1} \right)^2  &= \dfrac{{\left( {k + 1} \right)\left[ {\left( {k + 1} \right) + 1} \right]\left[ {2\left( {k + 1} \right) + 1} \right]}}{6}
\end{align*}

%$P( k )$    is  	$1^2  + 2^2  +  \cdots  + k^2  = \dfrac{{k(k + 1)(2k + 1)}}{6}
%$.
%\vskip6pt
%$P( k+1 )$   	is  	$1^2  + 2^2  +  \cdots  + \left( {k + 1} \right)^2  = \dfrac{{\left( {k + 1} \right)\left[ {\left( {k + 1} \right) + 1} \right]\left[ {2\left( {k + 1} \right) + 1} \right]}}{6}$.
%\vskip6pt

\newpar
%The predicate $P( {k + 1} )$ was obtained by substituting  $n = k + 1$ into the equation for  $P( n )$.    
The key is to look at the left side of the equation for  $P( {k + 1} )$ and realize what this notation means.  It means that we are adding the squares of the first  $\left( {k + 1} \right)$ natural numbers.  This means that we can write
\[
1^2  + 2^2  + \, \cdots \, + \left( {k + 1} \right)^2  = 1^2  + 2^2  + \, \cdots \, + k^2  + \left( {k + 1} \right)^2.
\]
This shows us that the left side of the equation for  $P( {k + 1} )$ can be obtained from the left side of the equation for  $P( k )$ by adding  $\left( {k + 1} \right)^2 $.  This is the motivation for proving the inductive step in the following proof.
%\hbreak
%
\begin{proposition} \label{P:suminduction}
For each natural number  $n$, 
\[
1^2  + 2^2  + \, \cdots \, + n^2  = \dfrac{{n(n + 1)(2n + 1)}}{6}.
\]
\end{proposition}
%
\setcounter{equation}{0}
\begin{myproof}
We will use a proof by mathematical induction.  For each natural number $n$, we let
$P( n )$ be
\[
1^2  + 2^2  +  \cdots  + n^2  = \dfrac{{n(n + 1)(2n + 1)}}{6}.
\]
We first prove that $P ( 1 )$ is true.  Notice that  $\dfrac{{1\left( {1 + 1} \right)\left( {2 \cdot 1 + 1} \right)}}{6} = 1$.  This shows that
\[
1^2  = \dfrac{{1\left( {1 + 1} \right)\left( {2 \cdot 1 + 1} \right)}}{6},
\]
which proves that $P( 1 )$  is true.

\newpar
For the inductive step, we prove that for each $k \in \mathbb{N}$, if $P ( k )$ is true, then 
$P( k + 1 )$ is true.  So let  $k$  be a natural number and assume that  $P( k )$  is true.  That is, assume that
%
\begin{equation} \label{eq:suminduction1}
1^2  + 2^2  +  \cdots  + k^2  = \frac{{k(k + 1)(2k + 1)}}{6}.
\end{equation}
%
The goal now is to prove that  $P\left( {k + 1} \right)$ is true.  That is, it must be proved that
\begin{align} 
1^2  + 2^2  + \, \cdots \, + k^2 + (k + 1)^2  &= \frac{{(k + 1)\left[ {(k + 1) + 1} \right]\left[ {2(k + 1) + 1} \right]}}{6} \notag \\
                                        &= \frac{{(k + 1)\left( k + 2\right) \left( 2k + 3 \right)}}{6}. 
\label{eq:suminduction2}%
\end{align}
%
To do this, we add  $\left( {k + 1} \right)^2 $ to both sides of equation~(\ref{eq:suminduction1}) and algebraically rewrite the right side of the resulting equation.  This gives
%
\[
\begin{aligned}
  1^2  + 2^2  +  \cdots  + k^2  + (k + 1)^2  &= \frac{{k(k + 1)(2k + 1)}}{6} + (k + 1)^2  \\ 
                &= \frac{{k(k + 1)(2k + 1) + 6(k + 1)^2 }}{6} \\ 
                &= \frac{{(k + 1)\left[ {k(2k + 1) + 6(k + 1)} \right]}}{6} \\ 
                &= \frac{{(k + 1)\left( {2k^2  + 7k + 6} \right)}}{6} \\ 
                &= \frac{{(k + 1)(k + 2)(2k + 3)}}{6}. \\ 
                %&= \frac{{(k + 1)\left[ {(k + 1) + 1} \right]\left[ {2(k + 1) + 1} \right]}}{6}. \\ 
\end{aligned} 
\]
Comparing this result to equation~(\ref{eq:suminduction2}), we see that if  $P( k )$  is true, then  $P( {k + 1} )$ is true.  Hence, the inductive step has been established, and by  the Principle of Mathematical Induction, we have proved that for each natural number $n$, \linebreak
$1^2  + 2^2  + \, \cdots \, + n^2  = \dfrac{{n(n + 1)(2n + 1)}}{6}$.
\end{myproof}
\hbreak

\endinput

\input{subsections/section41/sub41-writing}
\subsection*{Summation Notation}
The result in Proposition~\ref{P:suminduction} could be written using summation notation as follows:
%
\begin{center}
For each natural number $n$, $\sum\limits_{j = 1}^n {j^2 }  = \dfrac{{n(n + 1)(2n + 1)}}{6}$.
\end{center}
%
\noindent
In this case, we use  $j$  for the index for the summation, and the notation 
%\[
$\sum\limits_{j = 1}^n {j^2 }$
%\]
tells us to add all the values of  $j^2 $ for  $j$  from  1  to  $n$, inclusive.  That is, 
\[
\sum\limits_{j = 1}^n {j^2 }  = 1^2  + 2^2  +  \cdots  + n^2.
\]
So in the proof of Proposition~\ref{P:suminduction}, we would let $P( n )$  be  
%\[
$\sum\limits_{j = 1}^n {j^2 }  = \dfrac{{n(n + 1)(2n + 1)}}{6}$,
%\]
and we would use the fact that for each natural number  $k$,
\[
\sum\limits_{j = 1}^{k + 1} {j^2 }  = \left( {\sum\limits_{j = 1}^k {j^2 } } \right) + \left( {k + 1} \right)^2 .
\]
\hbreak
\setcounter{equation}{0}

%
\begin{prog}[\textbf{An Example of a Proof by Induction}] \label{prog:indexample} \hfill
\begin{enumerate}
\item Calculate  $1 + 2 + 3 +  \cdots  + n$ and $\dfrac{n(n + 1)}{2}$ 
for several natural numbers  $n$. What do you observe?

\item Use mathematical induction to prove that 
$1 + 2 + 3 +  \cdots  + n = \dfrac{n \left( n + 1 \right)}{2}$.

To do this, let $P ( n )$ be the open sentence, 
``$1 + 2 + 3 +  \cdots  + n = \dfrac{n \left( n + 1 \right)}{2}$.''  For the basis step, notice that the equation $1 = \dfrac{1 \left( 1 + 1 \right)}{2}$ shows that 
$P( 1 )$ is true.  Now let $k$ be a natural number and assume that 
$P (k )$ is true.  That is, assume that
\[
1 + 2 + 3 +  \cdots  + k = \frac{k \left( k + 1 \right)}{2},
\]
and complete the proof.
\end{enumerate}
\end{prog}
\hbreak

\endinput

\subsection*{Some Comments about Mathematical Induction}
\setcounter{equation}{0}
\begin{enumerate}
\item The basis step is an essential part of a proof by induction.  See 
Exercise~(\ref{exer:basis}) for an example that shows that the basis step is needed in a proof by induction.


\item Exercise~(\ref{exer:circleregions}) provides an example that shows the inductive step is also an essential part of a proof by mathematical induction.

\item It is important to remember that the inductive step in an induction proof is a proof of a conditional statement.  Although we did not explicitly use the forward-backward process in the inductive step for Proposition~\ref{P:suminduction}, it was implicitly used in the discussion prior to Proposition~\ref{P:suminduction}.  The key question was, ``How does knowing the sum of the first  $k$  squares help us find the sum of the first  $\left( {k + 1} \right)$
 squares?''

%\item The proof in Activity~\ref{A:basisstep} is a legitimate proof of the proposition that if  $P(  k )$  is true, then  $P\left( {k + 1} \right)$ is true.  This is a true conditional statement.  The point is that even though this conditional statement is true, nothing has been proved about the individual statements  $P\left( 1 \right)$, $P\left( 2 \right)$, 
%$P\left( 3 \right)$, and so on.

\item When proving the inductive step in a proof by induction, the key question is, 
\begin{center}
How does knowing  $P(  k )$ help us prove  $P( {k + 1} )$?
\end{center}
\setcounter{equation}{0}
In Proposition~\ref{P:suminduction}, we were able to see that the way to answer this question was to add a certain expression to both sides of the equation given in  $P(  k )$. Sometimes the relationship between  $P(  k )$  and  $P( {k + 1} )$ is not as easy to see.  For example, in \typeu Activity~\ref*{PA:exploringstatements}, we explored the following proposition:
\begin{center}
For each natural number $n$,  4  divides  $\left( {5^n  - 1} \right)$.
\end{center}
%
This means that the open sentence, $P( n )$, is  ``4  divides  $\left( {5^n  - 1} \right)$.''  So in the inductive step, we assume  $k \in \mathbb{N}$ and that  4  divides  $\left( {5^k  - 1} \right)$.  This means that there exists an integer  $m$  such that
%
\begin{equation} \label{eq:5a}
5^k  - 1 = 4m.
\end{equation}
%
In the backward process, the goal is to prove that  4  divides  $\left( {5^{k + 1}  - 1} \right)$.  This can be accomplished if we can prove that there exists an integer  $s$  such that
\begin{equation} \label{eq:5b}
5^{k + 1}  - 1 = 4s.
\end{equation}
%
We now need to see if there is anything in equation~(\ref{eq:5a}) that can be used in equation~(\ref{eq:5b}).  The key is to find something in the equation $5^k - 1 = 4m$ that is related to something similar in the equation $5^{k + 1}  - 1 = 4s$.  In this case, we notice that
\[
5^{k + 1}  = 5 \cdot 5^k .
\]
So if we can solve $5^k  - 1 = 4m$  for  $5^k $, we could make a substitution for $5^k$.  This is done in the proof of the following proposition.
\end{enumerate}
%\hbreak
%
\begin{proposition} \label{P:divideinduction}
For every natural number  $n$,  4  divides  $\left( {5^n  - 1} \right)$.
\end{proposition}
%
\setcounter{equation}{0}
\begin{myproof}(Proof by Mathematical Induction)
For each natural number  $n$, let  $P( n )$ be ``4  divides  $\left( {5^n  - 1} \right)$.'' 
%\vskip10pt
%\noindent
We first prove that $P \left( 1 \right)$ is true.  Notice that when  
$n = 1$, $\left( {5^n  - 1} \right) = 4$.  Since  4  divides 4,  $P\left( 1 \right)$  is true.
%\vskip6pt
%\noindent

For the inductive step, we prove that for all $k \in \mathbb{N}$, if $P \left( k \right)$ is true, then $P \left( k + 1 \right)$ is true.  So let  $k$  be a natural number and assume that  $P(  k )$  is true.  That is, assume that
\[
4\text{  divides  }\left( {5^k  - 1} \right).
\]
This means that there exists an integer  $m$  such that  
\[
5^k  - 1 = 4m.
\]
Thus,
\begin{equation} \label{eq:5c}
5^k  = 4m + 1.
\end{equation}
In order to prove that  $P( k + 1 )$ is true, we must show that  4 divides $\left( {5^{k + 1}  - 1} \right)$.  Since  $5^{k + 1}  = 5 \cdot 5^k $, we can write
\begin{equation} \label{eq:5d}
5^{k + 1}  - 1 = 5 \cdot 5^k  - 1.
\end{equation}
We now substitute the expression for  $5^k $ from equation~(\ref{eq:5c})  into equation~(\ref{eq:5d}).  This gives
%
\begin{align}
  5^{k + 1}  - 1 &= 5 \cdot 5^k  - 1 \notag \\
                 &= 5( {4m + 1} ) - 1 \notag \\ 
                 &= ( {20m + 5} ) - 1 \notag \\ 
                 &= 20m + 4 \notag \\ 
                 &= 4( {5m + 1} ) \label{eq:5e} 
\end{align} 
%
Since  $\left( {5m + 1} \right)$ is an integer, equation~(\ref{eq:5e}) shows that 4 divides $\left( {5^{k + 1}  - 1} \right)$.  Therefore,  if  $P(  k )$ is true, then  $P( k + 1 )$ is true and the inductive step has been established.
Thus, by the Principle of Mathematical Induction, for every natural number  $n$,  4  divides  $\left( {5^n  - 1} \right)$.
\end{myproof}
%\hbreak

Proposition~\ref{P:divideinduction} was stated in terms of ``divides.''  We can use congruence to state a proposition that is equivalent to Proposition~\ref{P:divideinduction}.  The idea is that the sentence, 4 divides $\left(5^n - 1 \right)$ means that $\mod{ 5^n }{1}{4}$.  So the following proposition is equivalent to Proposition~\ref{P:divideinduction}.
\begin{proposition} \label{P:congruenceinduction}
For every natural number  $n$,  $\mod{ 5^n }{1}{4}$.
\end{proposition}
Since we have proved Proposition~\ref{P:divideinduction}, we have in effect proved 
Proposition~\ref{P:congruenceinduction}.  However, we could have proved 
Proposition~\ref{P:congruenceinduction} first by using the results in Theorem~\ref{T:propsofcong} on page~\pageref{T:propsofcong}.  This will be done in the next progress check.
\hbreak

\begin{prog}[\textbf{Proof of Proposition~\ref{P:congruenceinduction}}] \hfill \\
To prove Proposition~\ref{P:congruenceinduction}, we let $P(n)$ be $\mod{ 5^n }{1}{4}$ and notice that $P(1)$ is true since $\mod{5}{1}{4}$.  For the inductive step, let $k$ be a natural number and assume that $P(k)$ is true.  That is, assume that $\mod{ 5^k }{1}{4}$.
\begin{enumerate}
  \item What must be proved in order to prove that $P(k+1)$ is true?
  \item Since $5^{k+1} = 5 \cdot 5^k$, multiply both sides of the congruence $\mod{ 5^k }{1}{4}$ by 5.  The results in Theorem~\ref{T:propsofcong} on page~\pageref{T:propsofcong} justify this step.
  \item Now complete the proof that for each $k \in \N$, if $P(k)$ is true, then $P(k+1)$ is true and complete the induction proof of Proposition~\ref{P:congruenceinduction}.
\end{enumerate}
It might be nice to compare the proofs of Propositions~\ref{P:divideinduction} and~\ref{P:congruenceinduction} and decide which one is easier to understand.
\end{prog}


%\noindent
%\note  We proved Proposition~\ref{P:congruenceinduction} using mathematical induction so that we could practice constructing and writing a proof by induction.  However, there is another way to prove this result that uses the concept of congruence.  See Exercise~(\ref{exer:51-congruence}).
\hbreak

\endinput






%
%\begin{activity}[The Derivatives of an Exponential Function] \label{A:derivatives} \hfill
%
%Let  $a$  be a real number.  We will explore the derivatives of the function  \linebreak
%$f\left( x \right) = e^{ax} $.  By using the chain rule, we see that
%\[
%\frac{d}
%{{dx}}\left( {e^{ax} } \right) = ae^{ax}. 
%\]
%Recall that the second derivative of a function is the derivative of the derivative function.  Similarly, the third derivative is the derivative of the second derivative.
%
%\begin{enumerate}
%\item What is  $\dfrac{{d^2 }}{{dx^2 }}\left( {e^{ax} } \right)$, the second derivative of   
%$e^{ax} $?
%	
%\item What is  $\dfrac{{d^3 }}{{dx^3 }}\left( {e^{ax} } \right)$, the third derivative of   
%$e^{ax} $?
%
%\item Let  $n$  be a natural number.  Make a conjecture about the $n{\text{th}}$ derivative of the function  $f\left( x \right) = e^{ax} $.  That is,  what is  
%$\dfrac{{d^n }}{{dx^n }}\left( {e^{ax} } \right)$?  This conjecture should be written as a self-contained proposition including an appropriate quantifier.
%
%\item Use mathematical induction to prove that your conjecture.
%\end{enumerate}
%\end{activity}
%\hbreak
%%
%\begin{activity} [A Conjecture about Congruence Modulo 3] \label{A:mod3conjecture} \hfill
%
%In Section~\ref{S:directproof}, we defined congruence modulo  $n$  for a natural number  $n$.  For  $a,b \in \mathbb{Z}$,
%\[
%a \equiv b \pmod n \text{ means that } n  \text{ divides }\left( {a - b} \right).
%\]
%
%In Section~\ref{S:cases}, we used the Division Algorithm to prove that each integer is congruent, modulo  $n$, to precisely one of the integers  $0,1,2, \ldots ,n - 1$ (Corollary~\ref{C:congtorem}).
%
%\begin{enumerate}
%\item Find the value of  $r$  so that   $4 \equiv r \pmod 3$ and  $r \in \left\{ {0,1,2} \right\}$.
%
%\item Find the value of  $r$  so that   $4^2  \equiv r \pmod 3$ and  $r \in \left\{ {0,1,2} \right\}$.
%
%\item Find the value of  $r$  so that   $4^3  \equiv r \pmod 3$  and  $r \in \left\{ {0,1,2} \right\}$.
%
%\item For two other values of $n$, find the value of  $r$  so that   $4^n  \equiv r \pmod 3$  and  
%$r \in \left\{ {0,1,2} \right\}$.
%
%\item If  $n \in \mathbb{N}$, make a conjecture concerning the value of  $r$  where  \\
%$4^n  \equiv r \pmod 3$  and  $r \in \left\{ {0,1,2} \right\}$.  This conjecture should be written as a self-contained proposition including an appropriate quantifier.
%
%\item Use mathematical induction to prove your conjecture.
%
%\end{enumerate}
%\end{activity}
%\hbreak


\endinput

