\section*{Section \ref{S:directproof}}
\renewcommand{\labelenumi}{(\textbf{\alph{enumi}})}

\begin{list}{\bf{\ref{exer:sec31-prove}.}}
\item  \begin{list}{\bf{(a)}}
\item Since $a \mid b$ and $a \mid c$, there exist integers $m$ and $n$ such that $b = am$ and $c = an$.  Hence,
\begin{align*}
b - c &= am - an \\
      &= a(m - n)
\end{align*}
Since $m - n$ is an integer (by the closure properties of the integers), the last equation implies that $a$ divides $b - c$.

%Remember that to prove that $a \mid \left( b - c \right)$, you need to prove that there exists an integer $q$ such that $b - c = a \cdot q$.
\end{list}
\end{list}

\begin{list}{}
\item  \begin{list}{\bf{(b)}}
\item What do you need to do in order to prove that $n^3$ is odd?  Notice that if $n$ is an odd integer, then there exists an integer $k$ such that $n = 2k + 1$.  Remember that to prove that $n^3$ is an odd integer, you need to prove that there exists an integer $q$ such that $n^3 = 2q + 1$.

This can also be approached as follows:  If $n$ is odd, then  by Theorem~\ref{T:xyodd}, $n^2$ is odd.  Now use the fact that $n^3 = n \cdot n^2$.
\end{list}
\end{list}


\begin{list}{}
\item  \begin{list}{\bf{(c)}}
\item If 4 divides $(a - 1)$, then there exists an integer $k$ such that $a - 1 = 4k$ and so  
$a = 4k + 1$.  Use algebra to rewrite $\left( a^2 - 1 \right) = (4k + 1)^2 - 1$.
\end{list}
\end{list}


\begin{list}{\bf{2.}} 
\item \begin{enumerate}
\item The natural number $n = 9$ is a counterexample since $n$ is odd, $n > 3$, $n^2 - 1 = 80$ and 3 does not divide 80.
\addtocounter{enumi}{2}
\item The integer $a = 3$ is a counterexample since $a^2 - 1 = 8$ and $a - 1 = 2$.  Since 4 divides 8 and 4 does not divide 2, this is an example where the hypothesis of the conditional statement is true and the conclusion is false.
\end{enumerate}
\end{list}
 

\begin{list}{\bf{\ref{exer:3truefalse}.}}
\item \begin{enumerate} \setcounter{enumi}{1}
\item This statement is false.  One counterexample is $a = 3$ and $b = 2$ since this is an example where the hypothesis is true and the conclusion is false.
\addtocounter{enumi}{1}
\item This statement is false.  One counterexample is $n = 5$.  Since $n^2 - 4 = 21$ and $n - 2 = 3$, this is an example where the hypothesis of the conditional statement is true and the conclusion is false. 
\item Make sure you first try some examples.  How do you prove that an integer is an odd integer?
\item The following algebra may be useful.
\[
4 \left( 2m + 1 \right)^2 + 7 \left( 2m + 1 \right) + 6 = 16m^2 + 30m + 17. 
\]
\item This statement is false.  One counterexample is $a = 7$, $b = 1$, and $d = 2$.  Why is this a counterexample?
\end{enumerate}
\end{list}



\begin{list}{\bf{\ref{exer:diveach}.}}
\item \begin{list}{\bf{(a)}}  
\item If $xy = 1$, then $x$ and $y$ are both divisors of $1$, and the only divisors of $1$ are $-1$ and $1$.
\end{list}
\end{list}


\begin{list}{}
\item \begin{list}{\bf{(b)}}
\item Part~(a) is useful in proving this.
\end{list}
\end{list}



\begin{list}{\bf{\ref{exer:sec31-10}.}}
\item \textbf{Another hint}:  $\left( 4n + 3 \right) - 2 \left( 2n + 1 \right) = 1$.
\end{list}


\begin{list}{\bf{\ref{exer:congmod3}.}}
\item Assuming $a$ and $b$ are both congruent to 2 modulo 3, there exist integers $m$ and $n$ such that $a = 3m + 2$ and $b = 3n + 2$.  

 For part~(a), show that
\[
a + b - 1 = 3 \left( m + n + 1 \right).
\]
We can then conclude that 3 divides $(a + b) - 1$ and this proves that $\mod{a+b}{1}{3}$.

For part~(b), show that 
\[
a \cdot b - 1 = 3(3mn + 2m + 2n + 1).
\]
We can then conclude that 3 divides $a \cdot b- 1$ and this proves that $\mod{a \cdot b}{1}{3}$.
\end{list}

\begin{list}{\bf{\ref{exer:cong-props}.}}
\item \begin{description}
\item[(a)] Let $n \in \N$.  If $a$ is an integer, then $a - a = 0$ and $n$ divides 0.  Therefore, $\mod{a}{a}{n}$.
\item[(b)] Let  $n \in \mathbb{N}$, let $a, b \in \mathbb{Z}$ and assume that $\mod{a}{b}{n}$.  We will prove that $\mod{b}{a}{n}$.  Since $\mod{a}{b}{n}$, $n$ divides $(a - b)$ and so there exists an integer $k$ such that $a - b = nk$.  From this, we can show that
$b - a = n(-k)$ and so $n$ divides $(b - a)$.  Hence, if $\mod{a}{b}{n}$, then $\mod{b}{a}{n}$.
\end{description}
\end{list}



\begin{list}{\bf{\ref{exer:sec31-11}.}}
\item The assumptions mean that $n \mid \left( a-b \right)$ and that 
$n \mid \left( c-d \right)$.  Use these divisibility relations to obtain an expression that is equal to  $a$  and to obtain an expression that is equal to  $c$.  Then use algebra to rewrite the resulting expressions for $a + c$    and $a \cdot c$.
\end{list}
\hbreak
\renewcommand{\labelenumi}{\textbf{\arabic{enumi}.}}

\endinput
