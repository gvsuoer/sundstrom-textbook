\section*{Section \ref{S:direct}}

\begin{list}{\bf{\ref{exer:evenodd}.}}
\item \begin{list}{\bf{(a)}} \item 
\begin{tabular}[t]{|p{0.4in}|p{1.6in}|p{1.6in}|}
  \hline
  \textbf{Step}  &  \textbf{Know}  &  \textbf{Reason} \\ \hline
  $P$  &  $m$ is an even integer.  &  Hypothesis \\ \hline
  $P1$ &  There exists an integer $k$ such that $m = 2k$. &  Definition of an even integer \\ \hline
  $P2$  &  $m + 1 = 2k + 1$  &  Algebra \\ \hline
  $Q1$  &  There exists an integer $q$ such that $m +1 = 2q+1$.  &  Substitution of $k = q$ \\ \hline
  $Q$  &  $m + 1$ is an odd integer. &  Definition of an odd integer \\ \hline
\end{tabular}
\end{list}
\end{list}

%\begin{list}{}
%\item \begin{list}{\bf{(b)}} 
%\item Similar to Part~(a).  The main difference is that if there exist an integer $k$ such that 
%$m = 2k + 1$, then $m + 1 = \left(2k + 1 \right) + 1 = 2 \left( k + 1 \right)$.
%\end{list}
%\end{list}


\begin{list}{\bf{\ref{exer:evenoddadd}.}}
%\item \begin{list}{\bf{(a)}} 
%\item  We assume that $x$ and $y$ are even integers and will prove that $x+y$ is an even integer.  Since $x$ and $y$ are odd, there exist integers $m$ and $n$ such that 
%$x = 2m$ and $y = 2n$.  Then,
%\begin{align*}
%x + y &= 2m + 2n\\
%      &= 2 \left( m + n  \right)
%\end{align*}
%Since the integers are closed under addition, $\left( m + n \right)$ is an integer, and hence the last equation shows that $x + y$ is even.  Therefore, we have proven that if $x$ and $y$ are even integers, then $x+y$ is an even integer.
%\end{list}
%\end{list}
%
%
%\begin{list}{}
%\item \begin{list}{\bf{(b)}} 
%\item  We assume that $x$ is an even integer and $y$ is an odd integers and will prove that $x+y$ is an odd integer.  Since $x$ and $y$ are odd, there exist integers $m$ and $n$ such that 
%$x = 2m$ and $y = 2n+1$.  Then,
%\begin{align*}
%x + y &= \left( 2m \right) + \left( 2n + 1 \right) \\
%      &= 2m + 2n + 1 \\
%      &= 2 \left( m + n \right) + 1
%\end{align*}
%Since the integers are closed under addition, $\left( m + n + 1 \right)$ is an integer, and hence the last equation shows that $x + y$ is even.  Therefore, we have proven that if $x$ and $y$ are odd integers, then $x+y$ is an even integer.
%\end{list}
%\end{list}
%
%
%\begin{list}{}
\item \begin{list}{\bf{(c)}} 
\item  We assume that $x$ and $y$ are odd integers and will prove that $x+y$ is an even integer.  Since $x$ and $y$ are odd, there exist integers $m$ and $n$ such that 
$x = 2m + 1$ and $y = 2n+1$.  Then
\begin{align*}
x + y &= \left( 2m + 1 \right) + \left( 2n + 1 \right) \\
      &= 2m + 2n + 2 \\
      &= 2 \left( m + n + 1 \right).
\end{align*}
Since the integers are closed under addition, $\left( m + n + 1 \right)$ is an integer, and hence the last equation shows that $x + y$ is even.  Therefore, we have proven that if $x$ and $y$ are odd integers, then $x+y$ is an even integer.

%\begin{tabular}[t]{|p{0.4in}|p{1.6in}|p{1.6in}|}
%  \hline
%  \textbf{Step}  &  \textbf{Know}  &  \textbf{Reason} \\ \hline
%  $P$  &  $x$ and $y$ are odd integers.  &  Hypothesis \\ \hline
%  $P1$ &  There exist integers $m$ and $n$ such that $x = 2m+1$ and $y = 2n+1$.  &  Definition of an odd integer \\ \hline
%  $P2$  &  $x + y = \left( 2m+1 \right) + \left( 2n+1 \right)$  &  Substitution \\ \hline
%  $P3$  &  $x + y = 2m + 2n + 2$  &  Algebra  \\
%        &  $x + y = 2 \left( {m+n+1} \right)$  &  \\ \hline
%  $P4$  &  $\left( {m+n+1} \right)$ is an integer  &  Closure properties of the integers \\ \hline
%  $Q1$  &  There exists an integer $q$ such that $x + y = 2q$  & $q = m+n+1$  \\ \hline
%  $Q$  &  $x + y$ is an even integer. &  Definition of an even integer \\ \hline
%\end{tabular}
\end{list}
\end{list}


\begin{list}{\bf{\ref{exer:evenoddmult}.}}
\item \begin{list}{\bf{(a)}} \item
\begin{tabular}[t]{|p{0.4in}|p{1.6in}|p{1.6in}|}
  \hline
  \textbf{Step}  &  \textbf{Know}  &  \textbf{Reason} \\ \hline
  $P$  &  $m$ is an even integer and $n$ is an integer.  &  Hypothesis \\ \hline
  $P1$ &  There exists an integer $k$ $m = 2k$.    &  Definition of an even integer. \\ \hline
  $P2$  &  $m \cdot n = \left( 2k \right) n$  &  Substitution \\ \hline
  $P3$  &  $m \cdot n = 2 \left( {kn} \right)$  &  Algebra  \\ \hline
  $P4$  &  $\left( {kn} \right)$ is an integer  &  Closure properties of the integers \\ \hline
  $Q1$  &  There exists an integer $q$ such that $m \cdot n = 2q$  & $q = kn$.  \\ \hline
  $Q$  &  $m \cdot n$ is an even integer. &  Definition of an even integer. \\ \hline
\end{tabular}
\end{list}

\end{list}
%
%
\begin{list}{}
\item \begin{list}{\bf{(b)}}
\item Use Part (a) to prove this.
\end{list}
\end{list}

%\begin{list}{}
%\item \begin{list}{\bf{(c)}}
%\item Use Theorem~\ref{T:xyodd} to prove this.
%\end{list}
%\end{list}


\begin{list}{\bf{\ref{exer:5m+7}.}}
\item \begin{list}{\bf{(a)}} 
\item We assume that $m$ is an even integer and will prove that $5m + 7$ is an odd integer.  Since $m$ is an even integer, there exists an integer $k$ such that $m = 2k$.  Using substitution and algebra, we see that
\begin{align*}
5m + 7 &= 5(2k) + 7 \\
           &= 10k + 6 + 1 \\
           &= 2(5k + 3) + 1
\end{align*}
By the closure properties of the integers, we conclude that $5k + 3$ is an integer, and so the last equation proves that $5m + 7$ is an odd integer.

\textbf{Another proof}.  
By Part~(a) of Exercise~\ref{exer:evenoddadd}, $5m$ is an even integer.  Hence, by Part~(b) of 
Exercise~\ref{exer:evenoddadd}, $5m + 7$ is an even integer.
\end{list}
\end{list}
%
%\begin{list}{}
%\item \begin{list}{\bf{(b)}}
%\item We assume that $m$ is an odd integer and will prove that $5m + 7$ is an even integer.  By Theorem~\ref{T:xyodd}, $5m$ is an odd integer.  Hence, by Part~(c) of 
%Exercise~\ref{exer:evenoddadd}, $5m + 7$ is an even integer.
%\end{list}
%\end{list}
%
%\begin{list}{}
%\item \begin{list}{\bf{(c)}}
%\item We assume that $m$ and $n$ are odd integers and will prove that $mn + 7$ is an even integer.  By Theorem~\ref{T:xyodd}, $mn$ is an odd integer.  Hence, by Part~(c) of 
%Exercise~\ref{exer:evenoddadd}, $mn + 7$ is an even integer.
%\end{list}
%\end{list}


%\begin{list}{\bf{\ref{exer:3m2}.}}
%\item \begin{list}{\bf{(a)}} 
%\item We assume that $m$ is an even integer and will prove that $3m^2 + 2m + 3$ is an odd integer.  Since $m$ is even, there exists an integer $k$ such that $m = 2k$.  Hence,
%\begin{align*}
%3m^2 + 2m + 3 &= 3 \left( 2k \right)^2 + 2 \left( 2k \right) + 3 \\
%              &= 12k^2 + 4k + 3 \\
%              &= 2 \left( 6k^2 + 2k + 1 \right) + 1
%\end{align*}
%By the closure properties of the ingegers, $\left( 6k^2 + 2k + 1 \right)$ is an integer.  Hence, this proves that if $m$ is even, then $3m^2 + 2m + 3$ is an odd integer.
%\end{list}
%\end{list}
%

%\begin{list}{}
\begin{list}{\bf{\ref{exer:3m2}.}}
\item \begin{list}{\bf{(b)}} 
\item We assume that $m$ is an odd integer and will prove that $3m^2 + 7m + 12$ is an even integer.  Since $m$ is odd, there exists an integer $k$ such that $m = 2k +1$.  Hence,
\begin{align*}
3m^2 + 7m + 12 &= 3 \left( 2k + 1 \right)^2 + 7 \left( 2k + 1 \right) + 12 \\
              &= 12k^2 + 26k + 22 \\
              &= 2 \left( 6k^2 + 13k + 11 \right)
\end{align*}
By the closure properties of the integers, $\left( 6k^2 + 13k + 11 \right)$ is an integer.  Hence, this proves that if $m$ is odd, then $3m^2 + 7m + 12$ is an even integer.
\end{list}
\end{list}




\begin{list}{\bf{\ref{exer:sec12-7}.}}
\item \begin{list}{\bf{(a)}}
\item Prove that they are not zero and their quotient is equal to 1.
\end{list}
\end{list}
%
%\begin{list}{}
%\item \begin{list}{\bf{(b)}}
%\item Prove that it is greater than or equal to 0 and that it is less than or equal to 0.  Prove that its square is equal to 0.  Prove that its absolute value is equal to 0.
%\end{list}
%\end{list}
%
%\begin{list}{}
%\item \begin{list}{\bf{(c)}}
%\item Prove that the two lines have a common perpendicular.  Prove that a transversal cuts the lines so that the alternate interior angles are equal.
%\end{list}
%\end{list}

\begin{list}{}
\item \begin{list}{\bf{(d)}}
\item Prove that two of the sides have the same length.  Prove that the triangle has two congruent angles.  Prove that an altitude of the triangle is a perpendicular bisector of a side of the triangle.
\end{list}
\end{list}


%\begin{list}{\bf{\ref{exer:pythag}.}}
%\item We assume that $m$ is a real number and that $m$, $m+1$, and $m+2$ are the lengths of the three sides of a right triangle.  Then, by the Pythagorean Theorem, we see that
%\[
%m^2 + \left(m + 1 \right)^2 = \left( m + 2 \right)^2.
%\]
%Using algebra to rewrite this equation, we obtain
%\begin{align*}
%2m^2 + 2m + 1 &= m^2 + 4m + 4 \\
%m^2 - 2m - 3 &= 0 \\
%(m - 3)(m + 1) &= 0
%\end{align*}
%Sovling this last equation for $m$, we see that $m = 3$ or $m = -1$.  However, $-1$ is not a positive real number and hence, cannot be the length of a side of triangle.  Therefore, 
%$m = 3$.
%\end{list}


%\begin{list}{\bf{\ref{exer:sec12-8}.}}
%\item
%\begin{tabular}[t]{|p{0.4in}|p{1.6in}|p{1.6in}|}
%  \hline
%  \textbf{Step}  &  \textbf{Know}  &  \textbf{Reason} \\ \hline
%  $P$  &  $a$ and $b$ are non-negative real numbers and $a + b = 0$.  &  Hypothesis \\ \hline
%  $P1$ &  $\left( a + b \right) + \left( -b \right) = 0 + \left( -b \right)$        &  Add $\left( -b \right)$ to both sides of the equation. \\ \hline
%  $P2$ &  $a = -b$  &  Algebra. \\ \hline
%  $P3$ &  $a \geq 0$ and $-b \leq 0$ &  Hypothesis and $b \geq 0$. \\ \hline
%  $P4$ &  $a \geq 0$ and $a \leq 0$  &  Step $P3$ and substitution of $a = -b$. \\ \hline
%  $Q$  &  $a = 0$. &  Zero is the only real number greater than or equal to 0 and less than or equal to 0.\\ \hline
%\end{tabular}
%\end{list}


\begin{list}{\bf{\ref{exer:sec12-type}.}} 
\item \begin{list}{\bf{(a)}}
\item  Some examples of type 1 integers are $-5$, $-2$, 1, 4, 7, 10.
\end{list}
\end{list}

%\begin{list}{}
%\item \begin{list}{\bf{(b)}}
%\item  Some examples of type 2 integers are $-4$, $-1$, 2, 5, 8, 11.
%\end{list}
%\end{list}


\begin{list}{}
\item \begin{list}{\bf{(c)}}
\item All examples should indicate the proposition is true.
\end{list}
\end{list}


\begin{list}{\bf{\ref{exer:sec12-typeproof}.}}
\item \begin{list}{\bf{(a)}}
\item Let $a$ and $b$ be integers and assume that $a$ and $b$ are both type 1 integers.  Then, there exist integers $m$ and $n$ such that $a = 3m + 1$ and $b = 3n + 1$.  Now show that
\[
a + b = 3 \left( m + n \right) + 2.
\]
The closure properties of the integers imply that $m + n$ is an integer.  Therefore, the last equation tells us that $a + b$ is a type 2 integer.  Hence, we have proved that if $a$ and $b$ are both type 1 integers, then $a + b$ is a type 2 integer.
\end{list}
\end{list}


%\begin{list}{}
%\item \begin{list}{\bf{(b)}}
%\item Similar to Part~(a) except:   If there exist integers $m$ and $n$ such that $a = 3m + 2$ and $b = 3n + 2$, then
%\[
%\begin{aligned}
%a + b &= \left( 3m + 2 \right) + \left( 3n + 2 \right) \\
%      &= 3m + 3n + 4 \\
%      &= 3m + 3n + 3 + 1 \\
%      &= 3 \left( m + n + 1 \right) + 1
%\end{aligned}
%\]
%\end{list}
%\end{list}
%
%
%\begin{list}{}
%\item \begin{list}{\bf{(c)}}
%\item Similar to Part~(a) except:   If there exist integers $m$ and $n$ such that $a = 3m + 1$ and $b = 3n + 2$, then
%\[
%\begin{aligned}
%a \cdot b &= \left( 3m + 1 \right) \cdot \left( 3n + 2 \right) \\
%      &= 9mn + 6m + 3n + 2 \\
%      &= 3 \left( 3mn + 2m + n \right) + 2
%\end{aligned}
%\]
%\end{list}
%\end{list}
%
%
%\begin{list}{}
%\item \begin{list}{\bf{(d)}}
%\item Similar to Part~(a) except:   If there exist integers $m$ and $n$ such that $a = 3m + 2$ and $b = 3n + 2$, then
%\[
%\begin{aligned}
%a \cdot b &= \left( 3m + 2 \right) \cdot \left( 3n + 2 \right) \\
%      &= 9mn + 6m + 6n + 4 \\
%      &= 9mn + 6m + 6n + 3 + 1 \\
%      &= 3 \left( 3mn + + 2m + 2n + 1 \right) + 1
%\end{aligned}
%\]
%\end{list}
%\end{list}



%\begin{list}{\bf{\ref{exer:sec12-11}.}}
%\item \begin{list}{\bf{(a)}}
%\item $x = \dfrac{-b \pm \sqrt{b^2 - 4ac}}{2a}$
%\end{list}
%\end{list}
%
%\begin{list}{}
%\item \begin{list}{\bf{(b)}}
%\item The key is that if $a > 0$ and $c < 0$, then $-4ac > 0$ and hence 
%\[
%b^2 - 4ac = b^2 + \left( -4ac \right) > 0.
%\]
%In addition, $b^2 - 4ac > b^2$ and so, $\sqrt{b^2 - 4ac} > \left| b \right|$.  Thus, \\
%$-b + \sqrt{b^2 - 4ac} > 0$, and hence, $\dfrac{-b + \sqrt{b^2 - 4ac}}{2a} > 0$.
%\end{list}
%\end{list}


%\begin{list}{}
%\item \begin{list}{\bf{(c)}}
%\item The key is that if $\dfrac{b}{2} < \sqrt{ac}$, then $\dfrac{b^2}{4} < ac$, and this implies that $b^2 - 4ac < 0$.  
%\end{list}
%\end{list}



\hbreak
