\section*{Section~\ref{S:inversefunctions}}

\subsection*{Progress Check~\ref{prog:ordered}}
Neither set can be used to define a function.
\begin{enumerate}
\item The set $F$ does not satisfy the first condition of Theorem~\ref{T:functionasordered}.
\item The set $G$ does not satisfy the second condition of Theorem~\ref{T:functionasordered}.
\end{enumerate}



\subsection*{Progress Check~\ref{prog:exploringinverse}}
\begin{enumerate} \setcounter{enumi}{1}
\item \begin{multicols}{2}
$f^{ - 1}  = \left\{ {( {r, a} ), ( {p, b} ), ( {q, c} )} \right\}$

$g^{ - 1}  = \left\{ {( {p, a} ), ( {q, b} ), ( {p, c} )} \right\}$

$h^{ - 1}  = \left\{ {( {p, a} ), ( {q, b} ), ( {r, c} ), ( {q, d} )} \right\}$
\end{multicols}

\item \begin{enumerate}
\item $f^{ - 1} $ is a function from  $C$  to  $A$.  

\item $g^{ - 1} $ is not a function from  $C$  to  $A$  since  
$( {p, a} ) \in g^{ - 1} $and  $( {p, c} ) \in g^{ - 1} $.

\item $h^{ - 1} $  is not a function from  $C$  to  $B$  since  
$( {q, b} ) \in h^{ - 1} $  and  $( {q, d} ) \in h^{ - 1} $.
\end{enumerate}

\addtocounter{enumi}{1}
\item In order for the inverse of a function  $F:S \to T$ to be a function  from  $T$  to  $S$, the function  $F$  must be a bijection.
\end{enumerate}

\hbreak
\newpage

\endinput

