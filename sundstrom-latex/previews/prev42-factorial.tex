\begin{previewactivity}[\textbf{Exploring a Proposition about Factorials}] \label{PA:factorials} \hfill

\begin{defbox}{nfactorial}{If  $n$  is a natural number, we define \textbf{$\boldsymbol{n}$  factorial},
\index{factorial}%
 denoted by $n!$ \label{sym:factorial},  to be the product of the first  $n$  natural numbers.  In addition, we define  $0!$ to be equal to  1.}
\end{defbox} 
Using this definition, we see that
\begin{align*}
0! &= 1  &  3! &= 1 \cdot 2 \cdot 3 = 6 \\
1! &=1   &  4! &= 1 \cdot 2 \cdot 3 \cdot 4 = 24\\
2! &= 1 \cdot 2 = 2  &  5! &= 1 \cdot 2 \cdot 3 \cdot 4 \cdot 5 = 120.
\end{align*}
%
\noindent
In general, we write  $n! = 1 \cdot 2 \cdot 3   \cdots \left( {n - 1} \right) \cdot n$ or  $n! = n \cdot \left( {n - 1} \right)  \cdots 2 \cdot 1$.  Notice that  for any natural number $n$, $n! = n \cdot (n-1)!$.
%
\begin{enumerate}
\item Compute the values of $2^n$ and $n!$ for each natural number  $n$  with  $1 \leq n \leq 7$. 
\label{pa:5211}
\end{enumerate}
Now let $P(n)$ be the open sentence, ``$n! > 2^n$.''
\setcounter{oldenumi}{\theenumi}
\begin{enumerate} \setcounter{enumi}{\theoldenumi}
  \item Which of the statements $P(1)$ through $P(7)$ are true?
\item Based on the evidence so far, does the following proposition appear to be true or false? \label{pa:5212}
For each natural number  $n$ with $n \geq 4$,
$n! > 2^n $.
\end{enumerate}
\setcounter{equation}{0}
Let $k$ be a natural number with $k \geq 4$.  Suppose that we want to prove that if $P(k)$ is true, then $P(k+1)$ is true.  (This could be the inductive step in an induction proof.)  To do this, we would be assuming that $k! > 2^k$ and would need to prove that $(k+1)! > 2^{k+1}$.  Notice that if we multiply both sides of the inequality $k! > 2^k$ by $(k + 1)$, we obtain
\begin{equation} \label{pa:52inequality}
(k + 1)\cdot k! > (k + 1) 2^k. 
\end{equation}
\setcounter{oldenumi}{\theenumi}
\begin{enumerate} \setcounter{enumi}{\theoldenumi}
\item In the inequality in~(\ref{pa:52inequality}), explain why $(k + 1) \cdot k! = (k + 1)!$.
\item Now look at the right side of the inequality in~(\ref{pa:52inequality}).  Since we are assuming that $k \geq 4$, we can conclude that $(k+1) > 2$.  Use this to help explain why $(k + 1)2^k > 2^{k+1}$.
\item Now use the inequality in~(\ref{pa:52inequality}) and the work in steps~(4) and~(5) to explain why 
$(k+1)! > 2^{k+1}$.
\end{enumerate}

%\item If the proposition in Part~(\ref{pa:5212}) is true, construct a proof.  If it is false, rewrite it (by adding a condition on the natural number  $n$) so that the new proposition appears to be true (based on the data in Part~(\ref{pa:5211})).
\end{previewactivity}
\hbreak

\endinput

