\section{Proof by Contradiction}\label{S:contradiction}
%\markboth{Chapter~\ref{C:proofs}. Constructing Proofs}{\ref{S:contradiction}. Proof by Contradiction}
\setcounter{previewactivity}{0}
%
\begin{previewactivity}[\textbf{Proof by Contradiction}]\label{PA:contradicton} \hfill
\index{proof!by contradiction}%

On page~\pageref{D:tautology} in Section~\ref{S:logop}, we defined
%\begin{defbox}{D:tautology}{A \textbf{tautology}
%\index{tautology}%
% is a propositional expression that yields a true statement regardless of what statements replace its variables. A \textbf{contradiction}
%\index{contradiction}%
% is a propositional expression that yields a false statement regardless of what statements replace its variables.}
%\end{defbox}
a \textbf{tautology}
\index{tautology}%
 to be a compound statement $S$ that is true for all possible combinations of truth values of the component statements that are part of $S$.  We also defined  \textbf{contradiction}
\index{contradiction}%
 to be a compound statement that is false for all possible combinations of truth values of the component statements that are part of $S$.


That is, a tautology is necessarily true in all circumstances, and a contradiction is necessarily false in all circumstances.

\begin{enumerate}
\item Use truth tables to explain why  $\left( {P \vee \mynot  P} \right)$ is a tautology and $\left( {P \wedge \mynot  P} \right)$ is a contradiction.
%\item Use a truth table to explain why  $\left( {P \wedge \mynot  P} \right)$ is a contradiction.
\end{enumerate}

Another method of proof that is frequently used in mathematics is a 
\textbf{proof by contradiction.}
\index{proof!by contradiction|(}%
  This method is based on the fact that a statement  $X$  can only be true or false (and not both).  The idea is to prove that the statement $X$  is true by showing that it cannot be false.  This is done by assuming that  $X$  is false and proving that this leads to a contradiction.  (The contradiction often has the form  
$\left( {R \wedge \mynot  R} \right)$, where  $R$  is some statement.)  When this happens, we can conclude that the assumption that the statement  $X$  is false is incorrect and hence $X$  cannot be false.  Since it cannot be false, then 
$X$  must be true.

A logical basis for the contradiction method of proof is the tautology
\[
\left[ \mynot  X \to C \right] \to X,
\]
where $X$ is a statement and $C$ is a contradiction.  The following truth table establishes this tautology.

$$
\BeginTable
    \BeginFormat
    | c | c | c | c | c |
    \EndFormat
     \_6
      | $X$ | $C$ \|6 $\mynot X$ | $\mynot X \to C$ | $(\mynot X \to C) \to X$ | \\+22 \_6
      | T   |  F  \|6  F | T | T | \\ 
      | F   |  F  \|6  T | F | T | \\ \_6
 \EndTable
$$

\noindent
This tautology shows that  if  $\mynot X$  leads to a contradiction, then  $X$  must be true.  The previous truth table also shows that the statement $\mynot  X \to C$  is logically equivalent to  $X$.  This means that if we have proved that  
$\mynot X$ leads to a contradiction, then we have proved statement  $X$.  So if we want to prove a statement $X$ using a proof by contradiction, we assume that $\mynot X$ is true and show that this leads to a contradiction.  

When we try to prove  the conditional statement, \lq\lq If $P$ then $Q$\rq\rq~using a proof by contradiction, we must assume that  $P \to Q$ is false and show that this leads to a contradiction.   
\begin{enumerate} \setcounter{enumi}{1}
  \item Use a truth table to show that $\mynot  \left( {P \to Q} \right)$ is logically equivalent to  
$P \wedge \mynot  Q$.
\end{enumerate}
The preceding logical equivalency shows that when we assume that  \mbox{$P \to Q$} is false, we are assuming that  $P$  is true and  $Q$  is false.  If we can prove that this leads to a contradiction, then we have shown that $\mynot  \left( {P \to Q} \right)$ is false and hence that  $P \to Q$ is true.

\begin{enumerate} \setcounter{enumi}{2}
\item Give a counterexample to show that the following statement is false.
\label{LA:contradiction2-1}%
  \begin{center}
   For each real number  $x$, $\dfrac{1}{x(1 - x)} \geq 4$.
  \end{center}
\item When a statement is false, it is sometimes possible to add an assumption that will yield a true statement. This is usually done by using a conditional statement.  So instead of working with the statement in~(\ref{LA:contradiction2-1}), we will work with a related statement that is obtained by adding an assumption (or assumptions) to the hypothesis.
  \begin{center}
   For each real number  $x$, if $0 < x < 1$, then $\dfrac{1}{x(1 - x)} \geq 4$.
  \end{center}
To begin a proof by contradiction for this statement, we need to assume the negation of the statement.  To do this, we need to negate the entire statement, including the quantifier.  Recall that the negation of a statement with a universal quantifier is a statement that contains an existential quantifier.  (See Theorem~\ref{T:negations} on page~\pageref{T:negations}.)  With this in mind, carefully write down all assumptions made at the beginning of a proof by contradiction for this statement.
\end{enumerate}
\index{proof!by contradiction|)}%



\end{previewactivity}
\hbreak
%
\endinput

\begin{previewactivity}[\textbf{Constructing a Proof by Contradiction}]\label{LA:contradicton2} \hfill \\
Consider the following proposition:

\newpar
\textbf{Proposition}.  For all real numbers  $x$  and  $y$, if  $x \ne y$, $x > 0,\text{ and }y > 0$, then  $\dfrac{x}{y} + \dfrac{y}{x} > 2.$

\newpar
To start a proof by contradiction, we assume that this statement is false; that is, we assume the negation is true.  Because this is a statement with a universal quantifier, we assume that there exist real numbers $x$  and  $y$  such that  $x \ne y$, $x > 0, y > 0$ and that  
$\dfrac{x}{y} + \dfrac{y}{x} \leq 2.$  (Notice that the negation of the conditional sentence is a conjunction.)

For this proof by contradiction, we will only work with the know column of a know-show table.  This is because we do not have a specific goal.  The goal is to obtain some contradiction, but we do not know ahead of time what that contradiction will be.  Using our assumptions, we can perform algebraic operations on the inequality
\begin{equation} \label{LA:learn33-inequality}
\frac{x}{y} + \frac{y}{x} \leq 2
\end{equation}
until we obtain a contradiction.  
\begin{enumerate}
  \item Try the following algebraic operations on the inequality in~(\ref{LA:learn33-inequality}).  First, multiply both sides of the inequality by $xy$, which is a positive real number since $x > 0$ and $y > 0$.  Then, subtract $2xy$ from both sides of this inequality and finally, factor the left side of the resulting inequality.
  \item Explain why the last inequality you obtained leads to a contradiction.
\end{enumerate}
By obtaining a contradiction, we have proved that the proposition cannot be false, and hence, must be true.  
%A completed proof of this proposition is given in Proposition~\ref{P:contradiction} on page 
%\pageref{P:contradiction}.
\end{previewactivity}
\hbreak


\endinput

%
\subsection*{Writing Guidelines: Keep the Reader Informed}
\index{writing guidelines}%

A very important piece of information about a proof is the method of proof to be used.  So when we are going to prove a result using the contrapositive or a proof by contradiction, we indicate this at the start of the proof.

\begin{itemize}
  \item We will prove this result by proving the contrapositive of the statement.
  \item We will prove this statement using a proof by contradiction.
  \item We will use a proof by contradiction.
\end{itemize}
We have discussed the logic behind a proof by contradiction in the \typel activities for this section.  The basic idea for a proof by contradiction of a proposition is to assume the proposition is false and show that this leads to a contradiction.  We can then conclude that the proposition cannot be false, and hence, must be true.  When we assume a proposition is false, we are, in effect, assuming that its negation is true.  This is one reason why it is so important to be able to write negations of propositions quickly and correctly.  We will illustrate the process with the proposition discussed in \typeu Activity~\ref*{PA:contradicton}.

\begin{proposition}
For each real number  $x$, if $0 < x < 1$, then $\dfrac{1}{x(1 - x)} \geq 4$.
\end{proposition}

\setcounter{equation}{0}
\begin{myproof}
We will use a proof by contradiction.  So we assume that the proposition is false, or that there exists a real number $x$ such that $0 < x < 1$ and
\begin{equation} \label{eq:contradiction1}
\dfrac{1}{x(1 - x)} < 4.
\end{equation}
We note that since $0 < x < 1$, we can conclude that $x > 0$ and that $(1 - x)>0$.  Hence, $x(1 - x) >0$ and if we multiply both sides of inequality~(\ref{eq:contradiction1}) by $x(1 - x)$, we obtain
\[
1 < 4x(1 - x).
\]
We can now use algebra to rewrite the last inequality as follows:
\begin{align*}
1 &< 4x - 4x^2 \\
4x^2 - 4x + 1 &< 0 \\
(2x - 1)^2 &< 0
\end{align*}
However, $(2x - 1)$ is a real number and the last inequality says that a real number squared is less than zero.  This is a contradiction since the square of any real number must be greater than or equal to zero.  Hence, the proposition cannot be false, and we have proved that for each real number  $x$, if $0 < x < 1$, then $\dfrac{1}{x(1 - x)} \geq 4$.
\end{myproof}
\hbreak

\begin{prog}[\textbf{Starting a Proof by Contradiction}]\label{pr:start-con} \hfill \\
One of the most important parts of a proof by contradiction is the very first part, which is to state the assumptions that will be used in the proof by contradiction.  This usually involves writing a clear negation of the proposition to be proven.  Review De Morgan's Laws and the negation of a conditional statement in Section~\ref{S:logequiv}.  (See Theorem~\ref{T:logequiv} on page~\pageref{T:logequiv}.)  Also, review Theorem~\ref{T:negations} (on page~\pageref{T:negations}) and then write a negation of each of the following statements.  (Remember that a real number is ``not irrational'' means that the real number is rational.)

\begin{enumerate}
\item For each real number $x$, if $x$ is irrational, then $\sqrt[3]{x}$ is irrational.
\item For each real number $x$, $\left(x + \sqrt{2} \right)$ is irrational or 
$\left(-x + \sqrt{2} \right)$ is irrational.
\item For all integers $a$ and $b$, if 5 divides $ab$, then 5 divides $a$ or 5 divides $b$.
\item For all real numbers $a$ and $b$, if $a > 0$ and $b > 0$, then 
$\dfrac{2}{a} + \dfrac{2}{b} \ne \dfrac{4}{a + b}$.
\end{enumerate}
\end{prog}
\hbreak

\endinput

\subsection*{Important Note}
A proof by contradiction is often used to prove a conditional statement \linebreak
$P \to Q$  when a direct proof has not been found and it is relatively easy to form the negation of the proposition.  The advantage of a proof by contradiction is that we have an additional assumption with which to work (since we assume not only  $P$ but also  $\mynot  Q$).  The disadvantage is that there is no well-defined goal to work toward.  The goal is simply to obtain some contradiction.  There usually is no way of telling beforehand what that contradiction will be, so we have to stay alert for a possible absurdity.  Thus, when we set up a know-show table for a proof by contradiction, we really only work with the know portion of the table.  %This was illustrated in Example~\ref{E:contradiction} and Proposition~\ref{P:contradiction}.
\hbreak

\begin{prog}[\textbf{Exploration and a Proof by Contradiction}]\label{pr:exploreproof} \hfill \\
Consider the following proposition:
\begin{list}{}
  \item For each integer $n$, if  $n \equiv 2 \pmod 4$, then  
$n\not  \equiv 3 \pmod 6$.
\end{list}
\begin{enumerate}
  \item Determine at least five different integers that are congruent to  2  modulo  4, 
\label{pr:exploreproof1}%
 and determine at least five different integers that are congruent to  3  modulo  6.  Are there any integers that are in both of these lists?

  \item For this proposition, why does it seem reasonable to try a proof by contradiction?

  \item For this proposition, state clearly the assumptions that need to be made at the beginning of a proof by contradiction, and then use a proof by contradiction to prove this proposition.
\end{enumerate}
\end{prog}
\hbreak

\endinput

\subsection*{Proving that Something Does Not Exist}
In mathematics, we sometimes need to prove that something does not exist or that something is not possible.  Instead of trying to construct a direct proof, it is sometimes easier to use a proof by contradiction so that we can assume that the something exists.
%In symbolic form, we can say that we want to prove that
%\[
%\mynot \left( \exists x \in U \right) \left( P(x) \right),
%\]
%where $P(x)$ is some open sentence.  Instead of working with equivalent statement 
%$\left( \forall x \in U \right) \left( \mynot P(x) \right)$, it is sometimes easier to use a proof by contradiction and assume that $\left( \exists x \in U \right) \left( P(x) \right)$.  
For example, suppose we want to prove the following proposition:

\begin{proposition} \label{prop:persquare}
For all integers $x$ and $y$, if $x$ and $y$ are odd integers, then there does not exist an integer $z$ such that $x^2 + y^2 = z^2$.
\end{proposition}
Notice that the conclusion involves trying to prove that an integer with a certain property does not exist.  If we use a proof by contradiction, we can assume that such an integer $z$ exists.  This gives us more with which to work.
\begin{prog} \label{prog:persquare}
Complete the following proof of Proposition~\ref{prop:persquare}:
\begin{myproof}
We will use a proof by contradiction.  So we assume that there exist integers $x$ and $y$ such that $x$ and $y$ are odd and there exists an integer $z$ such that $x^2 + y^2 = z^2$.  Since $x$ and $y$ are odd, there exist integers $m$ and $n$ such that $x = 2m + 1$ and $y = 2n + 1$.
\begin{enumerate}
  \item Use the assumptions that $x$ and $y$ are odd to prove that $x^2 + y^2$ is even and hence, $z^2$ is even.  (See Theorem~\ref{T:n2odd} on page~\pageref{T:n2odd}.)
\end{enumerate}
We can now conclude that $z$ is even.  (See Theorem~\ref{T:n2odd} on page~\pageref{T:n2odd}.)  So there exists an integer $k$ such that $z = 2k$.  If we substitute for $x$, $y$, and $z$ in the equation $x^2 + y^2 = z^2$, we obtain
\[
(2m + 1)^2 + (2n + 1)^2 = (2k)^2.
\]
\setcounter{oldenumi}{\theenumi}
\begin{enumerate} \setcounter{enumi}{\theoldenumi}
  \item Use the previous equation to obtain a contradiction.  \hint One way is to use algebra to obtain an equation where the left side is an odd integer and the right side is an even integer. \qedhere
\end{enumerate}
\end{myproof}

\end{prog}
\hbreak

\endinput


\subsection*{Rational and Irrational Numbers}
One of the most important ways to classify real numbers is as a rational number or an irrational number.  Following is the definition of rational (and irrational) numbers given in Exercise~(\ref{exer:rational}) from Section~\ref{S:moremethods}.

\begin{defbox}{D:rational}{A real number  $x$ is defined to be a \textbf{rational number}
\index{rational numbers}%
 provided that there exist integers  $m$  and $n$  with  $n \ne 0$ such that $x = \dfrac{m}{n}$.  A real number that is not a rational number is called an \textbf{irrational number.}
\index{irrational numbers}%
}
\end{defbox}
This may seem like a strange distinction because most people are quite familiar with the rational numbers (fractions) but the irrational numbers seem a bit unusual.  However, there are many irrational numbers such as $\sqrt{2}$, $\sqrt{3}$, $\sqrt[3]{2}$, $\pi$, and the number $e$.  We are discussing these matters now because we will soon prove that $\sqrt{2}$ is irrational in Theorem~\ref{T:squareroot2}.


We use the symbol $\Q$ to stand for the set of rational numbers.  There is no standard symbol for the set of irrational numbers.  Perhaps one reason for this is because of the closure properties of the rational numbers.  We introduced closure properties in Section~\ref{S:prop}, and the rational numbers $\Q$ are closed under addition, subtraction, multiplication, and division by nonzero rational numbers.  This means that if $x, y \in \Q$, then
\begin{itemize}
  \item $x + y$, $x - y$, and $xy$ are in $\Q$; and
  \item If $y \ne 0$, then $\dfrac{x}{y}$ is in $\Q$.
\end{itemize}
The basic reasons for these facts are that if we add, subtract, multiply, or divide two fractions, the result is a fraction.  One reason we do not have a symbol for the irrational numbers is that the irrational numbers are not closed under these operations.  For example, we will prove that $\sqrt{2}$ is irrational in Theorem~\ref{T:squareroot2}. We then see that
\[
\sqrt{2} \sqrt{2} = 2 \quad \text{and} \quad \frac{\sqrt{2}}{\sqrt{2}} = 1,
\]
which shows that the product of irrational numbers can be rational and the quotient of irrational numbers can be rational.

It is also important to realize that every integer is a rational number since any integer can be written as a fraction.  For example, we can write $3 = \dfrac{3}{1}$.  In general, if $n \in \Z$, then $n = \dfrac{n}{1}$, and hence, $n \in \Q$.

Because the rational numbers are closed under the standard operations and the definition of an irrational number simply says that the number is not rational, we often use a proof by contradiction to prove that a number is irrational.  This is illustrated in the next proposition.

\begin{proposition}  For all real numbers $x$ and $y$, if $x$ is rational and $x \ne 0$ and $y$ is irrational, then $x \cdot y$ is irrational.
\end{proposition}

\begin{myproof}
We will use a proof by contradiction.  So we assume that there exist real numbers $x$ and $y$ such that $x$ is rational, $x \ne 0$,  $y$ is irrational, and $x \cdot y$ is rational.  Since $x \ne 0$, we can divide by $x$, and since the rational numbers are closed under division by nonzero rational numbers, we know that $\dfrac{1}{x} \in \Q$.  We now know that $x \cdot y$ and $\dfrac{1}{x}$ are rational numbers and since the rational numbers are closed under multiplication, we conclude that
\[
\frac{1}{x} \cdot \left( xy \right) \in \Q.
\]
However, $\dfrac{1}{x} \cdot \left( xy \right) = y$ and hence, $y$ must be a rational number.  Since a real number cannot be both rational and irrational, this is a contradiction to the assumption that $y$ is irrational.  We have therefore proved that for all real numbers $x$ and $y$, if $x$ is rational and $x \ne 0$ and $y$ is irrational, then $x \cdot y$ is irrational.
\end{myproof}

\hbreak
\endinput


\subsection*{The Square Root of 2 Is an Irrational Number}
%\begin{activity}[\textbf{The Square Root of 2 Is Irrational}] \hfill \\
The proof that the square root of  2 is an irrational number is one of the classic proofs in mathematics, and  every mathematics student should know this proof.  %This is why we are completing this proof through a ``guided activity'' rather than simply writing a proof.  The proposition can be stated as follows:  
This is why we will be doing some preliminary work with rational numbers and integers before completing the proof.  The theorem we will be proving can be stated as follows:

\setcounter{equation}{0}
\begin{theorem}\label{T:squareroot2}
If  $r$  is a real number such that  $r^2  = 2$, then  $r$  is an irrational number.
\end{theorem}
This is stated in the form of a conditional statement, but it basically means that  $\sqrt 2$
 is irrational (and that   $ - \sqrt 2$ is irrational).  That is,  $\sqrt 2$  cannot be written as a quotient of integers with the denominator not equal to zero.

In order to complete this proof, we need to be able to work with some basic facts that follow about rational numbers and even integers.  
\begin{enumerate}
%  \item Write a complete definition of a rational number.  See Exercise~(\ref{exer:sec32-rational}) from Section~\ref{S:moremethods} on 
%page~\pageref{exer:sec32-rational}.  Give examples of at least five different rational numbers.

  \item Each integer $m$ is a  rational number since $m$ can be written as $m = \dfrac{m}{1}$.

  \item Notice that $\dfrac{2}{3} = \dfrac{4}{6}$, since
\begin{align*}
\frac{4}{6} &= \frac{2 \cdot 2}{3 \cdot 2} = \frac{2}{2} \cdot \frac{2}{3} = \frac{2}{3}
\end{align*}
We can also show that $\dfrac{15}{12} = \dfrac{5}{4}$, $\dfrac{10}{-8} = \dfrac{-5}{4}$, and $\dfrac{-30}{-16} = \dfrac{15}{8}$ \label{LA:learn33-eq2}%

%  \item Are any of the rational numbers $\dfrac{-5}{4}$, $\dfrac{-10}{8}$, $\dfrac{18}{27}$, and $\dfrac{-8}{-12}$ equal? \label{q:4}%

%  \item What does it mean to say that a real number $r$ is an irrational number?  Explain by writing a precise negation of the definition of a rational number. 
\end{enumerate}
Item~(\ref{LA:learn33-eq2}) was included to illustrate the fact that a rational number can be written as a fraction in ``lowest terms'' with a positive denominator.  This means that any rational number can be written as a quotient $\dfrac{m}{n}$, where $m$ and $n$ are integers, $ n > 0 $, and $m$ and $n$ have no common factor greater than 1.  %This fact will be used in a proof by contradiction that the square root of 2 is an irrational number.  (Theorem~\ref{T:squareroot2})

\setcounter{oldenumi}{\theenumi}
\begin{enumerate} \setcounter{enumi}{\theoldenumi}
  \item If $n$ is an integer and $n^2$ is even, what can be conclude about $n$.  Refer to Theorem~\ref{T:n2odd} on page~\pageref{T:n2odd}.
\end{enumerate}
In a proof by contradiction of a conditional statement $P \to Q$, we assume the negation of this statement or 
$P \wedge \mynot Q$.  So in a proof by contradiction of Theorem~\ref{T:squareroot2}, we will assume that  $r$  is a real number,  $r^2  = 2$, and  $r$  is not irrational (that is, $r$  is rational).  



%\setcounter{oldenumi}{\theenumi}
%\begin{enumerate} \setcounter{enumi}{\theoldenumi}
%\item Complete the indicated steps in the proof of Theorem~\ref{T:squareroot2}.
%\end{enumerate}


\setcounter{equation}{0}
\setcounter{theorem}{19}
\begin{theorem}
If  $r$  is a real number such that  $r^2  = 2$, then  $r$  is an irrational number.
\end{theorem}
\begin{myproof}
We will use a proof by contradiction.  So we assume that the statement of the theorem is false.  That is, we assume that
\begin{list}{}
  \item  $r$ is a real number,  $r^2  = 2$, and  $r$  is a rational number.
\end{list}
\noindent
Since  $r$  is a rational number, there exist integers  $m$  and  $n$  with  $n > 0$ such that  
\[
r = \frac{m}{n}
\]
and  $m$  and  $n$  have no common factor greater than 1.  We will obtain a contradiction by showing that  $m$  and  $n$  must both be even.  Squaring both sides of the last equation and using the fact that $r^2 = 2$, we obtain
\begin{align}
  2 &= \frac{{m^2 }}{{n^2 }} \notag \\ 
  m^2  &= 2n^2. 
  \label{eq:3h}%  
\end{align} 
Equation~(\ref{eq:3h}) implies that $m^2$ is even, and hence, by Theorem~\ref{T:n2odd}, $m$ must be an even integer.  This means that there exists an integer $p$ such that $m = 2p$.  We can now substitute this into equation~(\ref{eq:3h}), which gives
\begin{align}
  \left( {2p} \right)^2  &= 2n^2 \notag \\ 
  4p^2  &= 2n^2. 
  \label{eq:3i}%  
\end{align} 
We can divide both sides of equation~(\ref{eq:3i}) by 2 to obtain $n^2 = 2p^2$.  Consequently, $n^2$ is even and we can once again use Theorem~\ref{T:n2odd} to conclude that $n$ is an even integer.

We have now established that both  $m$  and  $n$  are even.  This means that  2  is a common factor of  $m$  and  $n$, which contradicts the assumption that $m$  and  $n$  have no common factor greater than 1.  Consequently, the statement of the theorem cannot be false, and we have proved that if  $r$  is a real number such that  $r^2  = 2$, then  $r$  is an irrational number.
\end{myproof}
\hbreak


\endinput







%\setcounter{equation}{0}
%\begin{proposition}\label{P:contradiction}
%For all real numbers  $x$  and  $y$, if  $x \ne y$, $x > 0$, and $y > 0$, then  
%$\dfrac{x}{y} + \dfrac{y}{x} > 2.$
%\end{proposition}
%%
%\begin{myproof}
%This proposition will be proved using a proof by contradiction.  So we assume that there exist real numbers $x$  and  $y$  such that  $x \ne y$, $x > 0,\text{ and }y > 0$  and that
%
%\[
%\frac{x}
%{y} + \frac{y}{x} \leq 2.
%\]
%
%
%Since  $x$  and  $y$  are positive, we can multiply both sides of this inequality by  $xy$  to obtain
%
%\[
%x^2  + y^2  \leq 2xy.
%\]
%
%
%We will obtain a contradiction using the assumption that  $x \ne y$.  Using basic algebra, it is seen that
%\begin{align}
%  x^2  - 2xy + y^2  &\leq 0 \notag \\ 
%  \left( {x - y} \right)^2  &\leq 0. 
%  \label{eq:3f}% 
%\end{align} 
%However, one of the assumptions we made was that  $x \ne y$.  This implies that  $x - y \ne 0$
% and hence that
%\begin{equation}\label{eq:3g}
%\left( {x - y} \right)^2  > 0.
%\end{equation}
%
%Inequalities~(\ref{eq:3f})  and (\ref{eq:3g})   cannot both be true, and so we have obtained a contradiction.  Thus, our given statement cannot be false and we have proven that if  $x \ne y$, $x > 0,\text{ and }y > 0$, then  $\dfrac{x}{y} + \dfrac{y}{x} > 2$.  
%\end{myproof}
%\hbreak







\endinput



%\begin{activity}[A Proof by Contradiction]\label{A:lineareq} \hfill \\
%Consider the following proposition:
%
%\begin{list}{}
%  \item Let  $a$, $b$, and $c$  be integers.  If  3  divides  $a$,  3  divides  $b$,  and  $c \equiv 1 \pmod 3$, then the equation	
%\[
%ax + by = c
%\]
%has no solution in which both  $x$  and  $y$  are integers.
%\end{list}
%\vskip10pt
%\noindent
%Complete the following proof of this proposition:
%
%\vskip10pt
%\noindent 
%\textbf{\emph{Proof.}}  A proof by contradiction will be used.  So we assume that the statement is false.  That is, we assume that there exists integers $a$, $b$, and $c$ such that 3  divides both  $a$  and  $b$, that  $c \equiv 1 \pmod 3$,  and that  the equation
%\[
%ax + by = c
%\]
%has a solution in which both  $x$  and  $y$  are integers.
%So there exist integers  $m$  and  $n$  such that 
%\[
%am  + bn  = c.
%\]
%\hint  Now use the facts that  3  divides  $a$, 3  divides  $b$, and  
%$c \equiv 1 \pmod 3$.
%\end{activity}
%\hbreak
%
%\begin{activity}[Exploring a Quadratic Equation]\label{A:quadratic} \hfill \\
%Consider the following proposition:
%\begin{list}{}
%  \item For all integers $m$ and $n$, if $n$ is odd, then the equation
%   \[
%   x^2+2mx+2n=0
%   \]
%   has no integer solution for $x$.
%\end{list}
%
%\begin{enumerate}
%  \item What are the solutions of the equation  when  $m = 1$ and $n =  - 1$?  That is, what are the solutions of the equation  $x^2  + 2x - 2 = 0$?
%
%  \item What are the solutions of the equation  when  $m = 2$ and $n = 3$?  That is, what are the solutions of the equation  $x^2  + 4x + 2 = 0$?
%
% % \item What are the solutions of the equation  when  $m = 1$ and $n = 3$?  That is, what are the solutions of the equation  $x^2  + 2x + 6 = 0$?
%
%  \item Solve the resulting quadratic equation for at least two more examples using values of  
%$m$  and  $n$  that satisfy the hypothesis of the proposition.
%
%  \item For this proposition, why does it seem reasonable to try a proof by contradiction?
%
%  \item For this proposition, state clearly the assumptions that need to be made at the beginning of a proof by contradiction.
%
%  \item Use a proof by contradiction to prove this proposition.
%\end{enumerate}
%\end{activity}
%\hbreak





\endinput
