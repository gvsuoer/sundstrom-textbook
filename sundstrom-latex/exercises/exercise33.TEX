\section*{Exercises for Section~\ref{S:contradiction}}
\begin{enumerate}
%
%\item Review the following part of the summary for Chapter~\ref{C:proofs} whose title is, ``A Comparison of Direct Proofs, Proofs Using the Contrapositive, and Proofs by Contradiction.'' See page \pageref{SS:proofcompare}.


\item This exercise is intended to provide another rationale as to why a proof by contradiction works.
\label{exer:sec33-1}%

Suppose that we are trying to prove that a statement $P$ is true.  Instead of proving this statement, assume that we prove that the conditional statement ``If $\mynot P$, then $C$'' is true, where  $C$ is some contradiction.  Recall that a contradiction is a statement that is always false.
\begin{enumerate}
  \yitem In symbols, write a statement that is a disjunction and that is logically equivalent to $\mynot P \to C$.
\label{exer:contradictionproof}%

  \item Since we have proven that $\mynot P \to C$ is true, then the disjunction in Exercise~(\ref{exer:contradictionproof}) must also be true.  Use this to explain why the statement $P$ must be true.

  \item Now explain why $P$ must be true if we prove that the negation of  $P$  implies a contradiction. 
\end{enumerate}

\item Are the following statements true or false?  Justify each conclusion. \label{exer:sec32-truefalse}
\begin{enumerate}
\yitem For all integers $a$ and $b$, if $a$ is even and $b$ is odd, then 
$4$ does not divide $\left( a^2 + b^2 \right)$.
\yitem For all integers $a$ and $b$, if $a$ is even and $b$ is odd, then 
$6$ does not divide $\left( a^2 + b^2 \right)$.
\item For all integers $a$ and $b$, if $a$ is even and $b$ is odd, then 
$4$ does not divide $\left( a^2 + 2b^2 \right)$.
\yitem For all integers $a$ and $b$, if $a$ is odd and $b$ is odd, then 
$4$ divides $\left( a^2 + 3b^2 \right)$.
\end{enumerate}

\xitem Consider the following statement:
\label{exer:sec33-2}%

\begin{list}{}
  \item For each positive real number $r$, if  $r^2  = 18$, then  $r$  is irrational.
\end{list}
%
\begin{enumerate}
  \item If you were setting up a proof by contradiction for this statement, what would you assume?  Carefully write down all conditions that you would assume.

  \item Complete a proof by contradiction for this statement.  
%\hint  Do not attempt to mimic the proof that the square root of 2 is irrational (Theorem~\ref{T:squareroot2}).  You should still use the definition of a rational number but then use the fact that  
%$\sqrt {18}  = \sqrt {9 \cdot 2}  = \sqrt 9 \sqrt 2  = 3\sqrt 2 $.
\end{enumerate}

\item Prove that the cube root of 2 is an irrational number.  That is, prove that if $r$ is a real number such that $r^3 = 2$, then $r$ is an irrational number.


\xitem Prove the following propositions:
\label{exer:sec33-3}%
\begin{enumerate}
\item For all real numbers  $x$  and  $y$, if  $x$  is rational and  $y$  is irrational, then  
$x + y$  is irrational.

\item For all nonzero real numbers  $x$  and  $y$, if  $x$  is rational and  $y$  is irrational, then  
$xy$  is irrational.
\end{enumerate}

\item	Are the following statements true or false?  Justify each conclusion. 
\label{exer:sec33-4}%

\begin{enumerate}
  \yitem For each positive real number  $x$,  if  $x$  is irrational, then  $x^2 $  is irrational.
  \yitem For each positive real number  $x$,  if  $x$  is irrational, then  $\sqrt x $  is irrational.
  \item For every pair of real numbers $x$ and $y$, if $x + y$ is irrational, then $x$ is         irrational and $y$ is irrational.
  \item For every pair of real numbers $x$ and $y$, if $x + y$ is irrational, then $x$ is         irrational or $y$ is irrational.
\end{enumerate}


\item \begin{enumerate}
\item Give an example that shows that the sum of two irrational numbers can be a rational number.

\item Now explain why the following proof that $\left( \sqrt{2} + \sqrt{5} \right)$ is an irrational number is not a valid proof:  Since $\sqrt{2}$ and $\sqrt{5}$ are both irrational numbers, their sum is an irrational number.  Therefore, 
$\left( \sqrt{2} + \sqrt{5} \right)$ is an irrational number.

\note  You may even assume that we have proven that $\sqrt{5}$ is an irrational number.  (We have not proven this.)

\item Is the real number $\sqrt 2  + \sqrt 5 $ a rational number or  an irrational number?  Justify your conclusion.
\label{exer:sec33-10}%
\end{enumerate}

\item \begin{enumerate}
\item Prove that for each real number $x$, $\left(x + \sqrt{2} \right)$ is irrational or 
$\left(-x + \sqrt{2} \right)$ is irrational.

\item Generalize the proposition in Part~(a) for any irrational number (instead of just 
$\sqrt{2}$) and then prove the new proposition.
\end{enumerate}

\item Is the following statement true or false?
  \begin{list}{}
     \item For all positive real numbers $x$ and $y$, $\sqrt {x + y}  \leq \sqrt x  + \sqrt y $.
  \end{list}
%
        

\item Is the following proposition true or false?  Justify your conclusion.

\begin{list}{}
\item For each real number $x$, $x \left( 1 - x \right) \leq \dfrac{1}{4}$.
\end{list}



\xitem \begin{enumerate}
  \item Is the base 2 logarithm of 32, $\log _2 (32)$,  a rational number or an irrational number? Justify your conclusion. 

  \item Is the base 2 logarithm of 3, $\log _2 (3)$,  a rational number or an irrational number? Justify your conclusion. 
\end{enumerate}
\label{exer:sec33-9}%

%\item Is the real number $\sqrt 2  + \sqrt 5 $ a rational number or  an irrational number?  Justify your conclusion. \label{exer:sec33-10}


\xitem In Exercise~(\ref{exer:IVT}) in Section~\ref{S:moremethods}, we proved that there exists a real number solution to the equation $x^3  - 4x^2  = 7$.  Prove that there is no integer $x$ such that  $x^3  - 4x^2  = 7$.
\label{exer:sec33IVT}% 
\newpage

\item Prove each of the following propositions:
\label{exer:sec33-11}%

\begin{enumerate}
  \yitem For each real number  $\theta $, if  $0 < \theta  < \dfrac{\textstyle \pi }{\textstyle 2}$, then  $\left[ {\sin (\theta)  + \cos (\theta) } \right] > 1$.

\item For all real numbers  $a$  and  $b$, if  $a \ne 0$  and  $b \ne 0$, then  
$\sqrt{a^2 + b^2} \ne a + b$.

 \item If  $n$  is an integer greater than 2, then for all integers  $m$,  $n$  does not divide  $m$   or  $n + m \ne nm$.

\item For all real numbers $a$ and $b$, if $a > 0$ and $b > 0$, then 
\[
\frac{2}{a} + \frac{2}{b} \ne \frac{4}{a + b}.
\]
\end{enumerate}


\xitem Prove that there do not exist three consecutive natural numbers such that the cube of the largest is equal to the sum of the cubes of the other two.
\label{exer:sec35-10}%

%\hint Three consecutive natural numbers can be represented by $n$, $n + 1$, and 
%$n + 2$, where $n \in \mathbb{N}$, or three consecutive natural numbers can be represented by 
%$m - 1$, $m$, and $m + 1$, where $m \in \mathbb{N}$.



\item Three natural numbers $a$, $b$, and $c$ with $a < b < c$ are called a \textbf{Pythagorean triple}
\index{Pythagorean triple}%
 provided that $a^2 + b^2 = c^2$.  For example,  the numbers 3, 4, and 5 form a Pythagorean triple, and the numbers 5, 12, and 13 form a Pythagorean triple.

\begin{enumerate}
\item Verify that if $a = 20$, $b = 21$, and $c = 29$, then $a^2 + b^2 = c^2$, and hence, 20, 21, and 29 form a Pythagorean triple.

\item Determine two other Pythagorean triples.  That is, find integers $a, b$, and $c$ such that $a^2 + b^2 = c^2$.

\item Is the following proposition true or false?  Justify your conclusion.

For all integers $a$, $b$, and $c$, if $a^2 + b^2 = c^2$, then $a$ is even or $b$ is even.
\end{enumerate}

\item Consider the following proposition:  There are no integers $a$ and $b$ such that 
$b^2 = 4a + 2$.
\begin{enumerate}
\item Rewrite this statement in an equivalent form using a universal quantifier by completing the following:  
\begin{center}
For all integers $a$ and $b$, $\ldots .$
\end{center}
\item Prove the statement in Part~(a).
\end{enumerate}





\item Is the following statement true or false?  Justify your conclusion.

\begin{list}{}
  \item For each integer  $n$  that is greater than 1, if  $a$ is the smallest positive factor of  $n$ that is greater than 1, then  $a$  is prime.
\end{list}

See Exercise~(\ref{exer:prime}) in Section~\ref{S:quantifier} (page~\pageref{exer:prime}) for the definition of a prime number and the definition of a composite number.


\item A \textbf{magic square}
\index{magic square}%
is a square array of natural numbers whose rows, columns, and diagonals all sum to the same number.  For example, the following is a 3 by 3 magic square since the sum of the 3 numbers in each row is equal to 15, the sum of the 3 numbers in each column is equal to 15, and the sum of the 3 numbers in each diagonal is equal to 15.
\begin{center}
\begin{tabular}{| c | c | c |} \hline
8  &  3  &  4  \\ \hline
1  &  5  &  9  \\ \hline
6  &  7  &  2  \\ \hline
\end{tabular}
\end{center}
Prove that the following 4 by 4 square cannot be completed to form a magic square.
\begin{center}
\begin{tabular}{| c | c | c | c |} \hline
   &  1  &     &  2  \\ \hline
3  &  4  &  5  &     \\ \hline
6  &  7  &     &  8  \\ \hline
9  &     &  10 &     \\ \hline
\end{tabular}
\end{center}

%\hint  Assign each of the six blank cells in the square a name.  One possibility is to use $a$, 
%$b$, $c$, $d$, $e$, and $f$.


\item Using only the digits 1 through 9 one time each, is it possible to construct a 3 by 3 magic square with the digit 3 in the center square?  That is, is it possible to construct a magic square of the form

\begin{center}
\begin{tabular}{| c | c | c |} \hline
$a$  &  $b$  &  $c$  \\ \hline
$d$  &   3   &  $e$  \\ \hline
$f$  &  $g$  &  $h$  \\ \hline
\end{tabular}
\end{center}

\noindent
where $a, b, c, d, e, f, g, h$ are all distinct digits, none of which is equal to 3?  Either construct such a magic square or prove that it is not possible.


\item \textbf{Evaluation of proofs}  \hfill \\
See the instructions for Exercise~(\ref{exer:proofeval}) on 
page~\pageref{exer:proofeval} from Section~\ref{S:directproof}.

\begin{enumerate}
\item \textbf{Proposition}.  For each real number $x$, if $x$ is irrational and $m$ is an integer, then $mx$ is irrational.

\begin{myproof}
We assume that $x$ is a real number and is irrational.  This means that for all integers $a$ and $b$ with $b \ne 0$, $x \ne \dfrac{a}{b}$.  Hence, we may conclude that $mx \ne \dfrac{ma}{b}$ and, therefore, $mx$ is irrational.
\end{myproof}

\item \textbf{Proposition}.  For all real numbers $x$ and $y$, if $x$ is irrational and $y$ is rational, then $x + y$ is irrational.

\begin{myproof}
We will use a proof by contradiction.  So we assume that the proposition is false, which means that there exist real numbers $x$ and $y$ where $x \notin \Q$, $y \in \Q$, and $x + y \in \Q$.  Since the rational numbers are closed under subtraction and $x+y$ and $y$ are rational, we see that
\[
\left(x + y \right) - y \in \Q.
\]
However, $\left( x + y \right) - y = x$, and hence we can conclude that $x \in \Q$.  This is a contradiction to the assumption that $x \notin \Q$.  Therefore, the proposition is not false, and we have proven that for all real numbers $x$ and $y$, if $x$ is irrational and $y$ is rational, then $x + y$ is irrational. 
\end{myproof}


\item \textbf{Proposition}.  For each real number $x$, $x (1 - x) \leq \dfrac{1}{4}$.

\begin{myproof}
A proof by contradiction will be used.  So we assume the proposition is false.  This means that there exists a real number $x$ such that $x (1 - x) > \dfrac{1}{4}$.  If we multiply both sides of this inequality by 4, we obtain
 $4x (1 - x) > 1$.
However, if we let $x = 3$, we then see that
\begin{align*}
4x (1 - x) &> 1 \\
4 \cdot 3 (1 - 3) &> 1 \\
-12 &> 1
\end{align*}
The last inequality is clearly  a contradiction and so we have proved the proposition.
\end{myproof}
\end{enumerate}
\end{enumerate}


\subsection*{Explorations and Activities}
\setcounter{oldenumi}{\theenumi}
\begin{enumerate} \setcounter{enumi}{\theoldenumi}
\item  \textbf{A Proof by Contradiction}.   Consider the following proposition:

\noindent
\textbf{Proposition}. Let  $a$, $b$, and $c$  be integers.  If  3  divides  $a$,  3  divides  $b$,  and  $c \equiv 1 \pmod 3$, then the equation	
\[
ax + by = c
\]
has no solution in which both  $x$  and  $y$  are integers.

\newpar
Complete the following proof of this proposition:

\noindent
\textbf{\emph{Proof.}}  A proof by contradiction will be used.  So we assume that the statement is false.  That is, we assume that there exist integers $a$, $b$, and $c$ such that 3  divides both  $a$  and  $b$, that  $c \equiv 1 \pmod 3$,  and that  the equation
\[
ax + by = c
\]
has a solution in which both  $x$  and  $y$  are integers.
So there exist integers  $m$  and  $n$  such that 
\[
am  + bn  = c.
\]
\hint  Now use the facts that  3  divides  $a$, 3  divides  $b$, and  
$c \equiv 1 \pmod 3$.


\item \textbf{Exploring a Quadratic Equation}.  Consider the following proposition:

\noindent
\textbf{Proposition}. For all integers $m$ and $n$, if $n$ is odd, then the equation
   \[
   x^2+2mx+2n=0
   \]
   has no integer solution for $x$.

\begin{enumerate}
  \item What are the solutions of the equation  when  $m = 1$ and $n =  - 1$?  That is, what are the solutions of the equation  $x^2  + 2x - 2 = 0$?

  \item What are the solutions of the equation  when  $m = 2$ and $n = 3$?  That is, what are the solutions of the equation  $x^2  + 4x + 6 = 0$?

 % \item What are the solutions of the equation  when  $m = 1$ and $n = 3$?  That is, what are the solutions of the equation  $x^2  + 2x + 6 = 0$?

  \item Solve the resulting quadratic equation for at least two more examples using values of  
$m$  and  $n$  that satisfy the hypothesis of the proposition.

  \item For this proposition, why does it seem reasonable to try a proof by contradiction?

  \item For this proposition, state clearly the assumptions that need to be made at the beginning of a proof by contradiction.

  \item Use a proof by contradiction to prove this proposition.
\end{enumerate}
\end{enumerate}

\hbreak

\endinput
