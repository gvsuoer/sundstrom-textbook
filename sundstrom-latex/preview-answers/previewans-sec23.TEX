\documentclass[11pt]{article}
\usepackage{../../styles/activity}

\usepackage{xr}
\externaldocument{0-MR}

\lhead{}
%\chead{\textbf{\Large{\hspace{0pt}Breview Activities for Section~2.3}}\\\prevhead}
\bahead{2.3}
\rhead{}
\lfoot{}
\rfoot{}
\cfoot{\hspace{0pt}\scalebox{0.4}{\includegraphics{cc-by-nc-sa.eps}}}

\begin{document}

\input table
\subsection*{Beginning Activity 1 (Sets and Set Notation)}
Mathematicians use notation to convey their ideas.  So it is very important to learn and understand correct mathematical notation.  In particular, the concept of sets pervades much of mathematics and so understanding set notation is extremely important as is the ability to write set notation correctly.  We should make sure that we completely understand the notation used in the answers to these beginning activities.
\begin{enumerate}
  \item \begin{enumerate}
    \item The set of real numbers that are solutions of the equation $x^2 - 5x = 0$ is $\{0, 5 \}$.
    \item The set of natural numbers that are less than or equal to 10 is $\{1, 2, 3, 4, 5, 6, 7, 8, 9, 10 \}$.
    \item The set of integers that are greater than $-2$ is $\{ -1, 0, 1, 2,  \ldots \}$.
  \end{enumerate}

  \item $$
\BeginTable
\BeginFormat
| l | l |
\EndFormat
\_
| {\bf Set} | {\bf Some other elements of the set} | \\ \_
| $A = \{ 1, 4, 7, 10, \ldots \}$ | 13, 16, 19, 22, 25, 28, 31 | \\ \_
| $B = \{ 2, 4, 8, 16, \ldots \}$ | 32, 64, 128, 256, 512, 1024 | \\ \_
| $C = \{\ldots -8, -6, -4, -2, 0 \}$ | $-20, -18, -16, -14, -12, -10$ | \\ \_
| $D = \{ \ldots -9, -6, -3, 0, 3, 6, 9 \ldots \}$ | $-21, -18, -15, -12, 12, 15, 18, 21$ | \\ \_
\EndTable
$$
\end{enumerate}
\hbreak




\subsection*{Beginning Activity 2 (Variables)}
\begin{enumerate}
  \item \begin{enumerate}
    \item The equation $x^2 - 25 = 0$ becomes a true statement if $-5$ is substituted for $x$.
    \item The equation $x^2 - 25 = 0$ becomes a false statement if $\sqrt{5}$ is substituted for $x$.
\end{enumerate}

  \item The only real numbers that will make the sentence ``$y^2 - 2y - 15 = 0$ a true statement when substituted for $y$ are $-3$ and 5.

  \item The only natural number that will make the sentence ``$y^2 - 2y - 15 = 0$ a true statement when substituted for $y$ is 5.

  \item Any non-negative real number will make the sentence ``$\sqrt{x}$  is a real number'' a true statement when substituted for $x$.

  \item Any real number will make the sentence ``$\sin ^2 x + \cos ^2 x = 1$'' a true statement when substituted for  $x$.

%  \item Only the real numbers $\dfrac{{2 + \sqrt {24} }}{2}$  and   
%$\dfrac{{2 - \sqrt {24} }}{2}$ will make the sentence ``$y^2  - 2y - 5 = 0$'' a true statement when substituted for  $y$.

  \item The natural numbers that are perfect squares will make the sentence ``$\sqrt{n}$ is a natural number'' a true statement when substituted for $n$.  Perfect squares are natural numbers such as $1, 4, 9, 16,  \ldots $.

  \item Any real number greater than 3 will make the sentence 
  \[
   \int_0^y {t^2 dt > 9}
  \]
   a true statement when substituted for  $y$?
\end{enumerate}

\hbreak

\end{document}

