\begin{previewactivity}[\textbf{Using a Logical Equivalency}]\label{PA:logicalequiv} \hfill
\begin{enumerate}
\item Complete a truth table to show that  $\left( {P \vee Q} \right) \to R$
  is logically equivalent to  $\left( {P \to R} \right) \wedge \left( {Q \to R} \right)$.

\item Suppose that you are trying to prove a statement that is written in the form  
$\left( {P \vee Q} \right) \to R$.  Explain why you can complete this proof by writing separate and independent proofs of   $P \to R$ and   $Q \to R$. \label{PA:logicalequiv2}

\item Now consider the following proposition:

\noindent
\textbf{Proposition}.  For all integers $x$ and $y$, if $xy$ is odd, then $x$ is odd and $y$ is odd. \label{PA:logicalequiv3}

\noindent
Write the contrapositive of this proposition.

\item Now prove that if $x$ is an even integer, then $xy$ is an even integer.  Also, prove that if $y$ is an even integer, then $xy$ is an even integer. \label{PA:logicalequiv4}

\item Use the results proved in part~(\ref{PA:logicalequiv4}) and the explanation in part~(\ref{PA:logicalequiv2}) to explain why we have proved the contrapositive of the proposition in part~(\ref{PA:logicalequiv3}).
\end{enumerate}
\end{previewactivity}
\hbreak

\endinput

