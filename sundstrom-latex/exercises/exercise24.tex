\section*{Exercises for Section~\ref{S:quantifier}}
\begin{enumerate}
%
\xitem For each of the following, write the statement as an English sentence and then explain why the statement is false.
\label{exer:sec24-1}%
  \begin{enumerate}
    \item $\left(\exists x \in \mathbb{Q} \right) \left( x^2  - 3x - 7 = 0 \right)$.
    \item $\left( \exists x \in \mathbb{R}  \right) \left( x^2  + 1 = 0 \right)$.
    \item $\left( \exists m \in \N \right) \left( m^2 < 1 \right)$.
  \end{enumerate}
%
%
\item For each of the following, use a counterexample to show that the statement is false.  Then write the negation of the statement in English, without using symbols for quantifiers. \label{exer:sec24-2new}
\begin{enumerate}
    \yitem $\left( {\forall m \in \mathbb{Z}} \right)\left( {m^2 \text{ is even}} \right)$.
  \yitem $\left( {\forall x \in \mathbb{R}} \right)\left( {x^2  > 0} \right)$.
  \item For each real number $x$, $\sqrt{x} \in \R$.
  \item $\left( \forall m \in \Z \right) \left( \dfrac{m}{3} \in \Z \right)$.
    \item $\left( {\forall a \in \mathbb{Z}} \right)\left( {\sqrt {a^2 }  = a} \right)$.
%  \item For each positive integer $n$, $\sqrt{n} \in \R$.
    \yitem $\left( {\forall x \in \mathbb{R}} \right)\left( {\tan ^2 x + 1 = \sec ^2 x} \right)$.
\end{enumerate}

\item For each of the following statements
\label{exer:sec24-2}%
  \begin{itemize}
    \item Write the statement as an English sentence that does not use the symbols for quantifiers.
    \item Write the negation of the statement in symbolic form in which the negation symbol is not used.
    \item Write a useful negation of the statement in an English sentence that does not use the symbols for quantifiers.
  \end{itemize}
%
  \begin{enumerate}
    \yitem $\left( {\exists x \in \mathbb{Q}} \right)\left( {x > \sqrt 2 } \right)$.
    \item $\left( {\forall x \in \mathbb{Q}} \right)\left( {x^2  - 2 \ne 0} \right)$.
    \yitem $\left( {\forall x \in \mathbb{Z}} \right)\left( {x\text{ is even  or  }x\text{ is odd}} \right)$.
   \item $\left( {\exists x \in \mathbb{Q}} \right)\left( {\sqrt 2  < x < \sqrt 3 } \right)$.  \note  The sentence ``$\sqrt 2  < x < \sqrt 3$'' is actually a conjuction.  It means 
$\sqrt 2 < x$ and $x < \sqrt 3$.
   \yitem $\left( {\forall x \in \mathbb{Z}} \right)\left( {\text{If }x^2 \text{ is odd, then }x\text{ is odd}} \right)$.
  \item $\left( {\forall n \in \mathbb{N}} \right)$ [If  $n$ is a perfect square, then 
$\left( {2^n  - 1} \right)$  is not a prime number].
   \item $\left( {\forall n \in \mathbb{N}} \right)\left( {n^2  - n + 41} 
\text{ is a prime number} \right)$.
  \yitem $\left( {\exists x \in \mathbb{R}} \right) \left( {\cos ( {2x} ) = 2( {\cos x} )} \right)$.
 \end{enumerate}
%
\item \label{exer:sec24-3}%
Write each of the following statements as an English sentence that does not use the symbols for quantifiers.
\begin{multicols}{2}
\begin{enumerate}
  \yitem $\left( {\exists m} \in \Z \right)\left( {\exists n} \in \Z \right)\left( {m > n} \right)$
  \item $\left( {\exists m} \in \Z \right)\left( {\forall n} \in \Z \right)\left( {m > n} \right)$
  \item $\left( {\forall m} \in \Z \right)\left( {\exists n} \in \Z \right)\left( {m > n} \right)$
  \item $\left( {\forall m}\in \Z \right)\left( {\forall n} \in \Z \right)\left( {m > n} \right)$
  \yitem $\left( {\exists n} \in \Z \right)\left( {\forall m} \in \Z \right)\left( {m^2  > n} \right)$
  \item $\left( {\forall n} \in \Z \right)\left( {\exists m} \in \Z \right)\left( {m^2  > n} \right)$
\end{enumerate}
\end{multicols}
%
\xitem Write the negation of each statement in Exercise~(\ref{exer:sec24-3}) in symbolic form and as an English sentence that does not use the symbols for quantifiers.
\label{exer:sec24-4}%

\xitem Assume that the universal set is  $\mathbb{Z}$.  Consider the following sentence: \label{Exer:quantifier1}
\[
\left( {\exists t \in \mathbb{Z}} \right)\left( {t \cdot x = 20} \right).
\]
  \begin{enumerate}
    \item Explain why this sentence is an open sentence and not a statement.
    \item If  5  is substituted for  $x$, is the resulting sentence a statement?  If it is a statement, is the statement true or false?
    \item If  8  is substituted for  $x$, is the resulting sentence a statement?  If it is a statement, is the statement true or false?
    \item If  $-2$  is substituted for  $x$, is the resulting sentence a statement?  If it is a statement, is the statement true or false?
    \item What is the truth set of the open sentence  $\left( {\exists t \in \mathbb{Z}} \right)\left( {t \cdot x = 20} \right)$?
  \end{enumerate}


\item Assume that the universal set is  $\mathbb{R}$.  Consider the following sentence: \label{Exer:quantifier}
\[
\left( {\exists t \in \mathbb{R}} \right)\left( {t \cdot x = 20} \right).
\]
  \begin{enumerate}
    \item Explain why this sentence is an open sentence and not a statement.
    \item If  5  is substituted for  $x$, is the resulting sentence a statement?  If it is a statement, is the statement true or false?
    \item If  $\pi$  is substituted for  $x$, is the resulting sentence a statement?  If it is a statement, is the statement true or false?
    \item If  $0$  is substituted for  $x$, is the resulting sentence a statement?  If it is a statement, is the statement true or false?
    \item What is the truth set of the open sentence  $\left( {\exists t \in \mathbb{R}} \right)\left( {t \cdot x = 20} \right)$?
  \end{enumerate}

\item Let $\Z^*$ be the set of all nonzero integers.
\begin{enumerate}
  \item Use a counterexample to explain why the following statement is false:
\begin{center}
For each $x \in \Z^*$, there exists a $y \in \Z^*$ such that $xy = 1$.
\end{center}
  \item Write the statement in part~(a) in symbolic form using appropriate symbols for quantifiers.
  \item Write the negation of the statement in part~(b) in symbolic form using appropriate symbols for quantifiers.
  \item Write the negation from part~(c) in English without using the symbols for quantifiers.
\end{enumerate}


\item An integer $m$ is said to have the \emph{divides property} provided that for all integers $a$ and $b$, if $m$ divides $ab$, then $m$ divides $a$ or $m$ divides $b$.
\begin{enumerate}
  \item Using the symbols for quantifiers, write what it means to say that the integer $m$ has the divides property.
  \item Using the symbols for quantifiers, write what it means to say that the integer $m$ does not have the divides property.
  \item Write an English sentence stating what it means to say that the integer $m$ does not have the divides property.
\end{enumerate}


\item In calculus, we define a function  $f$  with domain  $\mathbb{R}$
  to be \textbf{strictly increasing}
\index{increasing, strictly}%
 provided that for all real numbers  $x$  and  $y$, $f\left( x \right) < f\left( y \right)$ whenever  $x < y$.
\label{exer:24-increasing}%
%
Complete each of the following sentences using the appropriate symbols for quantifiers:
  \begin{enumerate}
    \yitem A function  $f$  with domain  $\mathbb{R}$ is strictly increasing provided that 
$ \ldots .$

  \item A function  $f$  with domain  $\mathbb{R}$ is not strictly increasing provided \linebreak
that $ \ldots .$
\end{enumerate}

Complete the following sentence in English without using symbols for quantifiers:
\begin{enumerate}
\setcounter{enumii}{2}
\item A function  $f$  with domain  $\mathbb{R}$ is not strictly increasing provided 
\linebreak
that $ \ldots .$
  \end{enumerate}
%




\item In calculus, we define a function  $f$  to be \textbf{continuous}
\index{continuous}%
 at a real number  $a$  provided that for every  $\varepsilon  > 0$, there exists a  $\delta  > 0$ such that if  $\left| {x - a} \right| < \delta $, then  $\left| {f\left( x \right) - f\left( a \right)} \right| < \varepsilon $.
\label{exer:24-continuous}%

\note  The symbol  $\varepsilon $ is  the lowercase Greek letter epsilon,  and the symbol  $\delta $ is  the lowercase Greek letter delta.

Complete each of the following sentences using the appropriate symbols for quantifiers:
  \begin{enumerate}
    \item A function  $f$  is continuous at the real number  $a$  provided that $ \ldots .$
    \item A function  $f$  is not continuous at the real number  $a$  provided that $ \ldots .$
  \end{enumerate}

Complete the following sentence in English without using symbols for quantifiers:
\begin{enumerate}
\setcounter{enumii}{2}
  \item A function  $f$  is not continuous at the real number  $a$  provided that $ \ldots .$
\end{enumerate}


\item The following exercises contain definitions or results from more advanced mathematics courses.  Even though we may not understand all of the terms involved, it is still possible to recognize the structure of the given statements and write a meaningful negation of that statement. \label{exer:advanced}
\begin{enumerate}
  \item In abstract algebra, an operation $*$ on a set $A$ is called a \textbf{commutative operation} 
\index{commutative operation}%
 provided that for all $x, y \in A$, $x * y = y * x$.  Carefully explain what it means to say that an operation $*$ on a set $A$ is not a commutative operation.
  \item In abstract algebra, a \textbf{ring} 
\index{ring}%
consists of a nonempty set $R$ and two operations called addition and multiplication.  A nonzero element $a$ in a ring $R$ is called a \textbf{zero divisor} 
\index{zero divisor}%
provided that there exists a nonzero element $b$ in $R$ such that $ab = 0$ or $ba = 0$.  Carefully explain what it means to say that a nonzero element $a$ in a ring $R$ is not a zero divisor.


\item A set $M$ of real numbers is called a \textbf{neighborhood} 
\index{neighborhood}%
 of a real number $a$ provided that there exists a positive real number $\varepsilon$ such that the open interval 
$(a - \varepsilon, a + \varepsilon)$ is contained in $M$.  Carefully explain what it means to say that a set $M$ is not a neighborhood of a real number $a$.

  \item In advanced calculus, a sequence of real numbers $\left(x_1, x_2, \ldots, x_k, \ldots \right)$ is called a 
\textbf{Cauchy sequence} 
\index{Cauchy sequence}%
 provided that for each positive real number $\varepsilon$, there exists a natural number $N$ such that for all 
$m, n \in \N$, if $m > N$ and $n > N$, then $\left| x_n - x_m \right| < \varepsilon$.  Carefully explain what it means to say that the sequence of real numbers  $\left(x_1, x_2, \ldots, x_k, \ldots \right)$ is not a Cauchy sequence.
\end{enumerate}



%\item In abstract algebra, a \textbf{ring}
%\index{ring}%
% consists of a nonempty set $R$ and two operations called addition and multiplication.  These operations must satisfy certain properties.  If $a$ and $b$ are two elements in $R$, we write the product as $ab$.  A ring $R$ must have a so-called additive identity $0$ such that for each $a \in R$, $a + 0 = a = 0 + a$.  The element $0$ is also called the zero element of $R$.  
%
%\newpar
%A nonzero element $a$ in $R$ is called a \textbf{zero divisor}%
%\index{zero divisor}
%  provided that there exists a nonzero element $b$ in $R$ such that $ab = 0$.
%\begin{enumerate}
%  \item Using symbols for quantifiers, write what it means to say that the element $a$ in $R$ is a zero divisor.
%  \item Using symbols for quantifiers, write what it means to say that the element $a$ in $R$ is not a zero divisor.
%  \item Write an English sentence stating what it means to say that the element $a$ in $R$ is not a zero divisor.
%\end{enumerate}

\end{enumerate}




\subsection*{Explorations and Activities}
\setcounter{oldenumi}{\theenumi}
\begin{enumerate} \setcounter{enumi}{\theoldenumi} 
\item \textbf{Prime Numbers}.  \label{exer:prime} The following definition of a prime number is very important in many areas of mathematics.  We will use this definition at various places in the text.  It is introduced now as an example of how to work with a definition in mathematics. 
\begin{defbox}{D:prime}{A natural number  $p$  is  a \textbf{prime number}
\index{prime number}%
 provided that it is greater than 1 and the only natural numbers that are factors of  $p$  are  1  and  $p$.  A natural number other than 1 that is not a prime number is a \textbf{composite number}.
\index{composite number}%
  The number 1 is neither prime nor composite.}
\end{defbox}
Using the definition of a prime number, we see that  2, 3, 5, and  7  are prime numbers.  Also, 4  is a composite number since  $4 = 2 \cdot 2$;  10 is a composite number since  $10 = 2 \cdot 5$; and 60 is a composite number since $60 = 4 \cdot 15$.
\label{exer:prime}%
\begin{enumerate}
  \item Give examples of four natural numbers other than 2, 3, 5, and 7 that are prime numbers.
  \item Explain why a natural number  $p$  that is greater than 1 is a prime number provided that
\begin{center} For all  $d \in \mathbb{N}$, if  $d$ is a factor of $p$, then  $d = 1$  or  
$d = p$.
\end{center} 
  \item Give examples of four natural numbers that are composite numbers and explain why they are composite numbers.
  \item Write a useful description of what it means to say that a natural number is a composite number (other than saying that it is not prime).
\end{enumerate}


\item \textbf{Upper Bounds for Subsets of $\mathbb{R}$}.  \label{A:upper}
Let  $A$  be a subset of the real numbers.  A number  $b$  is called an \textbf{upper bound}
\index{upper bound}%
 for the set  $A$ provided that for each element  $x$  in $A$, $x \leq b$.

\begin{enumerate}
  \item Write this definition in symbolic form by completing the following:

Let  $A$  be a subset of the real numbers.  A number  $b$  is called an upper bound for the set  $A$ provided that $ \ldots .$

  \item Give examples of three different upper bounds for the set \\ 
$A = \left\{ x \in \mathbb{R} \mid 1 \leq x \leq 3 \right\}$.

  \item Does the set  $B = \left\{ x \in \mathbb{R} \mid x > 0 \right\}$ have an upper bound?  Explain.

  \item Give examples of three different real numbers that are not upper bounds for the set  
$A = \left\{ x \in \mathbb{R} \mid 1 \leq x \leq 3 \right\}$. \label{A:upper4}%

  \item Complete the following in symbolic form:  ``Let  $A$  be a subset of $\R$.  A number  $b$  is not an upper bound for the set  $A$   provided that $ \ldots .$''

  \item Without using the symbols for quantifiers, complete the following sentence:  ``Let  $A$  be a subset of $\R$.  A number  $b$  is not an upper bound for the set  $A$ provided that $ \ldots .$''  \label{A:upper6}%

  \item Are your examples in Part~(\ref{A:upper4}) consistent with your work in 
Part~(\ref{A:upper6})?  Explain.
\end{enumerate}

\item  \textbf{Least Upper Bound for a Subset of $\R$}. \label{exer:leastupper}
In Exercise~\ref{A:upper}, we introduced the definition of an upper bound for a subset of the real numbers.  Assume that we know this definition and that we know what it means to say that a number is not an upper bound for a subset of the real numbers.

Let  $A$  be a subset of  $\mathbb{R}$.  A real number  $\alpha $ is the 
\textbf{least upper bound}
\index{least upper bound}%
 for  $A$  provided that  $\alpha $  is an upper bound for  $A$, and if $\beta $ is an upper bound for  $A$, then  $\alpha  \leq \beta $.

\noindent
\textbf{Note:}  The symbol  $\alpha $ is  the lowercase Greek letter alpha,  and the symbol  
$\beta $ is  the lowercase Greek letter beta.

If we define  $P\left( x \right)$ to be ``$x$  is an upper bound for  $A$,'' then we can write the definition for least upper bound as follows:

A real number  $\alpha $ is the \textbf{least upper bound} for  $A$  provided that \\ $P\left( \alpha  \right) \wedge \left[ {\left( {\forall \beta  \in \mathbb{R}} \right)\left( {P\left( \beta  \right) \to \left( {\alpha  \leq \beta } \right)} \right)} \right]$.

\begin{enumerate}
  \item Why is a universal quantifier used for the real number  $\beta $?
  %\item How do we negate a conjunction?
  \item Complete the following sentence in symbolic form:  ``A real number  $\alpha $ is not the least upper bound for  $A$  provided that $ \ldots .$''
  \item Complete the following sentence as an English sentence:  ``A real number  
$\alpha $ is not the least upper bound for  $A$  provided that $ \ldots .$''
\end{enumerate}
\end{enumerate}
\hbreak
\endinput


  \item $\left( {\forall n \in \mathbb{N}} \right) \left[ \text{If } n \text{ is a perfect square, then } \left( {2^n  - 1} \right) \text{ is not a prime number} \right] $.





\begin{activity}[Upper Bounds for Subsets of $\mathbb{R}$]\label{A:upper}
Let  $A$  be a subset of the real numbers.  A number  $b$  is called an \textbf{upper bound} for the set  $A$ provided that for each element  $x$  in $A$, $x \leq b$.

\begin{enumerate}
  \item Write this definition in symbolic form by completing the following:

Let  $A$  be a subset of the real numbers.  A number  $b$  is called an upper bound for the set  $A$ provided $ \ldots $

  \item Give examples of three different upper bounds for the set \\ 
$A = \left\{ x \in \mathbb{R} \mid 1 \leq x \leq 3 \right\}$.

  \item Does the set  $B = \left\{ x \in \mathbb{R} \mid x > 0 \right\}$ have an upper bound?  Explain.

  \item Give examples of three different real numbers that are not upper bounds for the set  
$A = \left\{ x \in \mathbb{R} \mid 1 \leq x \leq 3 \right\}$. \label{A:upper4}

  \item Complete the following in symbolic form:  ``Let  $A$  be a subset of the real numbers.  A number  $b$  is not an upper bound for the set  $A$   provided $ \ldots $''

  \item Without using the symbols for quantifiers, complete the following sentence:  ``Let  $A$  be a subset of the real numbers.  A number  $b$  is not an upper bound for the set  $A$ provided $ \ldots $''  \label{A:upper6}

  \item Are your examples in Part~(\ref{A:upper4}) consistent with your work in Part~(\ref{A:upper6})?  Explain.
\end{enumerate}
\hbreak
\end{activity}
%
\begin{activity}[Least Upper Bound for a Subset of $\mathbb{R}$] \label{A:least}
In \\ Activity~\ref{A:upper}, we introduced the definition of an upper bound for a subset of the real numbers.  Assume we know this definition and that we know what it means to say that a number is not an upper bound for a subset of the real numbers.

Let  $A$  be a subset of  $\mathbb{R}$.  A real number  $\alpha $ is the \textbf{least upper bound} for  $A$  provided that  $\alpha $  is an upper bound for  $A$, and if $\beta $ is an upper bound for  $A$, then  $\alpha  \leq \beta $.

Note:  The symbol  $\alpha $ is  the lower case Greek letter ``alpha'',  and the symbol  $\beta $ is  the lower case Greek letter ``beta.''

If we define  $P\left( x \right)$ to be, ``$x$  is an upper bound for  $A$'', then we can write the definition for least upper bound as:

A real number  $\alpha $ is the \textbf{least upper bound} for  $A$  provided that \\ $P\left( \alpha  \right) \wedge \left[ {\left( {\forall \beta  \in \mathbb{R}} \right)\left( {P\left( \beta  \right) \to \left( {\alpha  \leq \beta } \right)} \right)} \right]$.
%
\begin{enumerate}
  \item Why is a universal quantifier used for the real number  $\beta $?
  \item How do we negate a conjunction?
  \item Complete the following sentence in symbolic form:  ``A real number  $\alpha $ is not the least upper bound for  $A$  provided that $ \ldots $''.
  \item Complete the following sentence as an English sentence:  ``A real number  
$\alpha $ is not the least upper bound for  $A$  provided that $ \ldots $''.
\end{enumerate}

\end{activity}
