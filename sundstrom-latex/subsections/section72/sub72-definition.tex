\subsection*{Definition of an Equivalence Relation}
In mathematics, as in real life, it is often convenient to think of two different things as being essentially the same.  For example, when you go to a store to buy a cold soft drink, the cans of soft drinks in the cooler are often sorted by brand and type of soft drink.  The Coca Colas are grouped together, the Pepsi Colas are grouped together, the Dr. Peppers are grouped together, and so on.  When we choose a particular can of one type of soft drink, we are assuming that all the cans are essentially the same.  Even though the specific cans of one type of soft drink are physically different, it makes no difference which can we choose.  In doing this, we are saying that the cans of one type of soft drink are equivalent, and we are using the mathematical notion of an equivalence relation.

An equivalence relation on a set is a relation with a certain combination of properties that allow us to sort the elements of the set into certain classes.  In this section, we will focus on the properties that define an equivalence relation, and in the next section, we will see how these properties allow us to sort or partition the elements of the set into certain classes.

\begin{defbox}{equivalencerelation}{Let  $A$  be a nonempty set.  A relation  $\sim$  on the set  $A$  is an \textbf{equivalence relation}
\index{equivalence relation}%
\index{relation!equivalence}%
 provided that  $\sim$  is reflexive, symmetric, and transitive.  For  $a, b \in A$, if  $\sim$  is an equivalence relation on  $A$  and  $a \sim b$, we say that  $\boldsymbol{a}$  \textbf{is equivalent to}  $\boldsymbol{b}$.}
\end{defbox}
Most of the examples we have studied so far have involved a relation on a small finite set.  For these examples, it was convenient to use a directed graph to represent the relation.  It is now time to look at some other type of examples, which may prove to be more interesting.  In these examples, keep in mind that there is a subtle difference between the reflexive property and the other two properties.  The reflexive property states that some ordered pairs actually belong to the relation  $R$, or some elements of  $A$  are related.  The reflexive property has a universal quantifier and, hence, we must prove that for all  $x \in A$,  $x \mathrel{R} x$.  Symmetry and transitivity, on the other hand, are defined by conditional sentences.  We often use a direct proof for these properties, and so we start by assuming the hypothesis and then showing that the conclusion must follow from the hypothesis.

\begin{example}[\textbf{A Relation that Is Not an Equivalence Relation}] \hfill \\
Let  $M$  be the relation on  $\mathbb{Z}$  defined as follows:

\begin{list}{}
\item For  $a, b \in \mathbb{Z}$,  $a \mathrel{M} b$  if and only if  $a$  is a multiple of  $b$.
\end{list}
\vskip6pt
\noindent
So $a \mathrel{M} b$ if and only if there exists a  $k \in \mathbb{Z}$ such that  
$a = b k$.

\begin{itemize}
\item The relation  $M$  is reflexive on  $\mathbb{Z}$ since for each  $x \in \mathbb{Z}$,  
$x = x \cdot 1$ and, hence, $x \mathrel{M} x$.

\item Notice that  $4 \mathrel{M} 2$, but  $2 \mathrel{\not \negthickspace M} 4$.  So there exist integers  $x$  and  $y$  such that  $x \mathrel{M} y$ but  
$y \mathrel{\not \negthickspace M} x$. Hence, the relation  $M$  is not symmetric.

\item Now assume that  $x \mathrel{M} y$ and  $y \mathrel{M} z$.  Then there exist integers  $p$  and  $q$  such that
\[
x = y  p\text{  and  }y = z q .
\]
Using the second equation to make a substitution in the first equation, we see that  
%\linebreak
$x = z \left( {p q} \right)$.  Since  $p q \in \mathbb{Z}$, we have shown that  $x$  is a multiple of  $z$ and hence  $x \mathrel{M} z$.  Therefore,  $M$  is a transitive relation.
\end{itemize}
\end{example}
\noindent
The relation $M$ is reflexive on $\Z$ and is transitive, but since $M$ is not symmetric, it is not an equivalence relation on $\Z$.
\hbreak

\begin{prog}[\textbf{A Relation that Is an Equivalence Relation}] \label{prog:example-equiv} \hfill \\
Define the relation  $\sim$  on  $\Q$  as follows:  For all $a, b \in \Q$,
$a \sim b$  if and only if  $a - b \in \Z $.  For example:
\begin{itemize}
  \item  $\dfrac{3}{4} \sim \dfrac{7}{4}$ since $\dfrac{3}{4} - \dfrac{7}{4} = -1$ and $-1 \in \Z$.
  \item $\dfrac{3}{4} \not  \sim \dfrac{1}{2}$ since $\dfrac{3}{4} - \dfrac{1}{2} = \dfrac{1}{4}$ and 
$\dfrac{1}{4} \notin \Z$.
\end{itemize}
To prove that $\sim$ is reflexive on $\Q$, we note that for all $a \in \Q$, $a - a = 0$.  Since $0 \in \Z$, we conclude that $a \sim a$.  Now prove that the relation $\sim$ is symmetric and transitive, and hence, that 
$\sim$ is an equivalence relation on $\Q$.
\end{prog}
\hbreak

\endinput
