\section*{Section \ref{S:infinitesets} Countable Sets}

\begin{enumerate}
\item All except Part (d) are true.


\item \begin{enumerate}
\item The function $f: \mathbb{N} \to F^+$ defined by $f \left( n \right) = 5n$ for all 
$n \in \mathbb{N}$ is a bijection.

\item The function $g: \mathbb{Z} \to F$ defined by $f \left( n \right) = 5n$ for all 
$n \in \mathbb{N}$ is a bijection.  Therefore $F \approx \mathbb{Z}$.  By 
Theorem~\ref{T:ZequivtoN}, $\mathbb{Z}$ is countably infinite and hence, $F$ is countably infinite.

\item Let $A = \left\{ \dfrac{1}{2^k} \mid k \in \mathbb{N} \right\}$.  Define 
$f: \mathbb{N} \to A$ by $f \left( n \right) = \dfrac{1}{2^n}$ for all $n \in \mathbb{N}$.  Prove that the function $f$ is a bijection.

\item Let $B = \left\{ n \in \mathbb{Z} \mid n \geq -10 \right\}$.  Prove that the function $g$ is a bijection, where $g: B \to \mathbb{N}$ by $g \left( x \right) = x + 11$ for all $x \in B$.

\item Define $f\x \N \to \N -\{4, 5, 6 \}$ by
\begin{equation} \notag
f( n ) = 
\begin{cases}
n                        &\text{if $n = 1$, $n = 2$, or $n = 3$} \\
f ( n + 3 )   &\text{if $n \geq 4$}.
\end{cases}
\end{equation}
Prove that the function $f$ is a bijection.

It is also possible to use Corollary~\ref{C:subsetofcountable} to conclude that $\mathbb{N} - \left\{ 4, 5, 6 \right\}$ is countable, but it must also be proved that 
$\mathbb{N} - \left\{ 4, 5, 6 \right\}$ cannot be finite.

\item  Let $A = \left\{ m \in \mathbb{Z} \mid m \equiv 2 \pmod 3 \right\} = 
\left\{ 3k + 2 \mid k \in \mathbb{Z} \right\}$.  Prove that the function $f: \mathbb{Z} \to A$ is a bijection, where $f \left( x \right) = 3x + 2$ for all $x \in \mathbb{Z}$.  This proves that 
$\mathbb{Z} \approx A$ and hence, $\mathbb{N} \approx A$.
\end{enumerate}



\item We will use a proof by contradiction.  So we assume that $A$ is an infinite set, $A \subseteq B$, and that $B$ is a finite set.   Since $A \subseteq B$ and $B$ is a finite set, Theorem~9.13 tells us that $A$ is a finite set.  This is a contradiction.  Therefore, $B$ is an infinite set.




\item Let $m, n \in \mathbb{N}$ and assume that $g \left( m \right) = g \left( n \right)$.  If 
$g \left( m \right) = g \left( n \right) = x$, then since $x \notin A$ and for each 
$n \in \mathbb{N}$, $f \left( n - 1 \right) \in A$, we conclude that $m = n = 1$.  If 
$g \left( m \right) = g \left( n \right) \ne x$, then 
$f \left( m - 1 \right) = f \left( n - 1 \right)$, and since $f$ is an injection, $m - 1 = n - 1$.  Hence, $m = n$ and $g$ is an injection.

Now let $y \in A \cup \left\{ x \right\}$.  If $y = x $, then $g \left( 1 \right) = x$.  If 
$y \ne x$, then since $f$ is a surjection, there exists an $m$ in $\mathbb{N}$ such that 
$f \left( m \right) = y$.  Then, $m + 1 \in \mathbb{N}$ and
\[
g \left( m + 1 \right) = f \left( m \right) = y.
\]
Therefore, $g$ is a surjection.

\item For each $n \in \mathbb{N}$, let $P \left( n \right)$ be ``If 
$\text{card} \left( B \right) = n$, then $A \cup B$ is a countably infinite set.

Notice that $P \left( 1 \right)$ is true by Exercise~(3).

Now let $k \in \mathbb{N}$ and assume that $P \left( k \right)$ is true.  Let $B$ be a finite set with $\text{card} \left( B \right) = k + 1$.  Let $x \in B$.  Then 
$\text{card} \left( B - \left\{ x \right\} \right) = k$.  Hence, by the inductive assumption, 
$A \cup \left( B - \left\{ x \right\} \right)$ is a countably infinite set.  Now,
\[
A \cup B = A \cup \left( B - \left\{ x \right\} \right) \cup \left\{ x \right\}.
\]
Since $A \cup \left( B - \left\{ x \right\} \right)$ is countably infinite, Exercise~(3) implies that $A \cup B$ is countably infinite.  This proves that if $P \left( k \right)$ is true, then 
$P \left( k + 1 \right)$ is true.

\item Let $m, n \in \mathbb{N}$ and assume that $h \left( n \right) = h \left( m \right)$. Then since $A$ and $B$ are disjoint, either $h \left( n \right)$ and $h \left( m \right)$ are both in $A$ or are both in $B$.  If they are both in $A$, then both $m$ and $n$ are odd and
\[
f \left( \frac{n + 1}{2} \right) = h \left( n \right) = h \left( m \right) = f \left( \frac{m + 1}{2} \right).
\]
Since $f$ is an injection, this implies that $\dfrac{n + 1}{2} = \dfrac{m + 1}{2}$ and hence that $n = m$.  Similary, if both $h \left( n \right)$ and $h \left( m \right)$ are in $B$, then $m$ and $n$ are even and $g \left( \dfrac{n}{2} \right) = g \left( \dfrac{m}{2} \right)$, and since $g$ is an injection, $\dfrac{n}{2} = \dfrac{m}{2}$ and $n = m$.  Therefore, $h$ is an injection.

Now let $y \in A \cup B$.  There are only two cases to consider:  $y \in A$ or $y \in B$.  If 
$y \in A$, then since $f$ is a surjection, there exists an $m \in \mathbb{N}$ such that 
$f \left( m \right) = y$.  Let $n = 2m - 1$.  Then $n$ is an odd natural number, 
$m = \frac{n + 1}{2}$,  and
\[
h \left( n \right) = f \left( \frac{n + 1}{2} \right) = f \left( m \right) = y.
\]
If $y \in B$, then since $g$ is a surjection, there exists an $m \in \mathbb{N}$ such that 
$g \left( m \right) = y$.  Let $n = 2m$.  Then $n$ is an even natural number, 
$m = \frac{n}{2}$,  and
\[
h \left( n \right) = g \left( \frac{n}{2} \right) = g \left( m \right) = y.
\]
Therefore, $h$ is a surjection.

\item By Theorem~\ref{T:positiverationals}, the set $\mathbb{Q}^+$ of positive rational numbers is countably infinite. So by Theorem~\ref{T:addonetocountable}, 
$\mathbb{Q}^+ \cup \left\{ 0 \right\}$is countably infinite.  Now let $\mathbb{Q}^-$ be the set of all negative rational numbers.  Then, $\mathbb{Q}^- \approx \mathbb{Q}^+$.  Since 
$\mathbb{Q} = \left( \mathbb{Q}^+ \cup \left\{ 0 \right\} \right) \cup \mathbb{Q}^-$, 
Theorem~\ref{T:unionofcountable}, $\mathbb{Q}$ is countably infinite.

\item Since $A - B \subseteq A$, the set $A - B$ is countable.  Assume $A - B$ is finite.  Since 
$A = \left( A - B \right) \cup B$, this would imply that $A$ is finite.  This is a contradiction and hence, $A - B$ is countably infinite.

\item \begin{enumerate}
\item Let $m, n, s, t \in \mathbb{N}$ and assume that 
$f \left( m, n \right) = f \left( s, t \right)$.  Then, 
\[
2^{m - 1} \left( 2n - 1 \right) = 2^{s - 1} \left( 2t - 1 \right).
\]
\begin{itemize}
\item If $m >s$, then $2^{m - s} \left( 2n - 1 \right) = 2t - 1$.  Since $2t - 1$ is an odd natural number, $2^{m - s}$ must be equal to 1.  Therefore, $m - s = 0$ and $m = s$.  This implies that $2n - 1 = 2t - 1$ and hence that $n = t$.  Hence, 
$\left( m, n \right) = \left( s, t \right)$.

\item If $m < s$, then a similar proof shows that $\left( m, n \right) = \left( s, t \right)$.

\item If $m = s$, then $2n - 1 = 2t - 1$ and $n = t$.  Therefore, 
$\left( m, n \right) = \left( s, t \right)$
\end{itemize}
This proves that $f$ is an injection.

\item Let $y \in \mathbb{N}$.  Then by Exercise~(\ref{exer:fundtheoremcons}) in Section~\ref{S:primefactorizations}, there exists an odd natural number $x$ and a non-negative integer $k$ such that $y = 2^k x$.  Since $x$ is odd, there exists a natural number $n$ such that 
$x = 2n - 1$.  Let $m = k + 1$.  Then $m \in \mathbb{N}$, $k = m - 1$, and
\[
\begin{aligned}
f \left( m, n \right) &= 2^{m - 1} \left( 2n - 1 \right) \\
                      &= 2^k x \\
                      &= y. \\
\end{aligned}
\]
Therefore, $f$ is a surjection.
\end{enumerate}

\item \begin{enumerate}
\item Since the function $f$ in Exercise~(8) is a bijection, 
$\mathbb{N} \times \mathbb{N} \approx \mathbb{N}$ and hence 
$\text{card} \left( \mathbb{N} \times \mathbb{N} \right) = \aleph_0$.

\item By Exercise~(\ref{exer:sec92-7}) in Section~\ref{S:finitesets}, 
$A \times B \approx \mathbb{N} \times \mathbb{N}$, and hence by Part~(a), $A \times B$ is countably infinite.
\end{enumerate}

\item To prove that $g$ is an injection, let $r, s \in \mathbb{N}$ with $r \ne s$.  We may assume that $r < s$ since one of the two numbers must be less than the other.  Then notice that 
$g \left( r \right) \in \left\{ g \left( 1 \right), g \left( 2 \right), \ldots, g \left( s-1 \right) \right\}$.  Since 
$g \left( s \right) \in B - \left\{ g \left( 1 \right), g \left( 2 \right), \ldots, g \left( s-1 \right) \right\}$, we conclude that $g \left( r \right) \ne g \left( s \right)$ and hence, $g$ is an injection.

To prove that $g$ is a surjection, let $b \in B$.  If $b$ is the smallest element in $B$, then 
$g \left( 1 \right) = b$.  If $b$ is not the smallest element in $B$,  then for some 
$k \in \mathbb{N}$, there will be $k$ natural numbers in $B$ that are less than $b$.  In this case, $g \left( k + 1 \right) = b$.  Hence, $g$ is a surjection.

\item Let $S$ be a countable set and assume that $A \subseteq S$.  There are two cases:  $A$ is finite or $A$ is infinite.  If $A$ is finite, then $A$ is countable.  If $A$ is infinite, let 
$f: S \to \mathbb{N}$ be a bijection and define $g: A \to f \left( A \right)$ by 
$g \left( x \right) = f \left( x \right)$, for each $x \in A$.  Then, $g$ is a bijection and so 
$A \approx f \left( A \right)$.  Since $f \left( A \right) \subseteq \mathbb{N}$, we see that 
$f \left( A \right)$ and $A$ are countable.

\item Let $A$ be the set of all rational numbers between 0 and 1.  Since
\[
F = \left\{ \frac{1}{2^k} \mid k \in \mathbb{N} \right\}
\]
is an infinite subset of $A$, $A$ is not a finite set.  So by Corollary~\ref{C:subsetofcountable}, $A$ is countably infinite.
\end{enumerate}



\subsection*{Explorations and Activities}
\setcounter{oldenumi}{\theenumi}
\begin{enumerate} \setcounter{enumi}{\theoldenumi}
\item \begin{enumerate}
\item 
$f \left( \dfrac{2}{3} \right) = {2^2} \cdot {3^1} = 12$ 

$f \left( \dfrac{5}{6} \right) = f \left( \dfrac{5}{2 \cdot 3} \right) = {5^2}\cdot {2 \cdot 3} = 150$ 

$f \left( 6 \right) = f(2 \cdot 3) = 2^2 \cdot 3^2 = 36$ 

$f \left( \dfrac{12}{25} \right) = f \left( \dfrac{2^2 \cdot 3}{5^2} \right) = {2^4 \cdot 3^2}\cdot {5^3}$ 

$f \left( \dfrac{375}{392} \right) = f \left( \dfrac{3 \cdot 5^3}{2^3 \cdot 7^2} \right) = {3^2 \cdot 5^6} \cdot {2^5 \cdot 7^3}$

$f \left( \dfrac{2^3 \cdot 11^3}{3 \cdot 5^4} \right) = 2^6 \cdot 11^6 \cdot 3 \cdot 5^7$.
%\end{multicols}

\begin{multicols}{2}
\item $f \left( 10 \right) = 100$

\item $f \left( \dfrac{2}{3} \right) = {2^2} \cdot {3^1} = 12$

\item $f \left( \dfrac{2^4 \cdot 17}{3^3 \cdot 13} \right) = 2^8 \cdot 3^5 \cdot 13 \cdot 17^2$
\end{multicols}

\item To prove that the function $f$ is an injection, we let $x, y \in \Q^+$ and assume that $f(x) = f(y)$.  We will write
\begin{equation}
x = \dfrac{p_1^{\alpha_1} p_2^{\alpha_2} \cdots p_r^{\alpha_r}}{q_1^{\beta_1} q_2^{\beta_2} \cdots q_s^{\beta_s}}
\end{equation}
where $p_1$, $p_2$, \ldots, $p_r$ are distinct prime numbers, $q_1$, $q_2$, \ldots, $q_s$ are distinct prime numbers, and \\
$\alpha_1$, $\alpha_2$, \ldots, $\alpha_r$ and $\beta_1$, $\beta_2$, \ldots, $\beta_s$ are natural numbers.   We will also write
\begin{equation}
y = \dfrac{u_1^{\gamma_1} u_2^{\gamma_2} \cdots u_m^{\gamma_m}}{v_1^{\delta_1} v_2^{\delta_2} \cdots v_n^{\delta_n}}
\end{equation}
where $u_1$, $u_2$, \ldots, $u_m$ are distinct prime numbers, $v_1$, $v_2$, \ldots, $v_n$ are distinct prime numbers, and \\
$\gamma_1$, $\gamma_2$, \ldots, $\gamma_m$ and $\delta_1$, $\delta_2$, \ldots, $\delta_n$ are natural numbers.  From the assumption that $f(x) = f(y)$, we conclude that
\begin{align*}
\left( p_1^{2 \alpha_1} p_2^{2 \alpha_2} \cdots p_r^{2 \alpha_r} \right) &\left( q_1^{2 \beta_1-1} q_2^{ \beta_2-1} \cdots q_s^{2 \beta_s-1} \right) \\
 &= 
\left( u_1^{2 \gamma_1} u_2^{2 \gamma_2} \cdots u_m^{2 \gamma_m} \right) \left( v_1^{2 \delta_1-1} v_2^{ \delta_2-1} \cdots v_n^{2 \delta_n-1} \right).
\end{align*}
Now, the Fundamental Theorem of Arithmetic tells us that the prime factoriztion of a natural number is unique.  Using this and the previous equation, we can conclude that 
$r = m$, $s = n$, and 
\begin{itemize}
\item For each natural number $k$ with $1 \leq k \leq r$, $p_k = u_k$ and 
$\alpha_k = \gamma_k$;
\item For each natural number $j$ with $1 \leq j \leq s$, $q_j = v_j$ and 
$\beta_j = \delta_j$.
\end{itemize}
Using the expressions for $x$ and $y$ in~(1) and~(2), we can then conclude that $x = y$.  This proves the function $f$ is an injection.


\item To prove that $f$ is a surjection, we let $n \in \N$.  We then write the prime factorization of $n$ as follows:
\[
n = \left( p_1^{\gamma_1} p_2^{\gamma_2} \cdots p_r^{\gamma_r} \right) \left( q_1^{\delta_1} q_2^{\delta_2} \cdots q_s^{\delta_s} \right),
\]
where for each natural number $k$ with $1 \leq k \leq r$, $\gamma_k$ is even, and for each $j$ with $1 \leq j \leq s$, $\delta_j$ is even.  So:
\begin{itemize}
\item For each natural number $k$ with 
$1 \leq k \leq r$, there exists $\alpha_k \in \N$ such that $\gamma_k = 2 \alpha_k$.
\item For each natural number $j$ with 
$1 \leq j \leq s$, there exists $\beta_j \in \N$ such that $\delta_j = 2 \beta_j - 1$.
\end{itemize}

We now let $x = \dfrac{p_1^{\alpha_1} p_2^{\alpha_2} \cdots p_r^{\alpha_r}}{q_1^{\beta_1} q_2^{\beta_2} \cdots q_s^{\beta_s}}$.  Then, $x \in \Q^+$ and
\begin{align*}
f \left( x \right) &= f \left( \frac{p_1^{\alpha_1} p_2^{\alpha_2} \cdots p_r^{\alpha_r}}{q_1^{\beta_1} q_2^{\beta_2} \cdots q_s^{\beta_s}} \right) \\
                   &= \left( p_1^{2 \alpha_1} p_2^{2 \alpha_2} \cdots p_r^{2 \alpha_r} \right) \left( q_1^{2 \beta_1-1} q_2^{ \beta_2-1} \cdots q_s^{2 \beta_s-1} \right) \\
                   &= \left( p_1^{\gamma_1} p_2^{\gamma_2} \cdots p_r^{\gamma_r} \right) \left( q_1^{\delta_1} q_2^{\delta_2} \cdots q_s^{\delta_s} \right) \\
                   &= n
\end{align*}
This proves that the function $f$ is a surjection.


\item The proofs in Parts~(5) and~(6), prove that the function $f$ is a bijection.  From this, we conclude that $\Q^+ \approx \N$.

\end{enumerate}

\end{enumerate}

\hbreak
\endinput
