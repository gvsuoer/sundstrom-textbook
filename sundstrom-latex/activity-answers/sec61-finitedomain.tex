\documentclass[11pt]{article}
\usepackage{c://pctex/activity}

\lhead{}
\chead{\textbf{\large{Exercise 8 -- Section 6.1\\Creating Functions with Finite Domains}}}
\rhead{}
\lfoot{\emph{Mathematical Reasoning: Writing and Proof, Third Ed.} \\Ted Sundstrom}
\cfoot{}
\rfoot{\copyright \the\year\, by Pearson Education, Inc.\\}


\begin{document}
\noindent
Let  $A = \left\{ {a, b, c, d} \right\}$, $B = \left\{ {a, b, c} \right\}$, and  
$C = \left\{ {s, t, u, v} \right\}$.  In each of the following exercises, draw an arrow diagram to represent your function when it is appropriate.

\begin{enumerate}
\item Define  $f:A \to C$  by  $f( a ) = s$, $f( b ) = t$, 
$f( c ) = u$, $f( d ) = v$.  The range of  $f$  is the set  $C$. 

\item Define  $f:A \to C$  by  $f( a ) = u$, $f( b ) = u$, 
$f( c ) = u$, $f( d ) = v$.  The range of  $f$  is the set   
$\left\{ {u, v} \right\}$.

\item It is not possible to have a function  $f:B \to C$ whose range is the set   $C$  since the range of such a function can contain at most 3 elements.

\item Define  $f:A \to C$  by  $f( a ) = u$, $f( b ) = u$, 
$f( c ) = u$, $f( d ) = u$.  The range of  $f$  is the set   
$\left\{ y \right\}$.

\item The function in Part (1) is a function  $f:A \to C$  that satisfies the following condition:

\begin{list}{}
\item For all  $x, y \in A$,  if  $x \ne y$, then  $f( x ) \ne f( y )$.
\end{list}

\item It is not possible to create a function  $f:A \to \left\{ {s, t, u} \right\}$ that satisfies the following condition:

\begin{list}{}
\item For all  $x, y \in A$,  if  $x \ne y$, then  $f( x ) \ne f( y )$.
\end{list}

If  $f( a ), f( b ), f( c )$ are all distinct values, then  
$f( d )$ must be equal to one of them.
\end{enumerate}


\end{document}
