\section*{Section \ref{S:cases}}
\renewcommand{\labelenumi}{(\textbf{\alph{enumi}})}

\begin{list}{\bf{\ref{exer:consecutive}.}}
\item Use the fact that $n^2 + n = n ( n+1 )$.
\end{list}


\begin{list}{\bf{\ref{exer:sec33-6}.}}
\item Do not use the quadratic formula.  Try a proof by contradiction.  If there exists a  solution of the equation $x^2 + x - u = 0$ that is an integer, then we can conclude that there exists an integer $n$ such that $n^2 + n - u = 0$.  Then,
\[
u = n \left( n + 1 \right).
\]
From Exercise~(1), we know that $n(n + 1)$ is even and hence, $u$ is even.  This contradicts the assumption that $u$ is odd.
\end{list}

\begin{list}{\bf{\ref{exer:sec35-5}.}}
\item If $n$ is an odd integer, then there exists an integer $m$ such that $n = 2m + 1$.  Use two cases: (1) $m$ is even; (2) $m$ is odd.  If $m$ is even, then there exists an integer $k$ such that $m = 2k$ and this means that $n = 2(2k) + 1$ or $n = 4k + 1$.  If $m$ is odd, then there exists an integer $k$ such that $m = 2k + 1$.  Then $n = 2(2k + 1) + 1$ or $n = 4k + 3$.
\end{list}


\begin{list}{\bf{\ref{exer:sec34-quadratic}.}}
\item If $a \in \Z$ and $a^2 = a$, then $a(a - 1) = 0$.  Since the product is equal to zero, at least one of the factors must be zero.  In the first case, $a = 0$.  In the second case, 
$a - 1 = 0$ or $a = 1$.
\end{list}

\begin{list}{\bf{\ref{exer:dividesproduct}.}}
\item \begin{list}{\bf{(c)}}
\item For all integers  $a$, $b$, and  $d$  with $d \ne 0$, if  $d$  divides the product  $ab $, then $d$  divides  $a$  or  $d$  divides  $b$.
\end {list}
\end{list}


\begin{list}{\bf{\ref{exer:sec34-6}.}}
\item \begin{enumerate}
\item The statement, for all integers $m$ and $n$, if 4 divides $\left(m^2 + n^2 - 1 \right)$, then $m$ and $n$ are consecutive integers, is false.  A counterexample is $m = 2$ and $n = 5$.

The statement, for all integers $m$ and $n$, if $m$ and $n$ are consecutive integers, then 4 divides $\left(m^2 + n^2 - 1 \right)$, is true.  To prove this, let $n = m + 1$.  Then
\[
m^2 + n^2 - 1 = 2m^2 + 2m = 2m(m + 1).
\]
We have proven the $m(m + 1)$ is even.  (See Exercise~(1).)  So this can be used to prove that 4 divides $\left(m^2 + n^2 - 1 \right)$.
\end{enumerate}
\end{list}



\begin{list}{\bf{\ref{exer:a2plus1not2n}.}}
\item Try a proof by contradiction with two cases:  $a$ is even or $a$ is odd.
\end{list}





\begin{list}{\bf{\ref{exer:absvalue}.}}
\item \begin{list}{\bf{(a)}}
\item  One way is to use three cases:  (i) $x > 0$; (ii) $x = 0$; and $x < 0$.  For the first case, $-x < 0$ and $\left| -x \right| = -( -x ) = x = \left| x \right|$.
\end{list}
\end{list}


\begin{list}{\bf{\ref{exer:absvaluex}.}}
\item \begin{list}{\bf{(a)}}
\item  For each real number $x$, $\left| x \right| \geq a$ if and only if $x \geq a$ or 
$x \leq -a$.
\end{list}
\end{list}
\hbreak
\renewcommand{\labelenumi}{\textbf{\arabic{enumi}.}}

\endinput
