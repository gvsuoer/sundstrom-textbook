\begin{previewactivity}[\textbf{The Definition of a Function}] \label{PA:functiondef} \hfill \\
The domain and codomain of the functions in Beginning Activity~\ref{PA:previousfunctions} is the set  $\R$ of all real numbers, or some subset of   $\R$.  In most of these cases, the way in which the function associates elements of the domain with elements of the codomain is by a rule determined by some mathematical expression.  For example, when we say that $f$  is the function such that
\[
f( x ) = \dfrac{x}{{x - 1}},
\]
then the algebraic rule that determines the output of the function  $f$  when the input is  $x$  is  
$\dfrac{x}{{x - 1}}$.  In this case, we would say that the domain of  $f$  is the set of all real numbers not equal to  1 since division by zero is not defined.

However, the concept of a function is much more general than this.  The way in which a function associates elements of the domain with elements of the codomain can have many different forms.  For example, the idea of associating with each person his or her birthday (month and day) can be thought of as a function with the domain being the set of all people and the codomain being the set of all days in a (leap) year.  So the domain and codomain can be any set and the input-output rule can be a formula, a graph, a table, a random process, a computer algorithm, or a verbal description.  We formally define the concept of a function as follows:

\begin{defbox}{function}{A \textbf{function}
\index{function}%
 from a set  $A$  to a set  $B$  is a rule that associates with every element  $x$  of the set  $A$  exactly one element of the set  $B$.  A function $f$ from  $A$  to  $B$ is also called a 
\textbf{mapping}
\index{mapping}%
 from  $A$  to  $B$.  

\newpar
When $f$ is a function from $A$ to $B$, we write $f \x A \to B$.  The set  $A$  is called the \textbf{domain}
\index{domain!of a function}%
\index{function!domain}%
of the function  $f$, and we write  $A = \text{dom}( f )$\!.\label{sym:domfunc}  The set  $B$ is called the \textbf{codomain}
\index{codomain}%
\index{function!codomain}%
of the function  $f$, and we write  $B = \text{codom}( f )$\!. \label{sym:codomain}}
\end{defbox}

\end{previewactivity}

\endinput
