\section*{Section \ref{S:reviewproofs} Review of Proof Methods}


\begin{enumerate}
\item If $(a, b)$ is on or inside the circle $(x - 1)^2 + (y - 2)^2 = 4$, then we know that 
$(a - 1)^2 + (b - 2)^2 \leq 4$, and so 
\[
(a - 1)^2 \leq 4 \qquad \text{and} \qquad (b - 2)^2 \leq 4.  
\]
This two inequalities imply that
\[
-2 \leq a - 1 \leq 2 \qquad \text{and} \qquad -2 \leq b - 2 \leq 4.
\]
We can then conclude that $-1 \leq a \leq 3$ and $0 \leq b \leq 4$.  We can now rewrite the inequality $(a - 1)^2 + (b - 2)^2 \leq 4$ as follows:
\begin{align*}
(a^2 - 2a + 1) + (b^2 - 4b + 4) &\leq 4 \\
a^2 + b^2 &\leq -1 + 2a + 4b \\
a^2 + b^2 &\leq -1 + 2(3) + 4(4) \\
a^2 + b^2 &\leq 21.
\end{align*}
This implies that $a^2 + b^2 < 26$ and hence, $(a, b)$ is inside the circle $x^2 + y^2 = 26$.


\item With $x^2 + y^2 = r^2$, we have the following:
\begin{enumerate}
\item $2x + 2y \dfrac{dy}{dx} = 0$.  Solving for $\dfrac{dy}{dx}$ gives 
$\dfrac{dy}{dx} = -\dfrac{x}{y}$.

\item If $(a, b)$ is on the circle, the slope of the line tangent to the circle at $(a, b)$ is 
$-\dfrac{a}{b}$.

\item The slope of the radius from the origin to the point $(a, b)$ is $\dfrac{b}{a}$.  So the product of this slope with the slope of the tangent line in~(b) is $-1$.  This means that the radius is perpendicular to the line tangent to the circle.
\end{enumerate}


\item \begin{enumerate}
\item  This statement is true.  If we assume that 3 does not divide $a$, then we can use two cases.  In the first case, assume $\mod{a}{1}{3}$ and then use congruence arithmetic to show that $\mod{\left(2a^2 + 1 \right)}{0}{3}$.  This proves that if $\mod{a}{1}{3}$, then 3 divides $\left(2a^2 + 1 \right)$.  The second case when $\mod{a}{2}{3}$ is done in a similar manner.

\item This statement is true.  Prove the contrapositive, which is ``For each integer $a$, if 3 divides $a$, then 3 does not divide $\left(2a^2 + 1 \right)$.''  Since 3 divides $a$, we know that $\mod{a}{0}{3}$ and we can use congruence arithmetic to prove that $\mod{\left(2a^2 + 1 \right)}{1}{3}$, and hence, 3 does not divide $\left(2a^2 + 1 \right)$.

\item By using parts~(a) and~(b), we see that this statement is true.
\end{enumerate}


\item Use a proof by contradiction.  Assume that there exists a real number $x$ and there exists an irrational number $q$ such that $(x + q)$ is rational and $(x - q)$ is rational.  Since the rational numbers are closed under subtraction, this implies that $(x + q) - (x - q) = 2q$ is a rational number.  However, if $2q$ is rational, then $q$ is rational.  This is a contradiction to the assumption that $q$ is irrational.


\item We know that $\sqrt{2}$ is irrational.  Now consider the real number $q = \sqrt{2}^{\sqrt{2}}$.  The real number $q$ is either rational or irrational.
\begin{itemize}
\item If $q$ is rational, then we can use $u = \sqrt{2}$ and $v = \sqrt{2}$ as an example where $u$ and $v$ are irrational and $u^v$ is rational.

\item If $q$ is irrational, then use $u = \sqrt{2}^{\sqrt{2}}$ and $v = \sqrt{2}$.  We then see that
\begin{align*}
u^v &= \left( \sqrt{2}^{\sqrt{2}} \right)^{\sqrt{2}} \\
    &= \sqrt{2}^{\left( \sqrt{2} \cdot \sqrt{2} \right)} \\
    &= \sqrt{2}^2 \\
    &= 2
\end{align*}
So in this case, we have an example where $u$ and $v$ are irrational and $u^v$ is rational.
\end{itemize}
These two cases prove that there exist irrational numbers $u$ and $v$ such that $u^v$ is rational.



\item  We are given that $a$ and $b$ are natural numbers and that $a^2 = b^3$.
\begin{enumerate}
\item Assume that $a$ is even.  Then there exists an integer $k$ such that $a = 2k$.  So,
\[
4k^2 = b^3.
\]
Hence, $b^3$ is even and by Exercise~(1), $b$ is even. So, there exists an integer $m$ such that $b = 2m$.  Since $a^2 = b^3$, we see that $4k^2 = 8m^3$.  Hence, $k^2$ is even.  By 
Theorem~\ref{T:n2isodd}, $k$ is even.  So, there exists an integer $q$ such that $k = 2q$ and since $a = 2k$, we see that $a = 4q$.  Hence, 4 divides $a$.

\item Since 4 divides $a$, there exist an integer $n$ such that $a = 4n$.  Using this, we see that $b^3 = 16n^2$. So, $b^3$ is even and hence $b$ is even and there exists an integer $m$ such that $b = 2m$.  This implies that
\[
\begin{aligned}
8m^3 &= 16n^2 \\
m^3 &= 2n^2 \\
\end{aligned}
\]
Hence, $m^3$ is even and so by Exercise~(1), $m$ is even.  Since $b = 2m$, we see that 4 divides $b$.

\item If 4 divides $b$, then since $a^2 = b^3$, we conclude that $a^2$ is even.  But this implies that $a$ is even (Theorem~\ref{T:n2isodd}).  Hence, by Part~(a), 4 divides $a$.  So, there exist integers $s$ and $t$ such that $b = 4s$ and $a = 4t$.  Hence,
\[
\begin{aligned}
16t^2 &= 64s^3 \\
t^2 &= 4s^3 \\
\end{aligned}
\]
So, $t^2$ is even and hence $t$ must be even.  Since $a = 4t$, we conclude that 8 divides $a$.

\item Use Parts~(a), (b), and~(c).

\item $a = 8$, $b = 4$.
\end{enumerate}

\item Let $a$ and $b$ be integers with $a \ne 0$.  We will prove the contrapositive.  So assume that the equation $ax^3 + bx + \left( b + a \right) = 0$ has a solution that is a natural number.  Let $n$ be a natural number that is a solution of this equation.  Then
\[
\begin{aligned}
an^3 + bn + \left( b + a \right) &= 0 \\
an^3 + a &= -bn - b \\
a \left(n + 1\right) \left(n^2 - n + 1 \right) &= -b \left( n + 1 \right) \\
a \left( n^2 - n + 1 \right) &= -b \\
\end{aligned}
\]
The last equation can be used to conclude that $a \mid b$, and this completes the proof of the contrapositive.


\item \begin{enumerate}
\item  Use a proof by contradiction.  So assume that there exist natural numbers $a$, $b$, and $c$ with $a < b < c$ such that $a$, $b$, and $c$ form a Pythagorean triple and 3 does not divide $a$ and 3 does not divide $b$.  Then by Exercise~(\ref{exer:sec34-4}) in Section~\ref{S:divalgo}, $\mod{a^2}{1}{3}$ and $\mod{b^2}{1}{3}$.  Since $c^2 = a^2 + b^2$, we can then conclude that $\mod{c^2}{2}{3}$.  However, using Exercise~(\ref{exer:sec34-4}) in Section~\ref{S:divalgo} in Section~\ref{S:divalgo} again, we can conclude that $\mod{c^2}{0}{3}$ or $\mod{c^2}{1}{3}$ and so we have a contradiction.

\item Use a proof by contradiction.  So assume that there exist natural numbers $a$, $b$, and $c$ with $a < b < c$ such that $a$, $b$, and $c$ form a Pythagorean triple and 5 does not divide $a$ and 5 does not divide $b$ and 5 does not divide $c$.  Then by Proposition~\ref{prop:congmod5}, we know that
\begin{itemize}
\item $\mod{a^2}{1}{5}$ or $\mod{a^2}{4}{5}$;
\item $\mod{b^2}{1}{5}$ or $\mod{b^2}{4}{5}$;
\item $\mod{c^2}{1}{5}$ or $\mod{c^2}{4}{5}$.
\end{itemize}
We will now use the fact that $c^2 = a^2 + b^2$ to also determine the value of $c^2$ modulo 5.  Using the values for $a^2$ and $b^2$ modulo 5 and the facts that $\mod{1 + 1}{2}{5}$, $\mod{1 + 4}{0}{5}$, $\mod{4 + 1}{0}{5}$, and $\mod{4 + 4}{3}{5}$, we see that $c^2$ is congruent to 0, 2, or 3 modulo 5.  However, this contradicts the earlier conclusion that $\mod{c^2}{1}{5}$ or $\mod{c^2}{4}{5}$.  So we conclude that 5 divides $a$ or 5 divides $b$ or 5 divides $c$.
\end{enumerate}


\item \begin{enumerate}
\item One such Pythagorean triple is 5, 12, 13.
\item One such Pythagorean triple is 7, 24, 25.
\item Let $m$ be an odd natural number that is greater than 1.  We then know that $m^2 - 1$ is an even natural number and hence, $\left( \dfrac{m^2 - 1}{2} \right)$ is a natural number.  We notice that
\begin{align*}
\frac{m^2 - 1}{2} + 1 &= \frac{m^2 - 1 + 2}{2} \\
                      &= \frac{m^2 + 1}{2}
\end{align*}
Hence, $\left( \dfrac{m^2 - 1}{2} \right)$ and $\left( \dfrac{m^2 + 1}{2} \right)$ are consecutive natural numbers.  We now prove that $m$, $\left( \dfrac{m^2 - 1}{2} \right)$, and $\left( \dfrac{m^2 + 1}{2} \right)$ form a Pythagorean triple.
\begin{align*}
m^2 + \left( \frac{m^2 - 1}{2} \right)^2 &= m^2 + \frac{m^4 - 2m^2 + 1}{4} \\
                                         &= \frac{4m^2 + m^4 -2m^2 + 1}{4} \\
                                         &= \frac{m^4 + 2m^2 + 1}{4} \\
                                         &= \left(  \frac{m^2 + 1}{2} \right)^2
\end{align*}
This proves that $m$, $\left( \dfrac{m^2 - 1}{2} \right)$, and $\left( \dfrac{m^2 + 1}{2} \right)$ form a Pythagorean triple.
\end{enumerate}


\item \begin{enumerate}
\item $50 = 3 + 47$, \quad $142 = 71 + 71$, \quad $150 = 11 + 139$.
%\item \begin{align*}
%50 &= 3 + 47  &  142 &= 71 + 71  &  150 &= 11 + 139
%\end{align*}

\item We assume Goldbach's conjecture is true and let $n$ be an integer that is greater than 5.

If $n$ is even, then $n \geq 6$ and so $n - 2 \geq 4$ and $n -2$ is even.  We can then use Goldbach's conjecture to conclude that there exist prime numbers $p$ and $q$ such that $n - 2 = p + q$.  Then
\[
n = 2 + p + q,
\]
and so $n$ can be written as the sum of three primes.

If $n$ is odd, then $n \geq 7$ and so $n - 3$ is an even integer that is greater than 4.  We can then use Goldbach's conjecture to conclude that there exist prime numbers $p$ and $q$ such that $n - 3 = p + q$.  Then
\[
n = 3 + p + q,
\]
and so $n$ can be written as the sum of three primes.

Since we have proved it for even and odd integers, we have proved that Goldbach's Conjecture implies that every integer greater than $5$ can be written as a sum of three primes.


\item We assume Goldbach's Conjecture is true and let $n$ be an odd integer that is greater than 7. We have seen in part~(b) that $n$ can be written as the sum of three primes.  So we write
\[
n = p + q + r,
\]
where $p$, $q$, and $r$ are prime numbers.  If all three of these primes are odd, then we are done.  Otherwise, since $n$ is odd, two of the primes must be even.  Since the only even prime number is 2, we will use $p = 2$ and $q = 2$ and so
\[
n = 2 + 2 + r,
\]
where $r$ is a prime and since $n > 7$, $r > 3$.  We now use $2 = 3 - 1$ and write
\begin{align*}
n &= (3 - 1) + (3 - 1) + r \\
n &= 3 + 3 + (r - 2)
\end{align*}
Now, $(r - 2)$ is an odd number and $r - 2 > 1$.  We now rewrite the previous equation as follows:
\begin{align*}
n & = 3 + (3 + (r - 2)) \\
n & = 3 + (r + 1)
\end{align*}
Since $r$ is odd and $r > 3$, we see that $r + 1$ is even and $r + 1 > 4$.  So if Goldbach's Conjecture is true, there exist prime numbers $u$ and $v$ such that $r + 1 = u + v$.  Since $(r + 1)$ is even, both $u$ and $v$ must be odd primes.  So we have
\[
n = r + u + v
\]
and $r$, $u$, and $v$ are odd primes.  This proves that every odd integer that is greater than 7 is the sum of three odd prime numbers.

\end{enumerate}



\item Let $p$ and $q$ be twin primes other than 3 and 5.  We will let $q = p + 2$.  We then see that
\begin{align*}
pq + 1 &= p(p + 2) + 1 \\
       &= p^2 + 2p + 1 \\
       &= (p + 1)^2
\end{align*}
This proves that $pq + 1$ is a perfect square.  We will now look at cases for $p$ based on congruence modulo 6.  We first note that since $p$ is an odd prime, then $p$ cannot be congruent to 0, 2, or 4 modulo 6.  In addition, since $p$ is an odd prime other than 3, $p$ cannot be congruent to 3 modulo 6.  Finally, if $\mod{p}{1}{6}$, then $\mod{q}{3}{6}$, and this is not possible.  So we conclude that $\mod{p}{5}{6}$.  Therefore, there exists an integer $k$ such that $p = 6k + 5$ and so $q = 6k + 7$.  We then see that
\begin{align*}
pq + 1 &= (6k + 5)(6k + 7) + 1 \\
       &= 36k^2 + 72k + 36 \\
       &= 36 \left( k^2 + 2k + 1 \right)
\end{align*}
This proves that 36 divides $pq + 1$.


\item \begin{enumerate}
\item Since $(a + b)^2 = a^2 + 2ab + b^2$, we see that 
\[
(a + b)^2 - \left( a^2 + b^2 \right) = 2ab.
\]
This implies that $\mod{(a + b)^2}{\left( a^2 + b^2 \right)}{2}$.

\item Since $(a + b)^3 = a^3 + 3a^2 b + 3a b^2 + b^3$, we see that 
\[
(a + b)^3 - \left( a^3 + b^3 \right) = 3 \left( a^2 b + a b^2 \right).
\]
This implies that $\mod{(a + b)^3}{\left( a^3 + b^3 \right)}{3}$.

\item This statement is false.  A counterexample is $a = 1$ and $b = 1$.  For these values, 
$\mod{(a + b)^4}{0}{4}$ and $\mod{\left( a^4 + b^4 \right)}{2}{4}$.

\item Since $(a + b)^5 = a^5 + 5a^4 b + 10 a^3 b^2 + 10 a^2 b^3 + 5a b^4 + b^5$, we see that 
\[
(a + b)^5 - \left( a^5 + b^5 \right) = 5 \left( a^4 b + 2a^3 b^2 + 2a^2 b^3+ a b^4 \right).
\]
This implies that $\mod{(a + b)^5}{\left( a^5 + b^5 \right)}{5}$.
\end{enumerate}



\item \begin{enumerate}
\item $f ' (x) = 3ax^2 + 2bx + c$ and $f '' (x) = 6ax + 2b$.

\item The critical points of the function $f$ correspond to solutions of the equation $f'(x) = 0$.  Since this is a quadratic equation, there are at most two solutions and so the function $f$ has at most two critical points.

\noindent
At at point of inflection, we have $f ''(x) = 0$.  The only solution for this equation is $x =  - \dfrac{b}{3a}$.  Also notice that
\begin{itemize}
\item If $a > 0$ and $x >  - \dfrac{b}{3a}$, then $3ax + b > 0$ and $6ax + 2b > 0$.  So $f''(x) > 0$ and the graph of $f$ is concave up.  If $a > 0$ and $x <  - \dfrac{b}{3a}$, then $3ax + b < 0$ and $6ax + 2b < 0$.  So $f''(x) < 0$ and the graph of $f$ is concave down.

\item If $a < 0$ and $x >  - \dfrac{b}{3a}$, then $3ax + b < 0$ and $6ax + 2b < 0$.  So $f''(x) < 0$ and the graph of $f$ is concave down.  If $a > 0$ and $x <  - \dfrac{b}{3a}$, then $3ax + b > 0$ and $6ax + 2b > 0$.  So $f''(x) > 0$ and the graph of $f$ is concave up.
\end{itemize}
Since the graph of $f$ changes concavity in both cases, we see that the only inflection point of the function $f$ is when $x =  - \dfrac{b}{3a}$.
\end{enumerate}
\end{enumerate}




\subsection*{Explorations and Activities}
\setcounter{oldenumi}{\theenumi}
\begin{enumerate} \setcounter{enumi}{\theoldenumi}
\item \begin{enumerate}
\item When $a = 2$ and $b = 2$, we see that $c^3 = 8$ and since $c$ is an integer, this is a contradiction.

\item If $a$ and $b$ are both odd prime numbers, then $a^3 + b^3$ is an even natural number.  This means that $c^3$ is even and hence, that $c$ is even.  However, $c$ is prime and so $c = 2$.  This is a contradiction since $c$ must be greater than both $a$ and $b$.

\item If one of $a$ and $b$ is even and the other is odd, then one of them must be equal to 2.  We can use $b = 2$.  Since $b = 2$ and $a$ is odd and $c^3 = a^3 + b^3$, we see that $c^3$ is odd and hence, $c$ is odd.  We now write $b^3 = 2 = c^3 - a^3$ and factor as follows:
\[
8 = (c - a) \left( c^2 + ac + a^2 \right).
\]
This implies that $\left( c^2 + ac + a^2 \right)$ is a factor of 8, and since $a$ and $c$ are both odd, 
$\left( c^2 + ac + a^2 \right)$ is odd.  The only odd factor of 8 is 1 and so we obtain
\[
c^2 + ac + a^2 = 1.
\]
This is a contradiction since $\left( c^2 + ac + a^2 \right) > 1$.
\end{enumerate}

\item \begin{enumerate}
\item The following table shows the possible values for an integer $x$ modulo 9 and $x^3$ modulo 9.
$$
\BeginTable
\BeginFormat
|c | c | c | c | c | c | c | c | c | c |
\EndFormat
\_
|$x$|0|1|2|3|4|5|6|7|8| \\+22 \_
|$x^3$|0|1|8|0|1|8|0|1|8| \\+33 \_
\EndTable
$$

\item For all integers $x$ and $y$, there are three possibilities for $x^3$ modulo 9 and three possibilities for $y^3$ modulo 9.  The following table shows the nine cases for $x^3 + y^3$ modulo 9.
$$
\BeginTable
\BeginFormat
|c | c |c|c|c|c|c|c|c|c|
\EndFormat
\_
|$x^3$|0|0|0|1|1|1|8|8|8| \\+33 \_
|$y^3$|0|1|8|0|1|8|0|1|8| \\+33 \_
|$x^3+y^3$|0|1|8|1|2|0|8|0|7| \\+33 \_
\EndTable
$$

\item The table in Part~(b) shows that for all integers $x$ and $y$, $x^3+y^3$ is congruent (modulo 9) to 0, 1, 2, 7, or 8.  So if $\mod{k}{3}{9}$, then $k$ cannot be written as the sum of the cubes of two integers.
\item The table in Part~(b) shows that for all integers $x$ and $y$, $x^3+y^3$ is congruent (modulo 9) to 0, 1, 2, 7, or 8.  So if $\mod{k}{4}{9}$, then $k$ cannot be written as the sum of the cubes of two integers.

\item The table in Part~(b) can be used to prove the following theorem:
\begin{list}{}
\item For each integer $k$, if $k$ is congruent (modulo 9) to 3, 4, 5, or 6, then $k$ cannot be written as the sum of the cubes of two integers.
\end{list}

\item Let $x$, $y$, and $z$ be integers.  From Part~(b), we see that there are five possibilities for $x^3+y^3$.  From Part~(a), there are three possibilities for $z^3$.  So we will use 15 cases to examine $x^3+y^3+z^3$ as is shown in the following table.
$$
\BeginTable
\BeginFormat
| c | c | c | 
\EndFormat
\_
| $x^3 + y^3$ | $z^3$ | $x^3+y^3+z^3$ | \\+33 \_
|0|0|0| \\+11 \_
|0|1|1| \\+11\_
|0|8|8| \\+11\_
|1|0|1| \\+11\_
|1|1|2| \\+11\_
|1|8|0| \\+11\_
|2|0|2| \\+11\_
|2|1|3| \\+11\_
|2|8|1| \\+11\_
|7|0|7| \\+11\_
|7|1|8| \\+11\_
|7|8|6| \\+11\_
|8|0|8| \\+11\_
|8|1|0| \\+11\_
|8|8|7| \\+11\_
\EndTable
$$

\item The table in Part~(7) shows that for all integers $x$, $y$, and $z$, $x^3+y^3+z^3$ is congruent (modulo 9) to 0, 1, 2, 3, 6, 7, or 8.  So if $\mod{k}{4}{9}$, then $k$ cannot be written as the sum of the cubes of three integers.

\item The table in Part~(f) can be used to prove the following theorem:
\begin{list}{}
\item For each integer $k$, if $k$ is congruent (modulo 9) to 4 or 5, then $k$ cannot be written as the sum of the cubes of three integers.
\end{list}

\end{enumerate}
\note There are, of course, other ways to prove the result in Part~(h).  For example, if we consider all possible cases for $x$, $y$, and $z$ using congruence modulo 9, there would be $9^3 = 729$ cases to consider.  Probably the most important thing is to first prove Part~(a).  Then there are 3 cases each for $x^3$, $y^3$, and $z^3$ resulting in $3^3 = 27$ cases.  By using Part~(b) and~(f), we end up with $9+15 = 24$ cases.  This is a slight improvement, but along the way, we also get the result in Part~(b).
\end{enumerate}
\hbreak
\endinput






