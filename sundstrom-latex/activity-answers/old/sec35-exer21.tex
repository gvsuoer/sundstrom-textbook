\documentclass[11pt]{article}
\usepackage{c://pctex/activity}

\lhead{}
\chead{\textbf{\large{Exercise 21 -- Section 3.5\\Using Contradiction to Prove a Case Is Not Possible}}}
\rhead{}
\lfoot{\emph{Mathematical Reasoning: Writing and Proof, Third Ed.} \\Ted Sundstrom}
\cfoot{}
\rfoot{\copyright 2009 by Pearson Education, Inc.\\}


\begin{document}
\begin{enumerate} \setcounter{enumi}{2}
\item The statement is false.  A counterexample is $a = 12$.
\item The examples from Part~(b) suggest this statement is true.

\newpar
\textbf{Proposition.}
Let  $a \in \mathbb{Z}$.  If  2  divides  $a$  and  3  divides  $a$, then  6  divides  $a$.

\noindent
\textbf{\emph{Proof}}:  Let  $a \in \mathbb{Z}$ and assume that  2  divides  $a$  and  3  divides  $a$.  We will prove that  6 divides  $a$.  Since  3  divides  $a$, there exists an integer  $n$  such that
\[
a = 3n.
\]
The integer  $n$  is either even or it is odd.  We will show that it must be even by obtaining a contradiction if it assumed to be odd.  So assume that  $n$  is odd. Then there exists an integer $m$ such that  $n = 2m + 1$.  Substituting this into the previous equation gives
\[
a = 6m + 3.
\]
However, this implies that $a = 2 \left( 3m + 1 \right) + 1$ and hence, $a$ must be odd.  This contradicts the assumption that 2 divides $a$.  This proves that $n$ cannot be odd, and hence must be even.

So, there exists an integer $k$ such that $n = 2k$.  Substituting this into the equation $a = 3n$ shows that $a = 6k$.  Hence, 6 divides $a$.  This proves that if 2 divides $a$ and 3 divides $a$, then 6 divides $a$. \qedsymbol
\end{enumerate}

\end{document}
