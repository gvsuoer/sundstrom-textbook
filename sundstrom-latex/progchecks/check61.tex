\section*{Section~\ref{S:introfunctions}}

\subsection*{Progress Check \ref{pr:images}}
\begin{minipage}{2.3in}
\begin{enumerate}
\item $f ( -3 ) = 24$ \\$f ( \sqrt{8} ) = 8 - 5 \sqrt{8}$
\item $g ( 2 ) = -6$, $g ( -2 ) = 14$
\item $\left\{ -1, 6 \right\}$
\end{enumerate}
\end{minipage}
\begin{minipage}{2.3in}
\begin{enumerate} \setcounter{enumi}{3}
\item $\left\{ -1, 6 \right\}$
\item $\left\{ \dfrac{5 + \sqrt{33}}{2}, \dfrac{5 - \sqrt{33}}{2} \right\}$
\item $\emptyset$
\end{enumerate}
\end{minipage}

\subsection*{Progress Check \ref{pr:codomainandrange}}
\begin{enumerate}
\item 
\begin{enumerate}
  \item The domain of the function $f$ is the set of all people.
  \item A codomain for the function $f$ is the set of all days in a leap year.
  \item This means that the range of the function  $f$  is equal to its codomain.
\end{enumerate}

\item 
\begin{enumerate}
  \item The domain of the function $s$ is the set of natural numbers.
  \item A codomain for the function $s$ is the set of natural numbers.
  \item This means that the range of  $s$  is not equal to the set of natural numbers.
\end{enumerate}
\end{enumerate}


\subsection*{Progress Check \ref{pr:graphreal}}
\begin{enumerate}
  \item $f(-1) \approx -3$ and $f(2) \approx -2.5$.
  \item Values of $x$ for which $f(x) = 2$ are approximately $-2.8, -1.9, 0.3, 1.2$, and 3.5.
  \item The range of $f$ appears to be the closed interval $[-3.2, 3.2]$ or \\ $\{ y \in \R \mid -3.2 \leq y \leq 3.2 \}$.
\end{enumerate}



\subsection*{Progress Check \ref{pr:arrow}}
Only the arrow diagram in Figure~(a) can be used to represent a function from $A$ to $B$.  The range of this function is the set $\{ a, b \}$.
\hbreak

\endinput

