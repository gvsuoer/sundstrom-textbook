\documentclass[11pt]{article}
\usepackage{../../styles/activity}

\usepackage{xr}
\externaldocument{0-MR}

\lhead{}
%\chead{\textbf{\Large{\hspace{0pt}Beginning Activities for Section~9.3}}\\\hspace{0pt}\emph{Mathematical Reasoning: Writing and Proof}}
\bahead{9.3}
\rhead{}
\lfoot{}
\rfoot{}
\cfoot{\hspace{0pt}\scalebox{0.4}{\includegraphics{cc-by-nc-sa.eps}}}
\graphicspath{{./epsfigs/}}

\begin{document}
\subsection*{Beginning Activity 1 (The Game of Dodge Ball)}

Player Two has a winning strategy.
\begin{itemize}
\item Player One must complete the first row of six boxes.  Player Two looks only at the first box in this row.  Whatever Player One put in the first box, Player Two puts the opposite symbol in the first box of Player Two's row.

\item Player One then completes the second row of six boxes.  Player Two looks only at the second box in this row.  Whatever Player One put in the second box, Player Two puts the opposite symbol in the second box of Player Two's row.
\end{itemize}
This is the pattern.  After Player One completes the $k$th row of six boxes,  Player Two looks only at the $k$th box in this row.  Whatever Player One put in the $k$th box, Player Two puts the opposite symbol in the $k$th box of Player Two's row.  Doing this for all six boxes, Player Two will create a string of six symbols that will not match any of Player One's six rows.


\newpar
\textbf{Applying the Winning Strategy to Lists of Real Numbers} \\
Following is a list of real numbers between 0 and 1.  Each real number is written as a decimal number.
\begin{align*}
a_1 &= 0.1234567890   &  a_6 &= 0.0103492222 \\
a_2 &= 0.3216400000   &  a_7 &= 0.0011223344 \\
a_3 &= 0.4321593333   &  a_8 &= 0.7077700022 \\
a_4 &= 0.9120930092   &  a_9 &= 0.2100000000 \\
a_5 &= 0.0000234102   & a_{10} &= 0.9870008943 
\end{align*}

\newpar
We consider the numbers to the right of the decimal point to be like the boxes in the game of Dodgeball.  So we will construct a number $0.b_1 b_2 b_3 b_4 b_5 b_6 b_7 b_8 b_9 b_{10}$ so that:
\begin{itemize}
  \item $b_1$ is not equal to the first digit to the right of the decimal in $a_1$;
  \item $b_2$ is not equal to the second digit to the right of the decimal in $a_2$;
  \item $b_3$ is not equal to the second digit to the right of the decimal in $a_3$;
\end{itemize}
and so on through $b_{10}$.  We will do this by choosing $b_k$ to be 3 if the $k^{th}$ digit to the right of the decimal place in $a_k$ is not equal to 3 and $b_k = 5$ if the $k^{th}$ digit to the right of the decimal place in $a_k$ is equal to 3.  (There are many other ways to do this.)  So we use
\[
b_k = 0.3333335335.
\]
We should be able to apply this method to longer lists of numbers.

\hbreak
\newpage




\subsection*{Beginning Activity 2 (Functions from a Set to Its Power Set)}
The work in this beginning activity is intended to help understand the proof of one of the most profound results of number theory, namely, Cantor's Theorem (Theorem 9.27) on page 483.
\begin{enumerate}
%\item 
%\begin{enumerate}
%\item $\mathcal{P} \left( A \right) = \left\{ \emptyset, \left\{ a \right\}, \left\{ b \right\}, 
%\left\{ a, b \right\} \right\}$
%
%\item $\mathcal{P} \left( B \right) = \left\{ \emptyset, \left\{ a \right\}, \left\{ b \right\}, 
%\left\{ c \right\}, \left\{ a, b \right\}, \left\{ a, c \right\}, 
%\left\{ b, c \right\}, \left\{ a, b, c \right\} \right\}$
%\end{enumerate}

\item \begin{enumerate}
\item \begin{multicols}{4}
$1 \in f \left( 1 \right)$

$2 \notin f \left( 2 \right)$

$3 \notin f \left( 3 \right)$

$4 \in f \left( 4 \right)$
\end{multicols}

\item $S = \left\{ x \in A \mid x \notin f \left( x \right) \right\} = \left\{ 2, 3 \right\}$

\item There does not exist a $t \in A$ such that $f \left( t \right) = S$.  So 
$S \notin \text{range} \left( f \right)$.
\end{enumerate}

\item \begin{enumerate}
\item $f \left( 1 \right) = \left\{ 2, 3, 4 \right\}$.  So, $1 \notin f \left( 1 \right)$.

\item $f \left( 2 \right) = \left\{ 1, 3, 4 \right\}$.  So, $2 \notin f \left( 2 \right)$.

\item $f \left( 3 \right) = \left\{ 1, 2, 4 \right\}$.  So, $3 \notin f \left( 3 \right)$.

\item $f \left( 4 \right) = \left\{ 1, 2, 3 \right\}$.  So, $4 \notin f \left( 4 \right)$.

\item $S = \left\{ x \in A \mid x \notin f \left( x \right) \right\} = \left\{ 1, 2, 3, 4 \right\}$

\item There does not exist a $t \in A$ such that $f \left( t \right) = S$.  So 
$S \notin \text{range} \left( f \right)$.
\end{enumerate}


\item \begin{enumerate}
\item $f \left( 1 \right) = \mathbb{N} - \left\{ 1 \right\} = \left\{ 2, 3, 4, 5, \ldots \right\}$.  So, $1 \notin f \left( 1 \right)$.

$f \left( 2 \right) = \mathbb{N} - \left\{ 4 \right\} = \left\{ 2, 3, 5, 6, 7, \ldots \right\}$.  So, $2 \in f \left( 2 \right)$.

$f \left( 3 \right) = \mathbb{N} - \left\{ 9, 3 \right\}$.  So, $3 \notin f \left( 3 \right)$.

$f \left( 4 \right) = \mathbb{N} - \left\{ 16, 8 \right\}$.  So, $4 \in f \left( 4 \right)$.

\item If $n > 3$, then $n^2 > 3n$ and $3n > n$.   Therefore, $n^2 > n$.  Also, since $n^2 > 3n$, we seen that $n^2 - 2n > 3n - 2n$ or $n^2 - 2n > n$.  Therefore, 
$n \notin \left\{ n^2, n^2 - 2n \right\}$, and hence $n \in f \left( n \right)$.


\item A natural number $x$ is not in $f \left( x \right)$ if and only if 
$x^2 = x$ or $x^2 - 2x = x$.  The only natural number solution of the first equation is $x = 1$, and the only natural number solution of the second equation is $x = 3$.  Consequently, 
$S = \left\{ x \in \mathbb{N} \mid x \notin f \left( x \right) \right\} = \left\{ 1, 3 \right\}$.

\item For each $t \in \mathbb{N}$, $f \left( t \right)$ is an infinite set.  Consequently,  There does not exist a $t \in \mathbb{N}$ such that $f \left( t \right) = S$.  So 
$S \notin \text{range} \left( f \right)$.
\end{enumerate}
\end{enumerate}
\hbreak



\end{document}
