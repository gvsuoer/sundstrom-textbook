\section*{Exercises 7.4}
%
\begin{enumerate} 
\item \label{exer:sec74-modtables} \begin{enumerate}
  \yitem Complete the addition and multiplication tables for  $\mathbb{Z}_4$.
  
  \yitem Complete the addition and multiplication tables for  $\mathbb{Z}_7$.

  \item Complete the addition and multiplication tables for  $\mathbb{Z}_8$.
\end{enumerate}

\item The set  $\mathbb{Z}_n $ contains  $n$  elements.  One way to solve an equation in  
$\mathbb{Z}_n $ is to substitute each of these  $n$  elements in the equation to check which ones are solutions.  In $\mathbb{Z}_n $, when parentheses are not used, we follow the usual order of operations, which means that multiplications are done first and then additions.  Solve each of the following equations: 
\label{exer:sec74-3}%

\begin{multicols}{2}
\begin{enumerate}
\yitem $[ x ]^2  = [ 1 ]$  in  $\mathbb{Z}_4 $

\item $[ x ]^2  = [ 1 ]$  in  $\mathbb{Z}_8 $

\item $[ x ]^4  = [ 1 ]$  in  $\mathbb{Z}_5 $	

\item $[ x ]^2  \oplus [ 3 ] \odot [ x ] = 
[ 3 ]$  in  $\mathbb{Z}_6 $

\yitem $[ x ]^2  \oplus [ 1 ] = [ 0 ]$  in  
$\mathbb{Z}_5 $

\item $[ 3 ] \odot [ x ] \oplus [ 2 ] = [ 0 ]$
  in  $\mathbb{Z}_5 $

\yitem $[ 3 ] \odot [ x ] \oplus [ 2 ] = [ 0 ]$
  in  $\mathbb{Z}_6 $

\item $[ 3 ] \odot [ x ] \oplus [ 2 ] = [ 0 ]$
  in  $\mathbb{Z}_9 $
\end{enumerate}
\end{multicols}

\xitem In each case, determine if the statement is true or false. 
\label{exer:sec74-4}%
\begin{enumerate}
\item For all  $[ a ] \in \mathbb{Z}_6 $, if  
      $[ a ] \ne [ 0 ]$, then there exists a  
      $[ b ] \in \mathbb{Z}_6 $ such that 
      $[ a ] \odot [ b ] = [ 1 ]$.

\item For all  $[ a ] \in \mathbb{Z}_5 $, if  
      $[ a ] \ne [ 0 ]$, then there exists a  
      $[ b ] \in \mathbb{Z}_5 $ such that 
      $[ a ] \odot [ b ] = [ 1 ]$.
\end{enumerate}

\item In each case, determine if the statement is true or false. 
\label{exer:sec74-5}%
\begin{enumerate}
\item For all  $[ a ], [ b ] \in \mathbb{Z}_6 $, if  
      $[ a ] \ne [ 0 ]$ and $[ b ] \ne [ 0 ]$, then 
      $[ a ] \odot [ b ] \ne [ 0 ]$.

\item For all  $[ a ], [ b ] \in \mathbb{Z}_5 $, if  
      $[ a ] \ne [ 0 ]$ and $[ b ] \ne [ 0 ]$, then 
      $[ a ] \odot [ b ] \ne [ 0 ]$.
\end{enumerate}

\item\label{exer:squaresinZ5} 
\begin{enumerate}
\yitem Prove the following proposition:  
\begin{list}{}
\item For each $[ a ] \in \Z_5$, if 
$[ a ] \ne [ 0 ]$, then $[ a ]^2 = [ 1 ]$ or 
$[ a ]^2 = [ 4 ]$.
\end{list}

\item Does there exist an integer $a$ such that $a^2 = 5,158,232,468,953,153$?  Use your work in Part~(a) to justify your conclusion.  Compare to Exercise~(\ref{exer:squaremod5}) in 
Section~\ref{S:divalgo}.
\end{enumerate}

\item Use mathematical induction to prove Proposition~\ref{P:poweroftenmod9}. 
\label{exer:poweroftenmod9}%
\begin{list}{}
\item If  $n$  is a nonnegative integer, then  $10^n  \equiv 1 \pmod 9$, and hence for the equivalence relation of congruence modulo  9, $[ {10^n } ] = [ 1 ]$.
\end{list}


\item Use mathematical induction to prove that if  $n$  is a nonnegative integer, then  
      $10^n  \equiv 1 \pmod 3$.  Hence, for congruence classes modulo 3,  if $n$  is a nonnegative integer, then 
      $[ {10^n } ] = [ 1 ]$.
\label{exer:poweroftenmod3}


\item Let  $n \in \mathbb{N}$ and let  $s( n )$ denote the sum of the digits of  $n$.  \label{exer:divtest3} So if we write
\[
n = \left( {a_k  \times 10^k } \right) + \left( {a_{k - 1}  \times 10^{k - 1} } \right) +  \cdots  + \left( {a_1  \times 10^1 } \right) + \left( {a_0  \times 10^0 } \right), 
\]
then $s( n ) = a_k  + a_{k - 1}  +  \cdots  + a_1  + a_0$.  Use the result in Exercise~(\ref{exer:poweroftenmod3}) to help prove each of the following:

\begin{enumerate}
  \item $[ n ] = [ {s( n )} ]$, using congruence classes modulo 3.

  \item $n \equiv s( n ) \pmod 3$.

  \item $3 \mid n$  if and only if \, $3 \mid s( n )$. \label{T:sumofdigitsmod9-3}
\index{divisibility test!for 3}%

\end{enumerate}

\item Use mathematical induction to prove that if  $n$  is an integer and  $n \geq 1$, then  $10^n  \equiv 0 \pmod 5$.  Hence, for congruence classes modulo 5,  if  $n$  is an integer and  $n \geq 1$, then  $[ {10^n } ] = [ 0 ]$.
\label{exer:poweroftenmod5}%

\item Let  $n \in \mathbb{N}$ and assume  \label{exer:divtest5}
\[
n = \left( {a_k  \times 10^k } \right) + \left( {a_{k - 1}  \times 10^{k - 1} } \right) +  \cdots  + \left( {a_1  \times 10^1 } \right) + \left( {a_0  \times 10^0 } \right)\!.
\]
Use the result in Exercise~(\ref{exer:poweroftenmod5}) to help prove each of the following:

\begin{enumerate}
  \item $[ n ] = [ {a_0 } ]$, using congruence classes modulo 5.
  
  \item $n \equiv a_0 \pmod 5$.

  \item $5 \mid n$  if and only if  $5 \mid a_0 $.
\index{divisibility test!for 5}%
\end{enumerate}

\item Use mathematical induction to prove that if  $n$  is an integer and  $n \geq 2$, then  $10^n  \equiv 0 \pmod 4$.  Hence, for congruence classes modulo 4,  if  $n$  is an integer and  $n \geq 2$, then  $[ {10^n } ] = [ 0 ]$. 
\label{exer:poweroftenmod4}

\item Let  $n \in \mathbb{N}$ and assume  \label{exer:divtest4}
\[
n = \left( {a_k  \times 10^k } \right) + \left( {a_{k - 1}  \times 10^{k - 1} } \right) +  \cdots  + \left( {a_1  \times 10^1 } \right) + \left( {a_0  \times 10^0 } \right)\!.
\]
Use the result in Exercise~(\ref{exer:poweroftenmod4}) to help prove each of the following:
\begin{enumerate}
  \item $[ n ] = [ {10a_1  + a_0 } ]$, using congruence classes modulo 4.

  \item $n \equiv \left( {10a_1  + a_0 } \right) \pmod 4$. 

  \item $4 \mid n$  if and only if  $4 \mid \left( {10a_1  + a_0 } \right)$.
\index{divisibility test!for 4}%

\end{enumerate}

\item Use mathematical induction to prove that if  $n$  is an integer and  $n \geq 3$, then  $10^n  \equiv 0 \pmod 8$.  Hence, for congruence classes modulo 8,  if  $n$  is an integer and  $n \geq 3$, then  $[ {10^n } ] = [ 0 ]$.
\label{exer:poweroftenmod8}%

\item Let  $n \in \mathbb{N}$ and assume  \label{exer:divtest8}
\[
n = \left( {a_k  \times 10^k } \right) + \left( {a_{k - 1}  \times 10^{k - 1} } \right) +  \cdots  + \left( {a_1  \times 10^1 } \right) + \left( {a_0  \times 10^0 } \right)\!.
\]
Use the result in Exercise~(\ref{exer:poweroftenmod8}) to help develop a divisibility test for  8.  Prove that your divisibility test is correct.

\item Use mathematical induction to prove that if  $n$  is a nonnegative integer  then  
$10^n  \equiv \left( -1 \right)^{n} \pmod {11}$.  Hence, for congruence classes modulo 11,  if  $n$  is a nonnegative integer, then  
$[ {10^n } ] = [ \left( -1 \right)^{n} ]$.
\label{exer:poweroftenmod11}%

\item Let  $n \in \mathbb{N}$ and assume  
\label{exer:divtest11}%
\[
n = \left( {a_k  \times 10^k } \right) + \left( {a_{k - 1}  \times 10^{k - 1} } \right) +  \cdots  + \left( {a_1  \times 10^1 } \right) + \left( {a_0  \times 10^0 } \right)\!.
\]
Use the result in Exercise~(\ref{exer:poweroftenmod11}) to help prove each of the following:

\begin{enumerate}
  \item $n \equiv \sum\limits_{j = 0}^k {\left( { - 1} \right)^j a_j } \pmod {11}$.

  \item $[ n ] = [ {\sum\limits_{j = 0}^k {\left( { - 1} \right)^j a_j } }         ]$, using congruence classes modulo 11.


  \item $11$ divides $n$  if and only if  
        $11$ divides $\sum\limits_{j = 0}^k {\left( { - 1} \right)^j a_j }$.
\index{divisibility test!for 11}%
\end{enumerate}

\item\label{exer:sec74cong3} 
\begin{enumerate} \yitem Prove the following proposition:
\begin{list}{}
\item For all $[ a ], [ b ] \in \mathbb{Z}_3$,  if  
$[ a ]^2 + [ b ]^2 = [ 0 ]$, then $[ a ] = 0$ and 
$[ b ] = [ 0 ]$. \label{exer:sec74cong3a}
\end{list}

\item Use Exercise~(\ref{exer:sec74cong3a}) to prove the following proposition:
\begin{list}{}
\item Let  $a, b \in \mathbb{Z}$.  If  
$\left( {a^2  + b^2 } \right) \equiv 0 \pmod 3$, then  
$a \equiv 0 \pmod 3$  and  $b \equiv 0 \pmod 3$.
\end{list}\label{exer:sec74cong3b}

\item Use Exercise~(\ref{exer:sec74cong3b}) to prove the following proposition:

\begin{list}{}
\item For all  $a, b \in \mathbb{Z}$, if  3  divides  $\left( {a^2  + b^2 } \right)$, then  3              divides  $a$  and  3  divides  $b$.
\end{list}
\end{enumerate}

\item Prove the following proposition:

\begin{list}{}
\item For each  $a \in \mathbb{Z}$, if there exist integers  $b$  and  $c$  such that  
      $a = b^4  + c^4 $, then the units digit of  $a$  must be 0, 1, 2, 5, 6, or 7.
\end{list}

\item Is the following proposition true or false?  Justify your conclusion. \label{exer:sec74-15}
\begin{list}{}
  \item Let  $n \in \mathbb{Z}$.  If  $n$ is odd, then  $8 \mid \left( {n^2  - 1} \right)$.          \hint  What are the possible values of  $n$  (mod 8)?
\end{list}

\item Prove the following proposition:
\begin{list}{}
\item Let  $n \in \mathbb{N}$.  If  $n \equiv 7 \pmod 8$, then  $n$  is not the sum of three squares.  That is, there do not exist natural numbers  $a$, $b$, and  $c$ such that  \linebreak
$n = a^2  + b^2  + c^2 $.
\end{list}
\end{enumerate}


\subsection*{Explorations and Activities}
\setcounter{oldenumi}{\theenumi}
\begin{enumerate} \setcounter{enumi}{\theoldenumi} 
%\item \textbf{Divisibility Tests for 3 and 4}.  \label{exer:otherdivtests}
%Let  $n \in \mathbb{N}$ and let  $s( n )$ denote the sum of the digits of  $n$.  So if we write
%\[
%n = \left( {a_k  \times 10^k } \right) + \left( {a_{k - 1}  \times 10^{k - 1} } \right) +  \cdots  + \left( {a_1  \times 10^1 } \right) + \left( {a_0  \times 10^0 } \right), 
%\]
%then $s( n ) = a_k  + a_{k - 1}  +  \cdots  + a_1  + a_0$.
%
%\begin{enumerate}
%
%\item Prove each of the following:\label{A:otherdivtests2}
%
%\begin{enumerate}
%  \item $[ n ] = [ {10a_1  + a_0 } ]$, using congruence classes modulo 4.
%
%  \item $n \equiv \left( {10a_1  + a_0 } \right) \pmod 4$.
%
%  \item $4 \mid n$  if and only if  $4 \mid \left( {10a_1  + a_0 } \right)$.
%\end{enumerate}
%\index{divisibility test!for 4}%
%
%\item Use mathematical induction to prove that if  $n$  is a nonnegative integer, then  
%      $10^n  \equiv 1 \pmod 3$.  Hence, for congruence classes modulo 3,  if $n$  is a nonnegative integer, then  $[ {10^n } ] = [ 1 ]$.
%
%\item Prove each of the following:
%
%%\enlargethispage{\baselineskip}
%\begin{enumerate}
%  \item $[ n ] = [ {S( n )} ]$, using congruence classes modulo 3.
%
%  \item $n \equiv S( n ) \pmod 3$.
%
%  \item $3 \mid n$  if and only if \, $3 \mid S( n )$. \label{T:sumofdigitsmod9-3}
%\end{enumerate}
%\index{divisibility test!for 3}%
%\end{enumerate}


\item \textbf{Using Congruence Modulo 4}.  \label{exer:usingcongruencemod4}
The set  $\mathbb{Z}_n $ is a finite set, and hence one way to prove things about  $\mathbb{Z}_n $
is to simply use the  $n$  elements in  $\mathbb{Z}_n $ as the  $n$  cases for a proof using cases.   For example, if  $n \in \mathbb{Z}$, then in  $\mathbb{Z}_4 $, 
$[ n ] = [ 0 ]$, $[ n ] = [ 1 ]$, 
$[ n ] = [ 2 ]$, or $[ n ] = [ 3 ]$.

\begin{enumerate}
\item Prove that if  $n \in \mathbb{Z}$, then in  $\mathbb{Z}_4 $,  
$[ n ]^2  = [ 0 ]$  or  $[ n ]^2  = [ 1 ]$.  Use this to conclude that in  $\mathbb{Z}_4 $,  $[ {n^2 } ] = [ 0 ]$  or  $[ {n^2 } ] = [ 1 ]$.

\item Translate the equations  $\left[ {n^2 } \right] = [ 0 ]$ and   
$\left[ {n^2 } \right] = [ 1 ]$ in  $\mathbb{Z}_4 $ into congruences 
modulo 4.

\item Use a result in Exercise~(\ref{exer:divtest4}) to determine the value of  $r$  so that  
$r \in \mathbb{Z}$,  $0 \leq r < 3$, and  
\[
{104 \  257 \ 833 \ 259} \equiv r \pmod 4\!.
\]
That is,  $[ {104 \ 257 \ 833 \ 259} ] = [ r ]$ in  $\mathbb{Z}_4 $.

\item Is the natural number  $104 \ 257 \ 833 \ 259$  a perfect square?  Justify your conclusion.

\end{enumerate}


\end{enumerate}

\hbreak

\endinput


