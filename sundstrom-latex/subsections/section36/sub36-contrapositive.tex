\subsection*{Proof of a Conditional Statement \mathversion{bold} $\left( P \to Q \right)$ Using the Contrapositive}
%\textbf{{Proof of} {\mathversion{bold} $P \to Q$} Using the Contrapositive}
\index{contrapositive}%
\index{proof!contrapositive}%

\begin{itemize}
\item \textbf{When is it indicated}?  This type of proof is often used when both the hypothesis and the conclusion are stated in the form of negations.  This often works well if the conclusion contains the operator ``or'';  that is, if the conclusion is in the form of a disjunction.  In this case, the negation will be a conjunction.

\item \textbf{Description of the process}.  We prove the logically equivalent statement  
$\mynot  Q \to \mynot  P$.  The forward-backward method is used to prove  
$\mynot  Q \to \mynot  P$.  That is, we work forward from   $\mynot  Q$  and backward from  
$\mynot P$.  

\item \textbf{Why the process makes sense}.  When we prove  $\mynot  Q \to \mynot  P$, we are also proving  $P \to Q$  because these two statements are logically equivalent.  When we prove the contrapositive of   $P \to Q$, we are doing a direct proof of  $\mynot  Q \to \mynot P$.  So we assume  $\mynot  Q$  because, when doing a direct proof, we assume the hypothesis, and  $\mynot  Q$  is the hypothesis of the contrapositive.  We must show  $\mynot  P$  because it is the conclusion of the contrapositive.
\end{itemize}

\endinput
