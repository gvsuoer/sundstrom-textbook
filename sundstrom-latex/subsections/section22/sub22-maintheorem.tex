
As we will see, it is often difficult to construct a direct proof for a conditional statement of the form $P \to \left( {Q \vee R} \right)$.  The logical equivalency in Progress Check~\ref{pr:workingeq2} gives us another way to attempt to prove a statement of the form  \linebreak
$P \to \left( {Q \vee R} \right)$.  The advantage of the equivalent form,  
$\left( {P \wedge \mynot  Q} \right) \to R$, 
is that we have an additional assumption, $\mynot  Q$,  in the hypothesis.  This gives us more information with which to work.

%\begin{itemize}
%  \item The first logical equivalency is basically the contrapositive of  \\
%$P \to \left( {Q \vee R} \right)$.
%  \item The advantage of the equivalent form,  $\left( {P \wedge \mynot  Q} \right) \to R
%$, is that we have an additional assumption, $\mynot  Q$,  in the hypothesis.  This just gives us more information with which to work.
%\end{itemize}

Theorem~\ref{T:logequiv} states some of the most frequently used logical equivalencies used when writing mathematical proofs.  

%\pagebreak
\begin{nametheorem}[\textbf{Important Logical Equivalencies}]\label{T:logequiv} \hfill \\
For statements $P$, $Q$, and $R$,
$$
\BeginTable
\BeginFormat
| l | l |
\EndFormat

" \textbf{De Morgan's Laws}   "  $\mynot  \left( {P \wedge Q} \right) \equiv \;\mynot  P \vee \mynot  Q$ " \\ 
 "                       " $\mynot  \left( {P \vee Q} \right) \equiv \;\mynot  P \wedge \mynot  Q$ " \\+02
\index{De Morgan's Laws!for statements}%

" \textbf{Conditional Statements} "  $P \to Q \equiv \;\mynot  Q \to \mynot  P$  (contrapositive) " \\
"   "  $P \to Q \equiv \;\mynot  P \vee Q$ " \\
"   "  $\mynot  \left( {P \to Q} \right) \equiv P \wedge \mynot  Q$ " \\+02

" \textbf{Biconditional Statement} "  
$\left( {P \leftrightarrow Q} \right) \equiv \left( {P \to Q} \right) \wedge \left( {Q \to P} \right)$ " \\+02

" \textbf{Double Negation} "  $\mynot  \left( {\mynot  P} \right) \equiv P$ \\+02

" \textbf{Distributive Laws} "  
$P \vee \left( {Q \wedge R} \right) \equiv \left( {P \vee Q} \right) \wedge \left( {P \vee R} \right)$  " \\
"   "  $P \wedge \left( {Q \vee R} \right) \equiv \left( {P \wedge Q} \right) \vee \left( {P \wedge R} \right)$ " \\+02

" \textbf{Conditionals with}  " $P \to \left( {Q \vee R} \right) \equiv \left( {P \wedge \mynot  Q} \right) \to R$ " \\
" \textbf{Disjunctions}    "  $\left( {P \vee Q} \right) \to R\; \equiv \;\left( {P \to R} \right) \wedge \left( {Q \to R} \right)$ " \\
\EndTable
$$
\index{distributive laws!for statements}%
\end{nametheorem}


%\begin{theorem}[Important Logical Equivalencies] \hfill \label{T:logequiv}
%\begin{tabular}[h]{l l}
%\textbf{De Morgan's Laws}   &  $\mynot  \left( {P \wedge Q} \right) \equiv \;\mynot  P \vee \mynot  Q$ \\ 
%                        & $\mynot  \left( {P \vee Q} \right) \equiv \;\mynot  P \wedge \mynot  Q$ \\
%  &  \\
%\textbf{Conditional Statements} &  $P \to Q \equiv \;\mynot  Q \to \mynot  P$  (contrapositive) \\
%   &  $P \to Q \equiv \;\mynot  P \vee Q$ \\
%   &  $\mynot  \left( {P \to Q} \right) \equiv P \wedge \mynot  Q$ \\
%  &  \\
%%\end{tabular}
%
%%\begin{tabular}[h]{l l}
%\textbf{Biconditional Statement} &  
%$\left( {P \leftrightarrow Q} \right) \equiv \left( {P \to Q} \right) \wedge \left( {Q \to P} \right)$ \\
%  &  \\
%\textbf{Double Negation} &  $\mynot  \left( {\mynot  P} \right) \equiv P$ \\
%  &  \\
%\textbf{Distributive Laws} &  
%$P \vee \left( {Q \wedge R} \right) \equiv \left( {P \vee Q} \right) \wedge \left( {P \vee R} \right)$  \\
%  &  $P \wedge \left( {Q \vee R} \right) \equiv \left( {P \wedge Q} \right) \vee \left( {P \wedge R} \right)$ \\
%  &  \\
%\textbf{Conditionals with}  & $P \to \left( {Q \vee R} \right) \equiv \left( {P \wedge \mynot  Q} \right) \to R$ \\
%\textbf{Disjunctions}    &  $\left( {P \vee Q} \right) \to R\; \equiv \;\left( {P \to R} \right) \wedge \left( {Q \to R} \right)$  \\
%\end{tabular}
%\end{theorem}

We have already established many of these equivalencies.  Others will be established in the exercises.


\hbreak
